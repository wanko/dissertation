\chapter{Introduction}\label{sec:introduction}
\begin{itemize}
  \item
   Answer Set Programming (ASP~\cites{baral02a,lifschitz08b,gekakasc12a}) is a declarative approach to solving knowledge-intensive combinatorial (optimization) problems.
  \item 
   Its main draw is a powerful and human-readable first-order language based on rules
   that is able to express problems up to the second order of the polynomial hierarchy
   and supports concepts like defaults and reachability naturally.
  \item
   That is, ASP is a non-monotonic reasoning framework.
  \item
   The modelling language allows for the formulation of uniform problem encodings of even complex problems in but a few lines of code.
  \item 
   Those encodings can then be applied to specific problem instances.
  \item 
   A uniform problem encoding in the well-defined language of ASP also offers so-called elaboration tolerance,
   that is, a small change in the requirements of the problem induces a small change in the encoding of said problem.
  \item
   Both encoding and instance can then be solved by powerful ASP systems that map them to their answer sets~\cite{gellif88b}.
  \item
   The most common approach to ASP solving is a grounding step followed by solving step.
  \item 
   That is, the problem encoding is instantiated and transformed into a propositional representation
   and then processed by a search algorithm based on SAT solving~\cite{baseseti09a}.
  \item
   ASP has been applied to a wide variety of problems, among them 
   systems biology~\cite{kascsivi13a},
   planning~\cite{gekaknsc11a},
   package configuration~\cite{gekasc11c},
   a NASA space shuttle controller~\cite{nobagewaba01a}, or
   scheduling at the Swiss railway company~\cite{abjoossctowa21a}.
  \item
   However, many real-life applications have a heterogeneous nature that requires capabilities beyond ASP.
  \item
   Consider for instance,
   bio-informatics~\cite{frscscsiwa18a},
   hardware synthesis~\cite{newascha18b},
   or train scheduling~\cite{abjoossctowa19a}.
  \item
   All of the above display aspects suitable to ASP in that topological requirements, like reachability, have to be met,
   but they also require more fine-grained calculations to address scheduling or biological equilibria.
  \item
   The short-comings of plain ASP in this case are two-fold,
   first, language constructs of ASP do not naturally support these inferences,
   second, the grounding process leads to an explosion in problem size or is even impossible when considering equations over integers or reals.
  \item
   Mixed solving technology has been proposed to accommodate those short-comings inspired by the neighboring field of Satisfiability modulo Theories (SMT;~\cite{niolti06a}).
  \item
   For instance, Constraint ASP (CASP;~\cite{lierler14a}) combines ASP with Constraint Processing (CP;~\cite{dechter03a}).
\end{itemize}

\begin{itemize}
  \item
  \begin{figure}
  \centering
  \begin{tikzpicture}[
      x=3cm,y=2.5cm,
      >=stealth',
      big box/.style={draw,minimum width=2cm,minimum height=1.3cm,inner sep=0},
      small box/.style={draw,minimum width=2cm,minimum height=1cm,inner sep=0},
      auto,
    ]
    \begin{scope}[every node/.style={big box}]
      \node (a) at (0,1) {Problem};
      \node (b) at (0,0) {\shortstack{Logic\\Program}};
      \node (e) at (3,0) {\shortstack{Stable\\Models}};
      \node (f) at (3,1) {Solution};
    \end{scope}
    \begin{scope}[every node/.style={small box}]
      \node (c) at (1,0) {Grounder};
      \node (d) at (2,0) {Solver};
    \end{scope}
    \node[above left, anchor=south west,inner sep=0,yshift=.1cm] (label) at (c.north west){\clingo};
    \node[draw,fit=(c) (d) (label),inner sep=.2cm] {};
    \path[->]
      (a) edge node [yshift=.2cm] {Modeling} (b)
      (b) edge (c)
      (c) edge node [swap,yshift=-.75cm] {Solving} (d)
      (d) edge (e)
      (e) edge node [yshift=.2cm] {Interpreting} (f);
  \end{tikzpicture}
  \caption{ASP solving process}%
  \label{fig:asp-in-a-nutshell}
\end{figure}

  The ASP solving process can be summarized by the steps depicted in \cref{fig:asp-in-a-nutshell}.
  \item \cite{cafascwa20a}
  \begin{itemize}
    \item
    When it comes to extending ASP with foreign reasoning methods,
the design often follows the algorithmic framework of SMT and leaves semantic aspects behind.
For instance, a popular approach is to combine ASP with Constraint Processing (CP;~\cite{dechter03a}),
also referred to as Constraint ASP (CASP;~\cite{lierler14a}).
This blends non-monotonic aspects of ASP with monotonic ones of CP but
fails to provide a homogeneous representational framework.
In particular, the knowledge representation capabilities of ASP, like defaults and aggregates, remain inapplicable to constraint variables.
    \item
    We addressed this in~\cite{cakaossc16a} by integrating ASP and CP in the uniform semantic framework called \emph{Here-and-There with constraints}~(\HTC).
    \item
This relies upon the logic of Here-and-There (\HT;~\cite{heyting30a}) along with
its non-monotonic extension, called Equilibrium Logic~\cite{pearce96a}.
    \item
    As an example,
consider the hybrid ASP rule%
\footnote{We put dots on top of braces, viz.~``$\agg{ \dotsc }$'', to indicate \emph{multisets}.}
\begin{gather}\label{eq:tax.sum}
  \mathit{total}(R) := \mathit{sum}\agg{ \, \mathit{tax}(P) : \mathit{lives}(P,R) \,  } \ \leftarrow \ \mathit{region}(R)
  \quad
\end{gather}
gathering the total tax revenue of each region $R$ by summing up the tax liabilities of the region's residents, $P$.
%
As a matter of fact,
the calculation of tax liability is highly complex, and relies on defaults and discounts to address incomplete information,
which nicely underlines the need for non-monotonic constraint variables.
  \end{itemize}
  \item \cite{cafascwa20b}
    \begin{itemize}
      \item 
      This integration is however often done in system-oriented ways that leave semantic aspects behind.
    \end{itemize}
  \item \cite{cafascwa21a}
    \begin{itemize}
      \item 
      However, over the same period, the syntax of logic programs has been continuously enriched,
      equipping ASP with a highly attractive modeling language.
      \item
      This development also brought about much more intricate semantics,
      as nicely reflected by the mathematical apparatus used in the original definition~\cite{gellif88b} and the one for
      capturing the input language of the ASP system \clingo~\cite{gehakalisc15a}.
      \item
      With the growing range of ASP applications in academia and industry~\cite{fafrsctate18a}, % ergele16a,
      we also witness the emergence of more and more hybrid ASP systems~\cite{lierler14a,kascwa17a},
      similar to the raise of SMT solving from plain SAT solving~\cite{baseseti09a}.
      \item 
      From a general perspective, the resulting paradigm of ASP modulo theories (\AMT) can be seen as a refinement of ASP,
      in which an external theory certifies some of a program's stable models.
      \item
      This idea has meanwhile been adapted to the setting of ASP in various ways, most prominently in
      Constraint ASP~\cite{lierler14a} extending ASP with linear equations over integers in various ways
      \cite{elposo04a,geossc09a,drewal10a,ostsch12a,roeireri15a,bakaossc16a,jakaosscscwa17a}.
      \item
      Beyond this, ASP systems like \clingo\ or \dlvhex~\cite{eigeiakarescwe18a} leave the choice of the specific
      theory largely open.
      \item 
      As attractive as this generality may be from an implementation point of view,
      it complicates the development of generic semantics that are meaningful to existing systems.
      \item
      We address this issue and show how the Logic of Here-and-There with constraints (\HTC;~\cite{cakaossc16a})
      can be used as a semantics for \clingo's theory reasoning framework.
      \item
      Thus, just like the plain Logic of Here-and-There (\HT;~\cite{heyting30a,pearce96a}) serves as the logic foundations of ASP,
      \HTC\ extends this to \AMT.
    \end{itemize}
\end{itemize}

\section{Selected contributions}

\section{Overall contributions}

\begin{itemize}
  \item list all my contributions + how others are using them
\end{itemize}

\begin{itemize}
  \item check the Promotionsordnung for requirements
  \item my papers: \cites{}
\end{itemize}
