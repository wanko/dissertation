\chapter{Introduction}\label{sec:introduction}

\begin{itemize}
  \item 
  Answer Set Programming (ASP) is a popular approach
  to solve knowledge-intense search and optimization problems in a declarative way~\cites{baral02a,gekakasc12a}.
  \item
  It is based on the non-monotonic stable model semantics
  tailored to support both closed and open world reasoning.
  \item 
  This makes ASP applicable to a wide range of reasoning tasks including tasks involving incomplete information.
  It features a simple yet powerful rule-based language
  that can express all problems up to the second level of the polynomial hierarchy.
  \item
  The language allows us to write uniform problem specifications that can be used to solve specific problem instances.
  \item
  Even complex problems can typically be modeled with a small number of generic rules using first order variables.
  \item
  Another important aspect of ASP is it's elaboration tolerance.
  It is often possible to add new rules to a problem specification or modify a few of them
  to adapt to changing requirements throughout the development of an application.
  \item 
  Both problem specification and instance can then be solved by high performance ASP systems.
  There are numerous applications in various domains that have successfully applied {ASP}.
  This includes, for example,  systems biology~\cite{kascsivi13a},
  planning~\cite{gekaknsc11a},
  package configuration~\cite{gekasc11c},
  a NASA space shuttle controller~\cite{nobagewaba01a}, or
  scheduling at the Swiss railway company~\cite{abjoossctowa21a}.
  \item
  \begin{figure}
  \centering
  \begin{tikzpicture}[
      x=3cm,y=2.5cm,
      >=stealth',
      big box/.style={draw,minimum width=2cm,minimum height=1.3cm,inner sep=0},
      small box/.style={draw,minimum width=2cm,minimum height=1cm,inner sep=0},
      auto,
    ]
    \begin{scope}[every node/.style={big box}]
      \node (a) at (0,1) {Problem};
      \node (b) at (0,0) {\shortstack{Logic\\Program}};
      \node (e) at (3,0) {\shortstack{Stable\\Models}};
      \node (f) at (3,1) {Solution};
    \end{scope}
    \begin{scope}[every node/.style={small box}]
      \node (c) at (1,0) {Grounder};
      \node (d) at (2,0) {Solver};
    \end{scope}
    \node[above left, anchor=south west,inner sep=0,yshift=.1cm] (label) at (c.north west){\clingo};
    \node[draw,fit=(c) (d) (label),inner sep=.2cm] {};
    \path[->]
      (a) edge node [yshift=.2cm] {Modeling} (b)
      (b) edge (c)
      (c) edge node [swap,yshift=-.75cm] {Solving} (d)
      (d) edge (e)
      (e) edge node [yshift=.2cm] {Interpreting} (f);
  \end{tikzpicture}
  \caption{ASP solving process}%
  \label{fig:asp-in-a-nutshell}
\end{figure}

  The ASP solving process can be summarized by the steps depicted in \cref{fig:asp-in-a-nutshell}.
\end{itemize}

\section{Selected contributions}

\section{Overall contributions}

\begin{itemize}
  \item list all my contributions + how others are using them
\end{itemize}

\begin{itemize}
  \item check the Promotionsordnung for requirements
  \item my papers: \cites{}
\end{itemize}
