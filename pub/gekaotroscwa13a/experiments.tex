
\section{Experiments}\label{sec:experiments}

We implemented our approach as a dedicated heuristic module within the ASP solver \textit{clasp}
(2.1; available at \cite{hclasp}).
We consider \textit{moms} \cite{pretolani96a} as initial heuristic $h_0$ and
\textit{vsids} \cite{momazhzhma01a} as heuristic function $h_i$.
Accordingly, the sign heuristic \textsl{s} is set to the one associated with \textit{vsids}.
As base configuration, we use \textit{clasp} with options \texttt{--heu=vsids} and \texttt{--init-moms}.
%
To take effect,
the heuristic atoms as well as their contained atoms must be made visible to the solver via \texttt{\#show} directives.
Once the option \texttt{--heu=domain} is passed to \texttt{clasp},
it extracts the necessary information from the symbol table and applies the heuristic modifications
when it comes to non-deterministic assignments.
%
Our experiments ran under Linux on dual Xeon E5520 quad-core processors with $2.26$GHz and $48$GB RAM.
Each run was restricted to 600s CPU time.
Timeouts account for 600s and performed choices.

% ------------------------------------------------------------
\newcommand{\rvsids}[4]{\multicolumn{3}{c|}{#1$s$ (#2)}}
\newcommand{\cvsids}[4]{\multicolumn{3}{c|}{#4}}
\newcommand{\domheu}[3]{#1\%&(#2)&#3\%}
% ------------------------------------------------------------
\begin{table}[t]
  \centering\small
  \begin{tabular}{|@{\,}r@{\,}|@{}r@{}@{}r@{}@{}r@{}|@{}r@{}@{}r@{}@{}r@{}|@{}r@{}@{}r@{}@{}r@{}|}
    \hline
    \multicolumn{1}{|c|}{Setting}                    & \multicolumn{3}{@{}c@{}|}{\textit{Labyrinth}}& \multicolumn{3}{@{}c@{}|}{\textit{Sokoban}}   & \multicolumn{3}{@{}c@{}|}{\textit{Hanoi Tower}} \\
    \hline
    \multicolumn{1}{|@{\;}l@{}|}{\textit{base configuration}} & \rvsids{9,108}{14}{5,908,451}{24,545,667}    & \rvsids{2,844}{3}{13,799,878}{19,371,267}     & \rvsids{9,137}{11}{34,126,406}{41,016,235}      \\   
                                                     & \cvsids{9,108}{14}{5,908,451}{24,545,667}    & \cvsids{2,844}{3}{13,799,878}{19,371,267}     & \cvsids{9,137}{11}{34,126,406}{41,016,235}      \\
    \hline
    \hpre{a}{\texttt{init}}{\texttt{2}}              & \domheu{95}{12}{94}                          & \domheu{91}{\textbf{1}}{84}                   & \domheu{85}{9}{89}                              \\
    \hpre{a}{\texttt{factor}}{\texttt{4}}            & \domheu{\textbf{78}}{\textbf{8}}{30}         & \domheu{120}{\textbf{1}}{107}                 & \domheu{109}{11}{110}                           \\
    \hpre{a}{\texttt{factor}}{\texttt{16}}           & \domheu{\textbf{78}}{10}{23}                 & \domheu{120}{\textbf{1}}{107}                 & \domheu{109}{11}{110}                           \\
    \hpre{a}{\texttt{level}}{\texttt{1}}             & \domheu{90}{12}{\textbf{5}}                  & \domheu{119}{2}{91}                           & \domheu{126}{15}{120}                           \\\cline{1-1}
    \hpre{f}{\texttt{init}}{\texttt{2}}              & \domheu{103}{14}{123}                        & \domheu{\textbf{74}}{2}{\textbf{71}}          & \domheu{97}{10}{109}                            \\
    \hpre{f}{\texttt{factor}}{\texttt{2}}            & \domheu{98}{12}{49}                          & \domheu{116}{3}{134}                          & \domheu{\textbf{55}}{\textbf{6}}{\textbf{70}}   \\
    \hpre{f}{\texttt{sign}}{\texttt{-1}}             & \domheu{94}{13}{89}                          & \domheu{105}{\textbf{1}}{100}                 & \domheu{92}{12}{92}                             \\
    \hline
  \end{tabular}
  \caption{Selection from evaluation of heuristic modifiers}
  \label{tab:modifiers}
\end{table}
% ------------------------------------------------------------
%
To begin with,
we report on a systematic study comparing single heuristic modifications.
A selection of best results is given in Table~\ref{tab:modifiers};
full results are available at~\cite{hclasp}.
We focus on well-known ASP planning benchmarks in order to contrast heuristic modifications on comparable problems:
\textit{Labyrinth}, \textit{Sokoban}, and \textit{Hanoi Tower}, each comprising 32 instances from the third ASP competition~\cite{contest11a}.%
\footnote{All instances are satisfiable except for one third in \textit{Sokoban}.}
We contrast the aforementioned base configuration with 38 heuristic modifications,
(separately) promoting the choice of actions (\emph{a}) and fluents (\emph{f}) via the heuristic modifiers
\texttt{factor} (1,2,4,8,16),
\texttt{init} (2,4,8,16),
\texttt{level} (1,-1),
\texttt{sign} (1,-1),
as well as attributing values to \texttt{factor}, \texttt{init}, and \texttt{level} by ascending and descending time points.
%
The first line of Table~\ref{tab:modifiers} gives the sum of times, timeouts, and choices obtained by the base configuration on all 32 instances of each problem class.
The results of the two configurations using \texttt{factor,1} differ from these figures in the low per mille range, demonstrating that the infrastructure supporting heuristic
modifications does not lead to a loss in performance.
The seven configurations in Table~\ref{tab:modifiers} yield best values in at least one category (indicated in boldface).
We express the accumulated times and choices as percentage wrt the base configuration; timeouts are total.
We see that the base configuration can always be dominated by a heuristic modification.
However, the whole spectrum of modifiers is needed to accomplish this.
In other words, there is no dominating heuristic modifier and each problem class needs a customized heuristic.
Looking at \textit{Labyrinth}, we observe that a preferred choice of action occurrences ($a$) pays off.
The stronger this is enforced, the fewer choices are made.
However, the extremely low number of choices with \texttt{level} does not result in less time or timeouts
(compared to a ``lighter'' \texttt{factor}-based enforcement).
While with \texttt{level} \emph{all} choices are made on heuristically modified atoms,
both \texttt{factor}-based modifications result in only 43\% such choices and thus leave much more room to the solver's heuristic.
For a complement, \textit{a},\texttt{init},2 as well as the base configuration (with \textit{a},\texttt{factor},1) 
make 14\% of their choices on heuristically modified atoms
(though the former produces in total 6\% less choices than the latter).
Similar yet less extreme behaviors are observed on the two other classes.
With \textit{Hanoi Tower}, a slight preference of fluents yields a strictly dominating configuration,
whereas no dominating improvement was observed with \textit{Sokoban}.

% ------------------------------------------------------------
\newcommand{\data}[2]{&\,\ignorespaces#1$s$&(\ignorespaces#2)\,}
% ------------------------------------------------------------
\begin{table}[t]
  \centering\small
  \begin{tabular}{|@{}l@{}|@{}r@{}@{}r@{}|@{}r@{}@{}r@{}|@{}r@{}@{}r@{}@{}r@{}@{}r@{}|}
    \hline
    \multicolumn{1}{|@{}c@{}|}{Setting} & \multicolumn{2}{@{}c@{}|}{\textit{Diagnosis}} & \multicolumn{2}{@{}c@{}|}{\textit{Expansion}} & \multicolumn{2}{@{}c@{}}{\textit{Repair (H)}} & \multicolumn{2}{@{}c@{}|}{\textit{Repair (S)}}\\
    \hline
    \textit{base config.}            \data{        111.1}{        115}\data{        161.5}{      100}\data{       101.3}{       113}\data{        33.3}{        27}\\
    \hline
    \texttt{s,-1}                    \data{        324.5}{        407}\data{         7.6}{         3}\data{         8.4}{         5}\data{         3.1}{\textbf{0}}\\
    \texttt{s,-1} \, \texttt{f,2}    \data{        310.1}{        387}\data{         7.4}{\textbf{2}}\data{         3.5}{\textbf{0}}\data{         3.2}{         1}\\
    \texttt{s,-1} \, \texttt{f,8}    \data{        305.9}{        376}\data{         7.7}{\textbf{2}}\data{         3.1}{\textbf{0}}\data{         2.9}{\textbf{0}}\\
    \texttt{s,-1} \, \texttt{l,1}    \data{\textbf{76.1}}{\textbf{83}}\data{\textbf{6.6}}{\textbf{2}}\data{\textbf{0.8}}{\textbf{0}}\data{         2.2}{         1}\\
\multicolumn{1}{|r@{}|}{\texttt{l,1}}\data{         77.3}{         86}\data{        12.9}{         5}\data{         3.4}{\textbf{0}}\data{\textbf{2.1}}{\textbf{0}}\\
    \hline
  \end{tabular}
  \caption{Abductive problems with optimization}
  \label{tab:opt}
\end{table}
% ------------------------------------------------------------
%
Next, we apply our heuristic approach to problems using abduction in combination with a \texttt{\#minimize} statement
minimizing the number of abducibles.
We consider
Circuit \textit{Diagnosis},
Metabolic Network \textit{Expansion},
and
Transcriptional Network \textit{Repair} (including two distinct experiments, \textit{H} and \textit{S}).
The first uses the ISCAS-85 benchmark circuits along with test cases generated as in \cite{sidiqqi11a};
this results in 790 benchmark instances.
The second one considers the completion of the metabolic network of \emph{E.coli} with reactions from \textit{MetaCyc} in view of generating target from seed metabolites~\cite{schthi09a}.
We selected the 450 most difficult benchmarks in the suite.
Finally, we consider repairing the transcriptional network of \emph{E.coli} from \textit{RegulonDB} in view of two distinct experiment series~\cite{geguivscsithve10a}.
Selecting the most difficult triple repairs provided us with 1000 instances.
%
Our results are summarized in Table~\ref{tab:opt}.
Each entry gives the average runtime and number of timeouts.
% Otherwise the experimental setting is as above.
%
Here, heuristic modifiers apply only to abducibles subject to minimization.
%
For supporting minimization,
we assign false to such abducibles (\texttt{s,-1})%
\footnote{Assigning \true\ instead leads to a deterioration of performance.}
and gradually increase the bias of their choice by imposing \texttt{factor} 2 and 8 (\texttt{f})
or enforce it via a \texttt{level} modifier (\texttt{l,1}).
%
The second last setting%
\footnote{This corresponds to using \hpre{a}{\texttt{false}}{1} for an abducible $a$.}
in Table~\ref{tab:opt} is the winner, leading to speedups of one to two orders of magnitude over the base configuration.
Interestingly, merely fixing the sign heuristics to \false\ leads at first to a deterioration of performance on \textit{Diagnosis} problems.
This is finally overcome by the constant improvement observed by gradually strengthening the bias of choosing abducibles.
The stronger the preference for abducibles, the faster the solver converges to an optimum solution.
This limited experiment already illustrates that sometimes the right combination of heuristic
modifiers yields the best result.

Finally, let us consider true PDDL planning problems.
For this, we selected 20 instances from the % and \textit{Freecell'02} in view of \textit{Freecell'00}.}
STRIPS domains of the 2000 and 2002 planning competition~\cite{icaps-competition}.%
\footnote{We discard \textit{Schedule'00} due to grounding issues.}
In turn, we translated these PDDL instances into facts via \textit{plasp}~\cite{gekaknsc11a}
and used a simple planning encoding with 15 different plan lengths (\verb+l=5,10,..,75+) to generate 3000 ASP instances.
%
Inspired yet different from \cite{rintanen12a}, we devised a dynamic heuristic that aims at propagating fluents' truth values backwards in time.
Attributing levels via \texttt{l-T+1} aims at proceeding depth-first from the goal fluents.
\begin{lstlisting}
_h(holds(F,T-1),true, l-T+1) :- holds(F,T).
_h(holds(F,T-1),false,l-T+1) :-
        fluent(F), time(T), not holds(F,T).
\end{lstlisting}
% ------------------------------------------------------------
\newcommand{\pdata}[3]{&\,\ignorespaces#1$s$&(\ignorespaces#2/\ignorespaces#3)}
\newcommand{\sdata}[3]{&\,\ignorespaces#1$s$&(\ignorespaces#3)}
% ------------------------------------------------------------
\begin{table}[t]
  \centering\scriptsize
  \begin{tabular}{|@{}r@{\,}|@{}r@{}@{}r@{}|@{}r@{}@{}r@{}|@{}r@{}@{}r@{}|r@{}@{}r|}
    \hline
   \multicolumn{1}{|@{}c@{\,}|}{Problem} & \multicolumn{2}{@{}c@{}|}{\textit{base}} & \multicolumn{2}{@{}c@{}|}{\textit{base}+\texttt{\_h}} & \multicolumn{2}{@{}c@{}|}{\textit{base (SAT)}} & \multicolumn{2}{@{}c@{}|}{\textit{\textit{base}+\texttt{\_h} (SAT)}} \\
    \hline
    \textit{Blocks'00}    \pdata{134.4}{ 180}{  61}\pdata{  9.2}{ 239}{  3}\sdata{163.2}{180}{59}\sdata{ 2.6}{239}{0}\\
    \textit{Elevator'00}  \pdata{  3.1}{ 279}{   0}\pdata{  0.0}{ 279}{  0}\sdata{  3.4}{279}{ 0}\sdata{ 0.0}{279}{0}\\
    \textit{Freecell'00}  \pdata{288.7}{ 147}{ 115}\pdata{184.2}{ 194}{ 74}\sdata{226.4}{147}{47}\sdata{52.0}{194}{0}\\
    \textit{Logistics'00} \pdata{145.8}{ 148}{  61}\pdata{115.3}{ 168}{ 52}\sdata{113.9}{148}{23}\sdata{15.5}{168}{3}\\
    \hline
    \textit{Depots'02}    \pdata{400.3}{  51}{ 184}\pdata{297.4}{ 115}{135}\sdata{389.0}{ 51}{64}\sdata{61.6}{115}{0}\\
    \textit{Driverlog'02} \pdata{308.3}{ 108}{ 143}\pdata{189.6}{ 169}{ 92}\sdata{245.8}{108}{61}\sdata{ 6.1}{169}{0}\\
    \textit{Rovers'02}    \pdata{245.8}{ 138}{ 112}\pdata{165.7}{ 179}{ 79}\sdata{162.9}{138}{41}\sdata{ 5.7}{179}{0}\\
    \textit{Satellite'02} \pdata{398.4}{  73}{ 186}\pdata{229.9}{ 155}{106}\sdata{364.6}{ 73}{82}\sdata{30.8}{155}{0}\\
    \textit{Zenotravel'02}\pdata{350.7}{ 101}{ 169}\pdata{239.0}{ 154}{116}\sdata{224.5}{101}{53}\sdata{ 6.3}{154}{0}\\
    \hline
    \textit{Total}          \pdata{252.8}{1225}{1031}\pdata{158.9}{1652}{657}\sdata{187.2}{1225}{430}\sdata{17.1}{1652}{3}\\
    \hline
  \end{tabular}
  \caption{Planning Competition Benchmarks '00 and '02}
  \label{tab:plan}
\end{table}%
% ------------------------------------------------------------
%
Our results are given in Table~\ref{tab:plan}.
Each entry gives the average runtime along with the number of (solved satisfiable instances and) timeouts (in columns two and three).
% Otherwise the experimental setting is as above.
Our heuristic amendment (\textit{base}+\texttt{\_h}) greatly improves over the base configuration in terms of runtime and timeouts.
On the overall set of benchmarks, it provides us with 427 more plans and 374 less timeouts.
As already observed by \cite{rintanen12a}, the heuristic effect is stronger on satisfiable instances.
This is witnessed by the two last columns restricting results to 1655 satisfiable instances solved by either system setup.
Our heuristic extension allows us to reduce the total number of timeouts from 430 to 3;
the reduction in solving time would be even more drastic with a longer timeout.

Interestingly, the previous dynamic heuristic has no overwhelming effect on our initial ASP planning problems.
An improvement was only observed on \textit{Hanoi Tower} problems (being susceptible to choices on fluents),
viz.\ `{54\%}(7)\,\textbf{57\%}' in terms of the format used in Table~\ref{tab:modifiers}.
However, restricting the heuristic to positive fluents by only using the first rule gives a substantial
improvement, namely `\textbf{19\%}(\textbf{2})\,{66\%}', in terms of runtime and timeouts.
A direct comparison of both heuristics shows that, although the latter performs 15\% more choices, 
it encounters 75\% fewer conflicts than the former.

%%% Local Variables: 
%%% mode: latex
%%% TeX-master: "paper"
%%% End: 
