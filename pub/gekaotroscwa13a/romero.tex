%% $Id: paper.tex 34991 2013-04-06 07:51:22Z torsten $
%% $HeadURL: https://svn.cs.uni-potsdam.de/svn/reposWV/Papers/ASPDomHeu/trunk/paper.tex $
\documentclass[letterpaper]{article}
\usepackage{aaai}
\usepackage{times,helvet,courier}
\usepackage{amsmath,amssymb}

\setlength{\pdfpagewidth}{8.5in}
\setlength{\pdfpageheight}{11in}
\usepackage{verbatim}

\usepackage{listings}
\lstset{aboveskip=\smallskipamount,belowskip=\smallskipamount}
\lstset{basicstyle=\ttfamily\small}
%% \newcommand{\gringo}{\textit{gringo}}
\newcommand{\clasp}{\textit{clasp}}
\newcommand{\clingo}{\textit{clingo}}
\newcommand{\asprin}{\textit{asprin}}
\newcommand{\asap}{\textit{teaspoon}}
\newcommand{\piclasp}{\textit{piclasp}}

\newcommand{\code}[1]{\lstinline[basicstyle=\ttfamily]{#1}}

\newcommand{\lw}[1]{\smash{\lower1.ex\hbox{#1}}}
\newcommand{\llw}[1]{\smash{\lower3.ex\hbox{#1}}}

%\newcommand{\dataCL}[5]{%
%  \code{#1} & #3 & #5 & #4
%}
%\newcommand{\dataCS}[5]{%
%  #3 & #5 & #4
%}

\newenvironment{tableC}{%
  \scriptsize
  \tabcolsep = 0.6mm
  \begin{tabular}[t]{l|rlr|rlr|rlr|rlr|rlr}\hline
    \multicolumn{1}{l|}{\llw{Instance}} &
    \multicolumn{3}{c|}{UD1} &
    \multicolumn{3}{c|}{UD2} &
    \multicolumn{3}{c|}{UD3} &
    \multicolumn{3}{c|}{UD4} &
    \multicolumn{3}{c}{UD5} \\
    & 
    \multicolumn{1}{c}{Best} & & \multicolumn{1}{c|}{\emph{tea-}} & 
    \multicolumn{1}{c}{Best} & & \multicolumn{1}{c|}{\emph{tea-}} & 
    \multicolumn{1}{c}{Best} & & \multicolumn{1}{c|}{\emph{tea-}} & 
    \multicolumn{1}{c}{Best} & & \multicolumn{1}{c|}{\emph{tea-}} & 
    \multicolumn{1}{c}{Best} & & \multicolumn{1}{c}{\emph{tea-}} \\
    & 
    known & & \emph{spoon} & 
    known & & \emph{spoon} & 
    known & & \emph{spoon} & 
    known & & \emph{spoon} & 
    known & & \emph{spoon} \\
    \hline
  }{%
    \hline
  \end{tabular}
}

\newenvironment{tableB}{%
  \scriptsize
  \tabcolsep = 0.7mm
%  \begin{tabular}[t]{|l|c|r|l|l|l|}\hline
  \begin{tabular}[t]{lcrlll}\hline
    Instance &
    Formulation &
    Time (sec.)\\
    \hline
  }{%
    \hline
  \end{tabular}
}
\newenvironment{tableL}{%
  \scriptsize
  \tabcolsep = 0.7mm
  \begin{tabular}[t]{l|rrrrrrrr|r}\hline
    \lw{Instance} &
    \lw{Time (sec.)} &
    \multicolumn{6}{c}{The best utility vector} &
    The sum of  &
    The best of basic\\
    &
    &
    $(S_1,$ & $S_4,$ & $S_2,$ & $S_7,$ & $S_6,$ & $S_3)$ &
    utility vector &
    and optimized \\
    \hline
  }{%
    \hline
  \end{tabular}
}

%%% Local Variables:
%%% mode: latex
%%% TeX-master: "paper"
%%% End:
 
% - paper-specifics ------------------------------------------------------------
\newcommand{\hpredicate}{\texttt{\_h}}
\newcommand{\hpred}[4]{\ensuremath{\hpredicate(#1,{#2},#3,#4)}}
\newcommand{\hpre}[3]{\ensuremath{\hpredicate(#1,{#2},#3)}}

% - syntax ---------------------------------------------------------------------
\newcommand{\naf}[1]{\ensuremath{{\mathtt{not}\,{#1}}}} % {\ensuremath{{\sim\!{#1}}}}

\newcommand{\head}[1]{\ensuremath{\mathit{head}(#1)}}
\newcommand{\body}[1]{\ensuremath{\mathit{body}(#1)}}

\newcommand{\atom}[1]{\ensuremath{\mathit{atom}(#1)}}

\newcommand{\poslits}[1]{\ensuremath{{#1}^+}}
\newcommand{\neglits}[1]{\ensuremath{{#1}^-}}

\newcommand{\pbody}[1]{\poslits{\body{#1}}}
\newcommand{\nbody}[1]{\neglits{\body{#1}}}

\newcommand{\PRG}{\ensuremath{P}}

\newcommand{\atbody}[2]{\ensuremath{\mathit{body}_{#1}(#2)}}

\newcommand{\ground}[1]{\ensuremath{\mathit{grd}(#1)}}

% - semantics ------------------------------------------------------------------
\newcommand{\SM}[1]{\ensuremath{\mathit{SM}(#1)}}

% - operators ------------------------------------------------------------------
\newcommand{\Cn}[1]{\ensuremath{\mathit{Cn}(#1)}}
\newcommand{\reduct}[2]{\ensuremath{#1^{#2}}}

\newcommand{\To}[1]{\ensuremath{T_{#1}}}
\newcommand{\T}[2]{\To{#1}#2}
\newcommand{\TiO}[2]{\To{#2}^{#1}}
\newcommand{\Ti}[3]{\TiO{#1}{#2}#3}

\newcommand{\BF}[1]{\ensuremath{\mathit{BF}(#1)}}
\newcommand{\CF}[1]{\ensuremath{\mathit{CF}(#1)}}
\newcommand{\CFIF}[1]{\ensuremath{\overleftarrow{\mathit{CF}}(#1)}}
\newcommand{\CFFI}[1]{\ensuremath{\overrightarrow{\mathit{CF}}(#1)}}
\newcommand{\CFX}[1]{\ensuremath{\mathit{CF}^x(#1)}}

\newcommand{\loops}[1]{\ensuremath{\mathit{loop}(#1)}}\message{*** loop already defined ***}
\newcommand{\ES}[2]{\ensuremath{\mathit{ES}_{\!#2}(#1)}}
\newcommand{\EB}[2]{\ensuremath{\mathit{EB}_{\!#2}(#1)}}
\newcommand{\LFM}[2]{\ensuremath{\mathit{LF}_{\!#2}(#1)}}
\newcommand{\LF}[1]{\ensuremath{\mathit{LF}(#1)}}

% - tableaux -------------------------------------------------------------------
\newcommand{\true}{\ensuremath{\boldsymbol{T}}}
\newcommand{\false}{\ensuremath{\boldsymbol{F}}}

\newcommand{\Tsigned}[1]{\ensuremath{\true{#1}}}\message{ *** RENAME *** }
\newcommand{\Fsigned}[1]{\ensuremath{\false{#1}}}

\newcommand{\TOP}[1]{\ensuremath{D_{#1}}}
\newcommand{\COP}[1]{\ensuremath{D^*_{#1}}}

\newcommand{\Proviso}[1]{\raisebox{-7pt}[0pt][0pt]{\ensuremath{(#1)}}}

\newcommand{\plit}[1]{\ensuremath{\boldsymbol{t}{#1}}}
\newcommand{\nlit}[1]{\ensuremath{\boldsymbol{f}{#1}}}

% - nogoods, assigmemts, etc. --------------------------------------------------
\newcommand{\domain}[1]{\ensuremath{\mathit{dom}(#1)}}

\newcommand{\ass}{\ensuremath{A}}

\newcommand{\tlits}[1]{\ensuremath{{#1}^{\true}}}
\newcommand{\flits}[1]{\ensuremath{{#1}^{\false}}}
\newcommand{\prefix}[2]{\ensuremath{#1[#2]}}

% \newcommand{\clno}[1]{\ensuremath{\delta(#1)}}
% \newcommand{\ClNo}[1]{\ensuremath{\Delta(#1)}}
% \newcommand{\nocl}[1]{\ensuremath{\gamma(#1)}}
% \newcommand{\NoCl}[1]{\ensuremath{\Gamma(#1)}}

\newcommand{\CN}[1]{\ensuremath{\Delta_{#1}}}
\newcommand{\LN}[1]{\ensuremath{\Lambda_{#1}}}

\newcommand{\dl}[0]{\ensuremath{\mathit{dl}}}
\newcommand{\dlevel}[1]{\ensuremath{\mathit{dlevel}(#1)}}
\newcommand{\opp}[1]{\ensuremath{\overline{#1}}}

\newcommand{\undef}[0]{\ensuremath{\circ}}
\newcommand{\trdef}[0]{\ensuremath{\times}}
\newcommand{\scc}[1]{\ensuremath{\mathit{scc}(#1)}}
\newcommand{\source}[1]{\ensuremath{\mathit{source}(#1)}}

% - incremental solving ----------------------------------------
\newcommand{\grounder}{\textsc{Ground}}
\newcommand{\addProgram}{\textsc{Add}}
\newcommand{\solver}{\textsc{Solve}}
\newcommand{\isolve}{\textsc{iSolve}}

% % - algorithm2e ----------------------------------------------------------------
% \DontPrintSemicolon

% \SetKw{ForSome}{for some}
% \SetKw{SuchThat}{such that}

% \SetKwInOut{Input}{Input}
% \SetKwInOut{Output}{Output}
% \SetKwInOut{Global}{Global}
% \SetKwInOut{Internal}{Internal}

% \SetKwFor{Let}{let}{in}{tel}
% \SetKwFor{Loop}{loop}{}{}

% \SetFuncSty{sc}
% \SetKwFunction{Select}{Select}
% \SetKwFunction{UnFoundedSet}{Unfounded\-Set}
% \SetKwFunction{Propagation}{Nogood\-Propagation}
% \SetKwFunction{ConflictAnalysis}{Conflict\-Analysis}

% \SetKwData{Grounder}{Grounder}
% \SetKwData{Solver}{Solver}

% \SetCommentSty{it}
% \SetKwComment{AlgoComm}{// }{}

% - systems ----------------------------------------------------------------------
% >>> NOT USED in BOOK <<<
\newcommand{\sysfont}{\textit}
\newcommand{\smodels}{\sysfont{smodels}}
\newcommand{\smodelsr}{\sysfont{smodels}$_r$}
\newcommand{\smodelscc}{\sysfont{smodels$_{\!cc}$}}
\newcommand{\dlv}{\sysfont{dlv}}
\newcommand{\nomorepp}{\sysfont{nomore++}}
\newcommand{\assat}{\sysfont{assat}}
\newcommand{\cmodels}{\sysfont{cmodels}}
\newcommand{\sag}{\sysfont{sag}}
\newcommand{\clasp}{\sysfont{clasp}}
\newcommand{\claspD}{\sysfont{claspD}}
\newcommand{\claspfolio}{\sysfont{claspfolio}}
\newcommand{\claspar}{\sysfont{claspar}}
\newcommand{\gringo}{\sysfont{gringo}}
\newcommand{\clingo}{\sysfont{clingo}}
\newcommand{\iclingo}{\sysfont{iclingo}}
\newcommand{\clingcon}{\sysfont{clingcon}}
\newcommand{\gecode}{\sysfont{gecode}}
\newcommand{\lparse}{\sysfont{lparse}}
\newcommand{\mchaff}{\sysfont{mchaff}}
\newcommand{\zchaff}{\sysfont{zchaff}}
\newcommand{\siege}{\sysfont{siege}}
\newcommand{\minisat}{\sysfont{minisat}}
\newcommand{\berkmin}{\sysfont{berkmin}}
\newcommand{\picosat}{\sysfont{picosat}}
\newcommand{\lptosat}{\sysfont{lp2sat}}
\newcommand{\lptodiff}{\sysfont{lp2diff}}
\newcommand{\lua}{\textit{lua}}

% % - latex ----------------------------------------------------------------------
% \newcounter{excounter}
% \newcommand{\labex}[1]{\refstepcounter{excounter}\label{#1}} % \index{\ensuremath{\PRG_{\ref{#1}}}}

%%% Local Variables: 
%%% mode: latex
%%% TeX-master: "paper"
%%% End: 


\pdfinfo{
/Title (Domain-specific Heuristics in Answer Set Programming)
/Author (M. Gebser, B. Kaufmann, R. Otero, J. Romero, T. Schaub, P. Wanko)
/Keywords(answer set programming, answer set solving, heuristics, conflict-driven clause learning) 
} 

\sloppy

\title{Domain-specific Heuristics in Answer Set Programming}

\author%
{%
  M.~Gebser$^*$
  \and
  B.~Kaufmann$^*$
  \and
  R.~Otero$^\star$
  \and 
  J.~Romero$^{*,\star}$
  \and 
  T.~Schaub$^*$%\thanks{Affiliated with Simon Fraser University, Canada, and Griffith University, Australia.}
  \and
  P.~Wanko$^*$
  \\
  \begin{tabular}{ccc}
    Institute for Informatics$^*$       & & Department of Computer Science$^\star$\\
    University of Potsdam               & & University of Corunna\\
    14482 Potsdam, Germany	        & & 15071 Corunna, Spain
  \end{tabular}
}

\date{}

\begin{document}

\maketitle
%% %\begin{abstract}
%	The design space for highly complex system level specifications of embedded systems is enormous as tasks may be mapped to different resources and messages may be routed over several links of the hardware platform. 
%	Furthermore, highly constrained requirements lead to many infeasible solutions that have to be sorted out. \emph{\ac{ASP}} in combination with variant background theories (\emph{\ac{ASPmT}}) has been shown to cope with such requirements very efficiently. However, especially in system level design, a fast \emph{\ac{DSE}} including optimization is crucial in order to steer the development towards optimal design points. In this paper, we therefore propose to couple the highly efficient constraint solving capabilities of \ac{ASP} with a \ac{DSE} including \emph{multi-objective optimization} in an additional background theory. Utilizing the possibility to work on \emph{partial assignments}, \ac{ASPmT} is able to prune entire infeasible and dominated regions from the search space early in the decision process. In the experimental section, we present and compare variant approaches and domain specific heuristics.
%\end{abstract}

\begin{abstract}
	An efficient \emph{\ac{DSE}} is imperative for the design of modern, highly complex embedded systems in order to steer the development towards optimal design points. The early evaluation of design decisions at system-level abstraction layer helps to find promising regions for subsequent development steps in lower abstraction levels by diminishing the complexity of the search problem. In recent works, symbolic techniques, especially \ac{ASPmT}, have been shown to find feasible solutions of highly complex system-level synthesis problems with non-linear constraints very efficiently. In this paper, we present a novel approach to a holistic system-level \ac{DSE} based on \ac{ASPmT}. To this end, we include additional background theories that concurrently guarantee compliance with hard constraints and perform the simultaneous optimization of several design objectives. %First experimental results show the applicability of our approach. %for large optimization of up to 170 tasks mapped to 3-dimensional hardware platforms. Furthermore, it outperforms current multi-objective optimization strategies of \ac{ASP} with respect to both diversity and convergence of found solutions.   %We present and investigate several strategies that show the applicability of our approach even for large problem instances. 
	We implement and compare our approach with a state-of-the-art preference handling framework for \ac{ASP}. Experimental results indicate that our proposed method produces better solutions with respect to both diversity and convergence to the true Pareto front.
\end{abstract}
\begin{abstract}
We introduce a general declarative framework for incorporating domain-specific heuristics into
ASP solving.
We accomplish this by extending the first-order modeling language of ASP by a distinguished
heuristic predicate.
The resulting heuristic information is processed as an equitable part of the logic program
and subsequently exploited by the solver when it comes to non-deterministically assigning a 
truth value to an atom.
We implemented our approach as a dedicated heuristic in the ASP solver \textit{clasp} and 
show its great prospect by an empirical evaluation.
\end{abstract}

%%% Local Variables: 
%%% mode: latex
%%% TeX-master: "paper"
%%% End: 


%% \section{Introduction}
\label{sec:introduction}
%In order to cope with the ever-increasing complexity of embedded systems, system level description are utilized to diminish the complexity of finding potentially good solutions which can then be used as initial starting points for further optimization in lower abstraction levels. On system level, applications are composed of granular tasks that exchange information over communication messages and form dependency relations between each other. The hardware architecture contains heterogeneous processing elements (e.g.~CPU, DSP, GPU) as well as a communication infrastructure like routers and links. Yet, the design space for such system level specifications of embedded systems is still enormous as tasks may be mapped to different computational resources and messages may be routed over several links of the communication infrastructure.\par 
%Furthermore, various hard constraints like maximum latency and energy consumption of the resulting systems have to be considered. That is, only a subset of all possible decisions leads to valid system implementations that conform to previously defined constraints which makes it even hard\footnote{In fact, the mapping problem is known to be $\mathcal{NP}$-hard \cite{Blickle1998}.} to find \emph{one} feasible solution. However, by encoding the problem symbolically (cf.~\cite{Haubelt2003}) and due to the technological advances in \ac{SAT}, various constraint solvers can be utilized to cope with the complexity. Especially, \emph{\acf{ASP}} has been shown to deal with such stringently constrained design problems very efficiently (e.g.~\cite{Andres2013}). Opposed to other symbolic techniques like \ac{SAT}, reachability can be expressed naturally in \ac{ASP} which fastens the routing sub-problem.\par 
%%\ac{ASP} stems from the area of knowledge representation and reasoning and is based on the \emph{stable model semantics}. 
%Finding one feasible solution is however often insufficient. Depending on the decisions that have been made, the qualitative properties (e.g.~latency, energy consumption, area requirements) of the resulting system implementation may vary considerably from solution to solution. Thus, a \acf{DSE} is imperative to find solutions with optimal properties. Usually, the objectives (i.e.~optimizing the individual properties) of \acp{MOOP} are conflicting with each other and no single optimal solution but a set of \emph{Pareto optimal} solutions exists. A Pareto optimal solution is characterized by the property that it is not dominated by (i.e.~not worse in all objectives than) any other solution. \par%That is, all Pareto optimal solutions are mutually non-dominated.\par 
%Commonly, meta-heuristics like \acp{MOEA} are utilized to solve \acp{MOOP}. They are based on natural processes and work on sets of solutions (populations) concurrently. Each solution is evaluated by a fitness function with respect to the objectives and the best solutions are combined to create novel solutions for subsequent generations. As the initial population is created by a randomized process, finding feasible solutions becomes a problem for stringently constraint environments. Moreover, because the search is generally not executed systematically but based on combining previously found solutions, \acp{MOEA} tend to run into saturation and stop finding novel solutions after an arbitrary number of iterations.\par
%In the paper at hand, we therefore propose an approach that utilizes an exact symbolic encoding for both the constraint solving and the design space exploration. Based on \ac{ASP}, we tightly integrate background theory solvers, known as \ac{ASPmT}, that handle (non-)linear objectives as well as Pareto filtering of found solutions. Furthermore, they are able to work on partial solutions to prune the search space from infeasible and dominated regions of design points early in the decision process. 
%The contribution of this paper is threefold:
%\begin{enumerate}
%	\item We present a universal framework for preference handling that is capable of both linear and non-linear objectives based on \ac{ASPmT}.
%	\item In order to combine various background theories for multi-objective optimization and constraint solving concurrently, we present various approaches.
%	\item Extensive experimental test instances show the advantages and disadvantages of the different approaches. 
%\end{enumerate}\par
%\textbf{Paper organization:} Related work will be covered in Sec.~\ref{sec:relatedwork}. Afterwards, the execution model that will be used throughout the paper is briefly described in Sec.~\ref{sec:model}. Section \ref{sec:framework} contains detailed information about our proposed preference handling framework. Experimental results are given in Sec.~\ref{sec:experiments} before Sec.~\ref{sec:conclusion} concludes the paper.

%Essentially, there are three approaches to explore the design space \cite{Pimentel2017}: First, meta-heuristics like evolutionary algorithms have been studied thoroughly in the past (e.g.~\cite{1,2,3,4,5}). Those techniques are inspired by the natural selection process and work on whole sets of solutions (populations) concurrently. Each solution is evaluated and the best are combined to create new solutions for the following generations. One major problem arises if, due to various hard constraints, only a small subset of design points is feasible. Because of their random nature, pure meta-heuristics tend to fail in finding feasible regions of the design space. \par 
%Therefore, the second approach type combines meta-heuristics with exact methods (e.g.~\cite{Neubauer2016,Haubelt2003,Lukasiewycz2012a}). That is, not the decision variables themselves but the heuristics that are used by the constraint solver are subject to the randomized exploration process. Every found design point is thereby guaranteed to be feasible.\par 
%Finally, exact methods have been developed to explore the design space systematically. While meta-heuristics normally only cover a limited portion of the design space, exact methods (e.g.~\cite{6,7,8,9}) such as \ac{ILP} and branch-and-bound algorithms are guaranteed to find the optimal solutions. \par
%However, the latter are often infeasible for real-world problems as the design space is simply too vast to evaluate every design point.
%However, finding even \emph{one} feasible solution that conforms to all constraints is an $\mathcal{NP}$-hard problem (cf.~\cite{Blickle1998}).
%One way to cope with such complexities is to represent such problems symbolically and utilize specialized solvers like \ac{SAT} (e.g.~\cite{Neubauer2016}), \ac{ILP} (e.g.~\cite{Lukasiewycz2008}), or \acf{ASP} (e.g.~\cite{Andres2013}). 
%In combination with variant background theories, known as \acf{ASPmT}, it is able to handle non-linear constraints like latency and energy calculations (\cite{Andres2015,Neubauer2017}). Bases on \ac{ASP}, the preference handling framework  that is able to compute preferred (optimal) solutions.

%\begin{itemize}
%	\item Partial solutions $\ldots$ dominance checks, infeasibility
%	\item MOEAs three problems: saturation, finding initial solutions, complete solutions
%	\item symbolic encoding
%\end{itemize}>>>>>>> .r56897


In order to cope with the ever-increasing complexity of embedded systems, system-level descriptions are utilized to diminish the complexity of finding potentially good solutions which can then be used as initial starting points for further optimization in lower abstraction levels. At system level, applications are composed of communicating tasks while the hardware architecture contains heterogeneous processing elements (e.g.~CPU, DSP, GPU) as well as a communication infrastructure like routers and links. 
%Yet, the design space for such system-level specifications of embedded systems is still enormous as tasks may be mapped to different computational resources and communication messages may be routed over several links of the communication infrastructure.\par 
%Furthermore, various hard constraints like maximum latency and energy consumption of the resulting systems have to be considered. That is, only a subset of all possible decisions leads to valid system implementations that conform to previously defined constraints which makes it even hard\footnote{In fact, the mapping problem is known to be $\mathcal{NP}$-hard \cite{Blickle1998}.} to find \emph{one} feasible solution. However, by encoding the problem symbolically (cf.~\cite{Haubelt2003}) and due to the technological advances in \ac{SAT}, various constraint solvers can be utilized to cope with the complexity. Especially, \emph{\acf{ASP}} has been shown to deal with such stringently constrained design problems very efficiently (e.g.~\cite{Andres2013}). Opposed to other symbolic techniques like \ac{SAT}, reachability can be expressed naturally in \ac{ASP} which fastens the routing sub-problem.\par 
%\ac{ASP} stems from the area of knowledge representation and reasoning and is based on the \emph{stable model semantics}. 
%Finding one feasible solution is however often insufficient. Depending on the decisions that have been made, the qualitative properties (e.g.~latency, energy consumption, area requirements) of the resulting system implementation may vary considerably from solution to solution. Thus, a \acf{DSE} is imperative to find solutions with optimal properties. Usually, the objectives (i.e.~optimizing the individual properties) of \acp{MOOP} are conflicting with each other and no single optimal solution but a set of \emph{Pareto optimal} solutions exists. A Pareto optimal solution is characterized by the property that it is not dominated by (i.e.~not worse in all objectives than) any other solution. \par%That is, all Pareto optimal solutions are mutually non-dominated.\par 

Depending on the decisions that have been made, the qualitative properties (e.g.~latency, energy consumption, area requirements) of the resulting system implementation may vary considerably from solution to solution resulting into a \ac{MOOP}. Thus, a \acf{DSE} is imperative to find solutions with optimal properties. \par
Essentially, \ac{DSE} approaches can be characterized into two types \cite{Pimentel2017}: First, (meta-)heuristics like evolutionary algorithms and ant colony optimization (e.g.~\cite{Thompson2013,Ferrandi2010}) and second, exact methods such as \ac{ILP} and branch-and-bound algorithms (e.g.~\cite{Lukasiewycz2008,Khalilzad2016}). \par 
Most of the works presented in the field of meta-heuristics extend basic techniques in order to respect domain specific characteristics. For example, in \cite{Thompson2013}, the authors extend genetic algorithms by utilizing domain knowledge. They state, that small differences in design decisions lead to similar system implementations and that symmetrical design points can be pruned. \par 
Another approach (e.g. \cite{Neubauer2016,Schlichter2006}) of handling the infeasibility problem is to integrate dedicated constraint solvers into a \ac{MOEA}. The work of Schlichter et al. \cite{Schlichter2006} integrates, for example, a \ac{SAT} solver into a \ac{MOEA}. Here, the decisions are not directly controlled by the randomized search algorithm of the \ac{MOEA} but the heuristic of the decision variables is subject to exploration. This way, solutions are guaranteed to be feasible.\par
Finally, fully exact methods have been developed to explore the design space systematically. While meta-heuristics normally only cover a limited portion of the design space, exact methods are guaranteed to find the optimal solutions. Nevertheless, for a long time those methods were restricted to single-objective optimization problems only. As one of the few exceptions, Lukasiewycz et al.  \cite{Lukasiewycz2008} present a complete multi-objective Pseudo-Boolean solver based on branch-and-bound algorithms. The results show that this technique is able to find the proven optimal solutions for small problems in a short time. However, exact methods are often replaced in favor of heuristic approaches as the complexity of large systems hinders reasonable employment of those techniques. \par
The disadvantage of using meta-heuristics, on the other hand, is that the initial population is created by a randomized process. Finding feasible regions becomes therefore a problem for stringently constraint environments. Moreover, because the search is generally not executed systematically but based on combining previously found solutions, \acp{MOEA} tend to run into saturation and stop finding novel solutions after a number of iterations.\par
As a remedy, by encoding the problem symbolically, recent advances of constraint solving technologies can be utilized to cope with the complexity of finding feasible solutions. Especially, \emph{\acf{ASP}} has been shown to deal with such stringently constrained design problems very efficiently (e.g.~\cite{Andres2013}). Opposed to other symbolic techniques like \ac{SAT}, reachability can be expressed naturally in \ac{ASP} which fastens the communication synthesis. However, one problem is that non-linear constraints cannot be easily expressed within \ac{ASP}. \par
In the paper at hand, we therefore propose an approach that utilizes an exact symbolic encoding for both constraint solving and design space exploration. To address the shortcomings of \ac{ASP}, we present specific background theory solvers to handle \emph{non-linear objectives} as well as Pareto filtering of found solutions. By utilizing the state-of-the-art \ac{ASP} solver clingo~5 \cite{gekakaosscwa16a}, these background theories can be tightly integrated into the solving process (\emph{\acf{ASPmT}}). This way, we are able to utilize conflict clauses on partial solutions to prune the search space from infeasible and dominated regions of design points early in the decision process. \par
Note that our methodology uses \emph{exact} search strategies with "\emph{any-time}" characteristic, i.e., canceling the search at any time returns an approximate Pareto set that strictly improves with increased solving time until the true Pareto front is reached.\par
%\textbf{Paper organization and contribution:} In the following, we will first reflect upon related work in Sec.~\ref{sec:relatedwork} before the considered specification model and the basics of \ac{ASPmT} are presented in Sec.~\ref{sec:model}. Section \ref{sec:framework} contains the main contribution of the work at hand. Here, we present our proposed universal framework for \acf{DSE} that is capable of multi-objective optimization of both linear and non-linear objectives. For the first time, various approaches for handling the Pareto filtering in a background theory will be presented.    Afterwards, in Sec.~\ref{sec:experiments}, the approaches are evaluated by a number of differently configured test instances. Finally, Sec.~\ref{sec:conclusion} concludes the paper.

%The contribution of this paper is threefold:
%\begin{enumerate}
%	\item We present a universal framework for preference handling that is capable of both linear and non-linear objectives based on \ac{ASPmT}.
%	\item In order to combine various background theories for multi-objective optimization and constraint solving concurrently, we present various approaches.
%	\item Extensive experimental test instances show the advantages and disadvantages of the different approaches. 
%\end{enumerate}\par
%\textbf{Paper organization:} Related work will be covered in Sec.~\ref{sec:relatedwork}. Afterwards, the execution model that will be used throughout the paper is briefly described in Sec.~\ref{sec:model}. Section \ref{sec:framework} contains detailed information about our proposed preference handling framework. Experimental results are given in Sec.~\ref{sec:experiments} before Sec.~\ref{sec:conclusion} concludes the paper.

\section{Introduction}\label{sec:introduction}

The success of modern Boolean constraint technology was greatly boosted by Satisfiability Testing
(SAT;~\cite{SATHandbook}).
Meanwhile, this technology has been taken up in many related areas, 
like
Answer Set Programming (ASP; \cite{baral02a}).
This is because it provides highly performant yet general-purpose solving techniques for addressing demanding combinatorial search problems.
%
Sometimes, it is however advantageous to take a more application-oriented approach
by including domain-specific information.
On the one hand, domain-specific knowledge can be added for improving deterministic assignments
through propagation.
And on the other hand, domain-specific heuristics can be used for making better non-deterministic assignments.

In what follows,
we introduce a general declarative framework for incorporating domain-specific heuristics into ASP solving.
The choice of ASP is motivated by its first-order modeling language offering an easy way to express and process
heuristic information.
To this end,
we use a dedicated predicate \hpredicate\
whose arguments allow us to express various modifications to the solver's heuristic treatment of atoms.
The respective heuristic rules are seamlessly processed as an equitable part of the logic program
and subsequently exploited by the solver when it comes to choosing an atom for a non-deterministic 
truth assignment.
%
For instance, the rule
\begin{lstlisting}
_h(occ(A,T),factor,T) :- action(A), time(T).
\end{lstlisting}
favors later action occurrences over earlier ones (via multiplication by \texttt{T}).
That is, when making a choice between two unassigned atoms \texttt{occ(a,2)} and \texttt{occ(b,3)},
the solver's heuristic value of \texttt{occ(a,2)} is doubled while that of \texttt{occ(b,3)} is tripled.
This results in a bias on the system heuristic that may or may not take effect.
%
Besides \texttt{factor}, our heuristic language extension offers the primitive heuristic modifiers
\texttt{init}, \texttt{level}, and \texttt{sign}, from which even further modifiers can be defined.
Our approach provides an easy and flexible access to the solver's heuristic,
aiming at its modification rather than its replacement.
Note that the effect of the modifications is generally dynamic,
unless the truth of a heuristic atom is determined during grounding (as with the rule above).
As a result, our approach offers a declarative framework for expressing domain-specific heuristics.
As such, it appears to be the first of its kind.

%%% Local Variables: 
%%% mode: latex
%%% TeX-master: "paper"
%%% End: 

%% \section{Curriculum-based Course Timetabling}\label{sec:cb-ctt}

As mentioned, we focus on the curriculum-based course timetabling
(CB-CTT) problems used in the ITC-2007 competition.
The problem description of CB-CTT presented here is based on 
\citep{DBLP:journals/anor/BonuttiCGS12}.

The CB-CTT instance consists mainly of
\textit{curricula},
\textit{courses},
\textit{rooms},
\textit{days}, and
\textit{periods} per day.
A curriculum is a set of courses that shares common students.
We refer to a pair of day and period as \textit{timeslot}.
%
The CB-CTT problem is defined as the task of assigning all lectures
of each course into a weekly timetable, 
subject to a given set of hard and soft constraints.
%
Hard constraints must be strictly satisfied.
Soft constraints are not necessarily satisfied,
but the sum of their violations should be minimal.
%
A \textit{feasible solution} of the problem is an assignment
so that the hard constraints are satisfied.
The objective of the problem is to find a feasible solution with minimal penalty.
%
The CB-CTT problem has the following hard constraints.
\begin{list}{}{}
\item \bm{$H_1$}. \textbf{Lectures}: 
  All lectures of each course must be scheduled, 
  and they must be assigned to distinct timeslots.
\item \bm{$H_2$}. \textbf{Conflicts}: 
  Lectures of courses in the same curriculum or taught by the same
  teacher must be all scheduled in different timeslots.
\item \bm{$H_3$}. \textbf{RoomOccupancy}: 
  Two lectures cannot take place in the same room in the same timeslot.
\item \bm{$H_4$}. \textbf{Availability}: 
  If the teacher of the course is unavailable to teach that course
  at a given timeslot, then no lecture of the course can be scheduled at
  that timeslot.
\end{list}
The CB-CTT problem has the following soft constraints.
\begin{list}{}{}
\item\bm{$S_1$}. \textbf{RoomCapacity}: 
  For each lecture, the number of students that attend the course must
  be less than or equal the number of seats of all the rooms that host
  its lectures. 
  The penalty points, reflecting the number of students above the
  capacity, are imposed on each violation.
\item\bm{$S_2$}. \textbf{MinWorkingDays}: 
  The lectures of each course must be spread into a given minimum
  number of days. 
  The penalty points, reflecting the number of days below the minimum,
  are imposed on each violation.
\item\bm{$S_3$}. \textbf{IsolatedLectures}: 
  Lectures belonging to a curriculum should be adjacent to each other
  in consecutive timeslots. For a given curriculum we account
  for a violation every time there is one lecture not adjacent to any
  other lecture within the same day. 
  Each isolated lecture in a curriculum counts as 1 violation.
\item\bm{$S_4$}. \textbf{Windows}: 
  Lectures belonging to a curriculum should not have time windows
  (periods without teaching) between them. 
  For a given
  curriculum we account for a violation every time there is one
  window between two lectures within the same day. 
  The penalty points, reflecting the length in periods of time window,
  are imposed on each violation.
\item\bm{$S_5$}. \textbf{RoomStability}: 
  All lectures of a course should be given in the same room. 
  The penalty points, reflecting the number of distinct rooms but the first, 
  are imposed on each violation.
\item\bm{$S_6$}. \textbf{StudentMinMaxLoad}: 
  For each curriculum the number of daily lectures should be within a
  given range. 
  The penalty points, reflecting the number of lectures below the minimum or above the
  maximum, are imposed on each violation.
\item\bm{$S_7$}. \textbf{TravelDistance}: 
  Students should have the time to move from one building to another
  one between two lectures. For a given curriculum we account for a
  violation every time there is an \textit{instantaneous move}: 
  two lectures in rooms located in different building in two adjacent
  periods within the same day. 
  Each instantaneous move in a curriculum counts as 1 violation.
\item\bm{$S_8$}. \textbf{RoomSuitability}:
  Some rooms may be not suitable for a given course because of the
  absence of necessary equipment.
  Each lecture of a course in an unsuitable room counts as 1
  violation.
\item\bm{$S_9$}. \textbf{DoubleLectures}:
  Some courses require that lectures in the same day are grouped
  together (\textit{double lectures}). For a course that requires grouped
  lectures, every time there is more than one lecture in one day, 
  a lecture non-grouped to another is not allowed. 
  Two lectures are grouped if they are adjacent and in the same room. 
  Each non-grouped lecture counts as 1 violation.
\end{list}

%%%%%%%%%%%%%%%%%%%%%%%%%%%%%%%%%%%%%%%%%%%%
\begin{table}
\centering
\caption{Problem Formulations}
\label{table:problem_formulations}
%\renewcommand{\arraystretch}{0.9}
%\tabcolsep = 3mm
\begin{tabular}[t]{l|ccccc}\hline
Constraint & UD1 & UD2 & UD3 & UD4 & UD5\\\hline
$H_1$. Lectures &  
H &  H &  H &  H & H\\
$H_2$. Conflicts &  
H &  H &  H &  H & H\\
$H_3$. RoomOccupancy &  
H &  H &  H &  H & H\\
$H_4$. Availability &  
H &  H &  H &  H & H\\
$S_1$. RoomCapacity &  
1 & 1 & 1 & 1  & 1 \\
$S_2$. MinWorkingDays &  
5 &  5 & - & 1 & 5 \\
$S_3$. IsolatedLectures &  
1 & 2 & - & - & 1 \\
$S_4$. Windows &  
- & - & 4 & 1 & 2\\
$S_5$. RoomStability &  
- & 1 & - & - & -\\
$S_6$. StudentMinMaxLoad &  
- & - & 2 & 1 & 2\\
$S_7$. TravelDistance &  
- & - & - & - & 2\\
$S_8$. RoomSuitability &  
- & - & 3 & H & -\\
$S_9$. DoubleLectures &  
- & - & - & 1 & -\\\hline
\end{tabular}
\end{table}
%%%%%%%%%%%%%%%%%%%%%%%%%%%%%%%%%%%%%%%%%%%%

A \textit{formulation} is defined as a specific set of soft constraints
together with the weights associated with each of them.
%
The five formulations UD1--UD5 have been proposed so far.
UD1 is the most basic formulation among them~\citep{DBLP:conf/patat/GasperoS02}.
UD2 is a well known formulation used in the ITC-2007 competition~\citep{GasperoMS/ITC2007}.
UD3, UD4, and UD5 have been recently proposed
to capture more different scenarios~\citep{DBLP:journals/anor/BonuttiCGS12}.
These formulations focus on 
student load (UD3), 
double lectures (UD4), and
travel cost (UD5), respectively.
%
The weights of soft constraints in each formulation is shown in 
Table~\ref{table:problem_formulations}.
The symbol `H' stands for inclusion in a formulation as hard constraint.
The symbol `-' stands for exclusion from a formulation.

In this paper, we formulate the CB-CTT problem as a single-objective
combinatorial optimization problem whose objective function is to
minimize the weighted sum of penalty points in the same manner as
ITC-2007, 
as well as a multi-criteria optimization problem based on lexicographic ordering.
Furthermore, we consider a multi-objective course timetabling problem
combining CB-CTT and Minimal Perturbation Problem.

%%% Local Variables:
%%% mode: latex
%%% TeX-master: "paper"
%%% End:



\section{Background}\label{sec:background}

We assume some basic familiarity with ASP, its semantics as well as its basic language constructs,
like normal rules, cardinality constraints, and optimization statements.
Although our examples are self-explanatory, we refer the reader for details to~\cite{gekakasc12a}.
%
For illustrating our approach, 
we consider selected rules of a simple planning encoding, following~\cite{lifschitz02a}.
We use predicates \texttt{action} and \texttt{fluent} to distinguish the corresponding entities.
The length of the plan is given by the constant \texttt{l}, which is used to fix all time points via
the statement \lstinline{time(1..l).}
Moreover, suppose our ASP encoding contains the rule
\begin{lstlisting}
1 { occ(A,T) : action(A) } 1 :- time(T).
\end{lstlisting}
stating that exactly one action occurs at each time step.
Also, it includes a frame axiom of the following form.%
\footnote{We use `\texttt{-}/1' to stand for classical negation.}
\begin{lstlisting}
holds(F,T) :- holds(F,T-1), not -holds(F,T).
\end{lstlisting}
In such a setting, actions and fluents are prime subjects to planning-specific heuristics.
As we show below, these can be elegantly expressed by heuristic statements about atoms formed from
predicates \texttt{occ} and \texttt{holds}, respectively.

For computing the stable models of a logic program, we use
a Boolean assignment, that is, a (partial) function mapping propositional variables in $\mathcal{A}$
to truth values \true\ and \false.
We represent such an assignment \ass\ as a set of signed literals of form $\Tsigned{a}$ or $\Fsigned{a}$,
standing for $a\mapsto\true$ and $a\mapsto\false$, respectively.
%
We access the true and false variables in \ass\ via
\(
\tlits{\ass}
=
\{a\in\mathcal{A}\mid\Tsigned{a}\in\ass\}
\)
and
\(
\flits{\ass}
=
\{a\in\mathcal{A}\mid\Fsigned{a}\in \ass\}
\), respectively.
%
\ass\ is conflicting, if $\tlits{\ass}\cap\flits{\ass}\neq\emptyset$;
\ass\ is total, if it is non-conflicting and $\tlits{\ass}\cup\flits{\ass}=\mathcal{A}$.
%
For generality, we represent Boolean constraints by \emph{nogoods}~\cite{dechter03}.
A nogood is a set $\{\sigma_1,\dots,\sigma_m\}$ of signed literals,
expressing that any assignment containing $\sigma_1,\dots,\sigma_m$ is inadmissible.
Accordingly,
a total assignment \ass\ is a \emph{solution} for a set~$\Delta$ of nogoods
if $\delta\not\subseteq\ass$ for all $\delta\in\Delta$.
%
While clauses can be directly mapped into nogoods,
logic programs are subject to a more involved translation.
For instance, 
an atom $a$ defined by two rules $a\,\texttt{:-}\,b,\naf{c}$ and $a\,\texttt{:-}\,d$
gives rise to three nogoods:
$\{\Tsigned{a},\Fsigned{x_{\{b,\naf{c}\}}},\Fsigned{x_{\{d\}}}\}$,
$\{\Fsigned{a},\Tsigned{x_{\{b,\naf{c}\}}}\}$, and
$\{\Fsigned{a},\Tsigned{x_{\{d\}}}\}$,
where $x_{\{b,\naf{c}\}}$ and $x_{\{d\}}$ are auxiliary variables for the bodies of the two previous rules.
Similarly, the body ${\{b,\naf{c}\}}$ leads to nogoods
$\{\Fsigned{x_{\{b,\naf{c}\}}},\Tsigned{b},\Fsigned{c}\}$,
$\{\Tsigned{x_{\{b,\naf{c}\}}},\Fsigned{b}\}$, and
$\{\Tsigned{x_{\{b,\naf{c}\}}},\Tsigned{c}\}$.
See~\cite{gekakasc12a} for full details.
%
Note that translating logic programs into nogoods adds auxiliary variables.
For simplicity, we restrict our formal elaboration to atoms in $\mathcal{A}$
(also because our approach leaves such internal variables unaffected anyway).

% Finally, we mention that
% we use \textit{max} to give the maximum value in a set and 
% \textit{argmax} to give the set of elements having the maximum value.


%%% Local Variables: 
%%% mode: latex
%%% TeX-master: "paper"
%%% End: 

%% 
\section{Conflict-driven constraint learning}\label{sec:cdcl}

Given that we are primarily interested in the heuristic machinery of a solver,
we only provide a high-level description of the basic decision algorithm for 
conflict-driven constraint learning (CDCL;~\cite{marsak99a,zamamoma01a})
in Figure~\ref{algo:cdcl}.
%
% ----------------------------------------------------------------------
\begin{figure}[t]
\newcommand{\ITEMHACK}{\itemindent=-5pt\itemsep=0pt\parsep=\itemsep}
\small
\hrule\vspace{2pt}
\noindent\textbf{loop}\\[-12pt]
  \begin{itemize}\ITEMHACK
  \item [] \textit{propagate}  
    \hfill// compute deterministic consequences
  \item [] \textbf{if} no conflict \textbf{then}
    \begin{itemize}\ITEMHACK
    \item [] \textbf{if} all variables assigned 
      \textbf{then} 
      \textbf{return} variable assignment
    \item [] \textbf{else}
      \textit{decide} 
      \hfill// non-deterministically assign some literal
    \end{itemize}
  \item [] \textbf{else} 
    \begin{itemize}\ITEMHACK
    \item [] \textbf{if} top-level conflict %found 
      \textbf{then} 
      \textbf{return} unsatisfiable
    \item [] \textbf{else}
      \begin{itemize}\ITEMHACK
      \item [] \textit{analyze}\hfill// analyze conflict and add a conflict constraint
      \item [] \textit{backjump}\hfill// undo assignments until conflict constraint is unit
      \end{itemize}
    \end{itemize}
  \end{itemize}
  \hrule  
  \caption{Basic decision algorithm: CDCL}
  \label{algo:cdcl}
\end{figure}
% ----------------------------------------------------------------------
CDCL starts by extending a (partial) {assignment} by deterministic (unit) propagation.
% Importantly, every derived literal is ``implied'' by some {nogood}
% % (set of literals that must not jointly be assigned), 
% which would be violated if the literal's complement were assigned.
Although propagation aims at forgoing nogood violations,
assigning a literal implied by one nogood may lead to the violation of another nogood;
this situation is called \emph{conflict}.
If the conflict can be resolved, % (the violated nogood contains backtrackable literals),
it is analyzed to identify a conflict constraint.
The latter represents a ``hidden'' conflict reason that is recorded and
guides backjumping to an earlier stage such that
the complement of some formerly assigned literal is implied by the conflict constraint,
thus triggering propagation.
Only when propagation finishes without conflict,
a (heuristically chosen) literal can be assigned % at a new \emph{decision level},
provided that the assignment at hand is partial,
while a {solution} % (total assignment not violating any nogood)
has been found otherwise.
% The eventual termination of CDCL is guaranteed
% by either returning a solution or encountering an unresolvable conflict
% (independent of (non-implied) decision literals).
%
See~\cite{SATHandbook} for details.

A characteristic feature of CDCL is its look-back based approach.
Central to this are conflict-driven mechanisms scoring variables according to their prior conflict involvement.
These scores guide heuristic choices regarding literal selection as well as constraint learning and deletion.

A decision heuristic is used to implement the non-deterministic assignment done via
\emph{decide} in the CDCL algorithm in Figure~\ref{algo:cdcl}.
In fact,
the selection of an atom along with its sign relies on two such functions:
\[
h: \mathcal{A}\to[0,+\infty)
\quad\text{and}\quad
s: \mathcal{A}\to\{\true,\false\}
\ .
\]
Both functions vary over time.
To capture this, we use $h_i$ and $s_i$ to denote the specific mappings in the $i$th iteration of CDCL's main loop.
Analogously, we use $\ass_i$ to represent the $i$th assignment (after \textit{propagation}).
We use $i=0$ to refer to the initialization of both functions via $h_0$ and $s_0$;
similarly, $A_0$ gives the initial assignment (after \textit{propagation}).

The following lines give a more detailed yet still high-level account of the non-deterministic assignment done by
\emph{decide} in the CDCL algorithm for $i\geq 1$ (and a given $h_0$):%
\footnote{\label{fn:ass}For clarity, we keep using indexes in this algorithmic setting although this is
  unnecessary in view of assignment operator `$:=$'.}
% --------------------------------------------------
\begin{enumerate}\itemindent 10pt
\item $h_i(a) := \alpha_i\times h_{i-1}(a) + \beta_i(a)$ \hfill for each $a\in\mathcal{A}\qquad$
\item $U:=\mathcal{A}\setminus (\tlits{\ass_{i-1}}\cup\flits{\ass_{i-1}})$
\item $C:= \textit{argmax}_{a\in U}h_i(a)$
\item $a:= \tau(C)$
\item $\ass_i := \ass_{i-1}\cup\{s_i(a)a\}$
\end{enumerate}
% --------------------------------------------------
The first line describes the development of the heuristic depending on a global decay parameter
$\alpha_i$ and a variable-specific scoring function $\beta_i$.
The set $U$ contains all atoms unassigned at step~$i$.
Among them, the ones with the highest heuristic value are collected in $C$.
Whenever $C$ contains several (equally scored) variables,
the solver must break the tie by selecting one atom $\tau(C)$ from $C$.
% Such tie-breaking is usually done by adhoc mechanisms.

Look-back based heuristics rely on information gathered during conflict analysis in CDCL.
Starting from some initial heuristic values in $h_0$,
the heuristic function is continued as in Item~1 above,
where
$\alpha_i\in{[0,1]}$ is a global parameter decaying the influence of past values
and
$\beta_i(a)$ gives the conflict score attributed to variable $a$ within conflict analysis.
The value of $\beta_i(a)$ can be thought of being 0 unless $a$ was scored by \textit{analyze} in CDCL.
Similarly, $\alpha_i$ usually equals 1 unless it was lowered at some system-specific point, 
such as after a \emph{restart}.
%
Occurrence-based heuristics like~\textit{moms}~\cite{pretolani96a} furnish initial heuristics.
%
Prominent look-back heuristics are
\textit{berkmin}~\cite{golnov02a}
and
\textit{vsids}~\cite{momazhzhma01a}.
% and \textit{vmtf}~\cite{ryan04a}.

For illustration,
let us look at a rough trace of atoms $a$, $b$, and $c$ in a fictive run of the CDCL algorithm.
\[
\begin{array}{|r|c|c@{\ }@{\ }c@{\,}@{\,}c@{\,}r@{\,}|c@{\,}@{\,}c@{\,}@{\,}c@{\,}r@{\,}|c|c|c@{\,}@{\,}c@{\,}@{\,}c@{\,}r@{\,}|}
  \hline
  i&\mathit{operation}&
  \multicolumn{4}{|c|}{\ass}&
  \multicolumn{4}{ c|}{h}&
                       s &
                       \alpha &
  \multicolumn{4}{ c|}{\beta}\\
  \hline
   &                  &a     &b    &c&...&a&b&c   &...&     &  &a&b&c&...\\
  \hline
  \hline
  0&                  &      &     & &   &0&1&1   &   &\true&1 &0&0&0&   \\
  \hline
  1&\mathit{propagate}&\false&     & &   &0&1&1   &   &\true&1 &0&0&0&   \\
   &\mathit{decide}   &\false&\true& &   &0&1&1   &   &\true&1 &0&0&0&   
\end{array}
\]
The initial heuristic $h_0$ prefers $b$, $c$ over $a$;
the sign heuristic $s$ constantly assigns \true.
Initial propagation assigns \false\ to $a$.
This leaves all heuristics unaffected.
When invoking \textit{decide}, we find $b$ and $c$ among the unassigned variables in $U$
(in Item~2 above).
Assuming the maximum value of $h_1$ to be 1, both are added to $C$.
This tie is broken by selecting $\tau(C)=b$ in $C$.
Given that the (constant) sign heuristic yields \true,
Item~5 adds signed literal \Tsigned{b} to the current assignment.

Next suppose we encounter a conflict involving $c$ at step~8.
This leads to an incrementation of $\beta_8(c)$.
\[
\begin{array}{|r|c|c@{\,}@{\,}c@{\,}@{\,}c@{\,}r@{\,}|c@{\,}@{\,}c@{\,}@{\,}c@{\,}r@{\,}|c|c|c@{\,}@{\,}c@{\,}@{\,}c@{\,}r@{\,}|}
   &                  &a     &b    &c     &...&a&b&c   &...&     &  &a&b&c&...\\
  \hline
  \hline
  8&\mathit{propagate}&\false&\true&\false&   &0&2&2   &   &\true&1 &0&0&0&   \\
   &\mathit{analyze}  &\false&\true&\false&   &0&2&2   &   &\true&1 &0&0&1&   \\
   &\mathit{backjump} &\false&     &      &   &0&2&3   &   &\true&1 &0&0&0&   \\
  \hline
  9&\mathit{propagate}&\false&     &      &   &0&2&3   &   &\true&1 &0&0&0&   \\
   &\mathit{decide}   &\false&     &\true &   &0&2&3   &   &\true&1 &0&0&0&
\end{array}
\]
As at step~1, $b$ and $c$ are unassigned after backjumping.
Unlike above, $c$ is now heuristically preferred to $b$ since it occurred more frequently within conflicts.

Without going into detail, 
we mention that at certain steps $i$,
parameter $\alpha_i$ is decreased for decaying the values of $h_i$ and
the conflict scores in $\beta_i$ are re-set (eg.~after \textit{analyze}).

Also, look-back based sign heuristics take advantage of previous information.
The common approach is to choose the polarity of a literal according to the higher
number of occurrences in recorded nogoods~\cite{momazhzhma01a}.
Another effective approach is \emph{progress saving}~\cite{pipdar07a},
caching truth values of (certain) retracted variables and reusing them for sign selection.

Although we focus on look-back heuristics,
we mention that look-ahead heuristics 
% are primarily used in DPLL-based solvers~\cite{davput60,dalolo62a}.
% They 
aim at shrinking the search space by selecting the (signed) variable offering most implications.
This approach relies on failed-literal detection~\cite{freeman95a} for counting the number of
propagations obtained by (temporarily) adding in turn the variable and its negation to the current
assignment.
This count can be used in Item~1 above for computing the values $\beta_i(a)$,
while all $\alpha_i$ are set to 0 (because no past information is taken into account).
% For instance, such an approach is used in \smodels~\cite{siniso02a} to select both the variable as
% well as its sign.

%%% Local Variables: 
%%% mode: latex
%%% TeX-master: "paper"
%%% End: 


\section{Conflict-driven constraint learning}\label{sec:cdcl}

Given that we are primarily interested in the heuristic machinery of a solver,
we only provide a high-level description of the basic decision algorithm for 
conflict-driven constraint learning (CDCL;~\cite{marsak99a,zamamoma01a})
in Figure~\ref{algo:cdcl}.
%
% ----------------------------------------------------------------------
\begin{figure}[t]
\newcommand{\ITEMHACK}{\itemindent=-5pt\itemsep=0pt\parsep=\itemsep}
\small
\hrule\vspace{2pt}
\noindent\textbf{loop}\\[-12pt]
  \begin{itemize}\ITEMHACK
  \item [] \textit{propagate}  
    \hfill// compute deterministic consequences
  \item [] \textbf{if} no conflict \textbf{then}
    \begin{itemize}\ITEMHACK
    \item [] \textbf{if} all variables assigned 
      \textbf{then} 
      \textbf{return} variable assignment
    \item [] \textbf{else}
      \textit{decide} 
      \hfill// non-deterministically assign some literal
    \end{itemize}
  \item [] \textbf{else} 
    \begin{itemize}\ITEMHACK
    \item [] \textbf{if} top-level conflict %found 
      \textbf{then} 
      \textbf{return} unsatisfiable
    \item [] \textbf{else}
      \begin{itemize}\ITEMHACK
      \item [] \textit{analyze}\hfill// analyze conflict and add a conflict constraint
      \item [] \textit{backjump}\hfill// undo assignments until conflict constraint is unit
      \end{itemize}
    \end{itemize}
  \end{itemize}
  \hrule  
  \caption{Basic decision algorithm: CDCL}
  \label{algo:cdcl}
\end{figure}
% ----------------------------------------------------------------------
CDCL starts by extending a (partial) {assignment} by deterministic (unit) propagation.
% Importantly, every derived literal is ``implied'' by some {nogood}
% % (set of literals that must not jointly be assigned), 
% which would be violated if the literal's complement were assigned.
Although propagation aims at forgoing nogood violations,
assigning a literal implied by one nogood may lead to the violation of another nogood;
this situation is called \emph{conflict}.
If the conflict can be resolved, % (the violated nogood contains backtrackable literals),
it is analyzed to identify a conflict constraint.
The latter represents a ``hidden'' conflict reason that is recorded and
guides backjumping to an earlier stage such that
the complement of some formerly assigned literal is implied by the conflict constraint,
thus triggering propagation.
Only when propagation finishes without conflict,
a (heuristically chosen) literal can be assigned % at a new \emph{decision level},
provided that the assignment at hand is partial,
while a {solution} % (total assignment not violating any nogood)
has been found otherwise.
% The eventual termination of CDCL is guaranteed
% by either returning a solution or encountering an unresolvable conflict
% (independent of (non-implied) decision literals).
%
See~\cite{SATHandbook} for details.

A characteristic feature of CDCL is its look-back based approach.
Central to this are conflict-driven mechanisms scoring variables according to their prior conflict involvement.
These scores guide heuristic choices regarding literal selection as well as constraint learning and deletion.

A decision heuristic is used to implement the non-deterministic assignment done via
\emph{decide} in the CDCL algorithm in Figure~\ref{algo:cdcl}.
In fact,
the selection of an atom along with its sign relies on two such functions:
\[
h: \mathcal{A}\to[0,+\infty)
\quad\text{and}\quad
s: \mathcal{A}\to\{\true,\false\}
\ .
\]
Both functions vary over time.
To capture this, we use $h_i$ and $s_i$ to denote the specific mappings in the $i$th iteration of CDCL's main loop.
Analogously, we use $\ass_i$ to represent the $i$th assignment (after \textit{propagation}).
We use $i=0$ to refer to the initialization of both functions via $h_0$ and $s_0$;
similarly, $A_0$ gives the initial assignment (after \textit{propagation}).

The following lines give a more detailed yet still high-level account of the non-deterministic assignment done by
\emph{decide} in the CDCL algorithm for $i\geq 1$ (and a given $h_0$):%
\footnote{\label{fn:ass}For clarity, we keep using indexes in this algorithmic setting although this is
  unnecessary in view of assignment operator `$:=$'.}
% --------------------------------------------------
\begin{enumerate}\itemindent 10pt
\item $h_i(a) := \alpha_i\times h_{i-1}(a) + \beta_i(a)$ \hfill for each $a\in\mathcal{A}\qquad$
\item $U:=\mathcal{A}\setminus (\tlits{\ass_{i-1}}\cup\flits{\ass_{i-1}})$
\item $C:= \textit{argmax}_{a\in U}h_i(a)$
\item $a:= \tau(C)$
\item $\ass_i := \ass_{i-1}\cup\{s_i(a)a\}$
\end{enumerate}
% --------------------------------------------------
The first line describes the development of the heuristic depending on a global decay parameter
$\alpha_i$ and a variable-specific scoring function $\beta_i$.
The set $U$ contains all atoms unassigned at step~$i$.
Among them, the ones with the highest heuristic value are collected in $C$.
Whenever $C$ contains several (equally scored) variables,
the solver must break the tie by selecting one atom $\tau(C)$ from $C$.
% Such tie-breaking is usually done by adhoc mechanisms.

Look-back based heuristics rely on information gathered during conflict analysis in CDCL.
Starting from some initial heuristic values in $h_0$,
the heuristic function is continued as in Item~1 above,
where
$\alpha_i\in{[0,1]}$ is a global parameter decaying the influence of past values
and
$\beta_i(a)$ gives the conflict score attributed to variable $a$ within conflict analysis.
The value of $\beta_i(a)$ can be thought of being 0 unless $a$ was scored by \textit{analyze} in CDCL.
Similarly, $\alpha_i$ usually equals 1 unless it was lowered at some system-specific point, 
such as after a \emph{restart}.
%
Occurrence-based heuristics like~\textit{moms}~\cite{pretolani96a} furnish initial heuristics.
%
Prominent look-back heuristics are
\textit{berkmin}~\cite{golnov02a}
and
\textit{vsids}~\cite{momazhzhma01a}.
% and \textit{vmtf}~\cite{ryan04a}.

For illustration,
let us look at a rough trace of atoms $a$, $b$, and $c$ in a fictive run of the CDCL algorithm.
\[
\begin{array}{|r|c|c@{\ }@{\ }c@{\,}@{\,}c@{\,}r@{\,}|c@{\,}@{\,}c@{\,}@{\,}c@{\,}r@{\,}|c|c|c@{\,}@{\,}c@{\,}@{\,}c@{\,}r@{\,}|}
  \hline
  i&\mathit{operation}&
  \multicolumn{4}{|c|}{\ass}&
  \multicolumn{4}{ c|}{h}&
                       s &
                       \alpha &
  \multicolumn{4}{ c|}{\beta}\\
  \hline
   &                  &a     &b    &c&...&a&b&c   &...&     &  &a&b&c&...\\
  \hline
  \hline
  0&                  &      &     & &   &0&1&1   &   &\true&1 &0&0&0&   \\
  \hline
  1&\mathit{propagate}&\false&     & &   &0&1&1   &   &\true&1 &0&0&0&   \\
   &\mathit{decide}   &\false&\true& &   &0&1&1   &   &\true&1 &0&0&0&   
\end{array}
\]
The initial heuristic $h_0$ prefers $b$, $c$ over $a$;
the sign heuristic $s$ constantly assigns \true.
Initial propagation assigns \false\ to $a$.
This leaves all heuristics unaffected.
When invoking \textit{decide}, we find $b$ and $c$ among the unassigned variables in $U$
(in Item~2 above).
Assuming the maximum value of $h_1$ to be 1, both are added to $C$.
This tie is broken by selecting $\tau(C)=b$ in $C$.
Given that the (constant) sign heuristic yields \true,
Item~5 adds signed literal \Tsigned{b} to the current assignment.

Next suppose we encounter a conflict involving $c$ at step~8.
This leads to an incrementation of $\beta_8(c)$.
\[
\begin{array}{|r|c|c@{\,}@{\,}c@{\,}@{\,}c@{\,}r@{\,}|c@{\,}@{\,}c@{\,}@{\,}c@{\,}r@{\,}|c|c|c@{\,}@{\,}c@{\,}@{\,}c@{\,}r@{\,}|}
   &                  &a     &b    &c     &...&a&b&c   &...&     &  &a&b&c&...\\
  \hline
  \hline
  8&\mathit{propagate}&\false&\true&\false&   &0&2&2   &   &\true&1 &0&0&0&   \\
   &\mathit{analyze}  &\false&\true&\false&   &0&2&2   &   &\true&1 &0&0&1&   \\
   &\mathit{backjump} &\false&     &      &   &0&2&3   &   &\true&1 &0&0&0&   \\
  \hline
  9&\mathit{propagate}&\false&     &      &   &0&2&3   &   &\true&1 &0&0&0&   \\
   &\mathit{decide}   &\false&     &\true &   &0&2&3   &   &\true&1 &0&0&0&
\end{array}
\]
As at step~1, $b$ and $c$ are unassigned after backjumping.
Unlike above, $c$ is now heuristically preferred to $b$ since it occurred more frequently within conflicts.

Without going into detail, 
we mention that at certain steps $i$,
parameter $\alpha_i$ is decreased for decaying the values of $h_i$ and
the conflict scores in $\beta_i$ are re-set (eg.~after \textit{analyze}).

Also, look-back based sign heuristics take advantage of previous information.
The common approach is to choose the polarity of a literal according to the higher
number of occurrences in recorded nogoods~\cite{momazhzhma01a}.
Another effective approach is \emph{progress saving}~\cite{pipdar07a},
caching truth values of (certain) retracted variables and reusing them for sign selection.

Although we focus on look-back heuristics,
we mention that look-ahead heuristics 
% are primarily used in DPLL-based solvers~\cite{davput60,dalolo62a}.
% They 
aim at shrinking the search space by selecting the (signed) variable offering most implications.
This approach relies on failed-literal detection~\cite{freeman95a} for counting the number of
propagations obtained by (temporarily) adding in turn the variable and its negation to the current
assignment.
This count can be used in Item~1 above for computing the values $\beta_i(a)$,
while all $\alpha_i$ are set to 0 (because no past information is taken into account).
% For instance, such an approach is used in \smodels~\cite{siniso02a} to select both the variable as
% well as its sign.

%%% Local Variables: 
%%% mode: latex
%%% TeX-master: "paper"
%%% End: 

%% \section{The {\asap} Approach}\label{sec:approach}

We begin with describing {\asap}'s fact format of CB-CTT instances and
then present a basic {\asap} encoding for solving the CB-CTT problems
\footnote{{\asap}: \textbf{T}im\textbf{E}tabling with \textbf{A}nswer \textbf{S}et \textbf{P}r\textbf{O}grammi\textbf{N}g}.
%%%%%%%%%%%%%%%%%%%%%%%%%
\subsection{Fact Format}
%\textbf{Fact Format.}
%%%%%%%%%%%%%%%%%%%%%%%%%
\begin{figure}[t]
\centering
\begin{minipage}[t]{.45\textwidth}
\lstinputlisting[caption={A toy instance of the \code{ectt} format},%
captionpos=b,frame=single,label=ex:toy.ectt,%
numbers=none,%
basicstyle=\ttfamily\scriptsize]{code_toy1_ectt.tex}
\end{minipage}\hfill
\begin{minipage}[t]{.45\textwidth}
\lstinputlisting[frame=single,numbers=none,%
basicstyle=\ttfamily\scriptsize]{code_toy2_ectt.tex}
\end{minipage}
\end{figure}
%
\lstinputlisting[float=t,caption={ASP facts representing the toy instance of Listing~\ref{ex:toy.ectt}},%
captionpos=b,frame=single,label=ex:toy.lp,%
%numbers=none,%
breaklines=true,%
columns=fullflexible,keepspaces=true,%
basicstyle=\ttfamily\scriptsize]{code_toy_lp.tex}
%
\lstinputlisting[float=t,caption={Solution (partial answer set) of the toy instance in UD2},%
captionpos=b,frame=single,label=ex:toy_out.lp,%
%numbers=none,%
breaklines=true,%
columns=fullflexible,keepspaces=true,%
basicstyle=\ttfamily\scriptsize]{code_toy_sol_lp.tex}
%%%%%%%%%%%%%%%%%%%%%%%%%
Listing~\ref{ex:toy.ectt} shows a toy instance of the \code{ectt}
format, which is a standard input format of CB-CTT 
instances~\citep{DBLP:journals/anor/BonuttiCGS12}.
The format has headers that represent basic entities, followed
by five blocks, 
\code{COURSES}, 
\code{ROOMS}, 
\code{CURRICULA}, 
\code{UNAVAILABILITY_CONSTRAINTS}, and 
\code{ROOM_CONSTRAINTS}.

ASP facts representing the toy instance are shown in
Listing~\ref{ex:toy.lp}.
There exists a one-to-one correspondence between ASP fact format and
the \code{ectt} format except for the \code{CURRICULA} block.
%
The facts in Line 1--2 correspond to the \code{ectt} headers and
express that
the instance named \texttt{Toy} consists of 
4 courses, 
3 rooms,
2 curricula,
8 unavailability constraints, and 
3 room constraints.
The weekly timetable consists of 
5 days and 4 periods per day, which start from 0.
The fact \code{min_max_daily_lectures(2,3)} expresses 
the minimum and maximum numbers of daily lectures 
for each curriculum, and is used to specify $S_6$.

Each fact of predicate \code{course/6} in Line 4--5
corresponds to a line of the \code{COURSES} block.
A fact \texttt{course($C$,$T$,$N$,$MWD$,$M$,$DL$)}
expresses that a course $C$ taught by a teacher $T$ 
consists of $N$ lectures, which must be spread into $MWD$ days.  
The number of students attending the course $C$ is $M$.  
The course $C$ requires double lectures if $DL=1$.  
%
Each fact of predicate \code{room/3} in Line 7 
corresponds to a line of the \code{ROOMS} block.
A fact \texttt{room($R$,$CAP$,$BLD$)} expresses that a
room $R$ in a building $BLD$ has a seating capacity of $CAP$.

A fact \texttt{curricula($CUR$, $C$)} in Line 9--10 expresses that
a course $C$ belongs to a curriculum $CUR$.
%
Each fact of predicate \code{unavailability_constraint/3} in Line
12--15 corresponds to a line of the 
\code{UNAVAILABILITY_CONSTRAINTS} block, and is used to specify $H_4$.
A fact \texttt{unavailability\_constraint($C$,$D$,$P$)}
expresses that a course $C$ is not available at a period $P$ on a day
$D$.
%
Each fact of predicate \code{room_constraint/2} in Line 17
corresponds to a line of the \code{ROOM_CONSTRAINTS} block, and
is used to specify $S_8$.
A fact \texttt{room\_constraint($C$,$R$)} expresses that a room $R$ is
not suitable for a course $C$.

Listing~\ref{ex:toy_out.lp} shows an optimal solution with zero penalty
of the toy instance in the UD2 formulation.
Each atom \texttt{assigned($C$,$R$,$D$,$P$)} is intended to
express that a lecture of a course $C$ is assigned to 
a room $R$ at a period $P$ on a day $D$.
We can observe from Line 1 that
the lectures of the course \texttt{SceCosC} are
assigned to the room \texttt{rB} at
the first period (\texttt{0}) on Thursday (\texttt{3}),
the third period (\texttt{2}) on Wednesday (\texttt{2}), and
the third period (\texttt{2}) on Friday (\texttt{4})


\subsection{First-Order Encoding}
%\textbf{First-Order Encoding.}
%%%%%%%%%%%%%%%%%%%%%%%%%
\lstinputlisting[float=t,caption={Encoding of hard constraints},%
captionpos=b,frame=single,label=en:ctt_hard2.lp4,%
%numbers=none,%
breaklines=true,%
columns=fullflexible,keepspaces=true,%
basicstyle=\ttfamily\scriptsize]{code_hard_lp.tex}
%%%%%%%%%%%%%%%%%%%%%%%%%
\lstinputlisting[float=t,caption={Encoding of soft constraints and objective function},%
captionpos=b,frame=single,label=en:ctt_soft.lp4,%
%numbers=none,%
breaklines=true,%
columns=fullflexible,keepspaces=true,%
basicstyle=\ttfamily\scriptsize]{code_soft_lp.tex}
%%%%%%%%%%%%%%%%%%%%%%%%%
The {\asap} encoding of hard constraints ($H_1$--$H_4$) is shown in 
Listing~\ref{en:ctt_hard2.lp4}.
The expressive power of ASP's modelling language enables us to express
each hard constraint individually by just one or two ASP rules.
%
As mentioned, the atom \texttt{assigned($C$,$R$,$D$,$P$)}
expresses that a lecture of a course $C$ is assigned to a room $R$ at
a period $P$ on a day $D$, and a solution is composed of 
a set of these assignments.
The atom \texttt{assigned($C$,$D$,$P$)} dropping $R$
from \texttt{assigned($C$,$R$,$D$,$P$)} is also introduced,
since we do not always have to take the room information into account 
to specify the hard constraints except $H_3$.

Given an instance expressed in our fact format,
the first four rules in Line 1--2 generate 
\code{c(C)}, 
\code{t(T)}, 
\code{r(R)}, and
\code{cu(Cu)} 
for each course \code{C}, teacher \code{T}, room \code{R}, and 
curriculum \code{Cu}.
%
The next two rules in Line 3 generate 
\code{d(0)} $\ldots$ \code{d(D-1)} and
\code{ppd(0)} $\ldots$ \code{ppd(P-1)} 
expressing that the days range from \code{0} to \code{D-1}, 
and the periods per day range from \code{0} to \code{P-1}.

For $H_1$,
the rule in Line 6,
for every course \code{C} having \code{N} lectures,
generates a set of candidate assignments
subject to the condition that 
there are exactly \code{N} lectures such that \code{assigned(C,D,P)} holds.

For $H_2$,
the rule in Line 9 enforces that,
for every teacher \code{T}, day \code{D}, and period \code{P},
there is at most one course \code{C} taught by \code{T}
such that \code{assigned(C,D,P)} holds.
In detail, 
if \code{t(T)}, \code{d(D)}, and \code{ppd(P)} hold,
this integrity constraint tells us that
the at-most-one constraint represented by 
`\verb+{ assigned(C,D,P) : course(C,T,_,_,_,_) } 1+'
must be true as well in order to prevent its body from being satisfied. 
In the similar way,
the rule in Line 10 enforces that,
for every curriculum \code{Cu}, day \code{D}, and period \code{P},
there is at most one course \code{C} that belongs to \code{Cu}
such that \code{assigned(C,D,P)} holds.

For $H_3$, 
if \code{assigned(C,D,P)} holds, 
the rule in Line 13 generates a solution candidate 
subject to the condition that there is exactly one room \code{R} such
that \code{assigned(C,R,D,P)} holds. 
The rule in Line 14 enforces that,
for every room \code{R}, day \code{D}, and period \code{P},
there is at most one course \code{C} such that 
\code{assigned(C,R,D,P)} holds.

For $H_4$,
the rule in Line 17 enforces that
a course \code{C} is not assigned at a period \code{P} on a day \code{D}
if \code{unavailability_constraint(C,D,P)} holds, since
the conjunction of literals in its body must not hold. 

The rule in Line 20 expresses that 
for each timeslot (\code{D} and \code{P})
the number of lectures assigned must be less than or equal to 
the number of rooms (\code{N}).
This rule is an implied constraint and can be omitted, but we keep it
as an additional rule for performance improvement of some problem
instances.

The {\asap} encoding of soft constraints ($S_1$--$S_9$) and an
objective function is shown in Listing~\ref{en:ctt_soft.lp4}.
We introduce a \textit{penalty atom}
\texttt{penalty($S_i$,$V$,$C$)}, which is intended to express
that a constraint $S_i$ is violated by $V$ and its penalty cost is $C$.
The constants denoted by \code{weight_of_*} indicate
the weights associated with each soft constraint defined in
Table~\ref{table:problem_formulations}.
%
Once again, each soft constraint $S_i$ is compactly expressed by
just one or two ASP rules in which the head is of the form
\texttt{penalty($S_i$,$V$,$C$)}, and a violation $V$ and its penalty
cost $C$ are detected and calculated respectively in the body.
That is, for each violation $V$ of $S_i$, 
an atom \texttt{penalty($S_i$,$V$,$C$)} is generated.
Optimal solutions can be obtained by
minimizing the number of penalty atoms in Line 49.

We explain $S_{1}$--$S_{3}$ that compose the basic UD1 formulation.
%
For $S_1$, 
the rule in Line 2--3,
for every course \code{C} that \code{N} students attend and
room \code{R} that has a seating capacity of \code{Cap},
generates a penalty atom with the cost of 
\code{(N-Cap)*weight_of_s1}
if a lecture of course \code{C} is assigned to the room \code{R}
whose seating capacity (\code{Cap}) is less than the number of
attendees (\code{N}).

For $S_2$,
the rule in Line 6 generates an auxiliary atom \code{working_day(C,D)}
which expresses that a course \code{C} is given on a day \code{D}, 
if \code{assigned(C,D,P)} holds.
The rule in Line 7--8,
for every course \code{C} whose lectures 
must be spread into \code{MWD} days,
generates a penalty atom with the cost of 
\code{(MWD-N)*weight_of_s2},
if the number of days (\code{N}) in which a course \code{C} spread
is less than \code{MWD}.

For $S_3$, 
the rule in Line 11 generates an auxiliary atom \code{scheduled_curricula(Cu,D,P)}
which expresses that 
a curriculum \code{Cu} is 
scheduled at a period \code{P} on a day \code{D},
if \code{assigned(C,D,P)} holds.
% Note that, by $H_2$,
% for every curriculum \code{Cu}, day \code{D}, and period \code{P},
% there must be at most one course \code{C} that belongs to \code{Cu}
% such that \code{assigned(C,D,P)} holds. 
The rule in Line 12--14,
for every curriculum \code{Cu}, day \code{D}, and period \code{P},
generates a penalty atom with the cost of \code{weight_of_s3},
if a curriculum \code{Cu} is scheduled at a period \code{P} on a
day \code{D}, but not at both \code{P-1} and \code{P+1} 
within the same day \code{D}.


%%% Local Variables:
%%% mode: latex
%%% TeX-master: "paper"
%%% End:


\section{Heuristic language elements}\label{sec:approach}

We express heuristic modifications via a set $\mathcal{H}$ of \emph{heuristic atoms} 
disjoint from $\mathcal{A}$.
Such a heuristic atom is formed from a dedicated predicate \hpredicate\ along with four arguments:
a (reified) atom $a\in\mathcal{A}$,
a heuristic modifier $m$,
and two  integers $v,p\in\mathbb{Z}$.
A heuristic modifier is used to manipulate the heuristic treatment of an atom $a$ via the 
modifier's value given by $v$.
The role of this value varies for each modifier.
We distinguish four primitive heuristic modifiers:
\begin{description}
\item [\texttt{init}] for initializing the heuristic value of $a$ with $v$,
\item [\texttt{factor}] for amplifying the heuristic value of $a$ by factor $v$,
\item [\texttt{level}] for ranking all atoms; the rank of $a$ is $v$,
\item [\texttt{sign}] for attributing the sign of $v$ as truth value to $a$.
\end{description}
While $v$ allows for changing an atom's heuristic behavior relative to \emph{other} atoms,
the second integer $p$ allows us to express a priority for disambiguating similar
heuristic modifications to the \emph{same} atom.
This is particularly important in our dynamic setting, where varying heuristic atoms may be obtained
in view of the current assignment.
For instance, the heuristic atoms
\hpred{b}{\mathtt{sign}}{1}{3}
and
\hpred{b}{\mathtt{sign}}{-1}{5}
aim at assigning opposite truth values to atom $b$.
This conflict can be resolved by preferring the heuristic modification with the higher priority,
viz.\ 5 in \hpred{b}{\mathtt{sign}}{-1}{5}.
Obviously such priorities can only support disambiguation but not resolve conflicting values sharing the same priority.

For accommodating priorities,
we define for an assignment \ass\ the \emph{preferred values} for modifier $m$ on atom $a$ as
\[
V_{a,m}(\ass)=
\textit{argmax}_{v\in\mathbb{Z}}\{p\mid\Tsigned{\hpred{a}{m}{v}{p}}\in\ass\}.
\]
Heuristic values are dynamic;
they are extracted from the current assignment and may thus vary during solving.
Note that $V_{a,m}(\ass)$ returns the singleton set $\{v\}$,
if the current assignment \ass\ contains a single true heuristic atom \hpred{a}{m}{v}{p} 
involving $a$ and $m$.
$V_{a,m}(\ass)$ is empty whenever there are no such heuristic atoms.
%
And whenever all heuristic atoms regarding $a$ and $m$ have the same priority $p$,
$V_{a,m}(\ass)$ is equivalent to
\(
\{v\mid\Tsigned{\hpred{a}{m}{v}{p}}\in\ass\}
\).

Here are a few examples.
We obtain % the preferred values
$V_{b,\mathtt{sign}}(\ass_1)=\{-1\}$
and
$V_{c,\mathtt{init}}(\ass_1)=\emptyset$
from assignment
\(
\ass_1=\{\Fsigned{a},\Tsigned{\hpred{b}{\mathtt{sign}}{1}{3}},\Tsigned{\hpred{b}{\mathtt{sign}}{-1}{5}}\}
\),
while assignment
\(
\ass_2=\{\Tsigned{\hpred{b}{\mathtt{sign}}{1}{3}},\Tsigned{\hpred{b}{\mathtt{sign}}{-1}{3}}\}
\)
results in $V_{b,\mathtt{sign}}(\ass_2)=\{1,-1\}$.

For ultimately resolving ambiguities among alternative values for heuristic modifiers, 
we propose for a set $V\subseteq\mathbb{Z}$ of integers the function $\nu(V)$ as
\[
%\nu(V)=
\mathit{max}\big(\{v\in V\!\mid v\geq 0\}\cup\{0\}\big)
+
\mathit{min}\big(\{v\in V\!\mid v\leq 0\}\cup\{0\}\big).
\]
Note that $\nu(\emptyset)=0$, attributing 0 the status of a neutral value.
Alternative options exist, like taking means or median of $V$ or even time specific criteria
relating to the emergence of values in the assignment.
%
In the above examples,
we get $\nu(V_{b,\mathtt{sign}}(\ass_1))=-1$ and $\nu(V_{b,\mathtt{sign}}(\ass_2))=0$.

Given this, we proceed by defining the \emph{domain-specific extension} $d$ to the heuristic function $h$ 
for $a\in\mathcal{A}$ as
\[
d_0(a)=\nu(V_{a,\mathtt{init}}(\ass_0))+h_0(a)
\]
and for $i\geq 1$
\[
d_i(a)=
\left\{
  \begin{array}{rl}
    \nu(V_{a,\mathtt{factor}}(\ass_i))\times h_i(a)&\text{if } V_{a,\mathtt{factor}}(\ass_i)\neq\emptyset
    \\
                                             h_i(a)&\text{otherwise}
  \end{array}
\right.
\]
First of all, it is important to note that $d$ is merely a modification and not a replacement of the
system heuristic $h$.
In fact, $d$ extends the range of $h$ to $(-\infty,+\infty)$.
Negative values serve as penalties.
The values of the \texttt{init} modifiers are added to $h_0$ in $d_0$.
The use of addition rather than multiplication allows us to override an initial value of 0.
Also, the higher the absolute value of the \texttt{init} modifier, the longer lasts its effect
(given the decay of heuristic values).
Unlike this, \texttt{factor} modifiers rely on multiplication because they aim at de- or increasing
conflict scores gathered during conflict analysis.
In view of $h$'s range,
a factor greater than 1 amplifies the score, a negative one penalizes the atom, and 0 resets the atom's score.
Enforcing a factor of 1 
% (for instance, through assigning a high priority)
transfers control back to the system heuristic $h$.

Heuristically modified logic programs are simply programs over $\mathcal{A}\cup\mathcal{H}$,
the original vocabulary extended by heuristic atoms (without restrictions).
As a first example,
let us extend our planning encoding by a rule favoring atoms expressing action occurrences close to
the goal situation.
\begin{lstlisting}
_h(occ(A,T),factor,T,0) :- action(A),time(T).
\end{lstlisting}
With \texttt{factor}, we impose a bias on the underlying heuristic function $h$.
Rather than comparing, for instance,
the plain values $h(\mathtt{occ(a,2)})$ and $h(\mathtt{occ(a,3)})$,
a decision is made by looking at $2\times h(\mathtt{occ(a,2)})$ and  $3\times h(\mathtt{occ(a,3)})$,
even though it still depends on $h$.
A further refined strategy may suggest considering climbing actions as early as possible.
\begin{lstlisting}
_h(occ(climb,T),factor,l-T,1) :- time(T).
\end{lstlisting}
Clearly, this rule conflicts with the more general rule above.
However, this conflict is resolved in favor of the more specific rule by attributing it a higher
priority (viz.~1 versus 0).

% Similar statements can be formulated with the \texttt{init} modifier in order to change the initial heuristic values.

For capturing a \emph{domain-specific extension} $t$ to the sign heuristic $s$,
we define for $a\in\mathcal{A}$ and $i\geq 0$:
\[
t_i(a)=
\left\{
  \begin{array}{rl}
    \true &\text{if }
           \nu(V_{a,\mathtt{sign}}(\ass_i))>0
           % \text{ and }
           % V_{a,\mathtt{sign}}(\ass_i)\neq\emptyset
           \\
    \false&\text{if }
           \nu(V_{a,\mathtt{sign}}(\ass_i))<0
           % \text{ and }
           % V_{a,\mathtt{sign}}(\ass_i)\neq\emptyset
           \\
    s_i(a)&\text{otherwise}
  \end{array}
\right.
\]
As with $d$ above, the extension $t$ to the sign heuristic is dynamic.
The sign of the modifier's preferred value determines the truth value to assign to an atom at hand.
No \texttt{sign} modifier (or enforcing a value of 0) leaves sign selection with the system's sign heuristic $s$.
%
For example, the heuristic rule
\begin{lstlisting}
_h(holds(F,T),sign,-1,0) :- fluent(F),time(T).
\end{lstlisting}
tells the solver to assign false to non-deterministically chosen fluents.
%
The next pair of rules is a further refinement of our strategy on climbing actions,
favoring their effective occurrence in the first half of the plan.
\begin{lstlisting}
_h(occ(climb,T),sign, 1,0) :- T<l/2,time(T).
_h(occ(climb,T),sign,-1,0) :- T>l/2,time(T).
\end{lstlisting}
Thus, while the atom $\mathtt{occ(climb,1)}$ is preferably made true,
false should rather be assigned to $\mathtt{occ(climb,l)}$.

Finally, for accommodating rankings induced by \texttt{level} modifiers,
we define for an assignment \ass\ and $\mathcal{A}'\subseteq\mathcal{A}$:
\[
\ell_\ass(\mathcal{A}')=\textit{argmax}_{a\in\mathcal{A}'}\nu(V_{a,\mathtt{level}}(\ass))
\]
The set $\ell_\ass(\mathcal{A}')$ gives all atoms in $\mathcal{A}'$ with the highest \texttt{level} values relative to
the current assignment \ass.
Similar to $d$ and $t$ above, this construction is also dynamic and the rank of atoms may vary during solving.
The function $\ell_\ass$ is then used to modify the selection of unassigned atoms in the above elaboration of \textit{decide}.
For this purpose, we replace Item~2 by
\(
U:=\ell_\ass(\mathcal{A}\setminus (\tlits{\ass}\cup\flits{\ass}))
\)
in order to restrict $U$ to unassigned atoms of (current) highest rank.
Unassigned atoms at lower levels are only considered once all atoms at higher levels have been assigned.
Atoms without an associated level default to level 0 because $\nu(\emptyset)=0$.
Hence, negative levels act as a penalty since the respective atoms are only taken into account
once all atoms with non-negative or no associated level have been assigned.

For a complementary example, 
consider a \texttt{level}-based formulation of the previous (\texttt{factor}-based) heuristic rule.
\begin{lstlisting}
_h(occ(A,T),level,T,0) :- action(A),time(T).
\end{lstlisting}
Unlike the above, $\mathtt{occ(a,2)}$ and $\mathtt{occ(a,3)}$ are now associated with different ranks,
which leads to strictly preferring $\mathtt{occ(a,3)}$ over $\mathtt{occ(a,2)}$ 
whenever both atoms are unassigned.
Hence, \texttt{level} modifiers partition the set of atoms and restrict $h$ to unassigned atoms at the highest level.

The previous replacement along with the above amendments of $h$ and $s$ through the domain-specific extensions $d$ and $t$
yields the following elaboration of CDCL's heuristic choice operation \textit{decide} for $i\geq 1$ (and given $d_0$).$^{\ref{fn:ass}}$
% --------------------------------------------------
\begin{enumerate}\addtocounter{enumi}{-1}\itemindent 10pt
\item $h_{i-1}(a) := d_{i-1}(a)$                         \hfill for each $a\in\mathcal{A}\qquad$
\item $h_i(a) := \alpha_i\times h_{i-1}(a) + \beta_i(a)$ \hfill for each $a\in\mathcal{A}\qquad$
\item $U:=\ell_{\ass_{i-1}}(\mathcal{A}\setminus (\tlits{\ass_{i-1}}\cup\flits{\ass_{i-1}}))$
\item $C:= \textit{argmax}_{a\in U}d_i(a)$
\item $a:= \tau(C)$
\item $\ass_i := \ass_{i-1}\cup\{t_i(a)a\}$
\end{enumerate}
% --------------------------------------------------
Although we formally model both $h$ and $d$ (as well as $s$ and $t$) as functions,
there is a substantial conceptual difference in practice in that $h$ is a system-specific data structure while $d$ is an associated method.
This is also reflected above, where $h$ is subject to assignments.
%
Item~0 makes sure that our heuristic modifications take part in the look-back based evolution in Item~1,
and are thus also subject to decay.
We added this as a separate line rather than integrating it into Item~1 in order to stress that our
modifications are modular in leaving the underlying heuristic machinery unaffected.
%
Item~2 gathers in $U$ all unassigned atoms of highest rank.
%
Among them, Item~3 collects in $C$ all atoms $a$ with a maximum heuristic value $d_i(a)$.
%
Since this is not guaranteed to yield a unique element, the system-specific tie-breaking function
$\tau$ is evoked to return a unique atom.
%
Finally, the modified sign heuristic $t_i$ determines a truth value for $a$, and the resulting
signed literal ${t_i(a)a}$ is added to the current assignment.

Note that so far all sample heuristic rules were \emph{static} in the sense that they are turned into
facts by the grounder and thus remain unchanged during solving.
Examples of dynamic heuristic rules are given at the end of next section.

Our simple heuristic language is easily extended by further heuristic atoms.
For instance, \hpred{a}{\mathtt{true}}{v}{p} and \hpred{a}{\mathtt{false}}{v}{p} have turned out to be useful in practice.
\begin{lstlisting}
_h(A,level,V,P) :- _h(A,true, V,P).
_h(A,sign, 1,P) :- _h(A,true, V,P).
_h(A,level,V,P) :- _h(A,false,V,P).
_h(A,sign,-1,P) :- _h(A,false,V,P).
\end{lstlisting}
%
For instance, the heuristic atom \hpred{a}{\mathtt{true}}{3}{3} expands to 
\hpred{a}{\mathtt{level}}{3}{3} and \hpred{a}{\mathtt{sign}}{1}{3},
expressing a preference for both making a decision on~$a$ and
assigning it to true.
On the other hand,
\hpred{a}{\mathtt{false}}{-3}{3} expands to 
\hpred{a}{\mathtt{level}}{-3}{3} and \hpred{a}{\mathtt{sign}}{-1}{3},
thus suggesting not to make a decision on~$a$ but to
assign it to false if there is no ``better'' decision variable.

Another shortcut of pragmatic value is the abstraction from specific priorities.
For this, we use the following rule.
\begin{lstlisting}
_h(A,M,V,#abs(V)) :- _h(A,M,V).
\end{lstlisting}
With it,
we can directly describe the heuristic restriction used in \cite{rintanen11a} to simulate planning
by iterated deepening $A^*$ \cite{korf85a} in SAT solving through limiting choices to action variables,
assigning those for time \texttt{T} before those for time \texttt{T+1}, and always assigning truth
value \texttt{true} (where \texttt{l} is a constant indicating the planning horizon):
\begin{lstlisting}
_h(occ(A,T),true,l-T) :- action(A), time(T).
\end{lstlisting}

Although we impose no restriction on the occurrence of heuristic atoms within logic programs,
it seems reasonable to require that the addition of rules containing heuristic atoms does not alter
the stable models of the original program.
That is, given a logic program $P$ over $\mathcal{A}$ and a set of rules $H$ over $\mathcal{A}\cup\mathcal{H}$,
we aim at a one-to-one correspondence between the stable models of $P$ and $P\cup H$ and
their identity upon projection on $\mathcal{A}$.
This property is guaranteed whenever heuristic atoms occur only in the head of rules and thus only
depend upon regular atoms.
In fact, so far, this class of rules turned out to be expressive enough to model all heuristics of interest,
including the ones presented in this paper.
It remains future work to see whether more sophisticated schemes, eg., involving recursion, are useful.

%%% Local Variables: 
%%% mode: latex
%%% TeX-master: "paper"
%%% End: 

%% \section{Experiments}\label{sec:experiments}
%
\begin{table}[t]
\caption{Comparison of approximation techniques by 
(a) runtime and timeouts,
(b) diversification quality, and
(c) minimum distance}
\small
\parbox{.32\linewidth}{\centering
\begin{tabular}{|l||r|r|}

\hline
Class & \textit{T} & \textit{TO}  \\ 
\hline
\Alabel{3} & \textbf{165} & \textbf{70} \\
\Alabel{3}-\textit{true} & 200 & 113 \\ 
\Alabel{3}-\textit{all} & 202 & 118 \\ 
\Alabel{3}-\textit{rd} & 277 & 280 \\ 
\Alabel{3}-\textit{pg} & 317 & 351\\
\Alabel{3}-\textit{pg-l-rd} & 354 & 442\\
\Alabel{3}-\textit{false} & 351 & 443 \\ 
\Alabel{3}-\textit{pg-l} & 351 & 443\\
\Alabel{2}-\textit{true} & 482 & 618\\
\Alabel{2}-\textit{rd} & 474 & 648\\
\Alabel{1} & 482 & 672\\
\Alabel{2}-\textit{dist-to} & 528 & 689\\
\Alabel{2}-\textit{all} & 515 & 696\\
\Alabel{2}-\textit{false} & 532 & 696\\
\Alabel{2}-\textit{pg} & 542 & 708\\
\Alabel{2}-\textit{dist} & 572 & 773\\
\hline
\end{tabular} 
}
\parbox{.32\linewidth}{\centering
\begin{tabular}{|l||r|r|}

\hline
Class & \textit{S} & \textit{avg}\\ 
\hline
\Alabel{1} & \textbf{15} & 0.13\\
\Alabel{2}-\textit{dist-to} & 14 & 0.14\\ 
\Alabel{2}-\textit{pg} & 13 & \textbf{0.18}\\ 
\Alabel{3}-\textit{pg-l} & 11 & 0.17\\
\Alabel{3}-\textit{pg-l-rd} & 10 & 0.16\\
\Alabel{2}-\textit{all}  & 10 & 0.15\\
\Alabel{2}-\textit{dist} & 8 & 0.07\\ 
\Alabel{2}-\textit{false} & 8 & 0.15\\ 
\Alabel{2}-\textit{true} & 7 & 0.12\\ 
\Alabel{3}-\textit{false} & 6 & 0.16\\ 
\Alabel{2}-\textit{rd} & 5 & 0.12\\ 
\Alabel{3}-\textit{all}  & 5 & 0.08 \\ 
\Alabel{3}-\textit{true} & 4 & 0.08 \\ 
\Alabel{3}-\textit{rd} & 2 & 0.09 \\ 
\Alabel{3}-\textit{pg} & 1 & 0.09\\
%\Alabel{3}-Hdyn & 1 & 0.09\\ 
\Alabel{3} & 0 & 0.06\\

\hline
\end{tabular} 
}
\parbox{.32\linewidth}{\centering
\begin{tabular}{|l||r|r|}

\hline
Class & \textit{S} & \textit{avg}\\ 
\hline
\Alabel{1} & \textbf{15} & 12.25\\
\Alabel{2}-\textit{dist-to} & 13 & 10.38\\
\Alabel{3}-\textit{pg-l-rd } & 13 & 11.82 \\
\Alabel{2}-\textit{dist} & 12 & 5.31\\
\Alabel{3}-\textit{pg-l} & 12 & 11.10\\
\Alabel{2}-\textit{pg} & 10 & \textbf{12.86}\\
\Alabel{2}-\textit{rd} & 9 & 8.77 \\
\Alabel{3}-\textit{all}  & 7 & 3.99 \\ 
\Alabel{3}-\textit{true} & 6 & 4.00 \\ 
\Alabel{3}-\textit{false} & 6 & 7.07 \\ 
\Alabel{2}-\textit{false} & 6 & 6.80\\
\Alabel{2}-\textit{all}  & 4 & 6.98\\
\Alabel{2}-\textit{true} & 3 & 5.31\\
\Alabel{3}-\textit{rd} & 2 & 6.43\\
\Alabel{3} & 2 & 4.28\\
%\Alabel{3}-Hdyn & 1 & 2.90\\ 
\Alabel{3}-\textit{pg} & 0 & 2.79\\
\hline
\end{tabular} 
}
\label{tab:time_comparison_small}
\label{tab:diverse_comparison_small}
\label{tab:min_dist_comparison_small}
\end{table}
%
In this section, we present experiments focusing on the \emph{approximation} techniques of the \asprin\ system for obtaining most dissimilar optimal
solutions. 
%
While \emph{enumeration} and \emph{replication} provide exact results, they need to calculate and store a possibly exponential number of optimal
models or deal with a large search space, respectively.
%
Those techniques are therefore not effective for most practical applications.
%
For Algorithm~\Alabel{2}, we considered the variations \textit{rd}, \textit{pg}, \textit{true}, \textit{false}, and \textit{all} .
%
In \textit{dist}, we issued no timeout for the computation of the partial interpretation, 
while in \textit{dist-to}, we set a timeout for this computation of half the total possible runtime.
%
For Algorithm~\Alabel{3}, we consider the variations that include no extra ASP computation, namely, 
\textit{rd}, \textit{pg}, \textit{true}, \textit{false}, and \textit{all} .
%
We also evaluated a version without any heuristic modification (named simply \Alabel{3}).
%
Furthermore, following \cite{nadel11a}, 
we considered a variation of \textit{pg}, viz.~\textit{pg-l}, 
where the atoms of the selected partial interpretation are given a higher priority, 
and \textit{pg-l-rd}, extending \textit{pg-l} by fixing initially a random sign to all atoms not appearing in the partial interpretation.

We gathered 186 instances from six different classes: \emph{Design Space exploration (DSE)} from~\cite{angeglharesc13a}, \emph{Timetabling (CTT)}
from~\cite{basotainsc13a}, \emph{Crossing minimization} from the ASP competition 2013, \emph{Metabolic network expansion} from \cite{schthi09a},
\emph{Biological network repair} from \cite{geguivscsithve10a} and \emph{Circuit Diagnosis} from~\cite{sidiqqi11a}.
Since we required instances with multiple optimal solutions, we exclusively focused on Pareto optimality. 
DSE and CTT are inherently multi-objective and therefore we could naturally define a Pareto preference for them. 
For the other classes, we turned single-objective into multi-objective optimization problems by distributing their optimization statements.
First, we split the atoms in the optimization statements into four or eight groups evenly. 
We chose for each group the same preference type, either cardinality or subset minimization, and aggregated them by means of Pareto preference.
We calculated optimal solutions regarding these Pareto preferences.
The same was done for CTT and DSE.
An instance was selected if for some Pareto preference ten optimal solutions could be obtained within 600 seconds by \asprin. 
This method generated 816 instances in total. 
We ran the benchmarks on a cluster of Linux machines with dual Xeon E5520 quad-core 2.26 GHz processors and 48 GB RAM. 
We restricted the runtime to 600 seconds and the memory usage to 20 GB RAM.

Since algorithms~\Alabel{1} and \Alabel{2} involve querying programs over preferences, 
we started by evaluating the different query techniques. 
%
For that, we executed \Alabel{1} with query methods \Qlabel{1} to \Qlabel{4} on all selected instances,
stopping after the first $\mathit{solveQuery}$ call was finished.
%
%We achieved that by first calculating an optimal solution and then finding another optimal solution fulfilling the query that the model has to be dissimilar.
The performance of query techniques \Qlabel{2}, \Qlabel{3}, and \Qlabel{4} was similar regarding runtime and only \Qlabel{1} was clearly worse.
We selected \Qlabel{4} for the remaining experiments due to its slightly lower runtime. 
For more detailed tables, we refer to~\cite{roscwa16b}. % \ref{sec:suptables}.

Next, we approximated four most diverse optimal models with methods \Alabel{1} to \Alabel{3}. 
%
We measured runtime and two quality measures.
The first, called diversification quality~\cite{nadel11a},
gives the sum of the Hamming distances among all pairs of solutions normalized to values between zero and one.
The second is the minimum distance among all pairs of solutions of a set in percent.
%
The solution set size of four was chosen because~\cite{shimazu01a} 
claims that three solutions is the optimal amount for a user,
and considering one additional solution provides further insight into the different quality measures. 
%
For all algorithms that do not use heuristics for diversification, 
we instead enabled heuristics preferring a negative sign for the atoms appearing in preference statements. 
This was observed in~\cite{brderosc15b} to improve performance.

Table~\ref{tab:time_comparison_small}(a) provides in column \textit{T} the average runtime and in column \textit{TO} the sum of timeouts. 
The different methods are ordered by the number of timeouts. 
The best results in a column are shown in bold. 
We see that \Alabel{3} is by far the fastest with 70 timeouts, solving 91\% of the instances. 
Heuristic variations of \Alabel{3} perform the best after that. 
Less invasive heuristics achieve similar runtimes with 113-118 timeouts.
More sophisticated heuristics perform worse at 349-443 timeouts.
In a range from 618 to 773 timeouts, non-heuristic methods solve the least instances by a significant margin.
The results are in tune with the nature of the methods. 
Heuristics modifying the solving process for diversity decrease the performance 
in comparison with solving heuristics aimed at performance, 
but not as much as more complex methods involving preferences over optimal models. 

In particular, non-heuristic methods show many timeouts. 
If we tried to analyze the quality of the solutions by assuming worst possible values for the instances that timed out,
the results would be dominated by these instances. 
To avoid that, we calculated a score independent of the runtime.
We considered all possible parings of the different methods. 
For each pair, we compared only instances where both found a solution set.
The method with better quality value for the majority of instances receives a point. 
Finally, we ordered the subsequent tables according to that score. 
 
In Table~\ref{tab:diverse_comparison_small}(b), for each method we see the score in column \textit{S}, and 
the average of the diversification quality (over the instances solved by the method) in column \textit{avg}. 
This way, we can examine the quality a method has achieved compared to other methods, and also the individual average quality.
\Alabel{1} has the best quality with a score of 15, followed by \Alabel{2}-\textit{dist-to}, \Alabel{2}-\textit{pg}, \Alabel{3}-\textit{pg-l} and \Alabel{3}-\textit{pg-l-rd}.
All of those techniques regard the whole previous solution set to calculate the next solution
and guide the solving strictly to diversity.
\Alabel{2}-\textit{pg}, \Alabel{3}-\textit{pg-l} and \Alabel{3}-\textit{pg-l-rd } are also the first, second and third place, respectively, for average diversification quality. 
Next, with scores ranging from 10-7, we see \Alabel{2} methods 
that do not take into account the whole previous set, 
or that were simply unable to find many solutions at all, as in the case of \Alabel{2}-\textit{dist}. 
Finally, we observe that \Alabel{3} variations only regarding the last solution or no previous information 
perform worst in score and average. 
In these cases, the heuristic does not seem to be strong enough to steer the solving to high quality solution sets, 
and \Alabel{3} uses no heuristic or optimization techniques to ensure diverse solutions.

In analogy to Table~\ref{tab:diverse_comparison_small}(b),
Table~\ref{tab:min_dist_comparison_small}(c) provides information for the minimum distance among the solutions. 
%
% The overall grouping of the methods is similar to Table~\ref{tab:diverse_comparison_small}(b). 
%
The best methods considering score and average minimum distance, 
viz.\ \Alabel{1}, \Alabel{2}-\textit{dist-to}, \Alabel{3}-\textit{pg-l-rd}, \Alabel{3}-\textit{pg-l}, \Alabel{2}-\textit{pg}, utilize information from the whole
previous solution set and have strict diversification techniques. 
%\comment{I cut the part about the different behavior of min distance and diversification. The data is not that clear and it saves space. Maybe if we have space left in the end...}

Overall, plain heuristic methods perform better in regards to runtime 
while more complex methods, depending on all previous solutions, lead to better quality. 
%
Furthermore, \Alabel{3}-\textit{pg-l-rd } and \Alabel{3}-\textit{pg-l} provide the best trade-off between performance and quality. 
%
While \Alabel{1}, \Alabel{2}-\textit{dist-to} and \Alabel{2}-\textit{pg} achieve higher quality, they could solve only 18\%, 16\% and 13\% of the instances. 
%
On the other hand, \Alabel{3}-\textit{pg-l-rd } and \Alabel{3}-\textit{pg-l} provide good diversification quality and minimum distance while solving 46\% of the instances. 
%
%\comment{this section is enough for general conclusion: plain heuristic: fast but bad, maxmin: slow but good, more complex heuristic: tradeoff}


%%% Local Variables: 
%%% mode: latex
%%% TeX-master: "paper"
%%% End: 


\section{Experiments}\label{sec:experiments}

We implemented our approach as a dedicated heuristic module within the ASP solver \textit{clasp}
(2.1; available at \cite{hclasp}).
We consider \textit{moms} \cite{pretolani96a} as initial heuristic $h_0$ and
\textit{vsids} \cite{momazhzhma01a} as heuristic function $h_i$.
Accordingly, the sign heuristic \textsl{s} is set to the one associated with \textit{vsids}.
As base configuration, we use \textit{clasp} with options \texttt{--heu=vsids} and \texttt{--init-moms}.
%
To take effect,
the heuristic atoms as well as their contained atoms must be made visible to the solver via \texttt{\#show} directives.
Once the option \texttt{--heu=domain} is passed to \texttt{clasp},
it extracts the necessary information from the symbol table and applies the heuristic modifications
when it comes to non-deterministic assignments.
%
Our experiments ran under Linux on dual Xeon E5520 quad-core processors with $2.26$GHz and $48$GB RAM.
Each run was restricted to 600s CPU time.
Timeouts account for 600s and performed choices.

% ------------------------------------------------------------
\newcommand{\rvsids}[4]{\multicolumn{3}{c|}{#1$s$ (#2)}}
\newcommand{\cvsids}[4]{\multicolumn{3}{c|}{#4}}
\newcommand{\domheu}[3]{#1\%&(#2)&#3\%}
% ------------------------------------------------------------
\begin{table}[t]
  \centering\small
  \begin{tabular}{|@{\,}r@{\,}|@{}r@{}@{}r@{}@{}r@{}|@{}r@{}@{}r@{}@{}r@{}|@{}r@{}@{}r@{}@{}r@{}|}
    \hline
    \multicolumn{1}{|c|}{Setting}                    & \multicolumn{3}{@{}c@{}|}{\textit{Labyrinth}}& \multicolumn{3}{@{}c@{}|}{\textit{Sokoban}}   & \multicolumn{3}{@{}c@{}|}{\textit{Hanoi Tower}} \\
    \hline
    \multicolumn{1}{|@{\;}l@{}|}{\textit{base configuration}} & \rvsids{9,108}{14}{5,908,451}{24,545,667}    & \rvsids{2,844}{3}{13,799,878}{19,371,267}     & \rvsids{9,137}{11}{34,126,406}{41,016,235}      \\   
                                                     & \cvsids{9,108}{14}{5,908,451}{24,545,667}    & \cvsids{2,844}{3}{13,799,878}{19,371,267}     & \cvsids{9,137}{11}{34,126,406}{41,016,235}      \\
    \hline
    \hpre{a}{\texttt{init}}{\texttt{2}}              & \domheu{95}{12}{94}                          & \domheu{91}{\textbf{1}}{84}                   & \domheu{85}{9}{89}                              \\
    \hpre{a}{\texttt{factor}}{\texttt{4}}            & \domheu{\textbf{78}}{\textbf{8}}{30}         & \domheu{120}{\textbf{1}}{107}                 & \domheu{109}{11}{110}                           \\
    \hpre{a}{\texttt{factor}}{\texttt{16}}           & \domheu{\textbf{78}}{10}{23}                 & \domheu{120}{\textbf{1}}{107}                 & \domheu{109}{11}{110}                           \\
    \hpre{a}{\texttt{level}}{\texttt{1}}             & \domheu{90}{12}{\textbf{5}}                  & \domheu{119}{2}{91}                           & \domheu{126}{15}{120}                           \\\cline{1-1}
    \hpre{f}{\texttt{init}}{\texttt{2}}              & \domheu{103}{14}{123}                        & \domheu{\textbf{74}}{2}{\textbf{71}}          & \domheu{97}{10}{109}                            \\
    \hpre{f}{\texttt{factor}}{\texttt{2}}            & \domheu{98}{12}{49}                          & \domheu{116}{3}{134}                          & \domheu{\textbf{55}}{\textbf{6}}{\textbf{70}}   \\
    \hpre{f}{\texttt{sign}}{\texttt{-1}}             & \domheu{94}{13}{89}                          & \domheu{105}{\textbf{1}}{100}                 & \domheu{92}{12}{92}                             \\
    \hline
  \end{tabular}
  \caption{Selection from evaluation of heuristic modifiers}
  \label{tab:modifiers}
\end{table}
% ------------------------------------------------------------
%
To begin with,
we report on a systematic study comparing single heuristic modifications.
A selection of best results is given in Table~\ref{tab:modifiers};
full results are available at~\cite{hclasp}.
We focus on well-known ASP planning benchmarks in order to contrast heuristic modifications on comparable problems:
\textit{Labyrinth}, \textit{Sokoban}, and \textit{Hanoi Tower}, each comprising 32 instances from the third ASP competition~\cite{contest11a}.%
\footnote{All instances are satisfiable except for one third in \textit{Sokoban}.}
We contrast the aforementioned base configuration with 38 heuristic modifications,
(separately) promoting the choice of actions (\emph{a}) and fluents (\emph{f}) via the heuristic modifiers
\texttt{factor} (1,2,4,8,16),
\texttt{init} (2,4,8,16),
\texttt{level} (1,-1),
\texttt{sign} (1,-1),
as well as attributing values to \texttt{factor}, \texttt{init}, and \texttt{level} by ascending and descending time points.
%
The first line of Table~\ref{tab:modifiers} gives the sum of times, timeouts, and choices obtained by the base configuration on all 32 instances of each problem class.
The results of the two configurations using \texttt{factor,1} differ from these figures in the low per mille range, demonstrating that the infrastructure supporting heuristic
modifications does not lead to a loss in performance.
The seven configurations in Table~\ref{tab:modifiers} yield best values in at least one category (indicated in boldface).
We express the accumulated times and choices as percentage wrt the base configuration; timeouts are total.
We see that the base configuration can always be dominated by a heuristic modification.
However, the whole spectrum of modifiers is needed to accomplish this.
In other words, there is no dominating heuristic modifier and each problem class needs a customized heuristic.
Looking at \textit{Labyrinth}, we observe that a preferred choice of action occurrences ($a$) pays off.
The stronger this is enforced, the fewer choices are made.
However, the extremely low number of choices with \texttt{level} does not result in less time or timeouts
(compared to a ``lighter'' \texttt{factor}-based enforcement).
While with \texttt{level} \emph{all} choices are made on heuristically modified atoms,
both \texttt{factor}-based modifications result in only 43\% such choices and thus leave much more room to the solver's heuristic.
For a complement, \textit{a},\texttt{init},2 as well as the base configuration (with \textit{a},\texttt{factor},1) 
make 14\% of their choices on heuristically modified atoms
(though the former produces in total 6\% less choices than the latter).
Similar yet less extreme behaviors are observed on the two other classes.
With \textit{Hanoi Tower}, a slight preference of fluents yields a strictly dominating configuration,
whereas no dominating improvement was observed with \textit{Sokoban}.

% ------------------------------------------------------------
\newcommand{\data}[2]{&\,\ignorespaces#1$s$&(\ignorespaces#2)\,}
% ------------------------------------------------------------
\begin{table}[t]
  \centering\small
  \begin{tabular}{|@{}l@{}|@{}r@{}@{}r@{}|@{}r@{}@{}r@{}|@{}r@{}@{}r@{}@{}r@{}@{}r@{}|}
    \hline
    \multicolumn{1}{|@{}c@{}|}{Setting} & \multicolumn{2}{@{}c@{}|}{\textit{Diagnosis}} & \multicolumn{2}{@{}c@{}|}{\textit{Expansion}} & \multicolumn{2}{@{}c@{}}{\textit{Repair (H)}} & \multicolumn{2}{@{}c@{}|}{\textit{Repair (S)}}\\
    \hline
    \textit{base config.}            \data{        111.1}{        115}\data{        161.5}{      100}\data{       101.3}{       113}\data{        33.3}{        27}\\
    \hline
    \texttt{s,-1}                    \data{        324.5}{        407}\data{         7.6}{         3}\data{         8.4}{         5}\data{         3.1}{\textbf{0}}\\
    \texttt{s,-1} \, \texttt{f,2}    \data{        310.1}{        387}\data{         7.4}{\textbf{2}}\data{         3.5}{\textbf{0}}\data{         3.2}{         1}\\
    \texttt{s,-1} \, \texttt{f,8}    \data{        305.9}{        376}\data{         7.7}{\textbf{2}}\data{         3.1}{\textbf{0}}\data{         2.9}{\textbf{0}}\\
    \texttt{s,-1} \, \texttt{l,1}    \data{\textbf{76.1}}{\textbf{83}}\data{\textbf{6.6}}{\textbf{2}}\data{\textbf{0.8}}{\textbf{0}}\data{         2.2}{         1}\\
\multicolumn{1}{|r@{}|}{\texttt{l,1}}\data{         77.3}{         86}\data{        12.9}{         5}\data{         3.4}{\textbf{0}}\data{\textbf{2.1}}{\textbf{0}}\\
    \hline
  \end{tabular}
  \caption{Abductive problems with optimization}
  \label{tab:opt}
\end{table}
% ------------------------------------------------------------
%
Next, we apply our heuristic approach to problems using abduction in combination with a \texttt{\#minimize} statement
minimizing the number of abducibles.
We consider
Circuit \textit{Diagnosis},
Metabolic Network \textit{Expansion},
and
Transcriptional Network \textit{Repair} (including two distinct experiments, \textit{H} and \textit{S}).
The first uses the ISCAS-85 benchmark circuits along with test cases generated as in \cite{sidiqqi11a};
this results in 790 benchmark instances.
The second one considers the completion of the metabolic network of \emph{E.coli} with reactions from \textit{MetaCyc} in view of generating target from seed metabolites~\cite{schthi09a}.
We selected the 450 most difficult benchmarks in the suite.
Finally, we consider repairing the transcriptional network of \emph{E.coli} from \textit{RegulonDB} in view of two distinct experiment series~\cite{geguivscsithve10a}.
Selecting the most difficult triple repairs provided us with 1000 instances.
%
Our results are summarized in Table~\ref{tab:opt}.
Each entry gives the average runtime and number of timeouts.
% Otherwise the experimental setting is as above.
%
Here, heuristic modifiers apply only to abducibles subject to minimization.
%
For supporting minimization,
we assign false to such abducibles (\texttt{s,-1})%
\footnote{Assigning \true\ instead leads to a deterioration of performance.}
and gradually increase the bias of their choice by imposing \texttt{factor} 2 and 8 (\texttt{f})
or enforce it via a \texttt{level} modifier (\texttt{l,1}).
%
The second last setting%
\footnote{This corresponds to using \hpre{a}{\texttt{false}}{1} for an abducible $a$.}
in Table~\ref{tab:opt} is the winner, leading to speedups of one to two orders of magnitude over the base configuration.
Interestingly, merely fixing the sign heuristics to \false\ leads at first to a deterioration of performance on \textit{Diagnosis} problems.
This is finally overcome by the constant improvement observed by gradually strengthening the bias of choosing abducibles.
The stronger the preference for abducibles, the faster the solver converges to an optimum solution.
This limited experiment already illustrates that sometimes the right combination of heuristic
modifiers yields the best result.

Finally, let us consider true PDDL planning problems.
For this, we selected 20 instances from the % and \textit{Freecell'02} in view of \textit{Freecell'00}.}
STRIPS domains of the 2000 and 2002 planning competition~\cite{icaps-competition}.%
\footnote{We discard \textit{Schedule'00} due to grounding issues.}
In turn, we translated these PDDL instances into facts via \textit{plasp}~\cite{gekaknsc11a}
and used a simple planning encoding with 15 different plan lengths (\verb+l=5,10,..,75+) to generate 3000 ASP instances.
%
Inspired yet different from \cite{rintanen12a}, we devised a dynamic heuristic that aims at propagating fluents' truth values backwards in time.
Attributing levels via \texttt{l-T+1} aims at proceeding depth-first from the goal fluents.
\begin{lstlisting}
_h(holds(F,T-1),true, l-T+1) :- holds(F,T).
_h(holds(F,T-1),false,l-T+1) :-
        fluent(F), time(T), not holds(F,T).
\end{lstlisting}
% ------------------------------------------------------------
\newcommand{\pdata}[3]{&\,\ignorespaces#1$s$&(\ignorespaces#2/\ignorespaces#3)}
\newcommand{\sdata}[3]{&\,\ignorespaces#1$s$&(\ignorespaces#3)}
% ------------------------------------------------------------
\setlength{\textfloatsep}{15pt plus 1.0pt minus 2.0pt}
\begin{table}[t]
  \centering\scriptsize
  \begin{tabular}{|@{}r@{\,}|@{}r@{}@{}r@{}|@{}r@{}@{}r@{}|@{}r@{}@{}r@{}|r@{}@{}r|}
    \hline
   \multicolumn{1}{|@{}c@{\,}|}{Problem} & \multicolumn{2}{@{}c@{}|}{\textit{base}} & \multicolumn{2}{@{}c@{}|}{\textit{base}+\texttt{\_h}} & \multicolumn{2}{@{}c@{}|}{\textit{base (SAT)}} & \multicolumn{2}{@{}c@{}|}{\textit{\textit{base}+\texttt{\_h} (SAT)}} \\
    \hline
    \textit{Blocks'00}    \pdata{134.4}{ 180}{  61}\pdata{  9.2}{ 239}{  3}\sdata{163.2}{180}{59}\sdata{ 2.6}{239}{0}\\
    \textit{Elevator'00}  \pdata{  3.1}{ 279}{   0}\pdata{  0.0}{ 279}{  0}\sdata{  3.4}{279}{ 0}\sdata{ 0.0}{279}{0}\\
    \textit{Freecell'00}  \pdata{288.7}{ 147}{ 115}\pdata{184.2}{ 194}{ 74}\sdata{226.4}{147}{47}\sdata{52.0}{194}{0}\\
    \textit{Logistics'00} \pdata{145.8}{ 148}{  61}\pdata{115.3}{ 168}{ 52}\sdata{113.9}{148}{23}\sdata{15.5}{168}{3}\\
    \hline
    \textit{Depots'02}    \pdata{400.3}{  51}{ 184}\pdata{297.4}{ 115}{135}\sdata{389.0}{ 51}{64}\sdata{61.6}{115}{0}\\
    \textit{Driverlog'02} \pdata{308.3}{ 108}{ 143}\pdata{189.6}{ 169}{ 92}\sdata{245.8}{108}{61}\sdata{ 6.1}{169}{0}\\
    \textit{Rovers'02}    \pdata{245.8}{ 138}{ 112}\pdata{165.7}{ 179}{ 79}\sdata{162.9}{138}{41}\sdata{ 5.7}{179}{0}\\
    \textit{Satellite'02} \pdata{398.4}{  73}{ 186}\pdata{229.9}{ 155}{106}\sdata{364.6}{ 73}{82}\sdata{30.8}{155}{0}\\
    \textit{Zenotravel'02}\pdata{350.7}{ 101}{ 169}\pdata{239.0}{ 154}{116}\sdata{224.5}{101}{53}\sdata{ 6.3}{154}{0}\\
    \hline
    \textit{Total}          \pdata{252.8}{1225}{1031}\pdata{158.9}{1652}{657}\sdata{187.2}{1225}{430}\sdata{17.1}{1652}{3}\\
    \hline
  \end{tabular}
  \caption{Planning Competition Benchmarks '00 and '02}
  \label{tab:plan}
\end{table}%
% ------------------------------------------------------------
%
Our results are given in Table~\ref{tab:plan}.
Each entry gives the average runtime along with the number of (solved satisfiable instances and) timeouts (in columns two and three).
% Otherwise the experimental setting is as above.
Our heuristic amendment (\textit{base}+\texttt{\_h}) greatly improves over the base configuration in terms of runtime and timeouts.
On the overall set of benchmarks, it provides us with 427 more plans and 374 less timeouts.
As already observed by \cite{rintanen12a}, the heuristic effect is stronger on satisfiable instances.
This is witnessed by the two last columns restricting results to 1655 satisfiable instances solved by either system setup.
Our heuristic extension allows us to reduce the total number of timeouts from 430 to 3;
the reduction in solving time would be even more drastic with a longer timeout.

Interestingly, the previous dynamic heuristic has no overwhelming effect on our initial ASP planning problems.
An improvement was only observed on \textit{Hanoi Tower} problems (being susceptible to choices on fluents),
viz.\ `{54\%}(7)\,\textbf{57\%}' in terms of the format used in Table~\ref{tab:modifiers}.
However, restricting the heuristic to positive fluents by only using the first rule gives a substantial
improvement, namely `\textbf{19\%}(\textbf{2})\,{66\%}', in terms of runtime and timeouts.
A direct comparison of both heuristics shows that, although the latter performs 15\% more choices, 
it encounters 75\% fewer conflicts than the former.

%%% Local Variables: 
%%% mode: latex
%%% TeX-master: "paper"
%%% End: 

%% 
\section{Discussion}\label{sec:discussion}

Various ways of adding domain-specific information have been explored in the literature.
%
A prominent approach is to implement forms of preferential reasoning
% , like reasoning wrt inclusion-minimal models, 
by directing choices through
a given partial order on literals~\cite{cacacale96a,rogima10a,giumar12a}.
%
To some degree, this can be simulated by heuristic modifiers like
\hpre{a}{\texttt{false}}{1}
that allow for computing a (single) inclusion-minimal model.
However, as detailed in \cite{rogima10a}, enumerating all such models needs additional constraints
or downstream tester programs.
Similarly,
\cite{balduccini11b} modifies the heuristic of the ASP solver \textit{smodels} to accommodate learning from smaller instances.
See also~\cite{falepf01a,falemari07a}.
Most notably,
\cite{rintanen12a} achieves impressive results in planning by equipping a SAT solver with
planning-specific heuristics.
%
All aforementioned approaches need customized changes to solver implementations.
%
Hence, it will be interesting to investigate how these approaches can be expressed and combined in
our declarative framework.
%
Declarative approaches to incorporating control knowledge can be found in heuristic planning.
For instance, \cite{backab00a} harness temporal logic formulas, while \cite{sierra04a} also uses
dedicated predicates for controlling backtracking in a forward planner.
%
However,
care must be taken when it comes to modifying a solver's heuristics.
Although it may lead to great improvements, it may just as well lead to a degradation of search.
In fact, the restriction of choice variables may result in exponentially larger search spaces~\cite{jajuni05a}.
This issue is reflected in our choice of heuristic modifiers, 
ranging from an \texttt{init}ial bias,
over a continued yet scalable one by \texttt{factor},
to a strict preference with \texttt{level}.

To sum up,
we introduced a declarative framework for incorporating domain-specific heuristics into ASP solving.
The seamless integration into ASP's input language provides us with a general and flexible tool for
expressing domain-specific heuristics.
As such, we believe it to be the first of its kind.
Our heuristic framework offers completely new possibilities of applying, experimenting, and studying
domain-specific heuristics in a uniform setting.
Our example heuristics merely provide first indications on the prospect of our approach,
but much more systematic empirical studies are needed to exploit its full power.


%%% Local Variables: 
%%% mode: latex
%%% TeX-master: "paper"
%%% End: 


\section{Discussion}\label{sec:discussion}

Various ways of adding domain-specific information have been explored in the literature.
%
A prominent approach is to implement forms of preferential reasoning
% , like reasoning wrt inclusion-minimal models, 
by directing choices through
a given partial order on literals~\cite{cacacale96a,rogima10a,giumar12a}.
%
To some degree, this can be simulated by heuristic modifiers like
\hpre{a}{\texttt{false}}{1}
that allow for computing a (single) inclusion-minimal model.
However, as detailed in \cite{rogima10a}, enumerating all such models needs additional constraints
or downstream tester programs.
Similarly,
\cite{balduccini11b} modifies the heuristic of the ASP solver \textit{smodels} to accommodate learning from smaller instances.
See also~\cite{falepf01a,falemari07a}.
Most notably,
\cite{rintanen12a} achieves impressive results in planning by equipping a SAT solver with
planning-specific heuristics.
%
All aforementioned approaches need customized changes to solver implementations.
%
Hence, it will be interesting to investigate how these approaches can be expressed and combined in
our declarative framework.
%
Declarative approaches to incorporating control knowledge can be found in heuristic planning.
For instance, \cite{backab00a} harness temporal logic formulas, while \cite{sierra04a} also uses
dedicated predicates for controlling backtracking in a forward planner.
%
However,
care must be taken when it comes to modifying a solver's heuristics.
Although it may lead to great improvements, it may just as well lead to a degradation of search.
In fact, the restriction of choice variables may result in exponentially larger search spaces~\cite{jajuni05a}.
This issue is reflected in our choice of heuristic modifiers, 
ranging from an \texttt{init}ial bias,
over a continued yet scalable one by \texttt{factor},
to a strict preference with \texttt{level}.

To sum up,
we introduced a declarative framework for incorporating domain-specific heuristics into ASP solving.
The seamless integration into ASP's input language provides us with a general and flexible tool for
expressing domain-specific heuristics.
As such, we believe it to be the first of its kind.
Our heuristic framework offers completely new possibilities of applying, experimenting, and studying
domain-specific heuristics in a uniform setting.
Our example heuristics merely provide first indications on the prospect of our approach,
but much more systematic empirical studies are needed to exploit its full power.


%%% Local Variables: 
%%% mode: latex
%%% TeX-master: "paper"
%%% End: 


%% \paragraph{Acknowledgments}

This work was partially funded 
by 
the German Science Foundation (DFG) under 
% DFG
grant 
%SCHA 550/8-3   % clasp et al
SCHA 550/9   % Inc/ReaASP
%SCHA 550/10-1  % Bio/clingcon
and
SCHA 550/11.  % DSE/Haubelt

%%% Local Variables: 
%%% mode: latex
%%% TeX-master: "paper"
%%% End: 

\smallskip\noindent\emph{Acknowledgments.}
%
This work was partly funded 
by 
% the German Science Foundation (DFG)
DFG grants 
SCHA 550/8-3 % clasp et al
and
SCHA 550/9-1.   % Inc/ReaASP
% SCHA 550/10-1  % Bio/clingcon

%%% Local Variables: 
%%% mode: latex
%%% TeX-master: "paper"
%%% End: 


\bibliographystyle{aaai}
% \bibliography{lit,akku,procs,local} % https://svn.cs.uni-potsdam.de/svn/reposWV/Papers/bibfiles/trunk
%% \begin{thebibliography}{10}

% \bibitem{Acuna2009}
% V.~Acu{\~{n}}a, F.~Chierichetti, V.~Lacroix, A.~Marchetti-Spaccamela, M.~Sagot, L.~Stougie.
% \newblock {Modes and cuts in metabolic networks: Complexity and algorithms}.
% \newblock {\em Biosystems}, 95(1):51--60, 2009.

\bibitem{anbole13a}
C.~Ans{\'o}tegui, M.~Bonet, J.~Levy.
\newblock {SAT}-based {MaxSAT} algorithms.
\newblock {\em Artificial Intelligence}, 196:77--105, 2013.

\bibitem{baral02a}
C.~Baral.
\newblock {\em Knowledge Representation, Reasoning and Declarative Problem Solving}.
\newblock Cambridge, 2003.

\bibitem{Becker2007}
S.~Becker, A.~Feist, M.~Mo, G.~Hannum, B.~Palsson, M.~Herrgard.
\newblock {Quantitative Prediction of Cellular Metabolism with Constraint-based Models: The COBRA Toolbox}.
\newblock {\em Nature Protocols}, 2(3):727--738, 2007.

\bibitem{coevgeprscsith13a}
G.~Collet, D.~Eveillard, M.~Gebser, S.~Prigent, T.~Schaub, A.~Siegel, S.~Thiele.
\newblock Extending the metabolic network of {E}ctocarpus siliculosus using answer set programming.
\newblock 
% In P.~Cabalar and T.~Son, editors, {\em Proceedings of the Twelfth International Conference on Logic Programming and Nonmonotonic Reasoning (LPNMR'13)}, volume 8148 of {\em Lecture Notes in Artificial Intelligence}, 
{\em Proceedings LPNMR},
245--256. Springer, 2013.

\bibitem{dantzig63a}
G.~Dantzig.
\newblock {\em Linear Programming and Extensions}.
\newblock Princeton, 1963.

% \bibitem{ebhahe04a}
% O.~Ebenhöh, T.~Handorf, R.~Heinrich.
% \newblock Structural analysis of expanding metabolic networks.
% \newblock {\em Genome Informatics}, 15(1):35--45, 2004.

\bibitem{Ebrahim2013}
A.~Ebrahim, J.~Lerman, B.~Palsson, D.~Hyduke.
\newblock {COBRApy: COnstraints-Based Reconstruction and Analysis for Python.}
\newblock {\em BMC Systems Biology}, 7:74, aug 2013.

\bibitem{gekakaosscwa16a}
M.~Gebser, R.~Kaminski, B.~Kaufmann, M.~Ostrowski, T.~Schaub, P.~Wanko.
\newblock Theory solving made easy with clingo~5.
\newblock 
% In M.~Carro and A.~King, editors, {\em Technical Communications of the Thirty-second International Conference on Logic Programming (ICLP'16)}, volume~52, 
{\em Technical Comm.\ ICLP},
2:1--2:15. 
% Open Access Series in Informatics
OASIcs, 2016.

\bibitem{gekakarosc15a}
M.~Gebser, R.~Kaminski, B.~Kaufmann, J.~Romero, T.~Schaub.
\newblock Progress in clasp series 3.
\newblock 
% In F.~Calimeri, G.~Ianni, and M.~Truszczy{\'n}ski, editors, {\em Proceedings of the Thirteenth International Conference on Logic Programming and Nonmonotonic Reasoning (LPNMR'15)}, volume 9345 of {\em Lecture Notes in Artificial Intelligence}, 
{\em Proceedings LPNMR},
368--383. Springer, 2015.

\bibitem{gellif91a}
M.~Gelfond, V.~Lifschitz.
\newblock Classical negation in logic programs and disjunctive databases.
\newblock {\em New Generation Computing}, 9:365--385, 1991.

\bibitem{haebhe05a}
T.~Handorf, O.~Ebenhöh, R.~Heinrich.
\newblock Expanding metabolic networks: Scopes of compounds, robustness, and evolution.
\newblock {\em J.\ of Molec.\ Evolution}, 61(4):498--512, 2005.

\bibitem{laten2014a}
M.~Latendresse.
\newblock Efficiently gap-filling reaction networks.
\newblock {\em BMC bioinformatics}, 15(1):225, 2014.

\bibitem{marzom16a}
C.~Maranas, A.~Zomorrodi.
\newblock {\em Optimiz.\ methods in metabolic networks}.
\newblock Wiley, 2016.

\bibitem{Orth2010}
J.~Orth, B.~Palsson.
\newblock {Systematizing the generation of missing metabolic knowledge.}
\newblock {\em Biotechnology and bioengineering}, 107(3):403--12, oct 2010.

% \bibitem{orthpa10a}
% J.~Orth, I.~Thiele, B.~Palsson.
% \newblock What is flux balance analysis?
% \newblock {\em Nature biotechnology}, 28(3):245--248, 2010.

\bibitem{ostsch12a}
M.~Ostrowski, T.~Schaub.
\newblock {ASP} modulo {CSP}: The clingcon system.
\newblock {\em Theory and Practice of Logic Programming}, 12(4-5):485--503, 2012.

% \bibitem{potassco}
% Potassco website.
% \newblock http://potassco.org.

\bibitem{prcodideetdaevthcabosito14a}
S.~Prigent, G.~Collet, S.~Dittami, L.~Delage, F.~{Ethis de Corny}, O.~Dameron, D.~Eveillard, S.~Thiele, J.~Cambefort, C.~Boyen, A.~Siegel, T.~Tonon.
\newblock The genome-scale metabolic network of ectocarpus siliculosus (ectogem): a resource to study brown algal physiology and beyond.
\newblock {\em The Plant Journal}, 80(2):367–381, 2014.

\bibitem{Prigent2017}
S.~Prigent, C~.Frioux, S.~Dittami, S.~Thiele, A.~Larhlimi, G.~Collet, F.~Gutknecht, J.~Got, D.~Eveillard, J.~Bourdon, F.~Plewniak, T.~Tonon, A.~Siegel.
\newblock {Meneco, a Topology-Based Gap-Filling Tool Applicable to Degraded Genome-Wide Metabolic Networks}.
\newblock {\em PLOS Computational Biology}, 13(1):e1005276, jan 2017.

\bibitem{Reed2003}
J.~Reed, T.~Vo, C.~Schilling, B.~Palsson.
\newblock {An expanded genome-scale model of Escherichia coli K-12 (iJR904 GSM/GPR).}
\newblock {\em Genome Biology}, 4(9):R54, 2003.

\bibitem{SatishKumar2007}
V.~{Satish Kumar}, M.~Dasika, C.~Maranas.
\newblock {Optimization based automated curation of metabolic reconstructions}.
\newblock {\em BMC Bioinformatics}, 8(1):212, 2007.

\bibitem{schthi09a}
T.~Schaub, S.~Thiele.
\newblock Metabolic network expansion with {ASP}.
\newblock
% In P.~Hill and D.~Warren, editors, {\em Proceedings of the Twenty-fifth International Conference on Logic Programming (ICLP'09)}, volume 5649 of {\em Lecture Notes in Computer Science}, 
{\em Proceedings ICLP},
312--326. Springer, 2009.

\bibitem{siniso02a}
P.~Simons, I.~Niemelä, T.~Soininen.
\newblock Extending and implementing the stable model semantics.
\newblock {\em Artificial Intelligence}, 138(1-2):181--234, 2002.

\bibitem{Thiele2014}
I.~Thiele, N.~Vlassis, R.~Fleming.
\newblock {fastGapFill: efficient gap filling in metabolic networks.}
\newblock {\em Bioinformatics}, 30(17):2529--2531, sep 2014.

\bibitem{Vitkin2012}
E.~Vitkin, T.~Shlomi.
\newblock {MIRAGE: a functional genomics-based approach for metabolic network model reconstruction and its application to cyanobacteria networks.}
\newblock {\em Genome Biology}, 13(11):R111, 2012.

\end{thebibliography}

%%% Local Variables:
%%% mode: latex
%%% TeX-master: "paper"
%%% End:

\begin{thebibliography}{}

\bibitem[\protect\citeauthoryear{Bacchus and Kabanza}{2000}]{backab00a}
Bacchus, F., and Kabanza, F.
\newblock 2000.
\newblock Using temporal logics to express search control knowledge for
  planning.
\newblock {\em Artificial Intelligence} 116(1-2):123--191.

\bibitem[\protect\citeauthoryear{Balduccini}{2011}]{balduccini11b}
Balduccini, M.
\newblock 2011.
\newblock Learning and using domain-specific heuristics in {ASP} solvers.
\newblock {\em AI Communic.} 24(2):147--164.

\bibitem[\protect\citeauthoryear{Baral}{2003}]{baral02a}
Baral, C.
\newblock 2003.
\newblock {\em Knowledge Representation, Reasoning and Declarative Problem
  Solving}.
\newblock Cambridge University Press.

\bibitem[\protect\citeauthoryear{Biere \bgroup et al\mbox.\egroup
  }{2009}]{SATHandbook}
Biere, A.; Heule, M.; {van Maaren}, H.; and Walsh, T., eds.
\newblock 2009.
\newblock {\em Handbook of Satisfiability}, volume 185 of {\em Frontiers in
  Artificial Intelligence and Applications}.
\newblock IOS Press.

\bibitem[\protect\citeauthoryear{Calimeri \bgroup et al\mbox.\egroup
  }{2011}]{contest11a}
Calimeri, F. et al.
\newblock 2011.
\newblock The third answer set programming competition: Preliminary report of
  the system competition track.
\newblock In Delgrande and Faber \shortcite{lpnmr11},  388--403.

\bibitem[\protect\citeauthoryear{Castell \bgroup et al\mbox.\egroup
  }{1996}]{cacacale96a}
Castell, T.; Cayrol, C.; Cayrol, M.; and {Le Berre}, D.
\newblock 1996.
\newblock Using the {D}avis and {P}utnam procedure for an efficient computation
  of preferred models.
\newblock In Wahlster, W., ed., {\em Proceedings of the Twelfth European
  Conference on Artificial Intelligence (ECAI'96)},  350--354.
\newblock John Wiley \& sons.

\bibitem[\protect\citeauthoryear{Dechter}{2003}]{dechter03}
Dechter, R.
\newblock 2003.
\newblock {\em Constraint Processing}.
\newblock Morgan Kaufmann.

\bibitem[\protect\citeauthoryear{Delgrande and Faber}{2011}]{lpnmr11}
Delgrande, J., and Faber, W., eds.
\newblock 2011.
\newblock {\em Proceedings of the Eleventh International Conference on Logic
  Programming and Nonmonotonic Reasoning (LPNMR'11)}. Springer.

\bibitem[\protect\citeauthoryear{{Di Rosa}, Giunchiglia, and
  Maratea}{2010}]{rogima10a}
{Di Rosa}, E.; Giunchiglia, E.; and Maratea, M.
\newblock 2010.
\newblock Solving satisfiability problems with preferences.
\newblock {\em Constraints} 15(4):485--515.

\bibitem[\protect\citeauthoryear{Faber \bgroup et al\mbox.\egroup
  }{2007}]{falemari07a}
Faber, W.; Leone, N.; Maratea, M.; and Ricca, F.
\newblock 2007.
\newblock Experimenting with look-back heuristics for hard {ASP} programs.
\newblock In Baral, C.; Brewka, G.; and Schlipf, J., eds., {\em Proceedings of
  the Ninth International Conference on Logic Programming and Nonmonotonic
  Reasoning (LPNMR'07)},  110--122.
\newblock Springer.

\bibitem[\protect\citeauthoryear{Faber, Leone, and Pfeifer}{2001}]{falepf01a}
Faber, W.; Leone, N.; and Pfeifer, G.
\newblock 2001.
\newblock Experimenting with heuristics for answer set programming.
\newblock In Nebel, B., ed., {\em Proceedings of the Seventeenth International
  Joint Conference on Artificial Intelligence (IJCAI'01)},  635--640.
\newblock Morgan Kaufmann.

\bibitem[\protect\citeauthoryear{Freeman}{1995}]{freeman95a}
Freeman, J.
\newblock 1995.
\newblock {\em Improvements to Propositional Satisfiability Search Algorithms}.
\newblock Ph.D. Dissertation, University of Pennsylvania.

\bibitem[\protect\citeauthoryear{Gebser \bgroup et al\mbox.\egroup
  }{2010}]{geguivscsithve10a}
Gebser, M.; Guziolowski, C.; Ivanchev, M.; Schaub, T.; Siegel, A.; Thiele, S.;
  and Veber, P.
\newblock 2010.
\newblock Repair and prediction (under inconsistency) in large biological
  networks with answer set programming.
\newblock In Lin, F., and Sattler, U., eds., {\em Proceedings of the Twelfth
  International Conference on Principles of Knowledge Representation and
  Reasoning (KR'10)},  497--507.
\newblock AAAI Press.

\bibitem[\protect\citeauthoryear{Gebser \bgroup et al\mbox.\egroup
  }{2011}]{gekaknsc11a}
Gebser, M.; Kaminski, R.; Knecht, M.; and Schaub, T.
\newblock 2011.
\newblock plasp: A prototype for {PDDL}-based planning in {ASP}.
\newblock In Delgrande and Faber \shortcite{lpnmr11},  358--363.

\bibitem[\protect\citeauthoryear{Gebser \bgroup et al\mbox.\egroup
  }{2012}]{gekakasc12a}
Gebser, M.; Kaminski, R.; Kaufmann, B.; Schaub, T.
\newblock 2012.
\newblock {\em Answer Set Solving in Practice}.
\newblock  Morgan and Claypool.
% Synthesis Lectures on Artificial Intelligence and Machine Learning.


\bibitem[\protect\citeauthoryear{Giunchiglia and Maratea}{2012}]{giumar12a}
Giunchiglia, E., and Maratea, M.
\newblock 2012.
\newblock Algorithms for solving satisfiability problems with qualitative
  preferences.
\newblock In Erdem, E.; Lee, J.; Lierler, Y.; and Pearce, D., eds., {\em
  Correct Reasoning: Essays on Logic-Based {AI} in Honour of {V}ladimir
  {L}ifschitz},
  327--344.
\newblock Springer.

\bibitem[\protect\citeauthoryear{Goldberg and Novikov}{2002}]{golnov02a}
Goldberg, E., and Novikov, Y.
\newblock 2002.
\newblock {BerkMin}: A fast and robust {SAT} solver.
\newblock In {\em Proceedings of the Fifth Conference on Design, Automation and
  Test in Europe (DATE'02)},  142--149.
\newblock IEEE Computer Society Press.

\bibitem[\protect\citeauthoryear{hclasp\ignorespaces}{\ignorespaces}]{hclasp}
\newblock hclasp. \texttt{http://www.cs.uni-potsdam.de/hclasp}.

\bibitem[\protect\citeauthoryear{ICAPS\ignorespaces}{\ignorespaces}]{icaps-competition}
\newblock ICAPS. \texttt{http://ipc.icaps-conference.org}.

\bibitem[\protect\citeauthoryear{J{\"a}rvisalo, Junttila, and
  Niemel{\"a}}{2005}]{jajuni05a}
J{\"a}rvisalo, M.; Junttila, T.; and Niemel{\"a}, I.
\newblock 2005.
\newblock Unrestricted vs restricted cut in a tableau method for {B}oolean
  circuits.
\newblock {\em Annals of Mathematics and Artificial Intelligence}
  44(4):373--399.

\bibitem[\protect\citeauthoryear{Korf}{1985}]{korf85a}
Korf, R.
\newblock 1985.
\newblock Depth-first iterative-deepening: An optimal admissible tree search.
\newblock {\em Artificial Intelligence} 27(1):97--109.

\bibitem[\protect\citeauthoryear{Lifschitz}{2002}]{lifschitz02a}
Lifschitz, V.
\newblock 2002.
\newblock Answer set programming and plan generation.
\newblock {\em Artificial Intelligence} 138(1-2):39--54.

\bibitem[\protect\citeauthoryear{Marques-Silva and Sakallah}{1999}]{marsak99a}
Marques-Silva, J., and Sakallah, K.
\newblock 1999.
\newblock {GRASP}: A search algorithm for propositional satisfiability.
\newblock {\em IEEE Transactions on Computers} 48(5):506--521.

\bibitem[\protect\citeauthoryear{Moskewicz \bgroup et al\mbox.\egroup
  }{2001}]{momazhzhma01a}
Moskewicz, M.; Madigan, C.; Zhao, Y.; Zhang, L.; and Malik, S.
\newblock 2001.
\newblock Chaff: Engineering an efficient {SAT} solver.
\newblock In {\em Proceedings of the Thirty-eighth Conference on Design
  Automation (DAC'01)},  530--535.
\newblock ACM Press.

\bibitem[\protect\citeauthoryear{Pipatsrisawat and Darwiche}{2007}]{pipdar07a}
Pipatsrisawat, K., and Darwiche, A.
\newblock 2007.
\newblock A lightweight component caching scheme for satisfiability solvers.
\newblock In Marques-Silva, J., and Sakallah, K., eds., {\em Proceedings of the
  Tenth International Conference on Theory and Applications of Satisfiability
  Testing (SAT'07)},
  294--299.
\newblock Springer.

\bibitem[\protect\citeauthoryear{Pretolani}{1996}]{pretolani96a}
Pretolani, D.
\newblock 1996.
\newblock Efficiency and stability of hypergraph {SAT} algorithms.
\newblock In Johnson, D., and Trick, M., eds., {\em DIMACS Series in Discrete
  Mathematics and Theoretical Computer Science}, volume~26,  479--498.
\newblock American Mathematical Society.

\bibitem[\protect\citeauthoryear{Rintanen}{2011}]{rintanen11a}
Rintanen, J.
\newblock 2011.
\newblock Planning with {SAT}, admissible heuristics and {A}$^*$.
\newblock In Walsh \shortcite{ijcai11},  2015--2020.

\bibitem[\protect\citeauthoryear{Rintanen}{2012}]{rintanen12a}
Rintanen, J.
\newblock 2012.
\newblock Planning as satisfiability: heuristics.
\newblock {\em Artificial Intelligence} 193:45--86.

\bibitem[\protect\citeauthoryear{Schaub and Thiele}{2009}]{schthi09a}
Schaub, T., and Thiele, S.
\newblock 2009.
\newblock Metabolic network expansion with {ASP}.
\newblock In Hill, P., and Warren, D., eds., {\em Proceedings of the
  Twenty-fifth International Conference on Logic Programming (ICLP'09)},  312--326.
\newblock Springer.

\bibitem[\protect\citeauthoryear{Siddiqi}{2011}]{sidiqqi11a}
Siddiqi, S.
\newblock 2011.
\newblock Computing minimum-cardinality diagnoses by model relaxation.
\newblock In Walsh \shortcite{ijcai11},  1087--1092.

\bibitem[\protect\citeauthoryear{Sierra-Santib{\'a}{\~n}ez}{2004}]{sierra04a}
Sierra-Santib{\'a}{\~n}ez, J.
\newblock 2004.
\newblock Heuristic planning: A declarative approach based on strategies for
  action selection.
\newblock {\em Artificial Intelligence} 153(1-2):307--337.

\bibitem[\protect\citeauthoryear{Walsh}{2011}]{ijcai11}
Walsh, T., ed.
\newblock 2011.
\newblock {\em Proceedings of the Twenty-second International Joint Conference
  on Artificial Intelligence (IJCAI'11)}. IJCAI/AAAI.

\bibitem[\protect\citeauthoryear{Zhang \bgroup et al\mbox.\egroup
  }{2001}]{zamamoma01a}
Zhang, L.; Madigan, C.; Moskewicz, M.; and Malik, S.
\newblock 2001.
\newblock Efficient conflict driven learning in a {B}oolean satisfiability
  solver.
\newblock In {\em Proceedings of the International Conference on Computer-Aided
  Design (ICCAD'01)},  279--285.
\newblock ACM Press.

\end{thebibliography}

%%% Local Variables: 
%%% mode: latex
%%% TeX-master: "paper"
%%% End: 

\end{document}
