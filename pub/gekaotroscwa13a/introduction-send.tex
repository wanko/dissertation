
\section{Introduction}\label{sec:introduction}

The success of modern Boolean constraint technology was greatly boosted by Satisfiability Testing
(SAT;~\cite{SATHandbook}).
Meanwhile, this technology has been taken up in many related areas, 
like
Answer Set Programming (ASP; \cite{baral02a}),
because it provides highly performant yet general-purpose solving techniques for addressing demanding combinatorial search problems.
%
Sometimes, it is however advantageous to take a more application-oriented approach
by including domain-specific information.
On the one hand, domain-specific knowledge can be added for improving deterministic assignments
through propagation.
And on the other hand, domain-specific heuristics can be used for making better non-deterministic assignments.

In what follows,
we introduce a general declarative framework for incorporating domain-specific heuristics into ASP solving.
The choice of ASP is motivated by its first-order modeling language offering an easy way to express and process
heuristic information.
To this end,
we use a dedicated predicate \hpredicate\
whose arguments allow us to express various modifications to the solver's heuristic treatment of atoms.
The respective heuristic rules are seamlessly processed as an equitable part of the logic program
and subsequently exploited by the solver when it comes to choosing an atom for a non-deterministic 
truth assignment.
%
For instance, the rule
\begin{lstlisting}
_h(occ(A,T),factor,T) :- action(A), time(T).
\end{lstlisting}
favors later action occurrences over earlier ones (via multiplication by \texttt{T}).
That is, when making a choice between the two unassigned atoms \texttt{occ(a,2)} and \texttt{occ(b,3)},
the solver's heuristic value of \texttt{occ(a,2)} is doubled while that of \texttt{occ(b,3)} is tripled.
This results in a bias on the system heuristic that may or may not take effect.
%
Besides \texttt{factor}, our heuristic language extension offers the primitive heuristic modifiers
\texttt{init}, \texttt{level}, and \texttt{sign}, from which even further modifiers can be defined.
Our approach provides an easy and flexible access to the solver's heuristic,
aiming at its modification rather than its replacement.
Note that the effect of the modifications is generally dynamic,
unless the truth of a heuristic atom is determined during grounding (as with the rule above).
As a result, our approach offers a declarative framework for expressing domain-specific heuristics. %,
%thereby abolishing the need for special-purpose implementations.
As such, it appears to be the first of its kind.

%%% Local Variables: 
%%% mode: latex
%%% TeX-master: "paper"
%%% End: 
