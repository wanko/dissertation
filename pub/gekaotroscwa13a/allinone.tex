%% $Id: paper.tex 34991 2013-04-06 07:51:22Z torsten $
%% $HeadURL: https://svn.cs.uni-potsdam.de/svn/reposWV/Papers/ASPDomHeu/trunk/paper.tex $
\documentclass[letterpaper]{article}
\usepackage{aaai}
\usepackage{times,helvet,courier}
\usepackage{amsmath,amssymb}

\setlength{\pdfpagewidth}{8.5in}
\setlength{\pdfpageheight}{11in}
\usepackage{verbatim}

\usepackage{listings}
\lstset{aboveskip=\smallskipamount,belowskip=\smallskipamount}
\lstset{basicstyle=\ttfamily\small}
%% % - rules etc --------------------------------------------------------------------
\newcommand{\naf}[1]{\ensuremath{{\sim}{#1}}}
\newcommand{\poslits}[1]{\ensuremath{#1^+}}
\newcommand{\neglits}[1]{\ensuremath{#1^-}}
\newcommand{\body}[1]{\ensuremath{B(#1)}} % {\ensuremath{\mathit{body}(#1)}}
\newcommand{\pbody}[1]{\poslits{\body{#1}}} % {\ensuremath{\mathit{body}^+(#1)}}
\newcommand{\nbody}[1]{\neglits{\body{#1}}} % {\ensuremath{\mathit{body}^-(#1)}}
\newcommand{\head}[1]{\ensuremath{h(#1)}} % {\ensuremath{\mathit{head}(#1)}}

\newcommand{\Tsign}{\ensuremath{\mathbf{T}}}
\newcommand{\Fsign}{\ensuremath{\mathbf{F}}}
\newcommand{\Tlit}[1]{\ensuremath{\Tsign #1}}
\newcommand{\Flit}[1]{\ensuremath{\Fsign #1}}
\newcommand{\Ass}{\ensuremath{\mathbf{A}}}
\newcommand{\DL}{\ensuremath{\mathit{DL}}}

\newcommand{\code}[1]{{\ttfamily #1}}
\newcommand{\codeClass}[2]{\code{#2}}

% - systems ----------------------------------------------------------------------
%
\newcommand{\sysfont}{\textit}
\newcommand{\acthex}{\sysfont{acthex}}
\newcommand{\asparagus}{\sysfont{asparagus}}
\newcommand{\aspic}{\sysfont{aspic}}
\newcommand{\aspmt}{\sysfont{aspmt}}
\newcommand{\asprin}{\sysfont{asprin}}
\newcommand{\assat}{\sysfont{assat}}
\newcommand{\berkmin}{\sysfont{berkmin}}
\newcommand{\claspD}{\sysfont{claspD}}
\newcommand{\claspar}{\sysfont{claspar}}
\newcommand{\claspfolio}{\sysfont{claspfolio}}
\newcommand{\clasp}{\sysfont{clasp}}
\newcommand{\clingcon}{\sysfont{clingcon}}
\newcommand{\clingo}{\sysfont{clingo}}
\newcommand{\cmodels}{\sysfont{cmodels}}
\newcommand{\coala}{\sysfont{coala}}
\newcommand{\dingo}{\sysfont{dingo}}
\newcommand{\dflat}{\sysfont{dflat}}
\newcommand{\dlvhex}{\sysfont{dlvhex}}
\newcommand{\dlv}{\sysfont{dlv}}
\newcommand{\ezcsp}{\sysfont{ezcsp}}
\newcommand{\ftolp}{\sysfont{f2lp}}
\newcommand{\gecode}{\sysfont{gecode}}
\newcommand{\gidl}{\sysfont{gidl}\xspace}
\newcommand{\gnt}{\sysfont{gnt}}
\newcommand{\gringo}{\sysfont{gringo}}
\newcommand{\iclingo}{\sysfont{iclingo}}
\newcommand{\idp}{\sysfont{idp}}
\newcommand{\inca}{\sysfont{inca}}
\newcommand{\jdlv}{\sysfont{jdlv}}
\newcommand{\lparse}{\sysfont{lparse}}
\newcommand{\lptodiff}{\sysfont{lp2diff}}
\newcommand{\lptosat}{\sysfont{lp2sat}}
\newcommand{\lctocasp}{\sysfont{lc2casp}}
\newcommand{\mchaff}{\sysfont{mchaff}}
\newcommand{\metasp}{\sysfont{metasp}}
\newcommand{\mingo}{\sysfont{mingo}}
\newcommand{\minisat}{\sysfont{minisat}}
\newcommand{\nomorepp}{\sysfont{nomore++}}
\newcommand{\oclingo}{\sysfont{oclingo}}
\newcommand{\piclasp}{\sysfont{piclasp}}
\newcommand{\picosat}{\sysfont{picosat}}
\newcommand{\plasp}{\sysfont{plasp}}
\newcommand{\quontroller}{\sysfont{quontroller}}
\newcommand{\rosoclingo}{\sysfont{rosoclingo}}
\newcommand{\sag}{\sysfont{sag}}
\newcommand{\satz}{\sysfont{satz}}
\newcommand{\siege}{\sysfont{siege}}
\newcommand{\smodelscc}{\sysfont{smodels$_{\!cc}$}}
\newcommand{\smodelsr}{\sysfont{smodels}$_r$}
\newcommand{\smodels}{\sysfont{smodels}}
\newcommand{\unclasp}{\sysfont{unclasp}}
\newcommand{\wasp}{\sysfont{wasp}}
\newcommand{\zchaff}{\sysfont{zchaff}}
\newcommand{\zzz}{\sysfont{z3}}

\newcommand{\theory}{\emph{Theory}}
\newcommand{\hybrid}{\sysfont{Hybrid}}

\newcommand{\aspif}{\sysfont{aspif}}

\newcommand{\python}{Python}
\newcommand{\lua}{Lua}
\newcommand{\cpp}{C++}
\newcommand{\C}{C}
\newcommand{\java}{Java}
\newcommand{\haskell}{Haskell}

\newacro{ILP}{Integer Linear Programming}
\newacro{SAT}{Boolean Satisfiability}
\newacro{ASP}{Answer Set Programming}
\newacro{DSE}{Design Space Exploration}
\newacro{ASPmT}{\ac{ASP} modulo Theories}
\newacro{MOEA}{multi-objective evolutionary algorithm}
\newacro{MOOP}{multi-objective optimization problem}
\newacro{QF--IDL}{qunatifier-free integer difference logic}

% - hacks ----------------------------------------------------------------------

\newcommand{\neghspace}{\!\!\!\!\!}

%%% Local Variables:
%%% mode: latex
%%% TeX-master: "paper"
%%% End:
 
% - paper-specifics ------------------------------------------------------------
\newcommand{\hpredicate}{\texttt{\_h}}
\newcommand{\hpred}[4]{\ensuremath{\hpredicate(#1,{#2},#3,#4)}}
\newcommand{\hpre}[3]{\ensuremath{\hpredicate(#1,{#2},#3)}}

% - syntax ---------------------------------------------------------------------
\newcommand{\naf}[1]{\ensuremath{{\mathtt{not}\,{#1}}}} % {\ensuremath{{\sim\!{#1}}}}

\newcommand{\head}[1]{\ensuremath{\mathit{head}(#1)}}
\newcommand{\body}[1]{\ensuremath{\mathit{body}(#1)}}

\newcommand{\atom}[1]{\ensuremath{\mathit{atom}(#1)}}

\newcommand{\poslits}[1]{\ensuremath{{#1}^+}}
\newcommand{\neglits}[1]{\ensuremath{{#1}^-}}

\newcommand{\pbody}[1]{\poslits{\body{#1}}}
\newcommand{\nbody}[1]{\neglits{\body{#1}}}

\newcommand{\PRG}{\ensuremath{P}}

\newcommand{\atbody}[2]{\ensuremath{\mathit{body}_{#1}(#2)}}

\newcommand{\ground}[1]{\ensuremath{\mathit{grd}(#1)}}

% - semantics ------------------------------------------------------------------
\newcommand{\SM}[1]{\ensuremath{\mathit{SM}(#1)}}

% - operators ------------------------------------------------------------------
\newcommand{\Cn}[1]{\ensuremath{\mathit{Cn}(#1)}}
\newcommand{\reduct}[2]{\ensuremath{#1^{#2}}}

\newcommand{\To}[1]{\ensuremath{T_{#1}}}
\newcommand{\T}[2]{\To{#1}#2}
\newcommand{\TiO}[2]{\To{#2}^{#1}}
\newcommand{\Ti}[3]{\TiO{#1}{#2}#3}

\newcommand{\BF}[1]{\ensuremath{\mathit{BF}(#1)}}
\newcommand{\CF}[1]{\ensuremath{\mathit{CF}(#1)}}
\newcommand{\CFIF}[1]{\ensuremath{\overleftarrow{\mathit{CF}}(#1)}}
\newcommand{\CFFI}[1]{\ensuremath{\overrightarrow{\mathit{CF}}(#1)}}
\newcommand{\CFX}[1]{\ensuremath{\mathit{CF}^x(#1)}}

\newcommand{\loops}[1]{\ensuremath{\mathit{loop}(#1)}}\message{*** loop already defined ***}
\newcommand{\ES}[2]{\ensuremath{\mathit{ES}_{\!#2}(#1)}}
\newcommand{\EB}[2]{\ensuremath{\mathit{EB}_{\!#2}(#1)}}
\newcommand{\LFM}[2]{\ensuremath{\mathit{LF}_{\!#2}(#1)}}
\newcommand{\LF}[1]{\ensuremath{\mathit{LF}(#1)}}

% - tableaux -------------------------------------------------------------------
\newcommand{\true}{\ensuremath{\boldsymbol{T}}}
\newcommand{\false}{\ensuremath{\boldsymbol{F}}}

\newcommand{\Tsigned}[1]{\ensuremath{\true{#1}}}\message{ *** RENAME *** }
\newcommand{\Fsigned}[1]{\ensuremath{\false{#1}}}

\newcommand{\TOP}[1]{\ensuremath{D_{#1}}}
\newcommand{\COP}[1]{\ensuremath{D^*_{#1}}}

\newcommand{\Proviso}[1]{\raisebox{-7pt}[0pt][0pt]{\ensuremath{(#1)}}}

\newcommand{\plit}[1]{\ensuremath{\boldsymbol{t}{#1}}}
\newcommand{\nlit}[1]{\ensuremath{\boldsymbol{f}{#1}}}

% - nogoods, assigmemts, etc. --------------------------------------------------
\newcommand{\domain}[1]{\ensuremath{\mathit{dom}(#1)}}

\newcommand{\ass}{\ensuremath{A}}

\newcommand{\tlits}[1]{\ensuremath{{#1}^{\true}}}
\newcommand{\flits}[1]{\ensuremath{{#1}^{\false}}}
\newcommand{\prefix}[2]{\ensuremath{#1[#2]}}

% \newcommand{\clno}[1]{\ensuremath{\delta(#1)}}
% \newcommand{\ClNo}[1]{\ensuremath{\Delta(#1)}}
% \newcommand{\nocl}[1]{\ensuremath{\gamma(#1)}}
% \newcommand{\NoCl}[1]{\ensuremath{\Gamma(#1)}}

\newcommand{\CN}[1]{\ensuremath{\Delta_{#1}}}
\newcommand{\LN}[1]{\ensuremath{\Lambda_{#1}}}

\newcommand{\dl}[0]{\ensuremath{\mathit{dl}}}
\newcommand{\dlevel}[1]{\ensuremath{\mathit{dlevel}(#1)}}
\newcommand{\opp}[1]{\ensuremath{\overline{#1}}}

\newcommand{\undef}[0]{\ensuremath{\circ}}
\newcommand{\trdef}[0]{\ensuremath{\times}}
\newcommand{\scc}[1]{\ensuremath{\mathit{scc}(#1)}}
\newcommand{\source}[1]{\ensuremath{\mathit{source}(#1)}}

% - incremental solving ----------------------------------------
\newcommand{\grounder}{\textsc{Ground}}
\newcommand{\addProgram}{\textsc{Add}}
\newcommand{\solver}{\textsc{Solve}}
\newcommand{\isolve}{\textsc{iSolve}}

% % - algorithm2e ----------------------------------------------------------------
% \DontPrintSemicolon

% \SetKw{ForSome}{for some}
% \SetKw{SuchThat}{such that}

% \SetKwInOut{Input}{Input}
% \SetKwInOut{Output}{Output}
% \SetKwInOut{Global}{Global}
% \SetKwInOut{Internal}{Internal}

% \SetKwFor{Let}{let}{in}{tel}
% \SetKwFor{Loop}{loop}{}{}

% \SetFuncSty{sc}
% \SetKwFunction{Select}{Select}
% \SetKwFunction{UnFoundedSet}{Unfounded\-Set}
% \SetKwFunction{Propagation}{Nogood\-Propagation}
% \SetKwFunction{ConflictAnalysis}{Conflict\-Analysis}

% \SetKwData{Grounder}{Grounder}
% \SetKwData{Solver}{Solver}

% \SetCommentSty{it}
% \SetKwComment{AlgoComm}{// }{}

% - systems ----------------------------------------------------------------------
% >>> NOT USED in BOOK <<<
\newcommand{\sysfont}{\textit}
\newcommand{\smodels}{\sysfont{smodels}}
\newcommand{\smodelsr}{\sysfont{smodels}$_r$}
\newcommand{\smodelscc}{\sysfont{smodels$_{\!cc}$}}
\newcommand{\dlv}{\sysfont{dlv}}
\newcommand{\nomorepp}{\sysfont{nomore++}}
\newcommand{\assat}{\sysfont{assat}}
\newcommand{\cmodels}{\sysfont{cmodels}}
\newcommand{\sag}{\sysfont{sag}}
\newcommand{\clasp}{\sysfont{clasp}}
\newcommand{\claspD}{\sysfont{claspD}}
\newcommand{\claspfolio}{\sysfont{claspfolio}}
\newcommand{\claspar}{\sysfont{claspar}}
\newcommand{\gringo}{\sysfont{gringo}}
\newcommand{\clingo}{\sysfont{clingo}}
\newcommand{\iclingo}{\sysfont{iclingo}}
\newcommand{\clingcon}{\sysfont{clingcon}}
\newcommand{\gecode}{\sysfont{gecode}}
\newcommand{\lparse}{\sysfont{lparse}}
\newcommand{\mchaff}{\sysfont{mchaff}}
\newcommand{\zchaff}{\sysfont{zchaff}}
\newcommand{\siege}{\sysfont{siege}}
\newcommand{\minisat}{\sysfont{minisat}}
\newcommand{\berkmin}{\sysfont{berkmin}}
\newcommand{\picosat}{\sysfont{picosat}}
\newcommand{\lptosat}{\sysfont{lp2sat}}
\newcommand{\lptodiff}{\sysfont{lp2diff}}
\newcommand{\lua}{\textit{lua}}

% % - latex ----------------------------------------------------------------------
% \newcounter{excounter}
% \newcommand{\labex}[1]{\refstepcounter{excounter}\label{#1}} % \index{\ensuremath{\PRG_{\ref{#1}}}}

%%% Local Variables: 
%%% mode: latex
%%% TeX-master: "paper"
%%% End: 


\pdfinfo{
/Title (Domain-specific Heuristics in Answer Set Programming)
/Author (M. Gebser, B. Kaufmann, R. Otero, J. Romero, T. Schaub, P. Wanko)
/Keywords(answer set programming, answer set solving, heuristics, conflict-driven clause learning) 
} 

\sloppy

\title{Domain-specific Heuristics in Answer Set Programming}

\author%
{%
  M.~Gebser$^*$
  \and
  B.~Kaufmann$^*$
  \and
  R.~Otero$^\star$
  \and 
  J.~Romero$^{*,\star}$
  \and 
  T.~Schaub$^*$%\thanks{Affiliated with Simon Fraser University, Canada, and Griffith University, Australia.}
  \and
  P.~Wanko$^*$
  \\
  \begin{tabular}{ccc}
    Institute for Informatics$^*$       & & Department of Computer Science$^\star$\\
    University of Potsdam               & & University of Corunna\\
    14482 Potsdam, Germany	        & & 15071 Corunna, Spain
  \end{tabular}
}

\date{}
\setlength\textfloatsep{ 19pt plus 2pt minus 4pt}
\begin{document}

\maketitle
%% \begin{abstract}

Answer Set Programming (ASP) is an approach to declarative problem
solving, combining a rich yet simple modeling language with high
performance solving capacities.
%
We here develop an ASP-based approach to
\textit{Curriculum-Based Course Timetabling} (CB-CTT),
one of the most widely studied course timetabling problems.
The resulting {\asap} system reads a CB-CTT instance 
of a standard input format and converts it into a set of ASP facts.
In turn, these facts are combined with a first-order encoding for CB-CTT solving, 
which can subsequently be solved by any off-the-shelf ASP systems.
%
We establish the competitiveness of our approach by empirically
contrasting it to the best known bounds obtained so far via
dedicated implementations. 
%
Furthermore, we extend the {\asap} system to multi-objective course timetabling
and consider \textit{minimal perturbation problems}.
%
\keywords{Educational Timetabling \and Course Timetabling \and Answer
  Set Programming \and 
  Multi-objective Optimization \and 
  Minimal Perturbation Problems}
% \PACS{PACS code1 \and PACS code2 \and more}
% \subclass{MSC code1 \and MSC code2 \and more}
\end{abstract}

%%% Local Variables:
%%% mode: latex
%%% TeX-master: "paper"
%%% End:

\begin{abstract}
We introduce a general declarative framework for incorporating domain-specific heuristics into
ASP solving.
We accomplish this by extending the first-order modeling language of ASP by a distinguished
heuristic predicate.
The resulting heuristic information is processed as an equitable part of the logic program
and subsequently exploited by the solver when it comes to non-deterministically assigning a 
truth value to an atom.
We implemented our approach as a dedicated heuristic in the ASP solver \textit{clasp} and 
show its great prospect by an empirical evaluation.
\end{abstract}

%%% Local Variables: 
%%% mode: latex
%%% TeX-master: "paper"
%%% End: 


%% \section{Introduction}\label{intro}

\textit{Educational timetabling}~\citep{%
DBLP:journals/eor/BurkeP02,%
Lewis2007:survey,%
DBLP:journals/air/Schaerf99}
is generally defined as the
task of assigning a number of events, such as lectures and examinations,
to a limited set of timeslots (and perhaps rooms), 
subject to a given set of hard and soft constraints.
Hard constraints must be strictly satisfied.
Soft constraints must not necessarily be satisfied but
the overall number of violations should be minimal.
%
The educational timetabling problems can be 
classified into three categories:
\textit{school timetabling},
\textit{examination timetabling}, and
\textit{course timetabling}.
In this paper, 
we focus on \textit{curriculum-based course timetabling} 
(CB-CTT;~\citep{Bettinelli2015}), 
one of the most studied course timetabling problems.
%as well as \textit{post-enrollment course timetabling}.

The CB-CTT problems have been used in the third track of 
the second international timetabling competition 
(ITC-2007;~\citep{%
  GasperoMS/ITC2007,%
  DBLP:journals/informs/McCollumSPMLPGQB10}).
%
A web portal\footnote{\texttt{http://tabu.diegm.uniud.it/ctt/}}
for CB-CTT has been actively maintained by the ITC-2007 organizers%
~\citep{DBLP:journals/anor/BonuttiCGS12}.
The web site provides necessary infrastructures for benchmarking
such as validators, data formats, problem instances,
solutions in different formulations (uploaded by researchers), 
and visualizers.
All problem instances on the web are based on
real data from various universities in Europe.
The best known bounds on the web have been obtained by 
state-of-the-art CB-CTT solving techniques
including the winner algorithm of ITC-2007:
metaheuristics-based algorithms~\citep{%
  DBLP:journals/heuristics/AbdullahTMM12,%
%  ClarkHL/PATAT2008,%
  DBLP:conf/patat/GasperoS02,%
  DBLP:journals/jmma/GasperoS06,%
  DBLP:journals/anor/Geiger12,%
  DBLP:journals/eor/LuH10},
Integer Programming~\citep{DBLP:journals/anor/LachL12},
hybrid methods~\citep{DBLP:journals/anor/Muller09},
SAT/MaxSAT~\citep{DBLP:journals/aicom/AchaN12},
and many others.

However, each method has strength and weakness.
Metaheuristics-based dedicated implementations can quickly find better
upper bounds, but cannot guarantee their optimality.
Although complete methods such as SAT can guarantee the optimality, 
it is costly to implement a dedicated encoder from the CB-CTT problems in SAT\@.
Integer Programming has been widely used for CB-CTT solving,
but in general it does not scale to large instances in complex formulations.
It is therefore particularly challenging to develop a universal
timetabling solver which can efficiently find optimal solutions 
as well as better bounds for a wide range of CB-CTT instances in
different formulations at present.

Answer Set Programming (ASP;~\citep{%
  baral03:cambridge,%
  DBLP:conf/iclp/GelfondL88,%
  DBLP:journals/amai/Niemela99})
is an approach to declarative problem solving.
Recent advances in ASP open up a successful direction to extend logic
programming to be both more expressive as well as more effective.
%
ASP provides a rich language and 
is well suited for modeling combinatorial (optimization) problems 
in Artificial Intelligence and Computer Science.
Recent remarkable improvements in the effectiveness of ASP systems
have encouraged researchers to use ASP for solving problems in diverse areas,
such as
automated planning,
constraint satisfaction,
model checking,
music composition,
robotics,
system biology,
etc~\citep{ergele16a}.
However, so far, little attention has been paid to using ASP for timetabling.

In this paper, we describe an ASP-based approach for solving the
CB-CTT problems and present the resulting {\asap} system.
The {\asap} system reads a CB-CTT instance 
of a standard input format~\citep{DBLP:journals/anor/BonuttiCGS12}
and converts it into ASP facts.
In turn, these facts are combined with a first-order encoding for
CB-CTT solving, which is subsequently solved by an off-the-shelf ASP
system, in our case {\clingo}.
% ~\footnote{%
% ASP system {\clingo} is a monolithic combination of the grounder {\gringo} with the solver {\clasp}.}.
Figure~\ref{fig:arch} shows the \asap\ architecture.
% ----------------------------------------------------------------------
\begin{figure}[t]
  \centering
%  \thicklines
  \setlength{\unitlength}{1.28pt}
  \small
  % \frame <- uncomment this to see the bounding box ;)
  {\begin{picture}(210,65)(4,-15)
    \put( 14, 20){\dashbox(26,24){\shortstack{CB-CTT\\Instance}}}
    \put( 50, 20){\framebox(40,24){\shortstack{Format\\Converter}}}
    \put(100, 20){\dashbox(20,24){\shortstack{ASP\\Facts}}}
    \put( 80,-10){\dashbox(40,24){\shortstack{ASP\\Encoding}}}
    \put(130, 20){\framebox(40,24){\clingo}}
    \put(180, 20){\dashbox(26,24){\shortstack{CB-CTT\\Solution}}}
    \put( 40, 32){\vector(1,0){10}}
    \put( 90, 32){\vector(1,0){10}}
    \put(120, 32){\vector(1,0){10}}
    \put(170, 32){\vector(1,0){10}}
    \put(120, +2){\line(1,0){4}}
    \put(124, +2){\line(0,1){30}}
  \end{picture}}
\caption{Architecture of \asap.}
\label{fig:arch}
\end{figure}
%%% Local Variables: 
%%% mode: latex
%%% TeX-master: "paper"
%%% End: 

% ----------------------------------------------------------------------

The high-level approach of ASP has obvious advantages.
First, the problems are solved by general-purpose ASP systems rather
than dedicated implementation.
Second,
the elaboration tolerance of ASP allows for easy maintenance and
modifications of encodings.
And finally,
it is easy to experiment with advanced techniques in ASP solving such
as core-guided optimization, domain heuristics, and 
portfolios of prefabricated expert configurations~\citep{gekakarosc15a}.
%
However, the question is whether the high-level approach of \asap\ matches the
performance of dedicated systems.
We empirically address this question by contrasting the performance of 
\asap\ with the best known bounds on the CB-CTT web portal
obtained by state-of-the-art CB-CTT solving techniques.

From the perspective of applying ASP to educational timetabling, 
an early work studied school timetabling with ASP~\citep{Faber98}.
Recently, we showed in previous work~\citep{basotainsc13a} that ASP's
modeling language is well-suited for course timetabling by providing
a compact encoding for CB-CTT solving.
However, at the same time, 
we observed that a simple branch-and-bound optimization strategy
is insufficient to decrease the upper bounds of large instances in complex formulations.
In this paper, we provide insights into how more advanced solving techniques can be used 
to overcome this practical issue.

The main contributions of this paper are as follows.
\begin{enumerate}
\item We present a basic ASP encoding for solving CB-CTT problems, 
  which is an enhancement of our previous encoding~\citep{basotainsc13a}. 
  This enhancement provides the ability to use advanced ASP solving
  techniques such as 
  core-guided optimization, 
  domain heuristics, 
  portfolios of prefabricated expert configurations,
  multi-criteria optimization based on lexicographic ordering, and
  multi-shot ASP solving~\citep{gekakarosc15a,gekaobsc15a}.
\item We extend the basic encoding in view of enhancing the
  scalability and flexibility of solving (multi-criteria) CB-CTT problems.
  The extended \asap\ encodings have the following features:
  \begin{itemize}
  \item A collection of optimized encodings for soft constraints
  \item Easy composition of different formulations
  \item Multi-criteria optimization based on lexicographic ordering
%  \item Reusing legacy timetables in multi-shot ASP solving
  \end{itemize}
\item Our empirical analysis considers all 61 instances in 5
  different formulations, which are publicly available from the CB-CTT
  portal ($61\times 5 = 305$ combinations in a total)~\footnote{As of July 20, 2017}.
  Overall, \asap\ managed to either improve or reproduce the best
  known bounds for 182 combinations (59.7\% in the total).
  In detail, \asap\ provided 54 better bounds, 
  16 new optima,
  and 128 same bounds, 
  35 of which were proven optimal for the first time.
  Furthermore,
  \asap\ was able to produce upper bounds for very large instances in
  the category \verb+erlangen+ with every formulation, and 24 of them
  were unsolvable before.
\item We also extend the {\asap} system to finding Pareto optimal solutions
  of multi-objective course timetabling and consider
  \textit{minimal perturbation problems}~\citep{%
    BartakMR03,MullerRB04,RudovaMM11,Phillips2016}
  by utilizing multi-shot ASP solving techniques~\citep{gekaobsc15a}.
\end{enumerate}
All in all, the proposed declarative approach represents a significant
contribution to the state-of-the-art for CB-CTT.

The rest of the paper is structured as follows.
Section~\ref{sec:cb-ctt} provides the problem description of CB-CTT\@.
Although we give a brief introduction to ASP and its basic language
constructs in Section~\ref{sec:asp}, 
we refer the reader to the literature~\citep{baral03:cambridge,gekakasc12a}
for a comprehensive treatment of ASP\@.
Section~\ref{sec:approach}
describes {\asap}'s fact format of CB-CTT instances and then 
presents a basic {\asap} encoding for solving CB-CTT problems.
Section~\ref{sec:ext} presents a variety of features of
extended {\asap} encodings for (multi-criteria) CB-CTT solving.
Section~\ref{system} provides a detailed empirical analysis of {\asap}
features and performance in contrast to 
the best known bounds obtained by state-of-the-art CB-CTT solving techniques.
Section~\ref{mpp} presents an extension of the {\asap} system to 
minimal perturbation problems in course timetabling. 
Finally, a conclusion is given in Section~\ref{conclusion}.

%%% Local Variables:
%%% mode: latex
%%% TeX-master: "paper"
%%% End:


\section{Introduction}\label{sec:introduction}

The success of modern Boolean constraint technology was greatly boosted by Satisfiability Testing
(SAT;~\cite{SATHandbook}).
Meanwhile, this technology has been taken up in many related areas, 
like
Answer Set Programming (ASP; \cite{baral02a}).
This is because it provides highly performant yet general-purpose solving techniques for addressing demanding combinatorial search problems.
%
Sometimes, it is however advantageous to take a more application-oriented approach
by including domain-specific information.
On the one hand, domain-specific knowledge can be added for improving deterministic assignments
through propagation.
And on the other hand, domain-specific heuristics can be used for making better non-deterministic assignments.

In what follows,
we introduce a general declarative framework for incorporating domain-specific heuristics into ASP solving.
The choice of ASP is motivated by its first-order modeling language offering an easy way to express and process
heuristic information.
To this end,
we use a dedicated predicate \hpredicate\
whose arguments allow us to express various modifications to the solver's heuristic treatment of atoms.
The respective heuristic rules are seamlessly processed as an equitable part of the logic program
and subsequently exploited by the solver when it comes to choosing an atom for a non-deterministic 
truth assignment.
%
For instance, the rule
\begin{lstlisting}
_h(occ(A,T),factor,T) :- action(A), time(T).
\end{lstlisting}
favors later action occurrences over earlier ones (via multiplication by \texttt{T}).
That is, when making a choice between two unassigned atoms \texttt{occ(a,2)} and \texttt{occ(b,3)},
the solver's heuristic value of \texttt{occ(a,2)} is doubled while that of \texttt{occ(b,3)} is tripled.
This results in a bias on the system heuristic that may or may not take effect.
%
Besides \texttt{factor}, our heuristic language extension offers the primitive heuristic modifiers
\texttt{init}, \texttt{level}, and \texttt{sign}, from which even further modifiers can be defined.
Our approach provides an easy and flexible access to the solver's heuristic,
aiming at its modification rather than its replacement.
Note that the effect of the modifications is generally dynamic,
unless the truth of a heuristic atom is determined during grounding (as with the rule above).
As a result, our approach offers a declarative framework for expressing domain-specific heuristics.
As such, it appears to be the first of its kind.

%%% Local Variables: 
%%% mode: latex
%%% TeX-master: "paper"
%%% End: 

%% 
\section{Answer Set Programming with Linear Constraints}\label{sec:background}

For encoding our hybrid problem,
we rely upon the theory reasoning capacities of the ASP system \clingo\ that allows us to extend ASP with linear constraints over reals
(as addressed in Linear Programming).
We confine ourselves below to features relevant to our application and refer the interested reader for details to~\citep{gekakaosscwa16a}.

As usual, a \emph{logic program} consists of \emph{rules} of the form
\begin{lstlisting}[mathescape=true,numbers=none]
   a$_0$ :- a$_1$,...,a$_m$,not a$_{m+1}$,...,not a$_n$
\end{lstlisting}
where each \lstinline[mathescape=true]{a$_i$} is either
a \emph{(regular) atom} of form \lstinline[mathescape=true]{p(t$_1$,...,t$_k$)}
where all \lstinline[mathescape=true]{t$_i$} are terms
or
a \emph{linear constraint atom} of form%
\footnote{In \clingo, theory atoms are preceded by `\texttt{\&}'.}
`\lstinline[mathescape=true]@&sum{w$_1$*x$_1$;$\dots$;w$_l$*x$_l$} <= k@'
that stands for the linear constraint
\(
w_1\cdot x_1+\dots+w_l\cdot x_l\leq k
\).
All \lstinline[mathescape=true]{w$_i$} and \lstinline[mathescape=true]{k} are finite sequences of digits with at most one dot%
\footnote{In the input language of \clingo, such sequences must be quoted to avoid clashes.}
and represent real-valued coefficients $w_i$ and $k$.
Similarly all \lstinline[mathescape=true]{x$_i$} stand for the real-valued variables $x_i$.
%
As usual, \lstinline[mathescape=true]{not} denotes (default) \emph{negation}.
A rule is called a \emph{fact} if $n=0$.

Semantically, a logic program induces a set of \emph{stable models},
being distinguished models of the program determined by stable models semantics~\citep{gellif91a}.
%
Such a stable model $X$ is an \emph{LC-stable model} of a logic program $P$,%
\footnote{This corresponds to the definition of $T$-stable models using a \emph{strict} interpretation of theory atoms~\citep{gekakaosscwa16a},
  and letting $T$ be the theory of linear constraints over reals.}
if there is an assignment of reals to all real-valued variables occurring in $P$ that
(i)     satisfies all linear constraints associated with linear constraint atoms in $P$ being     in $X$
and
(ii) falsifies all linear constraints associated with linear constraint atoms in $P$ being not in $X$.
%
For instance, the (non-ground) logic program containing the fact
`\lstinline[mathescape=true]{a("1.5").}'
along with the rule
`\lstinline[mathescape=true]@&sum{R*x} <= 7 :- a(R).@'
has the stable model
\par
\lstinline[mathescape=true]@$\{$a("1.5")$,\;$&sum{"1.5"*x}<=7$\}$@.
\\
This model is LC-stable since there is an assignment,
e.g.\ $\{x\mapsto 4.2\}$,
that satisfies the associated linear constraint `$1.5*x\leq 7$'.
We regard the stable model along with a satisfying real-valued assignment as a solution to a logic program containing linear constraint atoms.
\review{For a more detailed introduction of ASP extended with linear constraints, illustrated with more complex examples, we refer the interested reader to~\citep{jakaosscscwa17a}.}

To ease the use of ASP in practice,
several extensions have been developed.
First of all, rules with variables are viewed as shorthands for the set of their ground instances.
Further language constructs include
\emph{conditional literals} and \emph{cardinality constraints} \citep{siniso02a}.
The former are of the form
\lstinline[mathescape=true]{a:b$_1$,...,b$_m$},
the latter can be written as
\lstinline[mathescape=true]+s{d$_1$;...;d$_n$}t+,
where \lstinline{a} and \lstinline[mathescape=true]{b$_i$} are possibly default-negated (regular) literals  % for $0\leq i\leq m$,
and each \lstinline[mathescape=true]{d$_j$} is a conditional literal; % for $1\leq i\leq n$;
\lstinline{s} and \lstinline{t} provide optional lower and upper bounds on the number of satisfied literals in the cardinality constraint.
We refer to \lstinline[mathescape=true]{b$_1$,...,b$_m$} as a \emph{condition}.
%
The practical value of both constructs becomes apparent when used with variables.
For instance, a conditional literal like
\lstinline[mathescape=true]{a(X):b(X)}
in a rule's antecedent expands to the conjunction of all instances of \lstinline{a(X)} for which the corresponding instance of \lstinline{b(X)} holds.
%
Similarly,
\lstinline[mathescape=true]+2{a(X):b(X)}4+
is true whenever at least two and at most four instances of \lstinline{a(X)} (subject to \lstinline{b(X)}) are true.
%
Finally, objective functions minimizing the sum of weights $w_i$ subject to condition $c_i$ are expressed as
\lstinline[mathescape=true]!#minimize{$w_1$:$c_1$;$\dots$;$w_n$:$c_n$}!.
% \lstinline[mathescape=true]!#minimize{$w_1$@$l_1$:$c_1$;$\dots$;$w_n$@$l_n$:$c_n$}!.
% Lexicographically ordered objective functions are (optionally) distinguished via levels indicated by $l_i$.

In the same way,
the syntax of linear constraints offers several convenience features.
As above,
elements in linear constraint atoms can be conditioned,
viz.\par
`\lstinline[mathescape=true]@&sum{w$_1$*x$_1$:c$_1$;...;w$_l$*x$_l$:c$_n$} <= k@'
\\
where each \lstinline[mathescape=true]{c$_i$} is a condition.
% Also, linear constraints can be formed with further relations, viz.\
% \texttt{>=},
% \texttt{<},
% \texttt{>},
% \texttt{=},
% and
% \texttt{!=}.
Moreover, the theory language for linear constraints offers a domain declaration for real variables,
`\lstinline[mathescape=true]@&dom{lb..ub} = x@'
expressing that all values of \texttt{x} must lie between \texttt{lb} and \texttt{ub}.
And finally the maximization (or minimization) of an objective function can be expressed with
\lstinline[mathescape=true]@&maximize{w$_1$*x$_1$:c$_1$;...;w$_l$*x$_l$:c$_n$}@
(by \texttt{minimize}).
The full theory grammar for linear constraints over reals is available at~\url{https://potassco.org}.

%%% Local Variables:
%%% mode: latex
%%% TeX-master: "paper"
%%% End:


\section{Background}\label{sec:background}

We assume some basic familiarity with ASP, its semantics as well as its basic language constructs,
like normal rules, cardinality constraints, and optimization statements.
Although our examples are self-explanatory, we refer the reader for details to~\cite{gekakasc12a}.
%
For illustrating our approach, 
we consider selected rules of a simple planning encoding, following~\cite{lifschitz02a}.
We use predicates \texttt{action} and \texttt{fluent} to distinguish the corresponding entities.
The length of the plan is given by the constant \texttt{l}, which is used to fix all time points via
the statement \lstinline{time(1..l).}
Moreover, suppose our ASP encoding contains the rule
\begin{lstlisting}
1 { occ(A,T) : action(A) } 1 :- time(T).
\end{lstlisting}
stating that exactly one action occurs at each time step.
Also, it includes a frame axiom of the following form.%
\footnote{We use `\texttt{-}/1' to stand for classical negation.}
\begin{lstlisting}
holds(F,T) :- holds(F,T-1), not -holds(F,T).
\end{lstlisting}
In such a setting, actions and fluents are prime subjects to planning-specific heuristics.
As we show below, these can be elegantly expressed by heuristic statements about atoms formed from
predicates \texttt{occ} and \texttt{holds}, respectively.

For computing the stable models of a logic program, we use
a Boolean assignment, that is, a (partial) function mapping propositional variables in $\mathcal{A}$
to truth values \true\ and \false.
We represent such an assignment \ass\ as a set of signed literals of form $\Tsigned{a}$ or $\Fsigned{a}$,
standing for $a\mapsto\true$ and $a\mapsto\false$, respectively.
%
We access the true and false variables in \ass\ via
\(
\tlits{\ass}
=
\{a\in\mathcal{A}\mid\Tsigned{a}\in\ass\}
\)
and
\(
\flits{\ass}
=
\{a\in\mathcal{A}\mid\Fsigned{a}\in \ass\}
\), respectively.
%
\ass\ is conflicting, if $\tlits{\ass}\cap\flits{\ass}\neq\emptyset$;
\ass\ is total, if it is non-conflicting and $\tlits{\ass}\cup\flits{\ass}=\mathcal{A}$.
%
For generality, we represent Boolean constraints by \emph{nogoods}~\cite{dechter03}.
A nogood is a set $\{\sigma_1,\dots,\sigma_m\}$ of signed literals,
expressing that any assignment containing $\sigma_1,\dots,\sigma_m$ is inadmissible.
Accordingly,
a total assignment \ass\ is a \emph{solution} for a set~$\Delta$ of nogoods
if $\delta\not\subseteq\ass$ for all $\delta\in\Delta$.
%
While clauses can be directly mapped into nogoods,
logic programs are subject to a more involved translation.
For instance, 
an atom $a$ defined by two rules $a\,\texttt{:-}\,b,\naf{c}$ and $a\,\texttt{:-}\,d$
gives rise to three nogoods:
$\{\Tsigned{a},\Fsigned{x_{\{b,\naf{c}\}}},\Fsigned{x_{\{d\}}}\}$,
$\{\Fsigned{a},\Tsigned{x_{\{b,\naf{c}\}}}\}$, and
$\{\Fsigned{a},\Tsigned{x_{\{d\}}}\}$,
where $x_{\{b,\naf{c}\}}$ and $x_{\{d\}}$ are auxiliary variables for the bodies of the two previous rules.
Similarly, the body ${\{b,\naf{c}\}}$ leads to nogoods
$\{\Fsigned{x_{\{b,\naf{c}\}}},\Tsigned{b},\Fsigned{c}\}$,
$\{\Tsigned{x_{\{b,\naf{c}\}}},\Fsigned{b}\}$, and
$\{\Tsigned{x_{\{b,\naf{c}\}}},\Tsigned{c}\}$.
See~\cite{gekakasc12a} for full details.
%
Note that translating logic programs into nogoods adds auxiliary variables.
For simplicity, we restrict our formal elaboration to atoms in $\mathcal{A}$
(also because our approach leaves such internal variables unaffected anyway).

% Finally, we mention that
% we use \textit{max} to give the maximum value in a set and 
% \textit{argmax} to give the set of elements having the maximum value.


%%% Local Variables: 
%%% mode: latex
%%% TeX-master: "paper"
%%% End: 

%% 
\section{Conflict-driven constraint learning}\label{sec:cdcl}

Given that we are primarily interested in the heuristic machinery of a solver,
we only provide a high-level description of the basic decision algorithm for 
conflict-driven constraint learning (CDCL;~\cite{marsak99a,zamamoma01a})
in Figure~\ref{algo:cdcl}.
%
% ----------------------------------------------------------------------
\begin{figure}[t]
\newcommand{\ITEMHACK}{\itemindent=-5pt\itemsep=0pt\parsep=\itemsep}
\small
\hrule\vspace{2pt}
\noindent\textbf{loop}\\[-12pt]
  \begin{itemize}\ITEMHACK
  \item [] \textit{propagate}  
    \hfill// compute deterministic consequences
  \item [] \textbf{if} no conflict \textbf{then}
    \begin{itemize}\ITEMHACK
    \item [] \textbf{if} all variables assigned 
      \textbf{then} 
      \textbf{return} variable assignment
    \item [] \textbf{else}
      \textit{decide} 
      \hfill// non-deterministically assign some literal
    \end{itemize}
  \item [] \textbf{else} 
    \begin{itemize}\ITEMHACK
    \item [] \textbf{if} top-level conflict %found 
      \textbf{then} 
      \textbf{return} unsatisfiable
    \item [] \textbf{else}
      \begin{itemize}\ITEMHACK
      \item [] \textit{analyze}\hfill// analyze conflict and add a conflict constraint
      \item [] \textit{backjump}\hfill// undo assignments until conflict constraint is unit
      \end{itemize}
    \end{itemize}
  \end{itemize}
  \hrule  
  \caption{Basic decision algorithm: CDCL}
  \label{algo:cdcl}
\end{figure}
% ----------------------------------------------------------------------
CDCL starts by extending a (partial) {assignment} by deterministic (unit) propagation.
% Importantly, every derived literal is ``implied'' by some {nogood}
% % (set of literals that must not jointly be assigned), 
% which would be violated if the literal's complement were assigned.
Although propagation aims at forgoing nogood violations,
assigning a literal implied by one nogood may lead to the violation of another nogood;
this situation is called \emph{conflict}.
If the conflict can be resolved, % (the violated nogood contains backtrackable literals),
it is analyzed to identify a conflict constraint.
The latter represents a ``hidden'' conflict reason that is recorded and
guides backjumping to an earlier stage such that
the complement of some formerly assigned literal is implied by the conflict constraint,
thus triggering propagation.
Only when propagation finishes without conflict,
a (heuristically chosen) literal can be assigned % at a new \emph{decision level},
provided that the assignment at hand is partial,
while a {solution} % (total assignment not violating any nogood)
has been found otherwise.
% The eventual termination of CDCL is guaranteed
% by either returning a solution or encountering an unresolvable conflict
% (independent of (non-implied) decision literals).
%
See~\cite{SATHandbook} for details.

A characteristic feature of CDCL is its look-back based approach.
Central to this are conflict-driven mechanisms scoring variables according to their prior conflict involvement.
These scores guide heuristic choices regarding literal selection as well as constraint learning and deletion.

A decision heuristic is used to implement the non-deterministic assignment done via
\emph{decide} in the CDCL algorithm in Figure~\ref{algo:cdcl}.
In fact,
the selection of an atom along with its sign relies on two such functions:
\[
h: \mathcal{A}\to[0,+\infty)
\quad\text{and}\quad
s: \mathcal{A}\to\{\true,\false\}
\ .
\]
Both functions vary over time.
To capture this, we use $h_i$ and $s_i$ to denote the specific mappings in the $i$th iteration of CDCL's main loop.
Analogously, we use $\ass_i$ to represent the $i$th assignment (after \textit{propagation}).
We use $i=0$ to refer to the initialization of both functions via $h_0$ and $s_0$;
similarly, $A_0$ gives the initial assignment (after \textit{propagation}).

The following lines give a more detailed yet still high-level account of the non-deterministic assignment done by
\emph{decide} in the CDCL algorithm for $i\geq 1$ (and a given $h_0$):%
\footnote{\label{fn:ass}For clarity, we keep using indexes in this algorithmic setting although this is
  unnecessary in view of assignment operator `$:=$'.}
% --------------------------------------------------
\begin{enumerate}\itemindent 10pt
\item $h_i(a) := \alpha_i\times h_{i-1}(a) + \beta_i(a)$ \hfill for each $a\in\mathcal{A}\qquad$
\item $U:=\mathcal{A}\setminus (\tlits{\ass_{i-1}}\cup\flits{\ass_{i-1}})$
\item $C:= \textit{argmax}_{a\in U}h_i(a)$
\item $a:= \tau(C)$
\item $\ass_i := \ass_{i-1}\cup\{s_i(a)a\}$
\end{enumerate}
% --------------------------------------------------
The first line describes the development of the heuristic depending on a global decay parameter
$\alpha_i$ and a variable-specific scoring function $\beta_i$.
The set $U$ contains all atoms unassigned at step~$i$.
Among them, the ones with the highest heuristic value are collected in $C$.
Whenever $C$ contains several (equally scored) variables,
the solver must break the tie by selecting one atom $\tau(C)$ from $C$.
% Such tie-breaking is usually done by adhoc mechanisms.

Look-back based heuristics rely on information gathered during conflict analysis in CDCL.
Starting from some initial heuristic values in $h_0$,
the heuristic function is continued as in Item~1 above,
where
$\alpha_i\in{[0,1]}$ is a global parameter decaying the influence of past values
and
$\beta_i(a)$ gives the conflict score attributed to variable $a$ within conflict analysis.
The value of $\beta_i(a)$ can be thought of being 0 unless $a$ was scored by \textit{analyze} in CDCL.
Similarly, $\alpha_i$ usually equals 1 unless it was lowered at some system-specific point, 
such as after a \emph{restart}.
%
Occurrence-based heuristics like~\textit{moms}~\cite{pretolani96a} furnish initial heuristics.
%
Prominent look-back heuristics are
\textit{berkmin}~\cite{golnov02a}
and
\textit{vsids}~\cite{momazhzhma01a}.
% and \textit{vmtf}~\cite{ryan04a}.

For illustration,
let us look at a rough trace of atoms $a$, $b$, and $c$ in a fictive run of the CDCL algorithm.
\[
\begin{array}{|r|c|c@{\ }@{\ }c@{\,}@{\,}c@{\,}r@{\,}|c@{\,}@{\,}c@{\,}@{\,}c@{\,}r@{\,}|c|c|c@{\,}@{\,}c@{\,}@{\,}c@{\,}r@{\,}|}
  \hline
  i&\mathit{operation}&
  \multicolumn{4}{|c|}{\ass}&
  \multicolumn{4}{ c|}{h}&
                       s &
                       \alpha &
  \multicolumn{4}{ c|}{\beta}\\
  \hline
   &                  &a     &b    &c&...&a&b&c   &...&     &  &a&b&c&...\\
  \hline
  \hline
  0&                  &      &     & &   &0&1&1   &   &\true&1 &0&0&0&   \\
  \hline
  1&\mathit{propagate}&\false&     & &   &0&1&1   &   &\true&1 &0&0&0&   \\
   &\mathit{decide}   &\false&\true& &   &0&1&1   &   &\true&1 &0&0&0&   
\end{array}
\]
The initial heuristic $h_0$ prefers $b$, $c$ over $a$;
the sign heuristic $s$ constantly assigns \true.
Initial propagation assigns \false\ to $a$.
This leaves all heuristics unaffected.
When invoking \textit{decide}, we find $b$ and $c$ among the unassigned variables in $U$
(in Item~2 above).
Assuming the maximum value of $h_1$ to be 1, both are added to $C$.
This tie is broken by selecting $\tau(C)=b$ in $C$.
Given that the (constant) sign heuristic yields \true,
Item~5 adds signed literal \Tsigned{b} to the current assignment.

Next suppose we encounter a conflict involving $c$ at step~8.
This leads to an incrementation of $\beta_8(c)$.
\[
\begin{array}{|r|c|c@{\,}@{\,}c@{\,}@{\,}c@{\,}r@{\,}|c@{\,}@{\,}c@{\,}@{\,}c@{\,}r@{\,}|c|c|c@{\,}@{\,}c@{\,}@{\,}c@{\,}r@{\,}|}
   &                  &a     &b    &c     &...&a&b&c   &...&     &  &a&b&c&...\\
  \hline
  \hline
  8&\mathit{propagate}&\false&\true&\false&   &0&2&2   &   &\true&1 &0&0&0&   \\
   &\mathit{analyze}  &\false&\true&\false&   &0&2&2   &   &\true&1 &0&0&1&   \\
   &\mathit{backjump} &\false&     &      &   &0&2&3   &   &\true&1 &0&0&0&   \\
  \hline
  9&\mathit{propagate}&\false&     &      &   &0&2&3   &   &\true&1 &0&0&0&   \\
   &\mathit{decide}   &\false&     &\true &   &0&2&3   &   &\true&1 &0&0&0&
\end{array}
\]
As at step~1, $b$ and $c$ are unassigned after backjumping.
Unlike above, $c$ is now heuristically preferred to $b$ since it occurred more frequently within conflicts.

Without going into detail, 
we mention that at certain steps $i$,
parameter $\alpha_i$ is decreased for decaying the values of $h_i$ and
the conflict scores in $\beta_i$ are re-set (eg.~after \textit{analyze}).

Also, look-back based sign heuristics take advantage of previous information.
The common approach is to choose the polarity of a literal according to the higher
number of occurrences in recorded nogoods~\cite{momazhzhma01a}.
Another effective approach is \emph{progress saving}~\cite{pipdar07a},
caching truth values of (certain) retracted variables and reusing them for sign selection.

Although we focus on look-back heuristics,
we mention that look-ahead heuristics 
% are primarily used in DPLL-based solvers~\cite{davput60,dalolo62a}.
% They 
aim at shrinking the search space by selecting the (signed) variable offering most implications.
This approach relies on failed-literal detection~\cite{freeman95a} for counting the number of
propagations obtained by (temporarily) adding in turn the variable and its negation to the current
assignment.
This count can be used in Item~1 above for computing the values $\beta_i(a)$,
while all $\alpha_i$ are set to 0 (because no past information is taken into account).
% For instance, such an approach is used in \smodels~\cite{siniso02a} to select both the variable as
% well as its sign.

%%% Local Variables: 
%%% mode: latex
%%% TeX-master: "paper"
%%% End: 


\section{Conflict-driven constraint learning}\label{sec:cdcl}

Given that we are primarily interested in the heuristic machinery of a solver,
we only provide a high-level description of the basic decision algorithm for 
conflict-driven constraint learning (CDCL;~\cite{marsak99a,zamamoma01a})
in Figure~\ref{algo:cdcl}.
%
% ----------------------------------------------------------------------
\begin{figure}[t]
\newcommand{\ITEMHACK}{\itemindent=-5pt\itemsep=0pt\parsep=\itemsep}
\small
\hrule\vspace{2pt}
\noindent\textbf{loop}\\[-12pt]
  \begin{itemize}\ITEMHACK
  \item [] \textit{propagate}  
    \hfill// compute deterministic consequences
  \item [] \textbf{if} no conflict \textbf{then}
    \begin{itemize}\ITEMHACK
    \item [] \textbf{if} all variables assigned 
      \textbf{then} 
      \textbf{return} variable assignment
    \item [] \textbf{else}
      \textit{decide} 
      \hfill// non-deterministically assign some literal
    \end{itemize}
  \item [] \textbf{else} 
    \begin{itemize}\ITEMHACK
    \item [] \textbf{if} top-level conflict %found 
      \textbf{then} 
      \textbf{return} unsatisfiable
    \item [] \textbf{else}
      \begin{itemize}\ITEMHACK
      \item [] \textit{analyze}\hfill// analyze conflict and add a conflict constraint
      \item [] \textit{backjump}\hfill// undo assignments until conflict constraint is unit
      \end{itemize}
    \end{itemize}
  \end{itemize}
  \hrule  
  \caption{Basic decision algorithm: CDCL}
  \label{algo:cdcl}
\end{figure}
% ----------------------------------------------------------------------
CDCL starts by extending a (partial) {assignment} by deterministic (unit) propagation.
% Importantly, every derived literal is ``implied'' by some {nogood}
% % (set of literals that must not jointly be assigned), 
% which would be violated if the literal's complement were assigned.
Although propagation aims at forgoing nogood violations,
assigning a literal implied by one nogood may lead to the violation of another nogood;
this situation is called \emph{conflict}.
If the conflict can be resolved, % (the violated nogood contains backtrackable literals),
it is analyzed to identify a conflict constraint.
The latter represents a ``hidden'' conflict reason that is recorded and
guides backjumping to an earlier stage such that
the complement of some formerly assigned literal is implied by the conflict constraint,
thus triggering propagation.
Only when propagation finishes without conflict,
a (heuristically chosen) literal can be assigned % at a new \emph{decision level},
provided that the assignment at hand is partial,
while a {solution} % (total assignment not violating any nogood)
has been found otherwise.
% The eventual termination of CDCL is guaranteed
% by either returning a solution or encountering an unresolvable conflict
% (independent of (non-implied) decision literals).
%
See~\cite{SATHandbook} for details.

A characteristic feature of CDCL is its look-back based approach.
Central to this are conflict-driven mechanisms scoring variables according to their prior conflict involvement.
These scores guide heuristic choices regarding literal selection as well as constraint learning and deletion.

A decision heuristic is used to implement the non-deterministic assignment done via
\emph{decide} in the CDCL algorithm in Figure~\ref{algo:cdcl}.
In fact,
the selection of an atom along with its sign relies on two such functions:
\[
h: \mathcal{A}\to[0,+\infty)
\quad\text{and}\quad
s: \mathcal{A}\to\{\true,\false\}
\ .
\]
Both functions vary over time.
To capture this, we use $h_i$ and $s_i$ to denote the specific mappings in the $i$th iteration of CDCL's main loop.
Analogously, we use $\ass_i$ to represent the $i$th assignment (after \textit{propagation}).
We use $i=0$ to refer to the initialization of both functions via $h_0$ and $s_0$;
similarly, $A_0$ gives the initial assignment (after \textit{propagation}).

The following lines give a more detailed yet still high-level account of the non-deterministic assignment done by
\emph{decide} in the CDCL algorithm for $i\geq 1$ (and a given $h_0$):%
\footnote{\label{fn:ass}For clarity, we keep using indexes in this algorithmic setting although this is
  unnecessary in view of assignment operator `$:=$'.}
% --------------------------------------------------
\begin{enumerate}\itemindent 10pt
\item $h_i(a) := \alpha_i\times h_{i-1}(a) + \beta_i(a)$ \hfill for each $a\in\mathcal{A}\qquad$
\item $U:=\mathcal{A}\setminus (\tlits{\ass_{i-1}}\cup\flits{\ass_{i-1}})$
\item $C:= \textit{argmax}_{a\in U}h_i(a)$
\item $a:= \tau(C)$
\item $\ass_i := \ass_{i-1}\cup\{s_i(a)a\}$
\end{enumerate}
% --------------------------------------------------
The first line describes the development of the heuristic depending on a global decay parameter
$\alpha_i$ and a variable-specific scoring function $\beta_i$.
The set $U$ contains all atoms unassigned at step~$i$.
Among them, the ones with the highest heuristic value are collected in $C$.
Whenever $C$ contains several (equally scored) variables,
the solver must break the tie by selecting one atom $\tau(C)$ from $C$.
% Such tie-breaking is usually done by adhoc mechanisms.

Look-back based heuristics rely on information gathered during conflict analysis in CDCL.
Starting from some initial heuristic values in $h_0$,
the heuristic function is continued as in Item~1 above,
where
$\alpha_i\in{[0,1]}$ is a global parameter decaying the influence of past values
and
$\beta_i(a)$ gives the conflict score attributed to variable $a$ within conflict analysis.
The value of $\beta_i(a)$ can be thought of being 0 unless $a$ was scored by \textit{analyze} in CDCL.
Similarly, $\alpha_i$ usually equals 1 unless it was lowered at some system-specific point, 
such as after a \emph{restart}.
%
Occurrence-based heuristics like~\textit{moms}~\cite{pretolani96a} furnish initial heuristics.
%
Prominent look-back heuristics are
\textit{berkmin}~\cite{golnov02a}
and
\textit{vsids}~\cite{momazhzhma01a}.
% and \textit{vmtf}~\cite{ryan04a}.

For illustration,
let us look at a rough trace of atoms $a$, $b$, and $c$ in a fictive run of the CDCL algorithm.
\[
\begin{array}{|r|c|c@{\ }@{\ }c@{\,}@{\,}c@{\,}r@{\,}|c@{\,}@{\,}c@{\,}@{\,}c@{\,}r@{\,}|c|c|c@{\,}@{\,}c@{\,}@{\,}c@{\,}r@{\,}|}
  \hline
  i&\mathit{operation}&
  \multicolumn{4}{|c|}{\ass}&
  \multicolumn{4}{ c|}{h}&
                       s &
                       \alpha &
  \multicolumn{4}{ c|}{\beta}\\
  \hline
   &                  &a     &b    &c&...&a&b&c   &...&     &  &a&b&c&...\\
  \hline
  \hline
  0&                  &      &     & &   &0&1&1   &   &\true&1 &0&0&0&   \\
  \hline
  1&\mathit{propagate}&\false&     & &   &0&1&1   &   &\true&1 &0&0&0&   \\
   &\mathit{decide}   &\false&\true& &   &0&1&1   &   &\true&1 &0&0&0&   
\end{array}
\]
The initial heuristic $h_0$ prefers $b$, $c$ over $a$;
the sign heuristic $s$ constantly assigns \true.
Initial propagation assigns \false\ to $a$.
This leaves all heuristics unaffected.
When invoking \textit{decide}, we find $b$ and $c$ among the unassigned variables in $U$
(in Item~2 above).
Assuming the maximum value of $h_1$ to be 1, both are added to $C$.
This tie is broken by selecting $\tau(C)=b$ in $C$.
Given that the (constant) sign heuristic yields \true,
Item~5 adds signed literal \Tsigned{b} to the current assignment.

Next suppose we encounter a conflict involving $c$ at step~8.
This leads to an incrementation of $\beta_8(c)$.
\[
\begin{array}{|r|c|c@{\,}@{\,}c@{\,}@{\,}c@{\,}r@{\,}|c@{\,}@{\,}c@{\,}@{\,}c@{\,}r@{\,}|c|c|c@{\,}@{\,}c@{\,}@{\,}c@{\,}r@{\,}|}
   &                  &a     &b    &c     &...&a&b&c   &...&     &  &a&b&c&...\\
  \hline
  \hline
  8&\mathit{propagate}&\false&\true&\false&   &0&2&2   &   &\true&1 &0&0&0&   \\
   &\mathit{analyze}  &\false&\true&\false&   &0&2&2   &   &\true&1 &0&0&1&   \\
   &\mathit{backjump} &\false&     &      &   &0&2&3   &   &\true&1 &0&0&0&   \\
  \hline
  9&\mathit{propagate}&\false&     &      &   &0&2&3   &   &\true&1 &0&0&0&   \\
   &\mathit{decide}   &\false&     &\true &   &0&2&3   &   &\true&1 &0&0&0&
\end{array}
\]
As at step~1, $b$ and $c$ are unassigned after backjumping.
Unlike above, $c$ is now heuristically preferred to $b$ since it occurred more frequently within conflicts.

Without going into detail, 
we mention that at certain steps $i$,
parameter $\alpha_i$ is decreased for decaying the values of $h_i$ and
the conflict scores in $\beta_i$ are re-set (eg.~after \textit{analyze}).

Also, look-back based sign heuristics take advantage of previous information.
The common approach is to choose the polarity of a literal according to the higher
number of occurrences in recorded nogoods~\cite{momazhzhma01a}.
Another effective approach is \emph{progress saving}~\cite{pipdar07a},
caching truth values of (certain) retracted variables and reusing them for sign selection.

Although we focus on look-back heuristics,
we mention that look-ahead heuristics 
% are primarily used in DPLL-based solvers~\cite{davput60,dalolo62a}.
% They 
aim at shrinking the search space by selecting the (signed) variable offering most implications.
This approach relies on failed-literal detection~\cite{freeman95a} for counting the number of
propagations obtained by (temporarily) adding in turn the variable and its negation to the current
assignment.
This count can be used in Item~1 above for computing the values $\beta_i(a)$,
while all $\alpha_i$ are set to 0 (because no past information is taken into account).
% For instance, such an approach is used in \smodels~\cite{siniso02a} to select both the variable as
% well as its sign.

%%% Local Variables: 
%%% mode: latex
%%% TeX-master: "paper"
%%% End: 

%% 
\section{Heuristic language elements}\label{sec:approach}

We express heuristic modifications via a set $\mathcal{H}$ of \emph{heuristic atoms} 
disjoint from $\mathcal{A}$.
Such a heuristic atom is formed from a dedicated predicate \hpredicate\ along with four arguments:
a (reified) atom $a\in\mathcal{A}$,
a heuristic modifier $m$,
and two  integers $v,p\in\mathbb{Z}$.
A heuristic modifier is used to manipulate the heuristic treatment of an atom $a$ via the 
modifier's value given by $v$.
The role of this value varies for each modifier.
We distinguish four primitive heuristic modifiers:
\begin{description}
\item [\texttt{init}] for initializing the heuristic value of $a$ with $v$,
\item [\texttt{factor}] for amplifying the heuristic value of $a$ by factor $v$,
\item [\texttt{level}] for ranking all atoms; the rank of $a$ is $v$,
\item [\texttt{sign}] for attributing the sign of $v$ as truth value to $a$.
\end{description}
While $v$ allows for changing an atom's heuristic behavior relative to \emph{other} atoms,
the second integer $p$ allows us to express a priority for disambiguating similar
heuristic modifications to the \emph{same} atom.
This is particularly important in our dynamic setting, where varying heuristic atoms may be obtained
in view of the current assignment.
For instance, the heuristic atoms
\hpred{b}{\mathtt{sign}}{1}{3}
and
\hpred{b}{\mathtt{sign}}{-1}{5}
aim at assigning opposite truth values to atom $b$.
This conflict can be resolved by preferring the heuristic modification with the higher priority,
viz.\ 5 in \hpred{b}{\mathtt{sign}}{-1}{5}.
Obviously such priorities can only support disambiguation but not resolve conflicting values sharing the same priority.

For accommodating priorities,
we define for an assignment \ass\ the \emph{preferred values} for modifier $m$ on atom $a$ as
\[
V_{a,m}(\ass)=
\textit{argmax}_{v\in\mathbb{Z}}\{p\mid\Tsigned{\hpred{a}{m}{v}{p}}\in\ass\}.
\]
Heuristic values are dynamic;
they are extracted from the current assignment and may thus vary during solving.
Note that $V_{a,m}(\ass)$ returns the singleton set $\{v\}$,
if the current assignment \ass\ contains a single true heuristic atom \hpred{a}{m}{v}{p} 
involving $a$ and $m$.
$V_{a,m}(\ass)$ is empty whenever there are no such heuristic atoms.
%
And whenever all heuristic atoms regarding $a$ and $m$ have the same priority $p$,
$V_{a,m}(\ass)$ is equivalent to
\(
\{v\mid\Tsigned{\hpred{a}{m}{v}{p}}\in\ass\}
\).

Here are a few examples.
We obtain % the preferred values
$V_{b,\mathtt{sign}}(\ass_1)=\{-1\}$
and
$V_{c,\mathtt{init}}(\ass_1)=\emptyset$
from assignment
\(
\ass_1=\{\Fsigned{a},\Tsigned{\hpred{b}{\mathtt{sign}}{1}{3}},\Tsigned{\hpred{b}{\mathtt{sign}}{-1}{5}}\}
\),
while assignment
\(
\ass_2=\{\Tsigned{\hpred{b}{\mathtt{sign}}{1}{3}},\Tsigned{\hpred{b}{\mathtt{sign}}{-1}{3}}\}
\)
results in $V_{b,\mathtt{sign}}(\ass_2)=\{1,-1\}$.

For ultimately resolving ambiguities among alternative values for heuristic modifiers, 
we propose for a set $V\subseteq\mathbb{Z}$ of integers the function $\nu(V)$ as
\[
%\nu(V)=
\mathit{max}\big(\{v\in V\!\mid v\geq 0\}\cup\{0\}\big)
+
\mathit{min}\big(\{v\in V\!\mid v\leq 0\}\cup\{0\}\big).
\]
Note that $\nu(\emptyset)=0$, attributing 0 the status of a neutral value.
Alternative options exist, like taking means or median of $V$ or even time specific criteria
relating to the emergence of values in the assignment.
%
In the above examples,
we get $\nu(V_{b,\mathtt{sign}}(\ass_1))=-1$ and $\nu(V_{b,\mathtt{sign}}(\ass_2))=0$.

Given this, we proceed by defining the \emph{domain-specific extension} $d$ to the heuristic function $h$ 
for $a\in\mathcal{A}$ as
\[
d_0(a)=\nu(V_{a,\mathtt{init}}(\ass_0))+h_0(a)
\]
and for $i\geq 1$
\[
d_i(a)=
\left\{
  \begin{array}{rl}
    \nu(V_{a,\mathtt{factor}}(\ass_i))\times h_i(a)&\text{if } V_{a,\mathtt{factor}}(\ass_i)\neq\emptyset
    \\
                                             h_i(a)&\text{otherwise}
  \end{array}
\right.
\]
First of all, it is important to note that $d$ is merely a modification and not a replacement of the
system heuristic $h$.
In fact, $d$ extends the range of $h$ to $(-\infty,+\infty)$.
Negative values serve as penalties.
The values of the \texttt{init} modifiers are added to $h_0$ in $d_0$.
The use of addition rather than multiplication allows us to override an initial value of 0.
Also, the higher the absolute value of the \texttt{init} modifier, the longer lasts its effect
(given the decay of heuristic values).
Unlike this, \texttt{factor} modifiers rely on multiplication because they aim at de- or increasing
conflict scores gathered during conflict analysis.
In view of $h$'s range,
a factor greater than 1 amplifies the score, a negative one penalizes the atom, and 0 resets the atom's score.
Enforcing a factor of 1 
% (for instance, through assigning a high priority)
transfers control back to the system heuristic $h$.

Heuristically modified logic programs are simply programs over $\mathcal{A}\cup\mathcal{H}$,
the original vocabulary extended by heuristic atoms (without restrictions).
As a first example,
let us extend our planning encoding by a rule favoring atoms expressing action occurrences close to
the goal situation.
\begin{lstlisting}
_h(occ(A,T),factor,T,0) :- action(A),time(T).
\end{lstlisting}
With \texttt{factor}, we impose a bias on the underlying heuristic function $h$.
Rather than comparing, for instance,
the plain values $h(\mathtt{occ(a,2)})$ and $h(\mathtt{occ(a,3)})$,
a decision is made by looking at $2\times h(\mathtt{occ(a,2)})$ and  $3\times h(\mathtt{occ(a,3)})$,
even though it still depends on $h$.
A further refined strategy may suggest considering climbing actions as early as possible.
\begin{lstlisting}
_h(occ(climb,T),factor,l-T,1) :- time(T).
\end{lstlisting}
Clearly, this rule conflicts with the more general rule above.
However, this conflict is resolved in favor of the more specific rule by attributing it a higher
priority (viz.~1 versus 0).

% Similar statements can be formulated with the \texttt{init} modifier in order to change the initial heuristic values.

For capturing a \emph{domain-specific extension} $t$ to the sign heuristic $s$,
we define for $a\in\mathcal{A}$ and $i\geq 0$:
\[
t_i(a)=
\left\{
  \begin{array}{rl}
    \true &\text{if }
           \nu(V_{a,\mathtt{sign}}(\ass_i))>0
           % \text{ and }
           % V_{a,\mathtt{sign}}(\ass_i)\neq\emptyset
           \\
    \false&\text{if }
           \nu(V_{a,\mathtt{sign}}(\ass_i))<0
           % \text{ and }
           % V_{a,\mathtt{sign}}(\ass_i)\neq\emptyset
           \\
    s_i(a)&\text{otherwise}
  \end{array}
\right.
\]
As with $d$ above, the extension $t$ to the sign heuristic is dynamic.
The sign of the modifier's preferred value determines the truth value to assign to an atom at hand.
No \texttt{sign} modifier (or enforcing a value of 0) leaves sign selection with the system's sign heuristic $s$.
%
For example, the heuristic rule
\begin{lstlisting}
_h(holds(F,T),sign,-1,0) :- fluent(F),time(T).
\end{lstlisting}
tells the solver to assign false to non-deterministically chosen fluents.
%
The next pair of rules is a further refinement of our strategy on climbing actions,
favoring their effective occurrence in the first half of the plan.
\begin{lstlisting}
_h(occ(climb,T),sign, 1,0) :- T<l/2,time(T).
_h(occ(climb,T),sign,-1,0) :- T>l/2,time(T).
\end{lstlisting}
Thus, while the atom $\mathtt{occ(climb,1)}$ is preferably made true,
false should rather be assigned to $\mathtt{occ(climb,l)}$.

Finally, for accommodating rankings induced by \texttt{level} modifiers,
we define for an assignment \ass\ and $\mathcal{A}'\subseteq\mathcal{A}$:
\[
\ell_\ass(\mathcal{A}')=\textit{argmax}_{a\in\mathcal{A}'}\nu(V_{a,\mathtt{level}}(\ass))
\]
The set $\ell_\ass(\mathcal{A}')$ gives all atoms in $\mathcal{A}'$ with the highest \texttt{level} values relative to
the current assignment \ass.
Similar to $d$ and $t$ above, this construction is also dynamic and the rank of atoms may vary during solving.
The function $\ell_\ass$ is then used to modify the selection of unassigned atoms in the above elaboration of \textit{decide}.
For this purpose, we replace Item~2 by
\(
U:=\ell_\ass(\mathcal{A}\setminus (\tlits{\ass}\cup\flits{\ass}))
\)
in order to restrict $U$ to unassigned atoms of (current) highest rank.
Unassigned atoms at lower levels are only considered once all atoms at higher levels have been assigned.
Atoms without an associated level default to level 0 because $\nu(\emptyset)=0$.
Hence, negative levels act as a penalty since the respective atoms are only taken into account
once all atoms with non-negative or no associated level have been assigned.

For a complementary example, 
consider a \texttt{level}-based formulation of the previous (\texttt{factor}-based) heuristic rule.
\begin{lstlisting}
_h(occ(A,T),level,T,0) :- action(A),time(T).
\end{lstlisting}
Unlike the above, $\mathtt{occ(a,2)}$ and $\mathtt{occ(a,3)}$ are now associated with different ranks,
which leads to strictly preferring $\mathtt{occ(a,3)}$ over $\mathtt{occ(a,2)}$ 
whenever both atoms are unassigned.
Hence, \texttt{level} modifiers partition the set of atoms and restrict $h$ to unassigned atoms at the highest level.

The previous replacement along with the above amendments of $h$ and $s$ through the domain-specific extensions $d$ and $t$
yields the following elaboration of CDCL's heuristic choice operation \textit{decide} for $i\geq 1$ (and given $d_0$).$^{\ref{fn:ass}}$
% --------------------------------------------------
\begin{enumerate}\addtocounter{enumi}{-1}\itemindent 10pt
\item $h_{i-1}(a) := d_{i-1}(a)$                         \hfill for each $a\in\mathcal{A}\qquad$
\item $h_i(a) := \alpha_i\times h_{i-1}(a) + \beta_i(a)$ \hfill for each $a\in\mathcal{A}\qquad$
\item $U:=\ell_{\ass_{i-1}}(\mathcal{A}\setminus (\tlits{\ass_{i-1}}\cup\flits{\ass_{i-1}}))$
\item $C:= \textit{argmax}_{a\in U}d_i(a)$
\item $a:= \tau(C)$
\item $\ass_i := \ass_{i-1}\cup\{t_i(a)a\}$
\end{enumerate}
% --------------------------------------------------
Although we formally model both $h$ and $d$ (as well as $s$ and $t$) as functions,
there is a substantial conceptual difference in practice in that $h$ is a system-specific data structure while $d$ is an associated method.
This is also reflected above, where $h$ is subject to assignments.
%
Item~0 makes sure that our heuristic modifications take part in the look-back based evolution in Item~1,
and are thus also subject to decay.
We added this as a separate line rather than integrating it into Item~1 in order to stress that our
modifications are modular in leaving the underlying heuristic machinery unaffected.
%
Item~2 gathers in $U$ all unassigned atoms of highest rank.
%
Among them, Item~3 collects in $C$ all atoms $a$ with a maximum heuristic value $d_i(a)$.
%
Since this is not guaranteed to yield a unique element, the system-specific tie-breaking function
$\tau$ is evoked to return a unique atom.
%
Finally, the modified sign heuristic $t_i$ determines a truth value for $a$, and the resulting
signed literal ${t_i(a)a}$ is added to the current assignment.

Note that so far all sample heuristic rules were \emph{static} in the sense that they are turned into
facts by the grounder and thus remain unchanged during solving.
Examples of dynamic heuristic rules are given at the end of next section.

Our simple heuristic language is easily extended by further heuristic atoms.
For instance, \hpred{a}{\mathtt{true}}{v}{p} and \hpred{a}{\mathtt{false}}{v}{p} have turned out to be useful in practice.
\begin{lstlisting}
_h(A,level,V,P) :- _h(A,true, V,P).
_h(A,sign, 1,P) :- _h(A,true, V,P).
_h(A,level,V,P) :- _h(A,false,V,P).
_h(A,sign,-1,P) :- _h(A,false,V,P).
\end{lstlisting}
%
For instance, the heuristic atom \hpred{a}{\mathtt{true}}{3}{3} expands to 
\hpred{a}{\mathtt{level}}{3}{3} and \hpred{a}{\mathtt{sign}}{1}{3},
expressing a preference for both making a decision on~$a$ and
assigning it to true.
On the other hand,
\hpred{a}{\mathtt{false}}{-3}{3} expands to 
\hpred{a}{\mathtt{level}}{-3}{3} and \hpred{a}{\mathtt{sign}}{-1}{3},
thus suggesting not to make a decision on~$a$ but to
assign it to false if there is no ``better'' decision variable.

Another shortcut of pragmatic value is the abstraction from specific priorities.
For this, we use the following rule.
\begin{lstlisting}
_h(A,M,V,#abs(V)) :- _h(A,M,V).
\end{lstlisting}
With it,
we can directly describe the heuristic restriction used in \cite{rintanen11a} to simulate planning
by iterated deepening $A^*$ \cite{korf85a} in SAT solving through limiting choices to action variables,
assigning those for time \texttt{T} before those for time \texttt{T+1}, and always assigning truth
value \texttt{true} (where \texttt{l} is a constant indicating the planning horizon):
\begin{lstlisting}
_h(occ(A,T),true,l-T) :- action(A), time(T).
\end{lstlisting}

Although we impose no restriction on the occurrence of heuristic atoms within logic programs,
it seems reasonable to require that the addition of rules containing heuristic atoms does not alter
the stable models of the original program.
That is, given a logic program $P$ over $\mathcal{A}$ and a set of rules $H$ over $\mathcal{A}\cup\mathcal{H}$,
we aim at a one-to-one correspondence between the stable models of $P$ and $P\cup H$ and
their identity upon projection on $\mathcal{A}$.
This property is guaranteed whenever heuristic atoms occur only in the head of rules and thus only
depend upon regular atoms.
In fact, so far, this class of rules turned out to be expressive enough to model all heuristics of interest,
including the ones presented in this paper.
It remains future work to see whether more sophisticated schemes, eg., involving recursion, are useful.

%%% Local Variables: 
%%% mode: latex
%%% TeX-master: "paper"
%%% End: 


\section{Heuristic language elements}\label{sec:approach}

We express heuristic modifications via a set $\mathcal{H}$ of \emph{heuristic atoms} 
disjoint from $\mathcal{A}$.
Such a heuristic atom is formed from a dedicated predicate \hpredicate\ along with four arguments:
a (reified) atom $a\in\mathcal{A}$,
a heuristic modifier $m$,
and two  integers $v,p\in\mathbb{Z}$.
A heuristic modifier is used to manipulate the heuristic treatment of an atom $a$ via the 
modifier's value given by $v$.
The role of this value varies for each modifier.
We distinguish four primitive heuristic modifiers:
\begin{description}
\item [\texttt{init}] for initializing the heuristic value of $a$ with $v$,
\item [\texttt{factor}] for amplifying the heuristic value of $a$ by factor $v$,
\item [\texttt{level}] for ranking all atoms; the rank of $a$ is $v$,
\item [\texttt{sign}] for attributing the sign of $v$ as truth value to $a$.
\end{description}
While $v$ allows for changing an atom's heuristic behavior relative to \emph{other} atoms,
the second integer $p$ allows us to express a priority for disambiguating similar
heuristic modifications to the \emph{same} atom.
This is particularly important in our dynamic setting, where varying heuristic atoms may be obtained
in view of the current assignment.
For instance, the heuristic atoms
\hpred{b}{\mathtt{sign}}{1}{3}
and
\hpred{b}{\mathtt{sign}}{-1}{5}
aim at assigning opposite truth values to atom $b$.
This conflict can be resolved by preferring the heuristic modification with the higher priority,
viz.\ 5 in \hpred{b}{\mathtt{sign}}{-1}{5}.
Obviously such priorities can only support disambiguation but not resolve conflicting values sharing the same priority.

For accommodating priorities,
we define for an assignment \ass\ the \emph{preferred values} for modifier $m$ on atom $a$ as
\[
V_{a,m}(\ass)=
\textit{argmax}_{v\in\mathbb{Z}}\{p\mid\Tsigned{\hpred{a}{m}{v}{p}}\in\ass\}.
\]
Heuristic values are dynamic;
they are extracted from the current assignment and may thus vary during solving.
Note that $V_{a,m}(\ass)$ returns the singleton set $\{v\}$,
if the current assignment \ass\ contains a single true heuristic atom \hpred{a}{m}{v}{p} 
involving $a$ and $m$.
$V_{a,m}(\ass)$ is empty whenever there are no such heuristic atoms.
%
And whenever all heuristic atoms regarding $a$ and $m$ have the same priority $p$,
$V_{a,m}(\ass)$ is equivalent to
\(
\{v\mid\Tsigned{\hpred{a}{m}{v}{p}}\in\ass\}
\).

Here are a few examples.
We obtain % the preferred values
$V_{b,\mathtt{sign}}(\ass_1)=\{-1\}$
and
$V_{c,\mathtt{init}}(\ass_1)=\emptyset$
from assignment
\(
\ass_1=\{\Fsigned{a},\Tsigned{\hpred{b}{\mathtt{sign}}{1}{3}},\Tsigned{\hpred{b}{\mathtt{sign}}{-1}{5}}\}
\),
while assignment
\(
\ass_2=\{\Tsigned{\hpred{b}{\mathtt{sign}}{1}{3}},\Tsigned{\hpred{b}{\mathtt{sign}}{-1}{3}}\}
\)
results in $V_{b,\mathtt{sign}}(\ass_2)=\{1,-1\}$.

For ultimately resolving ambiguities among alternative values for heuristic modifiers, 
we propose for a set $V\subseteq\mathbb{Z}$ of integers the function $\nu(V)$ as
\[
%\nu(V)=
\mathit{max}\big(\{v\in V\!\mid v\geq 0\}\cup\{0\}\big)
+
\mathit{min}\big(\{v\in V\!\mid v\leq 0\}\cup\{0\}\big).
\]
Note that $\nu(\emptyset)=0$, attributing 0 the status of a neutral value.
Alternative options exist, like taking means or median of $V$ or even time specific criteria
relating to the emergence of values in the assignment.
%
In the above examples,
we get $\nu(V_{b,\mathtt{sign}}(\ass_1))=-1$ and $\nu(V_{b,\mathtt{sign}}(\ass_2))=0$.

Given this, we proceed by defining the \emph{domain-specific extension} $d$ to the heuristic function $h$ 
for $a\in\mathcal{A}$ as
\[
d_0(a)=\nu(V_{a,\mathtt{init}}(\ass_0))+h_0(a)
\]
and for $i\geq 1$
\[
d_i(a)=
\left\{
  \begin{array}{rl}
    \nu(V_{a,\mathtt{factor}}(\ass_i))\times h_i(a)&\text{if } V_{a,\mathtt{factor}}(\ass_i)\neq\emptyset
    \\
                                             h_i(a)&\text{otherwise}
  \end{array}
\right.
\]
First of all, it is important to note that $d$ is merely a modification and not a replacement of the
system heuristic $h$.
In fact, $d$ extends the range of $h$ to $(-\infty,+\infty)$.
Negative values serve as penalties.
The values of the \texttt{init} modifiers are added to $h_0$ in $d_0$.
The use of addition rather than multiplication allows us to override an initial value of 0.
Also, the higher the absolute value of the \texttt{init} modifier, the longer lasts its effect
(given the decay of heuristic values).
Unlike this, \texttt{factor} modifiers rely on multiplication because they aim at de- or increasing
conflict scores gathered during conflict analysis.
In view of $h$'s range,
a factor greater than 1 amplifies the score, a negative one penalizes the atom, and 0 resets the atom's score.
Enforcing a factor of 1 
% (for instance, through assigning a high priority)
transfers control back to the system heuristic $h$.

Heuristically modified logic programs are simply programs over $\mathcal{A}\cup\mathcal{H}$,
the original vocabulary extended by heuristic atoms (without restrictions).
As a first example,
let us extend our planning encoding by a rule favoring atoms expressing action occurrences close to
the goal situation.
\begin{lstlisting}
_h(occ(A,T),factor,T,0) :- action(A),time(T).
\end{lstlisting}
With \texttt{factor}, we impose a bias on the underlying heuristic function $h$.
Rather than comparing, for instance,
the plain values $h(\mathtt{occ(a,2)})$ and $h(\mathtt{occ(a,3)})$,
a decision is made by looking at $2\times h(\mathtt{occ(a,2)})$ and  $3\times h(\mathtt{occ(a,3)})$,
even though it still depends on $h$.
A further refined strategy may suggest considering climbing actions as early as possible.
\begin{lstlisting}
_h(occ(climb,T),factor,l-T,1) :- time(T).
\end{lstlisting}
Clearly, this rule conflicts with the more general rule above.
However, this conflict is resolved in favor of the more specific rule by attributing it a higher
priority (viz.~1 versus 0).

% Similar statements can be formulated with the \texttt{init} modifier in order to change the initial heuristic values.

For capturing a \emph{domain-specific extension} $t$ to the sign heuristic $s$,
we define for $a\in\mathcal{A}$ and $i\geq 0$:
\[
t_i(a)=
\left\{
  \begin{array}{rl}
    \true &\text{if }
           \nu(V_{a,\mathtt{sign}}(\ass_i))>0
           % \text{ and }
           % V_{a,\mathtt{sign}}(\ass_i)\neq\emptyset
           \\
    \false&\text{if }
           \nu(V_{a,\mathtt{sign}}(\ass_i))<0
           % \text{ and }
           % V_{a,\mathtt{sign}}(\ass_i)\neq\emptyset
           \\
    s_i(a)&\text{otherwise}
  \end{array}
\right.
\]
As with $d$ above, the extension $t$ to the sign heuristic is dynamic.
The sign of the modifier's preferred value determines the truth value to assign to an atom at hand.
No \texttt{sign} modifier (or enforcing a value of 0) leaves sign selection with the system's sign heuristic $s$.
%
For example, the heuristic rule
\begin{lstlisting}
_h(holds(F,T),sign,-1,0) :- fluent(F),time(T).
\end{lstlisting}
tells the solver to assign false to non-deterministically chosen fluents.
%
The next pair of rules is a further refinement of our strategy on climbing actions,
favoring their effective occurrence in the first half of the plan.
\begin{lstlisting}
_h(occ(climb,T),sign, 1,0) :- T<l/2,time(T).
_h(occ(climb,T),sign,-1,0) :- T>l/2,time(T).
\end{lstlisting}
Thus, while the atom $\mathtt{occ(climb,1)}$ is preferably made true,
false should rather be assigned to $\mathtt{occ(climb,l)}$.

Finally, for accommodating rankings induced by \texttt{level} modifiers,
we define for an assignment \ass\ and $\mathcal{A}'\subseteq\mathcal{A}$:
\[
\ell_\ass(\mathcal{A}')=\textit{argmax}_{a\in\mathcal{A}'}\nu(V_{a,\mathtt{level}}(\ass))
\]
The set $\ell_\ass(\mathcal{A}')$ gives all atoms in $\mathcal{A}'$ with the highest \texttt{level} values relative to
the current assignment \ass.
Similar to $d$ and $t$ above, this construction is also dynamic and the rank of atoms may vary during solving.
The function $\ell_\ass$ is then used to modify the selection of unassigned atoms in the above elaboration of \textit{decide}.
For this purpose, we replace Item~2 by
\(
U:=\ell_\ass(\mathcal{A}\setminus (\tlits{\ass}\cup\flits{\ass}))
\)
in order to restrict $U$ to unassigned atoms of (current) highest rank.
Unassigned atoms at lower levels are only considered once all atoms at higher levels have been assigned.
Atoms without an associated level default to level 0 because $\nu(\emptyset)=0$.
Hence, negative levels act as a penalty since the respective atoms are only taken into account
once all atoms with non-negative or no associated level have been assigned.

For a complementary example, 
consider a \texttt{level}-based formulation of the previous (\texttt{factor}-based) heuristic rule.
\begin{lstlisting}
_h(occ(A,T),level,T,0) :- action(A),time(T).
\end{lstlisting}
Unlike the above, $\mathtt{occ(a,2)}$ and $\mathtt{occ(a,3)}$ are now associated with different ranks,
which leads to strictly preferring $\mathtt{occ(a,3)}$ over $\mathtt{occ(a,2)}$ 
whenever both atoms are unassigned.
Hence, \texttt{level} modifiers partition the set of atoms and restrict $h$ to unassigned atoms at the highest level.

The previous replacement along with the above amendments of $h$ and $s$ through the domain-specific extensions $d$ and $t$
yields the following elaboration of CDCL's heuristic choice operation \textit{decide} for $i\geq 1$ (and given $d_0$).$^{\ref{fn:ass}}$
% --------------------------------------------------
\begin{enumerate}\addtocounter{enumi}{-1}\itemindent 10pt
\item $h_{i-1}(a) := d_{i-1}(a)$                         \hfill for each $a\in\mathcal{A}\qquad$
\item $h_i(a) := \alpha_i\times h_{i-1}(a) + \beta_i(a)$ \hfill for each $a\in\mathcal{A}\qquad$
\item $U:=\ell_{\ass_{i-1}}(\mathcal{A}\setminus (\tlits{\ass_{i-1}}\cup\flits{\ass_{i-1}}))$
\item $C:= \textit{argmax}_{a\in U}d_i(a)$
\item $a:= \tau(C)$
\item $\ass_i := \ass_{i-1}\cup\{t_i(a)a\}$
\end{enumerate}
% --------------------------------------------------
Although we formally model both $h$ and $d$ (as well as $s$ and $t$) as functions,
there is a substantial conceptual difference in practice in that $h$ is a system-specific data structure while $d$ is an associated method.
This is also reflected above, where $h$ is subject to assignments.
%
Item~0 makes sure that our heuristic modifications take part in the look-back based evolution in Item~1,
and are thus also subject to decay.
We added this as a separate line rather than integrating it into Item~1 in order to stress that our
modifications are modular in leaving the underlying heuristic machinery unaffected.
%
Item~2 gathers in $U$ all unassigned atoms of highest rank.
%
Among them, Item~3 collects in $C$ all atoms $a$ with a maximum heuristic value $d_i(a)$.
%
Since this is not guaranteed to yield a unique element, the system-specific tie-breaking function
$\tau$ is evoked to return a unique atom.
%
Finally, the modified sign heuristic $t_i$ determines a truth value for $a$, and the resulting
signed literal ${t_i(a)a}$ is added to the current assignment.

Note that so far all sample heuristic rules were \emph{static} in the sense that they are turned into
facts by the grounder and thus remain unchanged during solving.
Examples of dynamic heuristic rules are given at the end of next section.

Our simple heuristic language is easily extended by further heuristic atoms.
For instance, \hpred{a}{\mathtt{true}}{v}{p} and \hpred{a}{\mathtt{false}}{v}{p} have turned out to be useful in practice.
\begin{lstlisting}
_h(A,level,V,P) :- _h(A,true, V,P).
_h(A,sign, 1,P) :- _h(A,true, V,P).
_h(A,level,V,P) :- _h(A,false,V,P).
_h(A,sign,-1,P) :- _h(A,false,V,P).
\end{lstlisting}
%
For instance, the heuristic atom \hpred{a}{\mathtt{true}}{3}{3} expands to 
\hpred{a}{\mathtt{level}}{3}{3} and \hpred{a}{\mathtt{sign}}{1}{3},
expressing a preference for both making a decision on~$a$ and
assigning it to true.
On the other hand,
\hpred{a}{\mathtt{false}}{-3}{3} expands to 
\hpred{a}{\mathtt{level}}{-3}{3} and \hpred{a}{\mathtt{sign}}{-1}{3},
thus suggesting not to make a decision on~$a$ but to
assign it to false if there is no ``better'' decision variable.

Another shortcut of pragmatic value is the abstraction from specific priorities.
For this, we use the following rule.
\begin{lstlisting}
_h(A,M,V,#abs(V)) :- _h(A,M,V).
\end{lstlisting}
With it,
we can directly describe the heuristic restriction used in \cite{rintanen11a} to simulate planning
by iterated deepening $A^*$ \cite{korf85a} in SAT solving through limiting choices to action variables,
assigning those for time \texttt{T} before those for time \texttt{T+1}, and always assigning truth
value \texttt{true} (where \texttt{l} is a constant indicating the planning horizon):
\begin{lstlisting}
_h(occ(A,T),true,l-T) :- action(A), time(T).
\end{lstlisting}

Although we impose no restriction on the occurrence of heuristic atoms within logic programs,
it seems reasonable to require that the addition of rules containing heuristic atoms does not alter
the stable models of the original program.
That is, given a logic program $P$ over $\mathcal{A}$ and a set of rules $H$ over $\mathcal{A}\cup\mathcal{H}$,
we aim at a one-to-one correspondence between the stable models of $P$ and $P\cup H$ and
their identity upon projection on $\mathcal{A}$.
This property is guaranteed whenever heuristic atoms occur only in the head of rules and thus only
depend upon regular atoms.
In fact, so far, this class of rules turned out to be expressive enough to model all heuristics of interest,
including the ones presented in this paper.
It remains future work to see whether more sophisticated schemes, eg., involving recursion, are useful.

%%% Local Variables: 
%%% mode: latex
%%% TeX-master: "paper"
%%% End: 

%% \section{Experiments}\label{sec:experiments}
%
\begin{table}[t]
\caption{Comparison of approximation techniques by 
(a) runtime and timeouts,
(b) diversification quality, and
(c) minimum distance}
\small
\parbox{.32\linewidth}{\centering
\begin{tabular}{|l||r|r|}

\hline
Class & \textit{T} & \textit{TO}  \\ 
\hline
\Alabel{3} & \textbf{165} & \textbf{70} \\
\Alabel{3}-\textit{true} & 200 & 113 \\ 
\Alabel{3}-\textit{all} & 202 & 118 \\ 
\Alabel{3}-\textit{rd} & 277 & 280 \\ 
\Alabel{3}-\textit{pg} & 317 & 351\\
\Alabel{3}-\textit{pg-l-rd} & 354 & 442\\
\Alabel{3}-\textit{false} & 351 & 443 \\ 
\Alabel{3}-\textit{pg-l} & 351 & 443\\
\Alabel{2}-\textit{true} & 482 & 618\\
\Alabel{2}-\textit{rd} & 474 & 648\\
\Alabel{1} & 482 & 672\\
\Alabel{2}-\textit{dist-to} & 528 & 689\\
\Alabel{2}-\textit{all} & 515 & 696\\
\Alabel{2}-\textit{false} & 532 & 696\\
\Alabel{2}-\textit{pg} & 542 & 708\\
\Alabel{2}-\textit{dist} & 572 & 773\\
\hline
\end{tabular} 
}
\parbox{.32\linewidth}{\centering
\begin{tabular}{|l||r|r|}

\hline
Class & \textit{S} & \textit{avg}\\ 
\hline
\Alabel{1} & \textbf{15} & 0.13\\
\Alabel{2}-\textit{dist-to} & 14 & 0.14\\ 
\Alabel{2}-\textit{pg} & 13 & \textbf{0.18}\\ 
\Alabel{3}-\textit{pg-l} & 11 & 0.17\\
\Alabel{3}-\textit{pg-l-rd} & 10 & 0.16\\
\Alabel{2}-\textit{all}  & 10 & 0.15\\
\Alabel{2}-\textit{dist} & 8 & 0.07\\ 
\Alabel{2}-\textit{false} & 8 & 0.15\\ 
\Alabel{2}-\textit{true} & 7 & 0.12\\ 
\Alabel{3}-\textit{false} & 6 & 0.16\\ 
\Alabel{2}-\textit{rd} & 5 & 0.12\\ 
\Alabel{3}-\textit{all}  & 5 & 0.08 \\ 
\Alabel{3}-\textit{true} & 4 & 0.08 \\ 
\Alabel{3}-\textit{rd} & 2 & 0.09 \\ 
\Alabel{3}-\textit{pg} & 1 & 0.09\\
%\Alabel{3}-Hdyn & 1 & 0.09\\ 
\Alabel{3} & 0 & 0.06\\

\hline
\end{tabular} 
}
\parbox{.32\linewidth}{\centering
\begin{tabular}{|l||r|r|}

\hline
Class & \textit{S} & \textit{avg}\\ 
\hline
\Alabel{1} & \textbf{15} & 12.25\\
\Alabel{2}-\textit{dist-to} & 13 & 10.38\\
\Alabel{3}-\textit{pg-l-rd } & 13 & 11.82 \\
\Alabel{2}-\textit{dist} & 12 & 5.31\\
\Alabel{3}-\textit{pg-l} & 12 & 11.10\\
\Alabel{2}-\textit{pg} & 10 & \textbf{12.86}\\
\Alabel{2}-\textit{rd} & 9 & 8.77 \\
\Alabel{3}-\textit{all}  & 7 & 3.99 \\ 
\Alabel{3}-\textit{true} & 6 & 4.00 \\ 
\Alabel{3}-\textit{false} & 6 & 7.07 \\ 
\Alabel{2}-\textit{false} & 6 & 6.80\\
\Alabel{2}-\textit{all}  & 4 & 6.98\\
\Alabel{2}-\textit{true} & 3 & 5.31\\
\Alabel{3}-\textit{rd} & 2 & 6.43\\
\Alabel{3} & 2 & 4.28\\
%\Alabel{3}-Hdyn & 1 & 2.90\\ 
\Alabel{3}-\textit{pg} & 0 & 2.79\\
\hline
\end{tabular} 
}
\label{tab:time_comparison_small}
\label{tab:diverse_comparison_small}
\label{tab:min_dist_comparison_small}
\end{table}
%
In this section, we present experiments focusing on the \emph{approximation} techniques of the \asprin\ system for obtaining most dissimilar optimal
solutions. 
%
While \emph{enumeration} and \emph{replication} provide exact results, they need to calculate and store a possibly exponential number of optimal
models or deal with a large search space, respectively.
%
Those techniques are therefore not effective for most practical applications.
%
For Algorithm~\Alabel{2}, we considered the variations \textit{rd}, \textit{pg}, \textit{true}, \textit{false}, and \textit{all} .
%
In \textit{dist}, we issued no timeout for the computation of the partial interpretation, 
while in \textit{dist-to}, we set a timeout for this computation of half the total possible runtime.
%
For Algorithm~\Alabel{3}, we consider the variations that include no extra ASP computation, namely, 
\textit{rd}, \textit{pg}, \textit{true}, \textit{false}, and \textit{all} .
%
We also evaluated a version without any heuristic modification (named simply \Alabel{3}).
%
Furthermore, following \cite{nadel11a}, 
we considered a variation of \textit{pg}, viz.~\textit{pg-l}, 
where the atoms of the selected partial interpretation are given a higher priority, 
and \textit{pg-l-rd}, extending \textit{pg-l} by fixing initially a random sign to all atoms not appearing in the partial interpretation.

We gathered 186 instances from six different classes: \emph{Design Space exploration (DSE)} from~\cite{angeglharesc13a}, \emph{Timetabling (CTT)}
from~\cite{basotainsc13a}, \emph{Crossing minimization} from the ASP competition 2013, \emph{Metabolic network expansion} from \cite{schthi09a},
\emph{Biological network repair} from \cite{geguivscsithve10a} and \emph{Circuit Diagnosis} from~\cite{sidiqqi11a}.
Since we required instances with multiple optimal solutions, we exclusively focused on Pareto optimality. 
DSE and CTT are inherently multi-objective and therefore we could naturally define a Pareto preference for them. 
For the other classes, we turned single-objective into multi-objective optimization problems by distributing their optimization statements.
First, we split the atoms in the optimization statements into four or eight groups evenly. 
We chose for each group the same preference type, either cardinality or subset minimization, and aggregated them by means of Pareto preference.
We calculated optimal solutions regarding these Pareto preferences.
The same was done for CTT and DSE.
An instance was selected if for some Pareto preference ten optimal solutions could be obtained within 600 seconds by \asprin. 
This method generated 816 instances in total. 
We ran the benchmarks on a cluster of Linux machines with dual Xeon E5520 quad-core 2.26 GHz processors and 48 GB RAM. 
We restricted the runtime to 600 seconds and the memory usage to 20 GB RAM.

Since algorithms~\Alabel{1} and \Alabel{2} involve querying programs over preferences, 
we started by evaluating the different query techniques. 
%
For that, we executed \Alabel{1} with query methods \Qlabel{1} to \Qlabel{4} on all selected instances,
stopping after the first $\mathit{solveQuery}$ call was finished.
%
%We achieved that by first calculating an optimal solution and then finding another optimal solution fulfilling the query that the model has to be dissimilar.
The performance of query techniques \Qlabel{2}, \Qlabel{3}, and \Qlabel{4} was similar regarding runtime and only \Qlabel{1} was clearly worse.
We selected \Qlabel{4} for the remaining experiments due to its slightly lower runtime. 
For more detailed tables, we refer to~\cite{roscwa16b}. % \ref{sec:suptables}.

Next, we approximated four most diverse optimal models with methods \Alabel{1} to \Alabel{3}. 
%
We measured runtime and two quality measures.
The first, called diversification quality~\cite{nadel11a},
gives the sum of the Hamming distances among all pairs of solutions normalized to values between zero and one.
The second is the minimum distance among all pairs of solutions of a set in percent.
%
The solution set size of four was chosen because~\cite{shimazu01a} 
claims that three solutions is the optimal amount for a user,
and considering one additional solution provides further insight into the different quality measures. 
%
For all algorithms that do not use heuristics for diversification, 
we instead enabled heuristics preferring a negative sign for the atoms appearing in preference statements. 
This was observed in~\cite{brderosc15b} to improve performance.

Table~\ref{tab:time_comparison_small}(a) provides in column \textit{T} the average runtime and in column \textit{TO} the sum of timeouts. 
The different methods are ordered by the number of timeouts. 
The best results in a column are shown in bold. 
We see that \Alabel{3} is by far the fastest with 70 timeouts, solving 91\% of the instances. 
Heuristic variations of \Alabel{3} perform the best after that. 
Less invasive heuristics achieve similar runtimes with 113-118 timeouts.
More sophisticated heuristics perform worse at 349-443 timeouts.
In a range from 618 to 773 timeouts, non-heuristic methods solve the least instances by a significant margin.
The results are in tune with the nature of the methods. 
Heuristics modifying the solving process for diversity decrease the performance 
in comparison with solving heuristics aimed at performance, 
but not as much as more complex methods involving preferences over optimal models. 

In particular, non-heuristic methods show many timeouts. 
If we tried to analyze the quality of the solutions by assuming worst possible values for the instances that timed out,
the results would be dominated by these instances. 
To avoid that, we calculated a score independent of the runtime.
We considered all possible parings of the different methods. 
For each pair, we compared only instances where both found a solution set.
The method with better quality value for the majority of instances receives a point. 
Finally, we ordered the subsequent tables according to that score. 
 
In Table~\ref{tab:diverse_comparison_small}(b), for each method we see the score in column \textit{S}, and 
the average of the diversification quality (over the instances solved by the method) in column \textit{avg}. 
This way, we can examine the quality a method has achieved compared to other methods, and also the individual average quality.
\Alabel{1} has the best quality with a score of 15, followed by \Alabel{2}-\textit{dist-to}, \Alabel{2}-\textit{pg}, \Alabel{3}-\textit{pg-l} and \Alabel{3}-\textit{pg-l-rd}.
All of those techniques regard the whole previous solution set to calculate the next solution
and guide the solving strictly to diversity.
\Alabel{2}-\textit{pg}, \Alabel{3}-\textit{pg-l} and \Alabel{3}-\textit{pg-l-rd } are also the first, second and third place, respectively, for average diversification quality. 
Next, with scores ranging from 10-7, we see \Alabel{2} methods 
that do not take into account the whole previous set, 
or that were simply unable to find many solutions at all, as in the case of \Alabel{2}-\textit{dist}. 
Finally, we observe that \Alabel{3} variations only regarding the last solution or no previous information 
perform worst in score and average. 
In these cases, the heuristic does not seem to be strong enough to steer the solving to high quality solution sets, 
and \Alabel{3} uses no heuristic or optimization techniques to ensure diverse solutions.

In analogy to Table~\ref{tab:diverse_comparison_small}(b),
Table~\ref{tab:min_dist_comparison_small}(c) provides information for the minimum distance among the solutions. 
%
% The overall grouping of the methods is similar to Table~\ref{tab:diverse_comparison_small}(b). 
%
The best methods considering score and average minimum distance, 
viz.\ \Alabel{1}, \Alabel{2}-\textit{dist-to}, \Alabel{3}-\textit{pg-l-rd}, \Alabel{3}-\textit{pg-l}, \Alabel{2}-\textit{pg}, utilize information from the whole
previous solution set and have strict diversification techniques. 
%\comment{I cut the part about the different behavior of min distance and diversification. The data is not that clear and it saves space. Maybe if we have space left in the end...}

Overall, plain heuristic methods perform better in regards to runtime 
while more complex methods, depending on all previous solutions, lead to better quality. 
%
Furthermore, \Alabel{3}-\textit{pg-l-rd } and \Alabel{3}-\textit{pg-l} provide the best trade-off between performance and quality. 
%
While \Alabel{1}, \Alabel{2}-\textit{dist-to} and \Alabel{2}-\textit{pg} achieve higher quality, they could solve only 18\%, 16\% and 13\% of the instances. 
%
On the other hand, \Alabel{3}-\textit{pg-l-rd } and \Alabel{3}-\textit{pg-l} provide good diversification quality and minimum distance while solving 46\% of the instances. 
%
%\comment{this section is enough for general conclusion: plain heuristic: fast but bad, maxmin: slow but good, more complex heuristic: tradeoff}


%%% Local Variables: 
%%% mode: latex
%%% TeX-master: "paper"
%%% End: 


\section{Experiments}\label{sec:experiments}

We implemented our approach as a dedicated heuristic module within the ASP solver \textit{clasp}
(2.1; available at \cite{hclasp}).
We consider \textit{moms} \cite{pretolani96a} as initial heuristic $h_0$ and
\textit{vsids} \cite{momazhzhma01a} as heuristic function $h_i$.
Accordingly, the sign heuristic \textsl{s} is set to the one associated with \textit{vsids}.
As base configuration, we use \textit{clasp} with options \texttt{--heu=vsids} and \texttt{--init-moms}.
%
To take effect,
the heuristic atoms as well as their contained atoms must be made visible to the solver via \texttt{\#show} directives.
Once the option \texttt{--heu=domain} is passed to \texttt{clasp},
it extracts the necessary information from the symbol table and applies the heuristic modifications
when it comes to non-deterministic assignments.
%
Our experiments ran under Linux on dual Xeon E5520 quad-core processors with $2.26$GHz and $48$GB RAM.
Each run was restricted to 600s CPU time.
Timeouts account for 600s and performed choices.

% ------------------------------------------------------------
\newcommand{\rvsids}[4]{\multicolumn{3}{c|}{#1$s$ (#2)}}
\newcommand{\cvsids}[4]{\multicolumn{3}{c|}{#4}}
\newcommand{\domheu}[3]{#1\%&(#2)&#3\%}
% ------------------------------------------------------------
\begin{table}[t]
  \centering\small
  \begin{tabular}{|@{\,}r@{\,}|@{}r@{}@{}r@{}@{}r@{}|@{}r@{}@{}r@{}@{}r@{}|@{}r@{}@{}r@{}@{}r@{}|}
    \hline
    \multicolumn{1}{|c|}{Setting}                    & \multicolumn{3}{@{}c@{}|}{\textit{Labyrinth}}& \multicolumn{3}{@{}c@{}|}{\textit{Sokoban}}   & \multicolumn{3}{@{}c@{}|}{\textit{Hanoi Tower}} \\
    \hline
    \multicolumn{1}{|@{\;}l@{}|}{\textit{base configuration}} & \rvsids{9,108}{14}{5,908,451}{24,545,667}    & \rvsids{2,844}{3}{13,799,878}{19,371,267}     & \rvsids{9,137}{11}{34,126,406}{41,016,235}      \\   
                                                     & \cvsids{9,108}{14}{5,908,451}{24,545,667}    & \cvsids{2,844}{3}{13,799,878}{19,371,267}     & \cvsids{9,137}{11}{34,126,406}{41,016,235}      \\
    \hline
    \hpre{a}{\texttt{init}}{\texttt{2}}              & \domheu{95}{12}{94}                          & \domheu{91}{\textbf{1}}{84}                   & \domheu{85}{9}{89}                              \\
    \hpre{a}{\texttt{factor}}{\texttt{4}}            & \domheu{\textbf{78}}{\textbf{8}}{30}         & \domheu{120}{\textbf{1}}{107}                 & \domheu{109}{11}{110}                           \\
    \hpre{a}{\texttt{factor}}{\texttt{16}}           & \domheu{\textbf{78}}{10}{23}                 & \domheu{120}{\textbf{1}}{107}                 & \domheu{109}{11}{110}                           \\
    \hpre{a}{\texttt{level}}{\texttt{1}}             & \domheu{90}{12}{\textbf{5}}                  & \domheu{119}{2}{91}                           & \domheu{126}{15}{120}                           \\\cline{1-1}
    \hpre{f}{\texttt{init}}{\texttt{2}}              & \domheu{103}{14}{123}                        & \domheu{\textbf{74}}{2}{\textbf{71}}          & \domheu{97}{10}{109}                            \\
    \hpre{f}{\texttt{factor}}{\texttt{2}}            & \domheu{98}{12}{49}                          & \domheu{116}{3}{134}                          & \domheu{\textbf{55}}{\textbf{6}}{\textbf{70}}   \\
    \hpre{f}{\texttt{sign}}{\texttt{-1}}             & \domheu{94}{13}{89}                          & \domheu{105}{\textbf{1}}{100}                 & \domheu{92}{12}{92}                             \\
    \hline
  \end{tabular}
  \caption{Selection from evaluation of heuristic modifiers}
  \label{tab:modifiers}
\end{table}
% ------------------------------------------------------------
%
To begin with,
we report on a systematic study comparing single heuristic modifications.
A selection of best results is given in Table~\ref{tab:modifiers};
full results are available at~\cite{hclasp}.
We focus on well-known ASP planning benchmarks in order to contrast heuristic modifications on comparable problems:
\textit{Labyrinth}, \textit{Sokoban}, and \textit{Hanoi Tower}, each comprising 32 instances from the third ASP competition~\cite{contest11a}.%
\footnote{All instances are satisfiable except for one third in \textit{Sokoban}.}
We contrast the aforementioned base configuration with 38 heuristic modifications,
(separately) promoting the choice of actions (\emph{a}) and fluents (\emph{f}) via the heuristic modifiers
\texttt{factor} (1,2,4,8,16),
\texttt{init} (2,4,8,16),
\texttt{level} (1,-1),
\texttt{sign} (1,-1),
as well as attributing values to \texttt{factor}, \texttt{init}, and \texttt{level} by ascending and descending time points.
%
The first line of Table~\ref{tab:modifiers} gives the sum of times, timeouts, and choices obtained by the base configuration on all 32 instances of each problem class.
The results of the two configurations using \texttt{factor,1} differ from these figures in the low per mille range, demonstrating that the infrastructure supporting heuristic
modifications does not lead to a loss in performance.
The seven configurations in Table~\ref{tab:modifiers} yield best values in at least one category (indicated in boldface).
We express the accumulated times and choices as percentage wrt the base configuration; timeouts are total.
We see that the base configuration can always be dominated by a heuristic modification.
However, the whole spectrum of modifiers is needed to accomplish this.
In other words, there is no dominating heuristic modifier and each problem class needs a customized heuristic.
Looking at \textit{Labyrinth}, we observe that a preferred choice of action occurrences ($a$) pays off.
The stronger this is enforced, the fewer choices are made.
However, the extremely low number of choices with \texttt{level} does not result in less time or timeouts
(compared to a ``lighter'' \texttt{factor}-based enforcement).
While with \texttt{level} \emph{all} choices are made on heuristically modified atoms,
both \texttt{factor}-based modifications result in only 43\% such choices and thus leave much more room to the solver's heuristic.
For a complement, \textit{a},\texttt{init},2 as well as the base configuration (with \textit{a},\texttt{factor},1) 
make 14\% of their choices on heuristically modified atoms
(though the former produces in total 6\% less choices than the latter).
Similar yet less extreme behaviors are observed on the two other classes.
With \textit{Hanoi Tower}, a slight preference of fluents yields a strictly dominating configuration,
whereas no dominating improvement was observed with \textit{Sokoban}.

% ------------------------------------------------------------
\newcommand{\data}[2]{&\,\ignorespaces#1$s$&(\ignorespaces#2)\,}
% ------------------------------------------------------------
\begin{table}[t]
  \centering\small
  \begin{tabular}{|@{}l@{}|@{}r@{}@{}r@{}|@{}r@{}@{}r@{}|@{}r@{}@{}r@{}@{}r@{}@{}r@{}|}
    \hline
    \multicolumn{1}{|@{}c@{}|}{Setting} & \multicolumn{2}{@{}c@{}|}{\textit{Diagnosis}} & \multicolumn{2}{@{}c@{}|}{\textit{Expansion}} & \multicolumn{2}{@{}c@{}}{\textit{Repair (H)}} & \multicolumn{2}{@{}c@{}|}{\textit{Repair (S)}}\\
    \hline
    \textit{base config.}            \data{        111.1}{        115}\data{        161.5}{      100}\data{       101.3}{       113}\data{        33.3}{        27}\\
    \hline
    \texttt{s,-1}                    \data{        324.5}{        407}\data{         7.6}{         3}\data{         8.4}{         5}\data{         3.1}{\textbf{0}}\\
    \texttt{s,-1} \, \texttt{f,2}    \data{        310.1}{        387}\data{         7.4}{\textbf{2}}\data{         3.5}{\textbf{0}}\data{         3.2}{         1}\\
    \texttt{s,-1} \, \texttt{f,8}    \data{        305.9}{        376}\data{         7.7}{\textbf{2}}\data{         3.1}{\textbf{0}}\data{         2.9}{\textbf{0}}\\
    \texttt{s,-1} \, \texttt{l,1}    \data{\textbf{76.1}}{\textbf{83}}\data{\textbf{6.6}}{\textbf{2}}\data{\textbf{0.8}}{\textbf{0}}\data{         2.2}{         1}\\
\multicolumn{1}{|r@{}|}{\texttt{l,1}}\data{         77.3}{         86}\data{        12.9}{         5}\data{         3.4}{\textbf{0}}\data{\textbf{2.1}}{\textbf{0}}\\
    \hline
  \end{tabular}
  \caption{Abductive problems with optimization}
  \label{tab:opt}
\end{table}
% ------------------------------------------------------------
%
Next, we apply our heuristic approach to problems using abduction in combination with a \texttt{\#minimize} statement
minimizing the number of abducibles.
We consider
Circuit \textit{Diagnosis},
Metabolic Network \textit{Expansion},
and
Transcriptional Network \textit{Repair} (including two distinct experiments, \textit{H} and \textit{S}).
The first uses the ISCAS-85 benchmark circuits along with test cases generated as in \cite{sidiqqi11a};
this results in 790 benchmark instances.
The second one considers the completion of the metabolic network of \emph{E.coli} with reactions from \textit{MetaCyc} in view of generating target from seed metabolites~\cite{schthi09a}.
We selected the 450 most difficult benchmarks in the suite.
Finally, we consider repairing the transcriptional network of \emph{E.coli} from \textit{RegulonDB} in view of two distinct experiment series~\cite{geguivscsithve10a}.
Selecting the most difficult triple repairs provided us with 1000 instances.
%
Our results are summarized in Table~\ref{tab:opt}.
Each entry gives the average runtime and number of timeouts.
% Otherwise the experimental setting is as above.
%
Here, heuristic modifiers apply only to abducibles subject to minimization.
%
For supporting minimization,
we assign false to such abducibles (\texttt{s,-1})%
\footnote{Assigning \true\ instead leads to a deterioration of performance.}
and gradually increase the bias of their choice by imposing \texttt{factor} 2 and 8 (\texttt{f})
or enforce it via a \texttt{level} modifier (\texttt{l,1}).
%
The second last setting%
\footnote{This corresponds to using \hpre{a}{\texttt{false}}{1} for an abducible $a$.}
in Table~\ref{tab:opt} is the winner, leading to speedups of one to two orders of magnitude over the base configuration.
Interestingly, merely fixing the sign heuristics to \false\ leads at first to a deterioration of performance on \textit{Diagnosis} problems.
This is finally overcome by the constant improvement observed by gradually strengthening the bias of choosing abducibles.
The stronger the preference for abducibles, the faster the solver converges to an optimum solution.
This limited experiment already illustrates that sometimes the right combination of heuristic
modifiers yields the best result.

Finally, let us consider true PDDL planning problems.
For this, we selected 20 instances from the % and \textit{Freecell'02} in view of \textit{Freecell'00}.}
STRIPS domains of the 2000 and 2002 planning competition~\cite{icaps-competition}.%
\footnote{We discard \textit{Schedule'00} due to grounding issues.}
In turn, we translated these PDDL instances into facts via \textit{plasp}~\cite{gekaknsc11a}
and used a simple planning encoding with 15 different plan lengths (\verb+l=5,10,..,75+) to generate 3000 ASP instances.
%
Inspired yet different from \cite{rintanen12a}, we devised a dynamic heuristic that aims at propagating fluents' truth values backwards in time.
Attributing levels via \texttt{l-T+1} aims at proceeding depth-first from the goal fluents.
\begin{lstlisting}
_h(holds(F,T-1),true, l-T+1) :- holds(F,T).
_h(holds(F,T-1),false,l-T+1) :-
        fluent(F), time(T), not holds(F,T).
\end{lstlisting}
% ------------------------------------------------------------
\newcommand{\pdata}[3]{&\,\ignorespaces#1$s$&(\ignorespaces#2/\ignorespaces#3)}
\newcommand{\sdata}[3]{&\,\ignorespaces#1$s$&(\ignorespaces#3)}
% ------------------------------------------------------------
\begin{table}[t]
  \centering\scriptsize
  \begin{tabular}{|@{}r@{\,}|@{}r@{}@{}r@{}|@{}r@{}@{}r@{}|@{}r@{}@{}r@{}|r@{}@{}r|}
    \hline
   \multicolumn{1}{|@{}c@{\,}|}{Problem} & \multicolumn{2}{@{}c@{}|}{\textit{base}} & \multicolumn{2}{@{}c@{}|}{\textit{base}+\texttt{\_h}} & \multicolumn{2}{@{}c@{}|}{\textit{base (SAT)}} & \multicolumn{2}{@{}c@{}|}{\textit{\textit{base}+\texttt{\_h} (SAT)}} \\
    \hline
    \textit{Blocks'00}    \pdata{134.4}{ 180}{  61}\pdata{  9.2}{ 239}{  3}\sdata{163.2}{180}{59}\sdata{ 2.6}{239}{0}\\
    \textit{Elevator'00}  \pdata{  3.1}{ 279}{   0}\pdata{  0.0}{ 279}{  0}\sdata{  3.4}{279}{ 0}\sdata{ 0.0}{279}{0}\\
    \textit{Freecell'00}  \pdata{288.7}{ 147}{ 115}\pdata{184.2}{ 194}{ 74}\sdata{226.4}{147}{47}\sdata{52.0}{194}{0}\\
    \textit{Logistics'00} \pdata{145.8}{ 148}{  61}\pdata{115.3}{ 168}{ 52}\sdata{113.9}{148}{23}\sdata{15.5}{168}{3}\\
    \hline
    \textit{Depots'02}    \pdata{400.3}{  51}{ 184}\pdata{297.4}{ 115}{135}\sdata{389.0}{ 51}{64}\sdata{61.6}{115}{0}\\
    \textit{Driverlog'02} \pdata{308.3}{ 108}{ 143}\pdata{189.6}{ 169}{ 92}\sdata{245.8}{108}{61}\sdata{ 6.1}{169}{0}\\
    \textit{Rovers'02}    \pdata{245.8}{ 138}{ 112}\pdata{165.7}{ 179}{ 79}\sdata{162.9}{138}{41}\sdata{ 5.7}{179}{0}\\
    \textit{Satellite'02} \pdata{398.4}{  73}{ 186}\pdata{229.9}{ 155}{106}\sdata{364.6}{ 73}{82}\sdata{30.8}{155}{0}\\
    \textit{Zenotravel'02}\pdata{350.7}{ 101}{ 169}\pdata{239.0}{ 154}{116}\sdata{224.5}{101}{53}\sdata{ 6.3}{154}{0}\\
    \hline
    \textit{Total}          \pdata{252.8}{1225}{1031}\pdata{158.9}{1652}{657}\sdata{187.2}{1225}{430}\sdata{17.1}{1652}{3}\\
    \hline
  \end{tabular}
  \caption{Planning Competition Benchmarks '00 and '02}
  \label{tab:plan}
\end{table}%
% ------------------------------------------------------------
%
Our results are given in Table~\ref{tab:plan}.
Each entry gives the average runtime along with the number of (solved satisfiable instances and) timeouts (in columns two and three).
% Otherwise the experimental setting is as above.
Our heuristic amendment (\textit{base}+\texttt{\_h}) greatly improves over the base configuration in terms of runtime and timeouts.
On the overall set of benchmarks, it provides us with 427 more plans and 374 less timeouts.
As already observed by \cite{rintanen12a}, the heuristic effect is stronger on satisfiable instances.
This is witnessed by the two last columns restricting results to 1655 satisfiable instances solved by either system setup.
Our heuristic extension allows us to reduce the total number of timeouts from 430 to 3;
the reduction in solving time would be even more drastic with a longer timeout.

Interestingly, the previous dynamic heuristic has no overwhelming effect on our initial ASP planning problems.
An improvement was only observed on \textit{Hanoi Tower} problems (being susceptible to choices on fluents),
viz.\ `{54\%}(7)\,\textbf{57\%}' in terms of the format used in Table~\ref{tab:modifiers}.
However, restricting the heuristic to positive fluents by only using the first rule gives a substantial
improvement, namely `\textbf{19\%}(\textbf{2})\,{66\%}', in terms of runtime and timeouts.
A direct comparison of both heuristics shows that, although the latter performs 15\% more choices, 
it encounters 75\% fewer conflicts than the former.

%%% Local Variables: 
%%% mode: latex
%%% TeX-master: "paper"
%%% End: 

%% 
\section{Discussion}\label{sec:discussion}

We presented a comprehensive framework for computing diverse (or similar) solutions to logic programs with generic preferences
and implemented it in \asprin~2, available at~\cite{asprin}. %\comment{T: Make it available!}
To this end, we introduced a spectrum of different methods, among them, generalizations of existing work to the case of
programs with general preferences.
Hence, certain fragments of our framework provide implementations of the proposals in \cite{eiererfi13a,zhutru13a}.
While the latter had to resort to solver wrappers or even internal solver modifications,
\asprin\ heavily relies upon multi-shot solving that allows for an easy yet fine-grained control of reasoning processes.
Moreover, we provided several generic building blocks, such as 
\textit{maxmin} (and \textit{minmax}) preferences,
query-answering for programs with preferences,
preferences among optimal models,
and an automated approach for the guess and check methodology of~\cite{eitpol06a},
all of which are also of interest beyond diversification.
%
Finally, we took advantage of the uniform setting offered by \asprin~2 to conduct a 
comparative empirical analysis of the various methods for diversification.
Generally speaking,
there is a clear trade-off between performance and diversification quality, 
which allows for selecting the most appropriate method 
depending on the hardness of the application at hand.
%\comment{T: Last phrase is a bit weak\dots}
% \begin{itemize}
% \item System for diverse optimal models of logic program with preferences.
% \item Lifts previous approaches in ASP \cite{eiererfi13a} or with answer set optimization \cite{zhutru13a} to general preferences of \asprin.
% \item Variety of methods.
% \item Experimental evaluation shows that different methods differ wrt performance and diversity of solutions.
% \item There is a tradeoff between performance and diversity, which allows for selecting the most appropiate method at hand for the application at hand.
% \item Furthermore, four more general contributions to preferences: $maxmin$, automation of generate and test, query solving, and preferences over
% optimal models.
% \item Future work: Real world applications on DSS and TT.
% \item Future work: Apply SMT framework of \clingo\ 5. In \cite{eiererfi13a}, the modification of the solver was more efficient than ASP implementation.
%       We intend to do it in a principled (general?) way inside the new framework.
% \end{itemize}



%%% Local Variables: 
%%% mode: latex
%%% TeX-master: "paper"
%%% End: 


\section{Discussion}\label{sec:discussion}

Various ways of adding domain-specific information have been explored in the literature.
%
A prominent approach is to implement forms of preferential reasoning
% , like reasoning wrt inclusion-minimal models, 
by directing choices through
a given partial order on literals~\cite{cacacale96a,rogima10a,giumar12a}.
%
To some degree, this can be simulated by heuristic modifiers like
\hpre{a}{\texttt{false}}{1}
that allow for computing a (single) inclusion-minimal model.
However, as detailed in \cite{rogima10a}, enumerating all such models needs additional constraints
or downstream tester programs.
Similarly,
\cite{balduccini11b} modifies the heuristic of the ASP solver \textit{smodels} to accommodate learning from smaller instances.
See also~\cite{falepf01a,falemari07a}.
Most notably,
\cite{rintanen12a} achieves impressive results in planning by equipping a SAT solver with
planning-specific heuristics.
%
All aforementioned approaches need customized changes to solver implementations.
%
Hence, it will be interesting to investigate how these approaches can be expressed and combined in
our declarative framework.
%
Declarative approaches to incorporating control knowledge can be found in heuristic planning.
For instance, \cite{backab00a} harness temporal logic formulas, while \cite{sierra04a} also uses
dedicated predicates for controlling backtracking in a forward planner.
%
However,
care must be taken when it comes to modifying a solver's heuristics.
Although it may lead to great improvements, it may just as well lead to a degradation of search.
In fact, the restriction of choice variables may result in exponentially larger search spaces~\cite{jajuni05a}.
This issue is reflected in our choice of heuristic modifiers, 
ranging from an \texttt{init}ial bias,
over a continued yet scalable one by \texttt{factor},
to a strict preference with \texttt{level}.

To sum up,
we introduced a declarative framework for incorporating domain-specific heuristics into ASP solving.
The seamless integration into ASP's input language provides us with a general and flexible tool for
expressing domain-specific heuristics.
As such, we believe it to be the first of its kind.
Our heuristic framework offers completely new possibilities of applying, experimenting, and studying
domain-specific heuristics in a uniform setting.
Our example heuristics merely provide first indications on the prospect of our approach,
but much more systematic empirical studies are needed to exploit its full power.


%%% Local Variables: 
%%% mode: latex
%%% TeX-master: "paper"
%%% End: 


%% \paragraph{Acknowledgments}

This work was partially funded by DFG grants SCHA~550/9 and~11,
as well as the Academy of Finland grant 251170.

%%% Local Variables: 
%%% mode: latex
%%% TeX-master: "paper"
%%% End: 

\smallskip\noindent\emph{Acknowledgments.}
%
This work was partly funded 
by 
% the German Science Foundation (DFG)
DFG grants 
SCHA 550/8-3 % clasp et al
and
SCHA 550/9-1.   % Inc/ReaASP
% SCHA 550/10-1  % Bio/clingcon

%%% Local Variables: 
%%% mode: latex
%%% TeX-master: "paper"
%%% End: 


\bibliographystyle{aaai}
% \bibliography{lit,akku,procs,local} % https://svn.cs.uni-potsdam.de/svn/reposWV/Papers/bibfiles/trunk
%% \begin{thebibliography}{}

\bibitem[\protect\citeauthoryear{Banbara, Gebser, Inoue, Ostrowski, Peano,
  Schaub, Soh, Tamura, and Weise}{Banbara
  et~al\mbox{.}}{2015}]{bageinospescsotawe15a}
{\sc Banbara, M.}, {\sc Gebser, M.}, {\sc Inoue, K.}, {\sc Ostrowski, M.}, {\sc
  Peano, A.}, {\sc Schaub, T.}, {\sc Soh, T.}, {\sc Tamura, N.}, {\sc and} {\sc
  Weise, M.} 2015.
\newblock aspartame: Solving constraint satisfaction problems with answer set
  programming.
\newblock In {\em Proceedings of the Thirteenth International Conference on
  Logic Programming and Nonmonotonic Reasoning (LPNMR'15)}, {F.~Calimeri},
  {G.~Ianni}, {and} {M.~Truszczy{\'n}ski}, Eds. Lecture Notes in Artificial
  Intelligence, vol. 9345. Springer-Verlag, 112--126.

\bibitem[\protect\citeauthoryear{Banbara, Kaufmann, Ostrowski, and
  Schaub}{Banbara et~al\mbox{.}}{2017}]{bakaossc16a}
{\sc Banbara, M.}, {\sc Kaufmann, B.}, {\sc Ostrowski, M.}, {\sc and} {\sc
  Schaub, T.} 2017.
\newblock Clingcon: The next generation.
\newblock {\em Theory and Practice of Logic Programming\/}.
\newblock To appear.

\bibitem[\protect\citeauthoryear{Baral}{Baral}{2003}]{baral02a}
{\sc Baral, C.} 2003.
\newblock {\em Knowledge Representation, Reasoning and Declarative Problem
  Solving}.
\newblock Cambridge University Press.

\bibitem[\protect\citeauthoryear{Barrett, Sebastiani, Seshia, and
  Tinelli}{Barrett et~al\mbox{.}}{2009}]{baseseti09a}
{\sc Barrett, C.}, {\sc Sebastiani, R.}, {\sc Seshia, S.}, {\sc and} {\sc
  Tinelli, C.} 2009.
\newblock Satisfiability modulo theories.
\newblock In {\em Handbook of Satisfiability}, {A.~Biere}, {M.~Heule}, {H.~{van
  Maaren}}, {and} {T.~Walsh}, Eds. Frontiers in Artificial Intelligence and
  Applications, vol. 185. IOS Press, Chapter~26, 825--885.

\bibitem[\protect\citeauthoryear{Bartholomew and Lee}{Bartholomew and
  Lee}{2014}]{barlee14b}
{\sc Bartholomew, M.} {\sc and} {\sc Lee, J.} 2014.
\newblock System aspmt2smt: Computing {ASPMT} theories by {SMT} solvers.
\newblock In {\em Proceedings of the Fourteenth European Conference on Logics
  in Artificial Intelligence (JELIA'14)}, {E.~Ferm\'{e}} {and} {J.~Leite}, Eds.
  Lecture Notes in Artificial Intelligence, vol. 8761. Springer-Verlag,
  529--542.

\bibitem[\protect\citeauthoryear{Cabalar, Otero, and Pose}{Cabalar
  et~al\mbox{.}}{2000}]{caotpo00a}
{\sc Cabalar, P.}, {\sc Otero, R.}, {\sc and} {\sc Pose, S.} 2000.
\newblock Temporal constraint networks in action.
\newblock In {\em Proceedings of the Fourteenth European Conference on
  Artificial Intelligence (ECAI'00)}, {W.~Horn}, Ed. IOS Press, 543--547.

\bibitem[\protect\citeauthoryear{Carro and King}{Carro and
  King}{2016}]{iclp-lipics16}
{\sc Carro, M.} {\sc and} {\sc King, A.}, Eds. 2016.
\newblock {\em Technical Communications of the Thirty-second International
  Conference on Logic Programming (ICLP'16)}. Vol.~52. Open Access Series in
  Informatics (OASIcs).

\bibitem[\protect\citeauthoryear{Cotton and Maler}{Cotton and
  Maler}{2006}]{cotmal06a}
{\sc Cotton, S.} {\sc and} {\sc Maler, O.} 2006.
\newblock Fast and flexible difference constraint propagation for {DPLL (T)}.
\newblock In {\em Proceedings of the Ninth International Conference on Theory
  and Applications of Satisfiability Testing (SAT'06)}, {A.~Biere} {and}
  {C.~Gomes}, Eds. Lecture Notes in Computer Science, vol. 4121.
  Springer-Verlag, 170--183.

\bibitem[\protect\citeauthoryear{Crawford and Baker}{Crawford and
  Baker}{1994}]{crabak94a}
{\sc Crawford, J.} {\sc and} {\sc Baker, A.} 1994.
\newblock Experimental results on the application of satisfiability algorithms
  to scheduling problems.
\newblock In {\em Proceedings of the Twelfth National Conference on Artificial
  Intelligence (AAAI'94)}, {B.~Hayes-Roth} {and} {R.~Korf}, Eds. AAAI Press,
  1092--1097.

\bibitem[\protect\citeauthoryear{Dantzig}{Dantzig}{1963}]{dantzig63a}
{\sc Dantzig, G.} 1963.
\newblock {\em Linear Programming and Extensions}.
\newblock Princeton University Press.

\bibitem[\protect\citeauthoryear{{De Rosis}, Eiter, Redl, and Ricca}{{De Rosis}
  et~al\mbox{.}}{}]{roeireri15a}
{\sc {De Rosis}, A.}, {\sc Eiter, T.}, {\sc Redl, C.}, {\sc and} {\sc Ricca,
  F.}
\newblock Constraint answer set programming based on {HEX}-programs.

\bibitem[\protect\citeauthoryear{Drescher and Walsh}{Drescher and
  Walsh}{2010}]{drewal10a}
{\sc Drescher, C.} {\sc and} {\sc Walsh, T.} 2010.
\newblock A translational approach to constraint answer set solving.
\newblock {\em Theory and Practice of Logic Programming\/}~{\em 10,\/}~4-6,
  465--480.

\bibitem[\protect\citeauthoryear{Gebser, Kaminski, Kaufmann, Ostrowski, Schaub,
  and Wanko}{Gebser et~al\mbox{.}}{2016}]{gekakaosscwa16a}
{\sc Gebser, M.}, {\sc Kaminski, R.}, {\sc Kaufmann, B.}, {\sc Ostrowski, M.},
  {\sc Schaub, T.}, {\sc and} {\sc Wanko, P.} 2016.
\newblock Theory solving made easy with clingo~5.
\newblock See \citeN{iclp-lipics16}, 2:1--2:15.

\bibitem[\protect\citeauthoryear{Gebser, Kaminski, Kaufmann, and Schaub}{Gebser
  et~al\mbox{.}}{2014}]{gekakasc14b}
{\sc Gebser, M.}, {\sc Kaminski, R.}, {\sc Kaufmann, B.}, {\sc and} {\sc
  Schaub, T.} 2014.
\newblock \textit{Clingo} = {ASP} + control: Preliminary report.
\newblock In {\em Technical Communications of the Thirtieth International
  Conference on Logic Programming (ICLP'14)}, {M.~Leuschel} {and}
  {T.~Schrijvers}, Eds. Theory and Practice of Logic Programming, Online
  Supplement, vol. arXiv:1405.3694v1.
\newblock Available at \url{http://arxiv.org/abs/1405.3694v1}.

\bibitem[\protect\citeauthoryear{Gebser, Kaufmann, and Schaub}{Gebser
  et~al\mbox{.}}{2012}]{gekasc09c}
{\sc Gebser, M.}, {\sc Kaufmann, B.}, {\sc and} {\sc Schaub, T.} 2012.
\newblock Conflict-driven answer set solving: From theory to practice.
\newblock {\em Artificial Intelligence\/}~{\em 187-188}, 52--89.

\bibitem[\protect\citeauthoryear{Gelfond and Lifschitz}{Gelfond and
  Lifschitz}{1991}]{gellif91a}
{\sc Gelfond, M.} {\sc and} {\sc Lifschitz, V.} 1991.
\newblock Classical negation in logic programs and disjunctive databases.
\newblock {\em New Generation Computing\/}~{\em 9}, 365--385.

\bibitem[\protect\citeauthoryear{Goldberg}{Goldberg}{1991}]{goldberg91a}
{\sc Goldberg, D.} 1991.
\newblock What every computer scientist should know about floating-point
  arithmetic.
\newblock {\em ACM Computing Surveys (CSUR)\/}~{\em 23,\/}~1, 5--48.

\bibitem[\protect\citeauthoryear{Janhunen, Liu, and Niemelä}{Janhunen
  et~al\mbox{.}}{2011}]{jalini11a}
{\sc Janhunen, T.}, {\sc Liu, G.}, {\sc and} {\sc Niemelä, I.} 2011.
\newblock Tight integration of non-ground answer set programming and
  satisfiability modulo theories.
\newblock In {\em Proceedings of the First Workshop on Grounding and
  Transformation for Theories with Variables (GTTV'11)}, {P.~Cabalar},
  {D.~Mitchell}, {D.~Pearce}, {and} {E.~Ternovska}, Eds. 1--13.

\bibitem[\protect\citeauthoryear{Lierler and Susman}{Lierler and
  Susman}{2016}]{liesus16a}
{\sc Lierler, Y.} {\sc and} {\sc Susman, B.} 2016.
\newblock {SMT}-based constraint answer set solver {EZSMT} (system
  description).
\newblock See \citeN{iclp-lipics16}, 1:1--1:15.

\bibitem[\protect\citeauthoryear{Liu, Janhunen, and Niemelä}{Liu
  et~al\mbox{.}}{2012}]{lijani12a}
{\sc Liu, G.}, {\sc Janhunen, T.}, {\sc and} {\sc Niemelä, I.} 2012.
\newblock Answer set programming via mixed integer programming.
\newblock In {\em Proceedings of the Thirteenth International Conference on
  Principles of Knowledge Representation and Reasoning (KR'12)}, {G.~Brewka},
  {T.~Eiter}, {and} {S.~McIlraith}, Eds. AAAI Press, 32--42.

\bibitem[\protect\citeauthoryear{Simons, Niemelä, and Soininen}{Simons
  et~al\mbox{.}}{2002}]{siniso02a}
{\sc Simons, P.}, {\sc Niemelä, I.}, {\sc and} {\sc Soininen, T.} 2002.
\newblock Extending and implementing the stable model semantics.
\newblock {\em Artificial Intelligence\/}~{\em 138,\/}~1-2, 181--234.

\bibitem[\protect\citeauthoryear{Soh, Inoue, Tamura, Banbara, and
  Nabeshima}{Soh et~al\mbox{.}}{2010}]{sointabana10a}
{\sc Soh, T.}, {\sc Inoue, K.}, {\sc Tamura, N.}, {\sc Banbara, M.}, {\sc and}
  {\sc Nabeshima, H.} 2010.
\newblock A {SAT}-based method for solving the two-dimensional strip packing
  problem.
\newblock {\em Fundamenta Informaticae\/}~{\em 102,\/}~3-4, 467--487.

\bibitem[\protect\citeauthoryear{Taillard}{Taillard}{1993}]{taillard93a}
{\sc Taillard, E.} 1993.
\newblock Benchmarks for basic scheduling problems.
\newblock {\em European Journal of Operational Research\/}~{\em 64,\/}~2,
  278--285.

\bibitem[\protect\citeauthoryear{van Loon}{van Loon}{1981}]{loon81}
{\sc van Loon, J.} 1981.
\newblock Irreducibly inconsistent systems of linear inequalities.
\newblock In {\em European Journal of Operational Research}. Vol.~3. Elsevier
  Science, 283--288.

\end{thebibliography}

%%% Local Variables: 
%%% mode: latex
%%% TeX-master: "paper"
%%% End: 

\begin{thebibliography}{}

\bibitem[\protect\citeauthoryear{Bacchus and Kabanza}{2000}]{backab00a}
Bacchus, F., and Kabanza, F.
\newblock 2000.
\newblock Using temporal logics to express search control knowledge for
  planning.
\newblock {\em Artificial Intelligence} 116(1-2):123--191.

\bibitem[\protect\citeauthoryear{Balduccini}{2011}]{balduccini11b}
Balduccini, M.
\newblock 2011.
\newblock Learning and using domain-specific heuristics in {ASP} solvers.
\newblock {\em AI Communic.} 24(2):147--164.

\bibitem[\protect\citeauthoryear{Baral}{2003}]{baral02a}
Baral, C.
\newblock 2003.
\newblock {\em Knowledge Representation, Reasoning and Declarative Problem
  Solving}.
\newblock Cambridge University Press.

\bibitem[\protect\citeauthoryear{Biere \bgroup et al\mbox.\egroup
  }{2009}]{SATHandbook}
Biere, A.; Heule, M.; {van Maaren}, H.; and Walsh, T., eds.
\newblock 2009.
\newblock {\em Handbook of Satisfiability}, volume 185 of {\em Frontiers in
  Artificial Intelligence and Applications}.
\newblock IOS Press.

\bibitem[\protect\citeauthoryear{Calimeri \bgroup et al\mbox.\egroup
  }{2011}]{contest11a}
Calimeri, F. et al.
\newblock 2011.
\newblock The third answer set programming competition: Preliminary report of
  the system competition track.
\newblock In Delgrande and Faber \shortcite{lpnmr11},  388--403.

\bibitem[\protect\citeauthoryear{Castell \bgroup et al\mbox.\egroup
  }{1996}]{cacacale96a}
Castell, T.; Cayrol, C.; Cayrol, M.; and {Le Berre}, D.
\newblock 1996.
\newblock Using the {D}avis and {P}utnam procedure for an efficient computation
  of preferred models.
\newblock In Wahlster, W., ed., {\em Proceedings of the Twelfth European
  Conference on Artificial Intelligence (ECAI'96)},  350--354.
\newblock John Wiley \& sons.

\bibitem[\protect\citeauthoryear{Dechter}{2003}]{dechter03}
Dechter, R.
\newblock 2003.
\newblock {\em Constraint Processing}.
\newblock Morgan Kaufmann.

\bibitem[\protect\citeauthoryear{Delgrande and Faber}{2011}]{lpnmr11}
Delgrande, J., and Faber, W., eds.
\newblock 2011.
\newblock {\em Proceedings of the Eleventh International Conference on Logic
  Programming and Nonmonotonic Reasoning (LPNMR'11)}. Springer.

\bibitem[\protect\citeauthoryear{{Di Rosa}, Giunchiglia, and
  Maratea}{2010}]{rogima10a}
{Di Rosa}, E.; Giunchiglia, E.; and Maratea, M.
\newblock 2010.
\newblock Solving satisfiability problems with preferences.
\newblock {\em Constraints} 15(4):485--515.

\bibitem[\protect\citeauthoryear{Faber \bgroup et al\mbox.\egroup
  }{2007}]{falemari07a}
Faber, W.; Leone, N.; Maratea, M.; and Ricca, F.
\newblock 2007.
\newblock Experimenting with look-back heuristics for hard {ASP} programs.
\newblock In Baral, C.; Brewka, G.; and Schlipf, J., eds., {\em Proceedings of
  the Ninth International Conference on Logic Programming and Nonmonotonic
  Reasoning (LPNMR'07)},  110--122.
\newblock Springer.

\bibitem[\protect\citeauthoryear{Faber, Leone, and Pfeifer}{2001}]{falepf01a}
Faber, W.; Leone, N.; and Pfeifer, G.
\newblock 2001.
\newblock Experimenting with heuristics for answer set programming.
\newblock In Nebel, B., ed., {\em Proceedings of the Seventeenth International
  Joint Conference on Artificial Intelligence (IJCAI'01)},  635--640.
\newblock Morgan Kaufmann.

\bibitem[\protect\citeauthoryear{Freeman}{1995}]{freeman95a}
Freeman, J.
\newblock 1995.
\newblock {\em Improvements to Propositional Satisfiability Search Algorithms}.
\newblock Ph.D. Dissertation, University of Pennsylvania.

\bibitem[\protect\citeauthoryear{Gebser \bgroup et al\mbox.\egroup
  }{2010}]{geguivscsithve10a}
Gebser, M.; Guziolowski, C.; Ivanchev, M.; Schaub, T.; Siegel, A.; Thiele, S.;
  and Veber, P.
\newblock 2010.
\newblock Repair and prediction (under inconsistency) in large biological
  networks with answer set programming.
\newblock In Lin, F., and Sattler, U., eds., {\em Proceedings of the Twelfth
  International Conference on Principles of Knowledge Representation and
  Reasoning (KR'10)},  497--507.
\newblock AAAI Press.

\bibitem[\protect\citeauthoryear{Gebser \bgroup et al\mbox.\egroup
  }{2011}]{gekaknsc11a}
Gebser, M.; Kaminski, R.; Knecht, M.; and Schaub, T.
\newblock 2011.
\newblock plasp: A prototype for {PDDL}-based planning in {ASP}.
\newblock In Delgrande and Faber \shortcite{lpnmr11},  358--363.

\bibitem[\protect\citeauthoryear{Gebser \bgroup et al\mbox.\egroup
  }{2012}]{gekakasc12a}
Gebser, M.; Kaminski, R.; Kaufmann, B.; Schaub, T.
\newblock 2012.
\newblock {\em Answer Set Solving in Practice}.
\newblock  Morgan and Claypool.
% Synthesis Lectures on Artificial Intelligence and Machine Learning.


\bibitem[\protect\citeauthoryear{Giunchiglia and Maratea}{2012}]{giumar12a}
Giunchiglia, E., and Maratea, M.
\newblock 2012.
\newblock Algorithms for solving satisfiability problems with qualitative
  preferences.
\newblock In Erdem, E.; Lee, J.; Lierler, Y.; and Pearce, D., eds., {\em
  Correct Reasoning: Essays on Logic-Based {AI} in Honour of {V}ladimir
  {L}ifschitz},
  327--344.
\newblock Springer.

\bibitem[\protect\citeauthoryear{Goldberg and Novikov}{2002}]{golnov02a}
Goldberg, E., and Novikov, Y.
\newblock 2002.
\newblock {BerkMin}: A fast and robust {SAT} solver.
\newblock In {\em Proceedings of the Fifth Conference on Design, Automation and
  Test in Europe (DATE'02)},  142--149.
\newblock IEEE Computer Society Press.

\bibitem[\protect\citeauthoryear{hclasp\ignorespaces}{\ignorespaces}]{hclasp}
\newblock hclasp. \texttt{http://www.cs.uni-potsdam.de/hclasp}.

\bibitem[\protect\citeauthoryear{ICAPS\ignorespaces}{\ignorespaces}]{icaps-competition}
\newblock ICAPS. \texttt{http://ipc.icaps-conference.org}.

\bibitem[\protect\citeauthoryear{J{\"a}rvisalo, Junttila, and
  Niemel{\"a}}{2005}]{jajuni05a}
J{\"a}rvisalo, M.; Junttila, T.; and Niemel{\"a}, I.
\newblock 2005.
\newblock Unrestricted vs restricted cut in a tableau method for {B}oolean
  circuits.
\newblock {\em Annals of Mathematics and Artificial Intelligence}
  44(4):373--399.

\bibitem[\protect\citeauthoryear{Korf}{1985}]{korf85a}
Korf, R.
\newblock 1985.
\newblock Depth-first iterative-deepening: An optimal admissible tree search.
\newblock {\em Artificial Intelligence} 27(1):97--109.

\bibitem[\protect\citeauthoryear{Lifschitz}{2002}]{lifschitz02a}
Lifschitz, V.
\newblock 2002.
\newblock Answer set programming and plan generation.
\newblock {\em Artificial Intelligence} 138(1-2):39--54.

\bibitem[\protect\citeauthoryear{Marques-Silva and Sakallah}{1999}]{marsak99a}
Marques-Silva, J., and Sakallah, K.
\newblock 1999.
\newblock {GRASP}: A search algorithm for propositional satisfiability.
\newblock {\em IEEE Transactions on Computers} 48(5):506--521.

\bibitem[\protect\citeauthoryear{Moskewicz \bgroup et al\mbox.\egroup
  }{2001}]{momazhzhma01a}
Moskewicz, M.; Madigan, C.; Zhao, Y.; Zhang, L.; and Malik, S.
\newblock 2001.
\newblock Chaff: Engineering an efficient {SAT} solver.
\newblock In {\em Proceedings of the Thirty-eighth Conference on Design
  Automation (DAC'01)},  530--535.
\newblock ACM Press.

\bibitem[\protect\citeauthoryear{Pipatsrisawat and Darwiche}{2007}]{pipdar07a}
Pipatsrisawat, K., and Darwiche, A.
\newblock 2007.
\newblock A lightweight component caching scheme for satisfiability solvers.
\newblock In Marques-Silva, J., and Sakallah, K., eds., {\em Proceedings of the
  Tenth International Conference on Theory and Applications of Satisfiability
  Testing (SAT'07)},
  294--299.
\newblock Springer.

\bibitem[\protect\citeauthoryear{Pretolani}{1996}]{pretolani96a}
Pretolani, D.
\newblock 1996.
\newblock Efficiency and stability of hypergraph {SAT} algorithms.
\newblock In Johnson, D., and Trick, M., eds., {\em DIMACS Series in Discrete
  Mathematics and Theoretical Computer Science}, volume~26,  479--498.
\newblock American Mathematical Society.

\bibitem[\protect\citeauthoryear{Rintanen}{2011}]{rintanen11a}
Rintanen, J.
\newblock 2011.
\newblock Planning with {SAT}, admissible heuristics and {A}$^*$.
\newblock In Walsh \shortcite{ijcai11},  2015--2020.

\bibitem[\protect\citeauthoryear{Rintanen}{2012}]{rintanen12a}
Rintanen, J.
\newblock 2012.
\newblock Planning as satisfiability: heuristics.
\newblock {\em Artificial Intelligence} 193:45--86.

\bibitem[\protect\citeauthoryear{Schaub and Thiele}{2009}]{schthi09a}
Schaub, T., and Thiele, S.
\newblock 2009.
\newblock Metabolic network expansion with {ASP}.
\newblock In Hill, P., and Warren, D., eds., {\em Proceedings of the
  Twenty-fifth International Conference on Logic Programming (ICLP'09)},  312--326.
\newblock Springer.

\bibitem[\protect\citeauthoryear{Siddiqi}{2011}]{sidiqqi11a}
Siddiqi, S.
\newblock 2011.
\newblock Computing minimum-cardinality diagnoses by model relaxation.
\newblock In Walsh \shortcite{ijcai11},  1087--1092.

\bibitem[\protect\citeauthoryear{Sierra-Santib{\'a}{\~n}ez}{2004}]{sierra04a}
Sierra-Santib{\'a}{\~n}ez, J.
\newblock 2004.
\newblock Heuristic planning: A declarative approach based on strategies for
  action selection.
\newblock {\em Artificial Intelligence} 153(1-2):307--337.

\bibitem[\protect\citeauthoryear{Walsh}{2011}]{ijcai11}
Walsh, T., ed.
\newblock 2011.
\newblock {\em Proceedings of the Twenty-second International Joint Conference
  on Artificial Intelligence (IJCAI'11)}. IJCAI/AAAI.

\bibitem[\protect\citeauthoryear{Zhang \bgroup et al\mbox.\egroup
  }{2001}]{zamamoma01a}
Zhang, L.; Madigan, C.; Moskewicz, M.; and Malik, S.
\newblock 2001.
\newblock Efficient conflict driven learning in a {B}oolean satisfiability
  solver.
\newblock In {\em Proceedings of the International Conference on Computer-Aided
  Design (ICCAD'01)},  279--285.
\newblock ACM Press.

\end{thebibliography}

%%% Local Variables: 
%%% mode: latex
%%% TeX-master: "paper"
%%% End: 

\end{document}
