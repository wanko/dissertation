
\section{Background}\label{sec:background}

We assume some basic familiarity with ASP, its semantics as well as its basic language constructs,
like normal rules, cardinality constraints, and optimization statements.
Although our examples are self-explanatory, we refer the reader for details to~\cite{gekakasc12a}.
%
For illustrating our approach, 
we consider selected rules of a simple planning encoding, following~\cite{lifschitz02a}.
We use predicates \texttt{action} and \texttt{fluent} to distinguish the corresponding entities.
The length of the plan is given by the constant \texttt{l}, which is used to fix all time points via
the statement \lstinline{time(1..l).}
Moreover, suppose our ASP encoding contains the rule
\begin{lstlisting}
1 { occ(A,T) : action(A) } 1 :- time(T).
\end{lstlisting}
stating that exactly one action occurs at each time step.
Also, it includes a frame axiom of the following form.%
\footnote{We use `\texttt{-}/1' to stand for classical negation.}
\begin{lstlisting}
holds(F,T) :- holds(F,T-1), not -holds(F,T).
\end{lstlisting}
In such a setting, actions and fluents are prime subjects to planning-specific heuristics.
As we show below, these can be elegantly expressed by heuristic statements about atoms formed from
predicates \texttt{occ} and \texttt{holds}, respectively.

For computing the stable models of a logic program, we use
a Boolean assignment, that is, a (partial) function mapping propositional variables in $\mathcal{A}$
to truth values \true\ and \false.
We represent such an assignment \ass\ as a set of signed literals of form $\Tsigned{a}$ or $\Fsigned{a}$,
standing for $a\mapsto\true$ and $a\mapsto\false$, respectively.
%
We access the true and false variables in \ass\ via
\(
\tlits{\ass}
=
\{a\in\mathcal{A}\mid\Tsigned{a}\in\ass\}
\)
and
\(
\flits{\ass}
=
\{a\in\mathcal{A}\mid\Fsigned{a}\in\ass\}
\), respectively.
%
\ass\ is conflicting, if $\tlits{\ass}\cap\flits{\ass}\neq\emptyset$;
\ass\ is total, if it is non-conflicting and $\tlits{\ass}\cup\flits{\ass}=\mathcal{A}$.
%
For generality, we represent Boolean constraints by \emph{nogoods}~\cite{dechter03}.
A nogood is a set $\{\sigma_1,\dots,\sigma_m\}$ of signed literals,
expressing that any assignment containing $\sigma_1,\dots,\sigma_m$ is inadmissible.
Accordingly,
a total assignment \ass\ is a \emph{solution} for a set~$\Delta$ of nogoods
if $\delta\not\subseteq\ass$ for all $\delta\in\Delta$.
%
While clauses can be directly mapped into nogoods,
logic programs are subject to a more involved translation.
For instance, 
an atom $a$ defined by two rules $a\,\texttt{:-}\,b,\naf{c}$ and $a\,\texttt{:-}\,d$
gives rise to three nogoods:
$\{\Tsigned{a},\Fsigned{x_{\{b,\naf{c}\}}},\Fsigned{x_{\{d\}}}\}$,
$\{\Fsigned{a},\Tsigned{x_{\{b,\naf{c}\}}}\}$, and
$\{\Fsigned{a},\Tsigned{x_{\{d\}}}\}$,
where $x_{\{b,\naf{c}\}}$ and $x_{\{d\}}$ are auxiliary variables for the bodies of the two previous rules.
Similarly, the body ${\{b,\naf{c}\}}$ leads to nogoods
$\{\Fsigned{x_{\{b,\naf{c}\}}},\Tsigned{b},\Fsigned{c}\}$,
$\{\Tsigned{x_{\{b,\naf{c}\}}},\Fsigned{b}\}$, and
$\{\Tsigned{x_{\{b,\naf{c}\}}},\Tsigned{c}\}$.
See~\cite{gekakasc12a} for full details.
%
Note that translating logic programs into nogoods adds auxiliary variables.
For simplicity, we restrict our formal elaboration to atoms in $\mathcal{A}$
(also because our approach leaves such internal variables unaffected anyway).

% Finally, we mention that
% we use \textit{max} to give the maximum value in a set and 
% \textit{argmax} to give the set of elements having the maximum value.


%%% Local Variables: 
%%% mode: latex
%%% TeX-master: "paper"
%%% End: 
