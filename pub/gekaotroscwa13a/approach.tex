
\section{Heuristic language elements}\label{sec:approach}

We express heuristic modifications via a set $\mathcal{H}$ of \emph{heuristic atoms} 
disjoint from $\mathcal{A}$.
Such a heuristic atom is formed from a dedicated predicate \hpredicate\ along with four arguments:
a (reified) atom $a\in\mathcal{A}$,
a heuristic modifier $m$,
and two  integers $v,p\in\mathbb{Z}$.
A heuristic modifier is used to manipulate the heuristic treatment of an atom $a$ via the 
modifier's value given by $v$.
The role of this value varies for each modifier.
We distinguish four primitive heuristic modifiers:
\begin{description}
\item [\texttt{init}] for initializing the heuristic value of $a$ with $v$,
\item [\texttt{factor}] for amplifying the heuristic value of $a$ by factor $v$,
\item [\texttt{level}] for ranking all atoms; the rank of $a$ is $v$,
\item [\texttt{sign}] for attributing the sign of $v$ as truth value to $a$.
\end{description}
While $v$ allows for changing an atom's heuristic behavior relative to \emph{other} atoms,
the second integer $p$ allows us to express a priority for disambiguating similar
heuristic modifications to the \emph{same} atom.
This is particularly important in our dynamic setting, where varying heuristic atoms may be obtained
in view of the current assignment.
For instance, the heuristic atoms
\hpred{b}{\mathtt{sign}}{1}{3}
and
\hpred{b}{\mathtt{sign}}{-1}{5}
aim at assigning opposite truth values to atom $b$.
This conflict can be resolved by preferring the heuristic modification with the higher priority,
viz.\ 5 in \hpred{b}{\mathtt{sign}}{-1}{5}.
Obviously such priorities can only support disambiguation but not resolve conflicting values sharing the same priority.

For accommodating priorities,
we define for an assignment \ass\ the \emph{preferred values} for modifier $m$ on atom $a$ as
\[
V_{a,m}(\ass)=
\textit{argmax}_{v\in\mathbb{Z}}\{p\mid\Tsigned{\hpred{a}{m}{v}{p}}\in\ass\}.
\]
Heuristic values are dynamic;
they are extracted from the current assignment and may thus vary during solving.
Note that $V_{a,m}(\ass)$ returns the singleton set $\{v\}$,
if the current assignment \ass\ contains a single true heuristic atom \hpred{a}{m}{v}{p} 
involving $a$ and $m$.
$V_{a,m}(\ass)$ is empty whenever there are no such heuristic atoms.
%
And whenever all heuristic atoms regarding $a$ and $m$ have the same priority $p$,
$V_{a,m}(\ass)$ is equivalent to
\(
\{v\mid\Tsigned{\hpred{a}{m}{v}{p}}\in\ass\}
\).

Here are a few examples.
We obtain % the preferred values
$V_{b,\mathtt{sign}}(\ass_1)=\{-1\}$
and
$V_{c,\mathtt{init}}(\ass_1)=\emptyset$
from assignment
\(
\ass_1=\{\Fsigned{a},\Tsigned{\hpred{b}{\mathtt{sign}}{1}{3}},\Tsigned{\hpred{b}{\mathtt{sign}}{-1}{5}}\}
\),
while assignment
\(
\ass_2=\{\Tsigned{\hpred{b}{\mathtt{sign}}{1}{3}},\Tsigned{\hpred{b}{\mathtt{sign}}{-1}{3}}\}
\)
results in $V_{b,\mathtt{sign}}(\ass_2)=\{1,-1\}$.

For ultimately resolving ambiguities among alternative values for heuristic modifiers, 
we propose for a set $V\subseteq\mathbb{Z}$ of integers the function $\nu(V)$ as
\[
%\nu(V)=
\mathit{max}\big(\{v\in V\!\mid v\geq 0\}\cup\{0\}\big)
+
\mathit{min}\big(\{v\in V\!\mid v\leq 0\}\cup\{0\}\big).
\]
Note that $\nu(\emptyset)=0$, attributing 0 the status of a neutral value.
Alternative options exist, like taking means or median of $V$ or even time specific criteria
relating to the emergence of values in the assignment.
%
In the above examples,
we get $\nu(V_{b,\mathtt{sign}}(\ass_1))=-1$ and $\nu(V_{b,\mathtt{sign}}(\ass_2))=0$.

Given this, we proceed by defining the \emph{domain-specific extension} $d$ to the heuristic function $h$ 
for $a\in\mathcal{A}$ as
\[
d_0(a)=\nu(V_{a,\mathtt{init}}(\ass_0))+h_0(a)
\]
and for $i\geq 1$
\[
d_i(a)=
\left\{
  \begin{array}{rl}
    \nu(V_{a,\mathtt{factor}}(\ass_i))\times h_i(a)&\text{if } V_{a,\mathtt{factor}}(\ass_i)\neq\emptyset
    \\
                                             h_i(a)&\text{otherwise}
  \end{array}
\right.
\]
First of all, it is important to note that $d$ is merely a modification and not a replacement of the
system heuristic $h$.
In fact, $d$ extends the range of $h$ to $(-\infty,+\infty)$.
Negative values serve as penalties.
The values of the \texttt{init} modifiers are added to $h_0$ in $d_0$.
The use of addition rather than multiplication allows us to override an initial value of 0.
Also, the higher the absolute value of the \texttt{init} modifier, the longer lasts its effect
(given the decay of heuristic values).
Unlike this, \texttt{factor} modifiers rely on multiplication because they aim at de- or increasing
conflict scores gathered during conflict analysis.
In view of $h$'s range,
a factor greater than 1 amplifies the score, a negative one penalizes the atom, and 0 resets the atom's score.
Enforcing a factor of 1 
% (for instance, through assigning a high priority)
transfers control back to the system heuristic $h$.

Heuristically modified logic programs are simply programs over $\mathcal{A}\cup\mathcal{H}$,
the original vocabulary extended by heuristic atoms (without restrictions).
As a first example,
let us extend our planning encoding by a rule favoring atoms expressing action occurrences close to
the goal situation.
\begin{lstlisting}
_h(occ(A,T),factor,T,0) :- action(A),time(T).
\end{lstlisting}
With \texttt{factor}, we impose a bias on the underlying heuristic function $h$.
Rather than comparing, for instance,
the plain values $h(\mathtt{occ(a,2)})$ and $h(\mathtt{occ(a,3)})$,
a decision is made by looking at $2\times h(\mathtt{occ(a,2)})$ and  $3\times h(\mathtt{occ(a,3)})$,
even though it still depends on $h$.
A further refined strategy may suggest considering climbing actions as early as possible.
\begin{lstlisting}
_h(occ(climb,T),factor,l-T,1) :- time(T).
\end{lstlisting}
Clearly, this rule conflicts with the more general rule above.
However, this conflict is resolved in favor of the more specific rule by attributing it a higher
priority (viz.~1 versus 0).

% Similar statements can be formulated with the \texttt{init} modifier in order to change the initial heuristic values.

For capturing a \emph{domain-specific extension} $t$ to the sign heuristic $s$,
we define for $a\in\mathcal{A}$ and $i\geq 0$:
\[
t_i(a)=
\left\{
  \begin{array}{rl}
    \true &\text{if }
           \nu(V_{a,\mathtt{sign}}(\ass_i))>0
           % \text{ and }
           % V_{a,\mathtt{sign}}(\ass_i)\neq\emptyset
           \\
    \false&\text{if }
           \nu(V_{a,\mathtt{sign}}(\ass_i))<0
           % \text{ and }
           % V_{a,\mathtt{sign}}(\ass_i)\neq\emptyset
           \\
    s_i(a)&\text{otherwise}
  \end{array}
\right.
\]
As with $d$ above, the extension $t$ to the sign heuristic is dynamic.
The sign of the modifier's preferred value determines the truth value to assign to an atom at hand.
No \texttt{sign} modifier (or enforcing a value of 0) leaves sign selection with the system's sign heuristic $s$.
%
For example, the heuristic rule
\begin{lstlisting}
_h(holds(F,T),sign,-1,0) :- fluent(F),time(T).
\end{lstlisting}
tells the solver to assign false to non-deterministically chosen fluents.
%
The next pair of rules is a further refinement of our strategy on climbing actions,
favoring their effective occurrence in the first half of the plan.
\begin{lstlisting}
_h(occ(climb,T),sign, 1,0) :- T<l/2,time(T).
_h(occ(climb,T),sign,-1,0) :- T>l/2,time(T).
\end{lstlisting}
Thus, while the atom $\mathtt{occ(climb,1)}$ is preferably made true,
false should rather be assigned to $\mathtt{occ(climb,l)}$.

Finally, for accommodating rankings induced by \texttt{level} modifiers,
we define for an assignment \ass\ and $\mathcal{A}'\subseteq\mathcal{A}$:
\[
\ell_\ass(\mathcal{A}')=\textit{argmax}_{a\in\mathcal{A}'}\nu(V_{a,\mathtt{level}}(\ass))
\]
The set $\ell_\ass(\mathcal{A}')$ gives all atoms in $\mathcal{A}'$ with the highest \texttt{level} values relative to
the current assignment \ass.
Similar to $d$ and $t$ above, this construction is also dynamic and the rank of atoms may vary during solving.
The function $\ell_\ass$ is then used to modify the selection of unassigned atoms in the above elaboration of \textit{decide}.
For this purpose, we replace Item~2 by
\(
U:=\ell_\ass(\mathcal{A}\setminus (\tlits{\ass}\cup\flits{\ass}))
\)
in order to restrict $U$ to unassigned atoms of (current) highest rank.
Unassigned atoms at lower levels are only considered once all atoms at higher levels have been assigned.
Atoms without an associated level default to level 0 because $\nu(\emptyset)=0$.
Hence, negative levels act as a penalty since the respective atoms are only taken into account
once all atoms with non-negative or no associated level have been assigned.

For a complementary example, 
consider a \texttt{level}-based formulation of the previous (\texttt{factor}-based) heuristic rule.
\begin{lstlisting}
_h(occ(A,T),level,T,0) :- action(A),time(T).
\end{lstlisting}
Unlike the above, $\mathtt{occ(a,2)}$ and $\mathtt{occ(a,3)}$ are now associated with different ranks,
which leads to strictly preferring $\mathtt{occ(a,3)}$ over $\mathtt{occ(a,2)}$ 
whenever both atoms are unassigned.
Hence, \texttt{level} modifiers partition the set of atoms and restrict $h$ to unassigned atoms at the highest level.

The previous replacement along with the above amendments of $h$ and $s$ through the domain-specific extensions $d$ and $t$
yields the following elaboration of CDCL's heuristic choice operation \textit{decide} for $i\geq 1$ (and given $d_0$).$^{\ref{fn:ass}}$
% --------------------------------------------------
\begin{enumerate}\addtocounter{enumi}{-1}\itemindent 10pt
\item $h_{i-1}(a) := d_{i-1}(a)$                         \hfill for each $a\in\mathcal{A}\qquad$
\item $h_i(a) := \alpha_i\times h_{i-1}(a) + \beta_i(a)$ \hfill for each $a\in\mathcal{A}\qquad$
\item $U:=\ell_{\ass_{i-1}}(\mathcal{A}\setminus (\tlits{\ass_{i-1}}\cup\flits{\ass_{i-1}}))$
\item $C:= \textit{argmax}_{a\in U}d_i(a)$
\item $a:= \tau(C)$
\item $\ass_i := \ass_{i-1}\cup\{t_i(a)a\}$
\end{enumerate}
% --------------------------------------------------
Although we formally model both $h$ and $d$ (as well as $s$ and $t$) as functions,
there is a substantial conceptual difference in practice in that $h$ is a system-specific data structure while $d$ is an associated method.
This is also reflected above, where $h$ is subject to assignments.
%
Item~0 makes sure that our heuristic modifications take part in the look-back based evolution in Item~1,
and are thus also subject to decay.
We added this as a separate line rather than integrating it into Item~1 in order to stress that our
modifications are modular in leaving the underlying heuristic machinery unaffected.
%
Item~2 gathers in $U$ all unassigned atoms of highest rank.
%
Among them, Item~3 collects in $C$ all atoms $a$ with a maximum heuristic value $d_i(a)$.
%
Since this is not guaranteed to yield a unique element, the system-specific tie-breaking function
$\tau$ is evoked to return a unique atom.
%
Finally, the modified sign heuristic $t_i$ determines a truth value for $a$, and the resulting
signed literal ${t_i(a)a}$ is added to the current assignment.

Note that so far all sample heuristic rules were \emph{static} in the sense that they are turned into
facts by the grounder and thus remain unchanged during solving.
Examples of dynamic heuristic rules are given at the end of next section.

Our simple heuristic language is easily extended by further heuristic atoms.
For instance, \hpred{a}{\mathtt{true}}{v}{p} and \hpred{a}{\mathtt{false}}{v}{p} have turned out to be useful in practice.
\begin{lstlisting}
_h(A,level,V,P) :- _h(A,true, V,P).
_h(A,sign, 1,P) :- _h(A,true, V,P).
_h(A,level,V,P) :- _h(A,false,V,P).
_h(A,sign,-1,P) :- _h(A,false,V,P).
\end{lstlisting}
%
For instance, the heuristic atom \hpred{a}{\mathtt{true}}{3}{3} expands to 
\hpred{a}{\mathtt{level}}{3}{3} and \hpred{a}{\mathtt{sign}}{1}{3},
expressing a preference for both making a decision on~$a$ and
assigning it to true.
On the other hand,
\hpred{a}{\mathtt{false}}{-3}{3} expands to 
\hpred{a}{\mathtt{level}}{-3}{3} and \hpred{a}{\mathtt{sign}}{-1}{3},
thus suggesting not to make a decision on~$a$ but to
assign it to false if there is no ``better'' decision variable.

Another shortcut of pragmatic value is the abstraction from specific priorities.
For this, we use the following rule.
\begin{lstlisting}
_h(A,M,V,#abs(V)) :- _h(A,M,V).
\end{lstlisting}
With it,
we can directly describe the heuristic restriction used in \cite{rintanen11a} to simulate planning
by iterated deepening $A^*$ \cite{korf85a} in SAT solving through limiting choices to action variables,
assigning those for time \texttt{T} before those for time \texttt{T+1}, and always assigning truth
value \texttt{true} (where \texttt{l} is a constant indicating the planning horizon):
\begin{lstlisting}
_h(occ(A,T),true,l-T) :- action(A), time(T).
\end{lstlisting}

Although we impose no restriction on the occurrence of heuristic atoms within logic programs,
it seems reasonable to require that the addition of rules containing heuristic atoms does not alter
the stable models of the original program.
That is, given a logic program $P$ over $\mathcal{A}$ and a set of rules $H$ over $\mathcal{A}\cup\mathcal{H}$,
we aim at a one-to-one correspondence between the stable models of $P$ and $P\cup H$ and
their identity upon projection on $\mathcal{A}$.
This property is guaranteed whenever heuristic atoms occur only in the head of rules and thus only
depend upon regular atoms.
In fact, so far, this class of rules turned out to be expressive enough to model all heuristics of interest,
including the ones presented in this paper.
It remains future work to see whether more sophisticated schemes, eg., involving recursion, are useful.

%%% Local Variables: 
%%% mode: latex
%%% TeX-master: "paper"
%%% End: 
