
\section{Discussion}\label{sec:discussion}

Various ways of adding domain-specific information have been explored in the literature.
%
A prominent approach is to implement forms of preferential reasoning
% , like reasoning wrt inclusion-minimal models, 
by directing choices through
a given partial order on literals~\cite{cacacale96a,rogima10a,giumar12a}.
%
To some degree, this can be simulated by heuristic modifiers like
\hpre{a}{\texttt{false}}{1}
that allow for computing a (single) inclusion-minimal model.
However, as detailed in \cite{rogima10a}, enumerating all such models needs additional constraints
or downstream tester programs.
Similarly,
\cite{balduccini11b} modifies the heuristic of the ASP solver \textit{smodels} to accommodate learning from smaller instances.
See also~\cite{falepf01a,falemari07a}.
Most notably,
\cite{rintanen12a} achieves impressive results in planning by equipping a SAT solver with
planning-specific heuristics.
%
All aforementioned approaches need customized changes to solver implementations.
%
Hence, it will be interesting to investigate how these approaches can be expressed and combined in
our declarative framework.
%
Declarative approaches to incorporating control knowledge can be found in heuristic planning.
For instance, \cite{backab00a} harness temporal logic formulas, while \cite{sierra04a} also uses
dedicated predicates for controlling backtracking in a forward planner.
%
However,
care must be taken when it comes to modifying a solver's heuristics.
Although it may lead to great improvements, it may just as well lead to a degradation of search.
In fact, the restriction of choice variables may result in exponentially larger search spaces~\cite{jajuni05a}.
This issue is reflected in our choice of heuristic modifiers, 
ranging from an \texttt{init}ial bias,
over a continued yet scalable one by \texttt{factor},
to a strict preference with \texttt{level}.

To sum up,
we introduced a declarative framework for incorporating domain-specific heuristics into ASP solving.
The seamless integration into ASP's input language provides us with a general and flexible tool for
expressing domain-specific heuristics.
As such, we believe it to be the first of its kind.
Our heuristic framework offers completely new possibilities of applying, experimenting, and studying
domain-specific heuristics in a uniform setting.
Our example heuristics merely provide first indications on the prospect of our approach,
but much more systematic empirical studies are needed to exploit its full power.


%%% Local Variables: 
%%% mode: latex
%%% TeX-master: "paper"
%%% End: 
