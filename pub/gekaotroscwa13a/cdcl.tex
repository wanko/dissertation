
\section{Conflict-driven constraint learning}\label{sec:cdcl}

Given that we are primarily interested in the heuristic machinery of a solver,
we only provide a high-level description of the basic decision algorithm for 
conflict-driven constraint learning (CDCL;~\cite{marsak99a,zamamoma01a})
in Figure~\ref{algo:cdcl}.
%
% ----------------------------------------------------------------------
\begin{figure}[t]
\newcommand{\ITEMHACK}{\itemindent=-5pt\itemsep=0pt\parsep=\itemsep}
\small
\hrule\vspace{2pt}
\noindent\textbf{loop}\\[-12pt]
  \begin{itemize}\ITEMHACK
  \item [] \textit{propagate}  
    \hfill// compute deterministic consequences
  \item [] \textbf{if} no conflict \textbf{then}
    \begin{itemize}\ITEMHACK
    \item [] \textbf{if} all variables assigned 
      \textbf{then} 
      \textbf{return} variable assignment
    \item [] \textbf{else}
      \textit{decide} 
      \hfill// non-deterministically assign some literal
    \end{itemize}
  \item [] \textbf{else} 
    \begin{itemize}\ITEMHACK
    \item [] \textbf{if} top-level conflict %found 
      \textbf{then} 
      \textbf{return} unsatisfiable
    \item [] \textbf{else}
      \begin{itemize}\ITEMHACK
      \item [] \textit{analyze}\hfill// analyze conflict and add a conflict constraint
      \item [] \textit{backjump}\hfill// undo assignments until conflict constraint is unit
      \end{itemize}
    \end{itemize}
  \end{itemize}
  \hrule  
  \caption{Basic decision algorithm: CDCL}
  \label{algo:cdcl}
\end{figure}
% ----------------------------------------------------------------------
CDCL starts by extending a (partial) {assignment} by deterministic (unit) propagation.
% Importantly, every derived literal is ``implied'' by some {nogood}
% % (set of literals that must not jointly be assigned), 
% which would be violated if the literal's complement were assigned.
Although propagation aims at forgoing nogood violations,
assigning a literal implied by one nogood may lead to the violation of another nogood;
this situation is called \emph{conflict}.
If the conflict can be resolved, % (the violated nogood contains backtrackable literals),
it is analyzed to identify a conflict constraint.
The latter represents a ``hidden'' conflict reason that is recorded and
guides backjumping to an earlier stage such that
the complement of some formerly assigned literal is implied by the conflict constraint,
thus triggering propagation.
Only when propagation finishes without conflict,
a (heuristically chosen) literal can be assigned % at a new \emph{decision level},
provided that the assignment at hand is partial,
while a {solution} % (total assignment not violating any nogood)
has been found otherwise.
% The eventual termination of CDCL is guaranteed
% by either returning a solution or encountering an unresolvable conflict
% (independent of (non-implied) decision literals).
%
See~\cite{SATHandbook} for details.

A characteristic feature of CDCL is its look-back based approach.
Central to this are conflict-driven mechanisms scoring variables according to their prior conflict involvement.
These scores guide heuristic choices regarding literal selection as well as constraint learning and deletion.

A decision heuristic is used to implement the non-deterministic assignment done via
\emph{decide} in the CDCL algorithm in Figure~\ref{algo:cdcl}.
In fact,
the selection of an atom along with its sign relies on two such functions:
\[
h: \mathcal{A}\to[0,+\infty)
\quad\text{and}\quad
s: \mathcal{A}\to\{\true,\false\}
\ .
\]
Both functions vary over time.
To capture this, we use $h_i$ and $s_i$ to denote the specific mappings in the $i$th iteration of CDCL's main loop.
Analogously, we use $\ass_i$ to represent the $i$th assignment (after \textit{propagation}).
We use $i=0$ to refer to the initialization of both functions via $h_0$ and $s_0$;
similarly, $A_0$ gives the initial assignment (after \textit{propagation}).

The following lines give a more detailed yet still high-level account of the non-deterministic assignment done by
\emph{decide} in the CDCL algorithm for $i\geq 1$ (and a given $h_0$):%
\footnote{\label{fn:ass}For clarity, we keep using indexes in this algorithmic setting although this is
  unnecessary in view of assignment operator `$:=$'.}
% --------------------------------------------------
\begin{enumerate}\itemindent 10pt
\item $h_i(a) := \alpha_i\times h_{i-1}(a) + \beta_i(a)$ \hfill for each $a\in\mathcal{A}\qquad$
\item $U:=\mathcal{A}\setminus (\tlits{\ass_{i-1}}\cup\flits{\ass_{i-1}})$
\item $C:= \textit{argmax}_{a\in U}h_i(a)$
\item $a:= \tau(C)$
\item $\ass_i := \ass_{i-1}\cup\{s_i(a)a\}$
\end{enumerate}
% --------------------------------------------------
The first line describes the development of the heuristic depending on a global decay parameter
$\alpha_i$ and a variable-specific scoring function $\beta_i$.
The set $U$ contains all atoms unassigned at step~$i$.
Among them, the ones with the highest heuristic value are collected in $C$.
Whenever $C$ contains several (equally scored) variables,
the solver must break the tie by selecting one atom $\tau(C)$ from $C$.
% Such tie-breaking is usually done by adhoc mechanisms.

Look-back based heuristics rely on information gathered during conflict analysis in CDCL.
Starting from some initial heuristic values in $h_0$,
the heuristic function is continued as in Item~1 above,
where
$\alpha_i\in{[0,1]}$ is a global parameter decaying the influence of past values
and
$\beta_i(a)$ gives the conflict score attributed to variable $a$ within conflict analysis.
The value of $\beta_i(a)$ can be thought of being 0 unless $a$ was scored by \textit{analyze} in CDCL.
Similarly, $\alpha_i$ usually equals 1 unless it was lowered at some system-specific point, 
such as after a \emph{restart}.
%
Occurrence-based heuristics like~\textit{moms}~\cite{pretolani96a} furnish initial heuristics.
%
Prominent look-back heuristics are
\textit{berkmin}~\cite{golnov02a}
and
\textit{vsids}~\cite{momazhzhma01a}.
% and \textit{vmtf}~\cite{ryan04a}.

For illustration,
let us look at a rough trace of atoms $a$, $b$, and $c$ in a fictive run of the CDCL algorithm.
\[
\begin{array}{|r|c|c@{\ }@{\ }c@{\,}@{\,}c@{\,}r@{\,}|c@{\,}@{\,}c@{\,}@{\,}c@{\,}r@{\,}|c|c|c@{\,}@{\,}c@{\,}@{\,}c@{\,}r@{\,}|}
  \hline
  i&\mathit{operation}&
  \multicolumn{4}{|c|}{\ass}&
  \multicolumn{4}{ c|}{h}&
                       s &
                       \alpha &
  \multicolumn{4}{ c|}{\beta}\\
  \hline
   &                  &a     &b    &c&...&a&b&c   &...&     &  &a&b&c&...\\
  \hline
  \hline
  0&                  &      &     & &   &0&1&1   &   &\true&1 &0&0&0&   \\
  \hline
  1&\mathit{propagate}&\false&     & &   &0&1&1   &   &\true&1 &0&0&0&   \\
   &\mathit{decide}   &\false&\true& &   &0&1&1   &   &\true&1 &0&0&0&   
\end{array}
\]
The initial heuristic $h_0$ prefers $b$, $c$ over $a$;
the sign heuristic $s$ constantly assigns \true.
Initial propagation assigns \false\ to $a$.
This leaves all heuristics unaffected.
When invoking \textit{decide}, we find $b$ and $c$ among the unassigned variables in $U$
(in Item~2 above).
Assuming the maximum value of $h_1$ to be 1, both are added to $C$.
This tie is broken by selecting $\tau(C)=b$ in $C$.
Given that the (constant) sign heuristic yields \true,
Item~5 adds signed literal \Tsigned{b} to the current assignment.

Next suppose we encounter a conflict involving $c$ at step~8.
This leads to an incrementation of $\beta_8(c)$.
\[
\begin{array}{|r|c|c@{\,}@{\,}c@{\,}@{\,}c@{\,}r@{\,}|c@{\,}@{\,}c@{\,}@{\,}c@{\,}r@{\,}|c|c|c@{\,}@{\,}c@{\,}@{\,}c@{\,}r@{\,}|}
   &                  &a     &b    &c     &...&a&b&c   &...&     &  &a&b&c&...\\
  \hline
  \hline
  8&\mathit{propagate}&\false&\true&\false&   &0&2&2   &   &\true&1 &0&0&0&   \\
   &\mathit{analyze}  &\false&\true&\false&   &0&2&2   &   &\true&1 &0&0&1&   \\
   &\mathit{backjump} &\false&     &      &   &0&2&3   &   &\true&1 &0&0&0&   \\
  \hline
  9&\mathit{propagate}&\false&     &      &   &0&2&3   &   &\true&1 &0&0&0&   \\
   &\mathit{decide}   &\false&     &\true &   &0&2&3   &   &\true&1 &0&0&0&
\end{array}
\]
As at step~1, $b$ and $c$ are unassigned after backjumping.
Unlike above, $c$ is now heuristically preferred to $b$ since it occurred more frequently within conflicts.

Without going into detail, 
we mention that at certain steps $i$,
parameter $\alpha_i$ is decreased for decaying the values of $h_i$ and
the conflict scores in $\beta_i$ are re-set (eg.~after \textit{analyze}).

Also, look-back based sign heuristics take advantage of previous information.
The common approach is to choose the polarity of a literal according to the higher
number of occurrences in recorded nogoods~\cite{momazhzhma01a}.
Another effective approach is \emph{progress saving}~\cite{pipdar07a},
caching truth values of (certain) retracted variables and reusing them for sign selection.

Although we focus on look-back heuristics,
we mention that look-ahead heuristics 
% are primarily used in DPLL-based solvers~\cite{davput60,dalolo62a}.
% They 
aim at shrinking the search space by selecting the (signed) variable offering most implications.
This approach relies on failed-literal detection~\cite{freeman95a} for counting the number of
propagations obtained by (temporarily) adding in turn the variable and its negation to the current
assignment.
This count can be used in Item~1 above for computing the values $\beta_i(a)$,
while all $\alpha_i$ are set to 0 (because no past information is taken into account).
% For instance, such an approach is used in \smodels~\cite{siniso02a} to select both the variable as
% well as its sign.

%%% Local Variables: 
%%% mode: latex
%%% TeX-master: "paper"
%%% End: 
