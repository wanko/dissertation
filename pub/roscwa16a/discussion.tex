
\section{Discussion}\label{sec:discussion}

We presented a comprehensive framework for computing diverse (or similar) solutions to logic programs with generic preferences
and implemented it in \asprin~2, available at~\cite{asprin}. %\comment{T: Make it available!}
To this end, we introduced a spectrum of different methods, among them, generalizations of existing work to the case of
programs with general preferences.
Hence, certain fragments of our framework provide implementations of the proposals in \cite{eiererfi13a,zhutru13a}.
While the latter had to resort to solver wrappers or even internal solver modifications,
\asprin\ heavily relies upon multi-shot solving that allows for an easy yet fine-grained control of reasoning processes.
Moreover, we provided several generic building blocks, such as 
\textit{maxmin} (and \textit{minmax}) preferences,
query-answering for programs with preferences,
preferences among optimal models,
and an automated approach for the guess and check methodology of~\cite{eitpol06a},
all of which are also of interest beyond diversification.
%
Finally, we took advantage of the uniform setting offered by \asprin~2 to conduct a 
comparative empirical analysis of the various methods for diversification.
Generally speaking,
there is a clear trade-off between performance and diversification quality, 
which allows for selecting the most appropriate method 
depending on the hardness of the application at hand.
%\comment{T: Last phrase is a bit weak\dots}
% \begin{itemize}
% \item System for diverse optimal models of logic program with preferences.
% \item Lifts previous approaches in ASP \cite{eiererfi13a} or with answer set optimization \cite{zhutru13a} to general preferences of \asprin.
% \item Variety of methods.
% \item Experimental evaluation shows that different methods differ wrt performance and diversity of solutions.
% \item There is a tradeoff between performance and diversity, which allows for selecting the most appropiate method at hand for the application at hand.
% \item Furthermore, four more general contributions to preferences: $maxmin$, automation of generate and test, query solving, and preferences over
% optimal models.
% \item Future work: Real world applications on DSS and TT.
% \item Future work: Apply SMT framework of \clingo\ 5. In \cite{eiererfi13a}, the modification of the solver was more efficient than ASP implementation.
%       We intend to do it in a principled (general?) way inside the new framework.
% \end{itemize}



%%% Local Variables: 
%%% mode: latex
%%% TeX-master: "paper"
%%% End: 
