
\emph{Approximation.}
%
We describe the different implementations of the procedure 
$\mathit{solve}(P,\mathcal{X})$ outlined in Section \ref{sec:overview}. 

%
In Algorithm~\Alabel{1}, $\mathit{solve}(P,\mathcal{X})$ solves the
\emph{Most Distant Optimal Model} problem given the optimal stable models in $\mathcal{X}$,
applying the solution 
%with preferences over optimal models described in Section \ref{sec:basic}. %preferences}.
described at the end of Section \ref{sec:basic}. %preferences}.
%
%In algorithm A1 $\mathit{solve}(P,\mathcal{X})$ 
%returns an optimal model of $P$ most dissimilar to those in $\mathcal{X}$.
%For this, solutions in $\mathcal{X}$ are reified into facts \lstinline!holds(a,i)! as for the Enumeration method,
%and atoms in $P$ are reified via \lstinline!holds/1! predicate.
%Given \lstinline!n! previous optimal solutions, 
%the following preference statement establishes the preference:
%\begin{lstlisting}[numbers=none]
%sol(1..n).
%#preference(p,maxmin) { 
%  I,1,X : holds(X), not holds(X,I), sol(I); 
%  I,1,X : not holds(X), holds(X,I), sol(I)  }.
%\end{lstlisting}

In Algorithm~\Alabel{2}, $\mathit{solve}(P,\mathcal{X})$ first computes a partial interpretation $I$ 
distant to $\mathcal{X}$ in one of the following ways:
\begin{enumerate}
\item 
A random one (named $\mathit{rd}$).
\item 
A heuristically chosen one, following the $\mathit{pguide}$ heuristic from \cite{nadel11a} ($pg$):
for an atom $a$, $a$ is added to $I$ if it is true in $\mathcal{X}$ more often than false, 
$\neg a$ is added in the opposite case, and nothing happens if there is a tie.
%if an atom is true if among the solutions in $\mathcal{X}$ it is $false$ more times than $true$, 
%and it is false in the opposite case. In case of a tie, it does not appear in $I$.
\item 
One most distant to the solutions in $\mathcal{X}$ ($\mathit{dist}$), 
computed applying the solution to the \emph{Most Distant Model} problem described at the beginning of Section \ref{sec:basic}, %\ref{sec:maxmin},  
where the program $P$ is `\lstinline[mathescape=true]!$\{${holds($a$)}.$ \mid a \in \mathcal{A}\}$!'.
\item 
One complementary to the last computed optimal model $L$ 
taking into account either 
true ($\{ \neg a \ | \ a \in L\}$), 
false ($\{ a \ | \ a \notin L\}$), 
or both types of atoms ($\{ \neg a \ | \ a \in L\} \cup \{ a \ | \ a \notin L\}$). 
They are named $\mathit{true}$, $\mathit{false}$ and $\mathit{all}$, respectively.
\end{enumerate}
%
For selecting an optimal model closest to $I$, 
the technique is similar to the one for the \emph{Most Distant Optimal Model} problem:
we start with the logic program $P$ with preference statement $s$, 
and we add the rules 
\lm{$\{\;$holds($a$,0) :- $a$.${}\mid a \in \mathcal{A}\}$} reifying the atoms of $P$, 
the facts 
\lm{$\{\;$holds($a$,1).${}\mid a \in I\cap\mathcal{A}\}\cup\{\;$nholds($a$,1).${}\mid \neg a \in I, a \in \mathcal{A} \}$} reifying $I$,  
and define the following preference statement: 
\begin{lstlisting}
#preference(partial,less(cardinality)) {
  holds(X,0), nholds(X,1); not holds(X,0), holds(X,1) }.
\end{lstlisting}
Finally, we compute the optimal models of this program wrt $s$ and then \lm{partial}
using the method for preferences over optimal models described in Section \ref{sec:basic}. %\ref{sec:preferences}.

In~\Alabel{3},
we select a distant solution $I$ as we do for \Alabel{2}, 
and we add the same reifying rules, 
along with the following heuristic rules for approximating an optimal model of $P$ close to $I$:
\begin{lstlisting}
#heuristic hold(X,0) :  holds(X,1). [  1, sign ]
#heuristic hold(X,0) : nholds(X,1). [ -1, sign ]
\end{lstlisting}
For prioritizing the variables in $I$, we add another two heuristic rules like the previous ones, 
but replace both \lm{[ 1, sign ]} and \lm{[ -1, sign ]} by \lm{[ 1, level ]}, respectively.
%
%These basic heuristics can be further refined by adding 
%a dynamic heuristic that, when the current assignment is very close to a previous solution, 
%modifies the signs given by the heuristic to get away from that previous solution:
%\begin{lstlisting}
%#heuristic X.
%\end{lstlisting}

%%% Local Variables: 
%%% mode: latex
%%% TeX-master: "paper"
%%% End: 
