
\emph{Enumeration.}
%
With this technique, we first enumerate all optimal stable models of $P$ with \asprin\
and afterwards we find, among all those stable models, the $n$ most diverse.
%This method may be exponential in space, given that we may have to store 
%an exponential number of optimal stable models.
%
For the initial step, we use \asprin{}'s enumeration algorithm (see~\cite{brderosc15a}).
For the second, let $\mathcal{X}=\{ X_1, \ldots, X_m \}$ be the set of $m$ optimal stable models of $P$.
Then, the following encoding along with the facts $H_\mathcal{X}$ reifying $\mathcal{X}$ 
provides a correct and complete solution to the \emph{n Most Diverse Optimal Models} problem:
\begin{lstlisting}[mathescape=true,numbers=none]
$n$ { sol(1..$m$) } $n$.
#preference(enumeration,maxmin) { 
  (I,J),1,X : holds(X,I), not holds(X,J), sol(I), sol(J), I < J; 
  (I,J),1,X : not holds(X,I), holds(X,J), sol(I), sol(J), I < J;
  (I,J),#sup,0 : sol(I), not sol(J), I < J ;
  (I,J),#sup,0 : not sol(I), sol(J), I < J }.
#optimize(enumeration).  
\end{lstlisting}
The choice rule guesses \lstinline!n! solutions among \lstinline!m! in $\mathcal{X}$, 
and the \lstinline!enumeration! preference statement selects the optimal ones.
In \lstinline!enumeration!, there is a sum for every pair \lstinline!(I,J)! with \lstinline!I < J!.
If both \lstinline!I! and \lstinline!J! are chosen (first two preference elements) 
then the sum represents their actual distance.
In the other case (last two elements) the sum has the maximum possible value  in \asprin\
(viz.~\lstinline!#sup!).
This allows for comparing only sums of pairs \lstinline!(I,J)! of selected solutions.

\emph{Replication.}
%
With this technique  \asprin\ begins translating a normal logic program with preferences $P$ 
into a disjunctive logic program $D$ applying the \gc\ method.
%
Next, $D$ is reified onto $\mathcal{R}(D)$ and combined with a meta-encoding $\mathcal{M}_n$ 
replicating $D$:%
\footnote{The actual encoding handles the whole \clingo\ language~\cite{gekakasc14b} and is more involved.}
\begin{lstlisting}[mathescape=true]
sol(1..$n$).
holds(A,S) : head(R,A) :- rule(R); sol(S); holds(A,S) : body(R,pos,A);
                                       not holds(A,S) : body(R,neg,A).
\end{lstlisting}
%
The stable models of $\mathcal{M}_n \cup \mathcal{R}(D)$ correspond one to one 
to the elements of $\mathit{Opt}(P)^n$, 
where $\mathit{Opt}(P)$ stands for the set of all optimal models of $P$.
%
Further rules are added for having exactly one stable model for every set of $n$ optimal stable models, 
but we do not detail them here for space reasons.
%
Finally, adding the following preference and optimize statements results  
in a correct and complete solution to the \emph{n Most Diverse Optimal Models} problem:
\begin{lstlisting}
#preference(replication,maxmin) { 
  (I,J),1,X : hold(A,I), not hold(A,J), sol(I), sol(J), I < J ; 
  (I,J),1,X : not hold(A,I), hold(A,J), sol(I), sol(J), I < J }.
#optimize(replication).
\end{lstlisting}
The preference statement is similar to the one for Enumeration, 
but now the $n$ solutions are generated by the meta-encoding, 
and all of them are used for calculating the sums.

%%% Local Variables: 
%%% mode: latex
%%% TeX-master: "paper"
%%% End: 
