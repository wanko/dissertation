%
%In this section first we define and implement the preference type \textit{maxmin} (and \textit{minmax}), 
%and we show how it can be used for \emph{representing} the \emph{Distant Model} problem in \asprin.
%Next, we specify the method for solving guess and check problems, 
%and explain how it can be used for \emph{solving} the \emph{Distant Model} problem.
%The \emph{Distant Optimal Model} problem appears\comment{JR: rephrase} in our approximation algorithms, 
%For solving it we developed different methods, that we specify here, for solving queries over optimal models, 
%that are applied in the extension of \asprin\ for handling preferences over optimal models. 

We first show how the \emph{Most Distant (Optimal) Model} problem can be represented in \asprin\ 
using the new preference type \textit{maxmin}: % (and \textit{minmax}):
Given a logic program $P$ (with preferences) over $\mathcal{A}$, 
and a set $\mathcal{X}=\{ X_1, \ldots, X_m \}$ of (optimal) stable models of $P$, 
find an (optimal) stable model of $P$ that 
maximizes the minimum distance to the (optimal) stable models in $\mathcal{X}$.
%\comment{JR: Rephrase}
%
The \emph{Most Distant (Optimal) Model} is used by our approximation algorithms in Section~\ref{sec:advanced}.

\emph{Maxmin optimization in \asprin.}
%
%Let \lstinline[mathescape=true]!$D_\mathcal{A}=\{$domain($a$).$\mid a \in \mathcal{A}\}$! and 
Let $H_\mathcal{X}$ be the set of facts \lstinline[mathescape=true]!$\{$holds($a$,$i$).$\mid a \in X_i, X_i\in\mathcal{X} \}$!
reifying the stable models in $\mathcal{X}$,  
and let \lstinline!distance! be the following preference statement:
%
\begin{lstlisting}[mathescape=true]
#preference(distance,maxmin){
  I,1,X : holds(X,0), not holds(X,I), I=1..$m$;
  I,1,X : not holds(X,0), holds(X,I), I=1..$m$ }.
\end{lstlisting}
%
Then, the \emph{Most Distant Model} problem is solved by the following program with preferences:
\lstinline[mathescape=true]!$P \cup \{ $holds($a$,0) :- $a$. $\mid a \in \mathcal{A}\} \cup H_\mathcal{X} \cup \; \{$distance$\} \; \cup \{$#optimize(distance).$\}$!.
%
$P$ generates stable models that are reified with \lm{holds($a$,0)} for $a \in \mathcal{A}$. 
%
The preference statement \lstinline!distance! represents a \lstinline!maxmin! preference over $m$ sums, 
where the value of each sum \lstinline!I! (with \lstinline[mathescape=true]!I=1..$m$!)
amounts to the distance between the generated stable model and $X_\mathtt{I}$.
%
Finally, the optimize statement selects the optimal stable models wrt $\succ_\mathtt{distance}$.

Formally, the preference elements of preference type \textit{maxmin} have the restricted form
`\lm{s,w,t:B}'
%\[ s,w,t : B \]
where \lm{s}, \lm{w}, \lm{t} are terms, and \lm{B} is a condition.
%
Term \lm{s} names different sums,
whose value is specified by the rest of the element `\lm{w,t:B}'
(similar to aggregate elements).
%
%As usual, preference elements stand for their ground instantiations,
%and we define the semantics of \textit{maxmin} for a set $E$ of ground preference elements. 
For defining the semantics of \textit{maxmin}, 
preference elements stand for their ground instantiations, 
and we consider a set $E$ of such ground preference elements.
%
We say that \lm{s} is (the name of) a sum of $E$ if it is the first term of some preference element.
Given a stable model $X$ and a sum \lm{s} of $E$, the value of \lm{s} in $X$ is:
\[\textstyle
v(\mathtt{s},X)=\sum_{(\mathtt{w},\mathtt{t})\in\{\mathtt{w},\mathtt{t}\mid \mathtt{s,w,t:B}\in E, X\models \mathtt{B}\}}\mathtt{w}
\]
For a set $E$ of ground preference elements for preference statement \lm{p}, 
\textit{maxmin} defines the following preference relation:%
\footnote{For defining \textit{minmax}, we simply switch $\min$ by $\max$, and $>$ by $<$.}
\[
X \succ_\mathtt{p} Y \text{ if } \min \{ v(\mathtt{s},X) \mid \mathtt{s} \ \text{is a sum of } E \} > \min \{ v(\mathtt{s},Y) \mid \mathtt{s} \ \text{is a sum of } E \} 
\]
Applying this definition to the preference statement \lm{distance} gives the partial order $\succ_\mathtt{distance}$.
%Note that when there is only one sum, 
%\textit{maxmin} reduces to maximization of a sum (and minimization for \textit{minmax}).

%Now let's see how is $\succ_\mathtt{distance}$ specified in \asprin\ using the implementation of \textit{maxmin}.
%Now let's see how $\succ_\mathtt{distance}$ and \textit{maxmin} are implemented. 
%and how is it used to specify $\succ_\mathtt{distance}$.
%
In \asprin, partial orders $\succ$ are implemented by so-called \emph{preference programs}. 
%
For our example, we say that $Q$ is a \emph{preference program} for $\succ_\mathtt{distance}$ if it holds that
$X \succ_\mathtt{distance} Y$ iff $Q \cup \Holds{X} \cup \Holdsp{Y}$ is satisfiable, 
where \lm{$\Holds{X}=\{ $holds($a$). $\mid a \in X\}$}
and   $\Holdsp{Y}$\lm{$=\{ $holds'($a$). $\mid a \in Y\}$}.
%
In practice, the preference program $Q$ consists of three parts.

% \begin{itemize}
First, % \item
each preference statement is translated into a set of facts, and added to $Q$.
Our example preference statement \lm{distance} results in
\lm{preference(distance,maxmin)} 
and the instantiations of 
\lm{preference(distance,1,(I,X),for(t$_\mathtt{1}$),(I,1,X))} 
and 
\lm{preference(distance,2,(I,X),} \\ \lm{for(t$_\mathtt{2}$),(I,1,X))}
where \lm{t$_\mathtt{1}$} and \lm{t$_\mathtt{2}$} are terms 
standing for the conditions of the two non-ground preference elements.

Second, % \item 
$Q$ contains the implementation of the preference type $\mathtt{p}$,
consisting of rules defining an atom \lstinline[mathescape]!better($\mathtt{p}$)! 
that indicates whether $X\succ_{\mathtt{p}}Y$ holds for two stable models $X,Y$.
The sets $X$ and $Y$ are provided by \asprin\ in reified form via unary predicates \lstinline!holds! and \lstinline!holds'!\!.%
\footnote{That is, \lstinline[mathescape]!holds($\mathtt{a}$)!\! (or \lstinline!holds'!\!\!\lstinline[mathescape]!($\mathtt{a}$)!\!) is true iff $\mathtt{a}\!\in\!X$ (or $\mathtt{a}\!\in\!Y$).}
%
Further rules are added by \asprin\ to define \lstinline!holds! and \lstinline!holds'! for
the conditions appearing in the preference statement (\lm{t$_\mathtt{1}$} and \lm{t$_\mathtt{2}$} in our example).
%
The definition of \lstinline[mathescape]!better($\mathtt{p}$)! then draws upon the instances of both predicates for deciding $X\succ_{\mathtt{p}}Y$.
%
For the new preference type $\mathtt{maxmin}$ (being now part of \asprin\ 2's library),
we get the following rules:
%
\begin{lstlisting}
#program preference(maxmin).
sum(P,S) :- preference(P,maxmin), preference(P,_,_,_,(S,_,_)).

value(P,S,V) :- preference(P,maxmin), sum(P,S), 
  V = #sum { W,T : holds'(X), preference(P,_,_,for(X),(S,W,T)).

minvalue(P,V) :- preference(P,maxmin), V = #min { W : value(P,S,W) }.

better(P,S) :- preference(P,maxmin), sum(P,S), minvalue(P,V),
  V < #sum { W,T : holds(X), preference(P,_,_,for(X),(S,W,T)).

better(P) :- preference(P,maxmin), better(P,S) : sum(P,S).
\end{lstlisting}
Predicate \lm{sum/2} stores the sums \lm{S} of the preference statement \lm{P},  
while \lstinline!value/3! collects the value \lm{V} of every sum for the stable model $Y$, 
and \lstinline!minvalue/2! stores the minimum of them.
In the end, \lstinline{better(P)} is obtained if \lstinline{better(P,S)} holds for all sums \lm{S}, 
and this is the case whenever the value of the sums for the stable model $X$ 
is greater than the minimum value for the stable model $Y$.

Third, % \item
the constraint
`\lm{:- not better(distance).}'
% \begin{lstlisting}
% :- not better(distance).
% \end{lstlisting}
is added to $Q$, % by \asprin\ after reading the \lm{#optimize} statement: 
enforcing that 
the set of rules is satisfiable iff \lm{better(p)} is obtained, 
which is the case whenever $X \succ_\mathtt{distance} Y$. 
%\end{itemize}

We can show that for any preference statement \lm{p} of type \lm{maxmin}, 
the union of the above three sets of rules constitutes a preference program for $\succ_\mathtt{p}$.

%We note that this preference could also be implemented in \clingo\ using $\#minimize$ statements, 
%but a naive implementation would lead to large groundings:
%\begin{lstlisting}
%group(G) :- preference(P,maxmin), preference(P,_,_,_,(G,_,_)).
%dist(X)  :- group(G), 
%  X = #sum { W,T : holds(X), preference(P,_,_,for(X),(G,W,T))}.
% min(X)  :- X = #min { Y: dist(Y) }.
%#maximize { X : min(X) }.
%\end{lstlisting}
%Without going into details, note that the rule defining \lstinline!dist/1! would have a ground instance
%for every possible value of the sum.
%We leave as future work the investigation of other possible more compact encodings.
%
%
%%%%Predicate \texttt{valueh'(P,G,V)} collects the value \texttt{V} of group \texttt{G} in the stable model $\mathcal{H}'$, 
%%%%and \texttt{valueh'(P,V)} takes the minimum among these values. 
%%%\begin{itemize}
%%%\item
%%%The naive implementation of this preference in \clingo\ via $\#minimize$ statements leads to large groundings:
%%%\begin{lstlisting}
%%%group(G) :- preference(P,maxmin), preference(P,_,_,_,(G,_,_)).
%%%dist(X)  :- group(G), X = #sum { W,T : holds(X), preference(P,_,_,for(X),(G,W,T))}.
%%% min(X)  :- X = #min { Y: dist(Y) }.
%%%#maximize { X : min(X) }.
%%%\end{lstlisting}
%%%Let us assume that the \textit{maxmin} preference is specified directly via facts of predicates \texttt{preference/2} and \texttt{preference/5}, 
%%%and that \texttt{holds/1} is defined as in \asprin.
%%%Then the first line defines the groups of the \textit{maxmin} preference (with predicate \texttt{group/1}), 
%%%the second line stablishes when the distance to any group is \texttt{X} (with predicate \texttt{dist/1}), 
%%%and this is used in the third line to determine the minimum distance of any of the groups (with predicate \texttt{min/1}), 
%%%which is finally maximized in the last line.
%%%The large grounding is due to the second rule, 
%%%leading to a ground rule for every group 
%%%and for every possible value of the distance to that group.
%%%In the longer version of this paper we investigate other possible encodings, 
%%%and compare them with the \asprin\ implementation.
%%%\end{itemize}
%
%\emph{Maxmin Stable Models}:
%given a set $\mathcal{X}=\{ X_1, \ldots, X_m \}$ of stable models,
%find a subset $\mathcal{Y} \subseteq \mathcal{X}$ of size $n$ 
%such that the minimum distance among the stable models of $\mathcal{Y}$ is maximized.
%\begin{lstlisting}[mathescape=true]
%$n$ { sol(1..$m$) } $n$.
%#preference(p,maxmin) { 
%  (I,J),1,X : holds(X,I), not holds(X,J), sol(I), sol(J), I < J; 
%  (I,J),1,X : not holds(X,I), holds(X,J), sol(I), sol(J), I < J 
%  (I,J),#sup,X : not sol(I), sol(J), I=1..$m$, I < J; 
%  (I,J),#sup,X : sol(I), not sol(J), J=1..$m$, I < J; 
%}.
%#optimize(p).
%\end{lstlisting}
%The optimal stable models of $P \cup H_\mathcal{X}$ correspond to 
%the solutions of the \emph{Maxmin Stable Models} problem.
%While the choice rule guesses a set of \lstinline!n! solutions from $\mathcal{X}$,
%the preference statement \lstinline!p! selects the optimal ones.

%%% Local Variables: 
%%% mode: latex
%%% TeX-master: "paper"
%%% End: 
