\section{Supplementary complexity results} \label{sec:complexity}

\emph{$\Sigma^p_2$-hardness for queries over normal logic programs with preferences.}
%  
Let $P$ be a normal logic program with preferences, and \lm{q} an atom. 
%
We prove that the query problem of deciding whether 
there is an optimal stable model of $P$ that contains \lm{q} is $\Sigma^p_2$-hard.
%
We do this by a reduction from the problem of deciding 
the existence of a stable model of a disjunctive logic program $D$.
%
For this, consider the reified facts $\mathcal{R}(D)$ and the following meta-encoding $\mathcal{M}_D$:
\begin{lstlisting}
atom(A) :- head(_,A). atom(A) :- body(_,_,A).

{  holds(A) ; nholds(A) } :- atom(A).

:- not holds(A) : head(R,A) ;  holds(A) : body(R,pos,A); 
                              nholds(A) : body(R,neg,A).

nquery :-     holds(A),     nholds(A).
nquery :- not holds(A), not nholds(A), atom(A).
 query :- not nquery.

#preference(min_holds,subset) {
  holds(A)  : atom(A);
  nholds(A) : atom(A); not nholds(A) : atom(A)
}.
#optimize(min_holds).
\end{lstlisting}
It turns out that the disjunctive logic program $D$ has a stable model iff 
there is an optimal stable model of the normal logic program with preferences 
$\mathcal{R}(D) \cup \mathcal{M}_D$ that contains \lm{query}.
%
Intuitively, in an optimal stable model with \lm{query}, 
the \lm{holds/1} and \lm{nholds/1} atoms
represent the true and false atoms of a stable model of $D$, respectively.
%
Then the constraint rule guarantees that the stable model is a classical model of $D$, 
and the preference \lm{min_holds} enforces the minimality condition.

\emph{$F\Delta^p_3$-hardness of solving preferences over optimal models.}
%  
Let $P$ be a normal logic program and let $s$ and $t$ be two preference statements.
%
We prove that the problem of finding an optimal model of $P$ wrt $s$ and then $t$ is $F\Delta^p_3$-hard.
%
We do this by a reduction from the problem of finding 
an optimal stable model of a disjunctive logic program $D$ with weight minimization.
%
For this, consider the reified facts $\mathcal{R}(D)$
(where the pairs of weights and atoms to minimize $\langle w, a \rangle$ are reified by \lm{minimize(w,a)}) 
and the following meta-encoding $\mathcal{M}_O$:
\begin{lstlisting}
#preference(query,superset) { query }.

#preference(minimize,less(weight)) { W,A : holds(A), minimize(W,A) }.

#preference(overall,lexico) { 1 : name(query); 2 : name(minimize) }.

\end{lstlisting}
First, observe that the optimal stable models of 
$\mathcal{R}(D) \cup \mathcal{M}_D \cup \mathcal{M}_O \cup \{ \mathtt{\#reoptimize(query).} \}$
correspond to the (possibly non optimal) stable models of $D$  if \lm{query} belongs to them\footnote{Note
that either \lm{query} belongs to all of them, or it belongs to none.}.
%
If \lm{query} does not belong to them, 
then $D$ has no stable model\footnote{
Note that it cannot be the case that the program with preferences has no stable models, 
since making all \lm{holds/1} atoms true always leads to a stable model.}.
%
Replacing the reoptimize statement by `$\mathtt{\#reoptimize(overall)}$',
if \lm{query} belonged to the previous optimal models (which where stable models of $D$), 
then those models are now optimized at the second level wrt \lm{minimize}, 
and therefore the selected ones are also optimal models of $D$.
In the other case, we have again that $D$ has no stable model.
%
Summarizing, the optimal stable models of 
$\mathcal{R}(D) \cup \mathcal{M}_D \cup \mathcal{M}_O \cup \{ \mathtt{\#reoptimize(overall).} \}$
correspond to the optimal stable models of $D$ if \lm{query} belongs to them, 
and if not, $D$ has no stable model.


