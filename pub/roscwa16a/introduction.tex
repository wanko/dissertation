
\section{Introduction}\label{sec:introduction}
 
Answer Set Programming (ASP; \cite{baral02a}) has become a prime paradigm for solving combinatorial problems in Knowledge Representation and Reasoning. 
As a matter of fact,
such problems have an exponential number of solutions in the worst-case.
A first means to counterbalance this is to impose preference relations among solutions
to filter out optimal ones.
Often enough, this still leaves us with a large number of optimal models.
%
A typical example is the computation of Pareto frontiers for multi-objective optimization problems~\cite{pareto64a},
as we encounter in design space exploration~\cite{angeglharesc13a} or timetabling~\cite{basotainsc13a}.
%
Other examples include configuration, planning, and phylogeny, as discussed in~\cite{eiererfi13a}.
% 
This calls for computational support that allows for identifying small subsets of diverse solutions.
%
The computation of diverse stable models was first considered in ASP by \cite{eiererfi13a}.
The analogous problem regarding optimal stable models is addressed in \cite{zhutru13a} in the case of answer set optimization~\cite{brnitr03a}.
Beyond ASP, the computation of diverse solutions is also studied in CP~\cite{hehnocwa05a} and SAT~\cite{nadel11a}.

In this paper,
we introduce a comprehensive framework for computing diverse (or similar) solutions to logic programs with preferences.
One of its distinguishing factors is that it allows for dealing with aggregated (or plain) qualitative and quantitative preferences among stable models of logic programs.
This is accomplished by building on the preference handling capacities of \asprin~\cite{brderosc15a}.
The other appealing factor of our framework is that it covers a wide spectrum of methods for diversification.
Apart from new techniques, it also accommodates and generalizes existing approaches by lifting them to programs with preferences.
Interestingly, this is done by taking advantage of several existing basic ASP techniques that we automate and integrate in our framework.
The enabling factor of this is the recent advance of multi-shot ASP solving that allows for an easy yet fine-grained control of
ASP-based reasoning processes (cf.~\cite{gekakasc14b}).
In particular, this abolishes the need for internal solver modifications or singular solver wrappers that were often unavoidable in previous approaches.
We have implemented our approach as an extension to the preference handling framework \asprin.
The resulting system \asprin~2 is then used for an empirical evaluation contrasting several alternative approaches to
computing diverse solutions.
%
Last but not least,
note that although we concentrate on diversity,
our approach applies just as well to the dual concept of \emph{similarity}.
This is also reflected by its implementation supporting both settings.

%%% Local Variables: 
%%% mode: latex
%%% TeX-master: "paper"
%%% End: 
