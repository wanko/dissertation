
\section{Background}\label{sec:background}
 
In ASP, problems are described as (disjunctive) \emph{logic programs}, 
being sets of \emph{rules} of the form
\begin{lstlisting}[mathescape=true,numbers=none]
   a$_1$;...;a$_m$ :- a$_{m+1}$,...,a$_n$,not a$_{n+1}$,...,not a$_o$
\end{lstlisting}
where each \lstinline[mathescape=true]{a$_i$} is a propositional atom %for ${0}\leq{i}\leq{n}$
and
\lstinline[mathescape=true]{not} stands for \emph{default negation}.
%
We call a rule a \emph{fact} if $m=o=1$, 
\emph{normal} if $m=1$, and 
an \emph{integrity constraint} if $m=0$.
%
We may reify a rule $r$ with the set of facts 
\lm{
$R(r) = 
\{ $rule($r$)$ \} \cup 
\{ $head($r$,a$_i$)$ \mid 1 \leq i \leq m \} \cup
\{ $body($r$,pos,a$_i$)$ \mid m+1 \leq i \leq n \} \cup$}
\lm{$\{ $body($r$,neg,a$_i$)$ \mid n+1 \leq i \leq o \}$
}, and we reify 
a program by joining its reified rules.
%
Semantically, a logic program induces a collection of \emph{stable models},
which are distinguished models of the program determined by stable models semantics;
see \cite{gellif91a} for details.
 
To ease the use of ASP in practice, 
several extensions have been developed. 
First of all, rules with variables are viewed as shorthands for the set of their ground instances.
Further language constructs include
\emph{conditional literals} and \emph{cardinality constraints} \cite{siniso02a}.
The former are of the form
\lstinline[mathescape=true]{a:b$_1$,...,b$_m$},
the latter can be written as
\lstinline[mathescape=true]+s{c$_1$;...;c$_n$}t+,
where \lstinline{a} and \lstinline[mathescape=true]{b$_i$} are possibly default-negated literals  % for $0\leq i\leq m$,
and each \lstinline[mathescape=true]{c$_j$} is a conditional literal; % for $1\leq i\leq n$;
\lstinline{s} and \lstinline{t} provide lower and upper bounds on the number of satisfied literals in the cardinality constraint.
We refer to \lstinline[mathescape=true]{b$_1$,...,b$_m$} as a \emph{condition}.
%
The practical value of both constructs becomes apparent when used with variables.
For instance, a conditional literal like
\lstinline[mathescape=true]{a(X):b(X)}
in a rule's antecedent expands to the conjunction of all instances of \lstinline{a(X)} for which the corresponding instance of \lstinline{b(X)} holds.
%
Similarly,
\lstinline[mathescape=true]+2{a(X):b(X)}4+
is true whenever at least two and at most four instances of \lstinline{a(X)} (subject to \lstinline{b(X)}) are true.
%
%%%Finally, objective functions minimizing the sum of weights $w_j$ of conditional literals $c_j$ are expressed as
%%%\lstinline[mathescape=true]!#minimize{$w_1$:$c_1$,$\dots$,$w_n$:$c_n$}!.
%
Specifically, 
we rely in the sequel
on the input language of the ASP system \clingo~\cite{gekakasc14b};
further language constructs are explained on the fly.
 
In what follows, we go beyond plain ASP and deal with \emph{logic programs with preferences}.
%
More precisely,
we consider programs $P$ over some set $\mathcal{A}$ of atoms
along with a strict partial order ${\succ}\subseteq{\mathcal{A}\times\mathcal{A}} $ among their stable models. 
%and we restrict ourselves to order relations decidable in polynomial time.
Given two stable models $X,Y$ of $P$,
$X\succ Y$ means that $X$ is preferred to $Y$.
%
Then, a stable model $X$ of $P$ is \emph{optimal} wrt $\succ$,
if there is no other stable model $Y$ such that $Y\succ X$.
%
In what follows,
we often leave the concrete order implicit and simply refer to a program with preferences and its optimal stable models.
% \comment{T: Needed\dots?}
% Note that an empty order yields all stable models of a program.
% Hence, our contributions also apply to this base case without further mention.
%
We restrict ourselves to partial orders and distance measures (among pairs of stable models) that can be computed in polynomial time.
%
For simplicity, we focus on the Hamming distance, 
defined for two stable models $X,Y$ of a program $P$ over $\mathcal{A}$ as
\(
d(X,Y)
=
|(\mathcal{A} - X) - Y| + | X \cap Y |
\).
%
Given a logic program $P$ with preferences and a positive integer $n$,
we define % (similar to~\cite{eiererfi13a}) %in defining
a set $\mathcal{X}$ of optimal stable models of $P$ as \emph{most diverse},
if
\(
\min \{\, d(X,Y) \mid X, Y \in \mathcal{X},  X \neq Y \} 
> 
\min \{\, d(X,Y) \mid X, Y \in \mathcal{X}', X \neq Y \}
\)
for every other set $\mathcal{X}'$ of optimal stable models of $P$.
%
We are thus interested, following \cite{eiererfi13a}, in the problem \emph{n Most Diverse Optimal Models}:
%
Given a logic program $P$ with preferences and a positive integer $n$, 
find $n$ most diverse optimal stable models of $P$.

For representing logic programs with complex preferences and computing their optimal models,
we built upon the preference framework of \asprin~\cite{brderosc15a},
a system for dealing with aggregated qualitative and quantitative preferences.
%
In \asprin, the above mentioned \emph{preference relations} are represented by declarations of the form
\lstinline[mathescape]!#preference($\mathtt{p}$,$\mathtt{t}$){$\mathtt{t_1}$:$\mathtt{b_1}$,$\dots$,$\mathtt{t_n}$:$\mathtt{b_n}$}!
where $\mathtt{p}$ and $\mathtt{t}$ are the name and type of the preference relation, %respectively, 
and $\mathtt{t_i}$ and $\mathtt{b_i}$ are tuples of terms and conditions, respectively,\footnote{
See~\cite{brderosc15a} %and Section~\ref{sec:maxmin} 
for more general preference elements.}
%each $\mathtt{c_j}$ is a conditional literal
serving as arguments of $\mathtt{p}$.
%
The directive \lstinline[mathescape]!#optimize($\mathtt{p}$)! instructs \asprin\ to search for stable models that are optimal wrt the strict partial order
$\succ_{\mathtt{p}}$ associated with $\mathtt{p}$.
%
While \asprin\ already comes with a library of predefined primitive and aggregate preference types, like \texttt{subset} or \texttt{pareto},
respectively,
it also allows for adding customized preferences.
%
We illustrate this by implementing preference type \texttt{maxmin} in Section~\ref{sec:basic}.
%To this end,
%users provide rules defining an atom \lstinline[mathescape]!better($\mathtt{p}$)! that indicates whether $X\succ_{\mathtt{p}}Y$ holds 
%for two stable models $X,Y$.
%
%The sets $X$ and $Y$ are provided by \asprin\ in reified form via unary predicates \lstinline!holds! and \lstinline!holds'!\!.%
%\footnote{That is, \lstinline[mathescape]!holds($\mathtt{a}$)!\! (or \lstinline!holds'!\!\!\lstinline[mathescape]!($\mathtt{a}$)!\!) is true iff $\mathtt{a}\!\in\!X$ (or $\mathtt{a}\!\in\!Y$).}
%%
%The definition of \lstinline[mathescape]!better($\mathtt{p}$)! then draws upon the instances of both predicates for deciding $X\succ_{\mathtt{p}}Y$. 
%%
%Note that this delineates the expressiveness of preference handling in \asprin.
%Note that for representing order relations decidable in polynomial time, stratified rules suffice.

Finally,
we investigate whether the heuristic capacities of \clingo\ allow for boosting our approach.
%
In fact, \clingo~5 features heuristic directives of the form
`\lstinline[mathescape]!#heuristic c. [$k$,$m$]!'
where $c$ is a conditional atom,
$k$ is a term evaluating to an integer, and 
$m$ is a heuristic modifier among
\lstinline{init}, \lstinline{factor}, \lstinline{level},  or \lstinline{sign}. %, respectively.
%
The effect of the heuristic modifiers is to bias the score of \clasp's heuristic by
initially adding or multiplying the score,
prioritizing variables, or
preferably assigning a truth value, respectively.
%
The value of $k$ serves as argument to the respective modification.
%
%Modifiers \lstinline{true} and \lstinline{false} combine \lstinline{level} with \lstinline{sign} selection, respectively.
%
A more detailed description can be found in~\cite{gekaotroscwa13a}.

%%% Local Variables: 
%%% mode: latex
%%% TeX-master: "paper"
%%% End: 
