
\emph{Automatic Guess and Check in \clingo.}
%
Given a logic program $P$ over $\mathcal{A}$, and a preference statement $s$ with preference program $Q_s$, 
the optimal models of $P$ wrt $\succ_s$ correspond to the solutions of the \gc\ problem
$\langle P \cup R_\mathcal{A}',P \cup R_\mathcal{A} \cup Q_s\rangle$, 
where $R_X'$ stands for \lm{$\{$ holds'($a$) :- $\ a$. ${}\mid a \in X \}$}, 
and $R_X$ for \lm{$\{$ holds($a$) :- $\ a$. ${}\mid a \in X \}$} given some set $X$.%
\footnote{To avoid the conflict between the atoms of $P$ appearing in both the guesser and the checker, 
given a model $X$ of $P \cup R_{\mathcal{A}}'$, only the atoms of predicate \lm{holds'/1} in $X$ are passed to the checker.
In the system $\mathit{metagnc}$ this is declared via directive `\lstinline!#guess holds'/1.!'}
%
The guess program generates stable models $X$ of $P$ reified with \lstinline!holds'/1!,
while the check program looks for models better than $X$ wrt $s$ reified with \lstinline!holds/1!,
so that $X$ is optimal whenever the checker along with the \lstinline!holds'/1! atoms of $X$ becomes unsatisfiable.
%
This correspondence is the basis of a method for computing optimal models in \asprin, 
where the logic program with preferences is translated into a \gc\ problem, 
that $\mathit{metagnc}$ translates into a disjunctive logic program,
which is then solved by \clingo.
%
This allows, for example, for solving the \emph{Most Distant Model} problem using the logic program 
\lstinline[mathescape=true]!$P \cup \{ $holds($a$,0) :- $a$. $\mid a \in \mathcal{A}\} \cup H_\mathcal{X}$! 
and the preference statement \lstinline!distance!,
with the corresponding preference program comprising the three sets of rules described before.

%
In general, the G{\&}C framework \cite{eitpol06a} 
allows for representing $\Sigma^p_2$ problems in ASP, 
and solving them using the \textit{saturation} technique by Eiter and Gottlob in~\cite{eitgot95a}.
%
The idea is to re-express the problem as a positive disjunctive logic program, containing a special-purpose atom \lm{bot}.
%
Whenever \lm{bot} is obtained, saturation derives all atoms (belonging to a ``guessed'' model).
%
Intuitively, this is a way to materialize unsatisfiability.
%
We automatize this process in $\mathit{metagnc}$ by building on the meta-interpretation-based approach of~\cite{gekasc11b}.
%
For a \gc\ problem $\langle G, C \rangle$ over $\langle \mathcal{A}_G,\mathcal{A}_C\rangle$,
the idea is to reify the program
\lstinline[mathescape=true]!$C \cup \{ ${$a$}.$ \mid a \in \mathcal{A}_G \}$!
into the set of facts
\lstinline[mathescape=true]!$\mathcal{R}(C \cup \{ ${$a$}.$ \mid a \in \mathcal{A}_G \})$!.
%\lm{$\mathcal{R}(C \cup \{ ${$a$}.$ \mid a \in \mathcal{A}_G \})$}. 
%%%\footnote{This is done via option \texttt{--reify} in \textit{gringo}~3 (\textit{gringo}~4 implementation is forthcoming).}
%
The latter are combined with the meta-encoding $\mathcal{M}$ from~\cite{gekasc11b} implementing saturation.
%%%\footnote{The specific encoding, \texttt{metaD.lp}, is available at \url{http://www.cs.uni-potsdam.de/metasp}.}
%    
This leads to the positive disjunctive logic program:
\begin{lstlisting}[mathescape=true]
$\phantom{G}$$\mathcal{R}\big(C \cup \{ ${$a$}.$ \mid a \in \mathcal{A}_G \} \big) \cup\mathcal{M}$
\end{lstlisting}
%\begin{lstlisting}[mathescape=true]
%$\mathcal{R}\big(C \cup \{ ${$a$}.$ \mid a \in \mathcal{A}_G \} \big) \cup\mathcal{M}$
%\end{lstlisting}
This program has a stable model (excluding \lm{bot}) for each $X \subseteq \mathcal{A}_G$ such that $C \cup X$ is satisfiable, 
and it has a saturated stable model (including \lm{bot}) if there is no such $X$.
%This is part of the meta-encoding that works when $C$ is tight:
%\begin{lstlisting}
%true(A) ; false(A) :- atom(A).
%
%supported(A)   :- head(R,A); true(A)  : body(R,pos,A); 
%                             false(A) : body(R,neg,A).
%unsupported(R) :- rule(R);   true(A)  ; body(R,neg,A).
%unsupported(R) :- rule(R);   false(A) ; body(R,pos,A).
%unsupported(A) :- unsupported(R) : head(R,A).
%
%bot :- true(A), unsupported(A). true(A)  :- bot, atom(A).
%bot :- false(A),  supported(A). false(A) :- bot, atom(A).
%\end{lstlisting}
%For the case where $C$ has loops, further rules have to be added (see \cite{gekasc11b} for details).
%
%For representing a solution to the \gc\ problem, 
Next, we just have to add the generator program $G$, 
map the true and false atoms of $G$ to their counterparts in the positive disjunctive logic program 
(represented by predicates \lm{true/1} and \lm{false/1}, respectively),
and enforce the atom \lm{bot} to hold:
\begin{lstlisting}[mathescape=true]
$\phantom{G }$$\mathcal{R}\big(C \cup \{ ${$a$}.$ \mid a \in \mathcal{A}_G \} \big) \cup\mathcal{M} \; \cup $
$\phantom{G    \mathcal{R}\big( }$$G \cup \{ $true($a$) :- $a$.$ \mid a \in \mathcal{A}_G \} \cup \{ $false($a$) :- not $a$.$ \mid a \in \mathcal{A}_G \} \cup \{ $:- not bot.$\}$
\end{lstlisting}
%
%\begin{align*}
%G \cup {}&
%\mathcal{R}\big(C \cup \{ \{ a \} \mid a \in \mathcal{A_G} \} \big) \cup\mathcal{M}
%\cup{} \\& \{ \mathit{true}(a) \leftarrow a \mid a \in X \} \cup \{ \mathit{false}(a) \leftarrow not \ a \mid a \in X \} \cup\{\leftarrow \ not \ \mathit{bot}\}
%\ .  
%\end{align*}
The stable models of the resulting program correspond to the solutions of the \gc\ problem.
%\comment{JR: reformat, explain true(a) and false(a), and add a conclusion: the toolchain is automated. Also, $\mathcal{R}$ must be used always the same way in the paper.}
%For solving the problem posed in the previous section, 
%in \cite{brderosc15a} an iterative algorithm was introduced. 
%Given the logic program $P$ over $\mathcal{A}_P$, 
%and the preference program $Q$ for preference relation $\succ_{\mathtt{distance}}$, 
%the algorithm returns an optimal model of $P$ wrt $\mathcal{A}_P$, or $\bot$ if $P$ is unsatisfiable:
%\begin{enumerate}
%\item
%Step 1: Assign to $Y$ the result of solving $P$, and return $\bot$ if $P$ is unsatisfiable.
%\item
%Step 2:
%Assign $Y$ to $X$, and assign to $Y$ the result of solving the program
%$P \cup Q \cup R_\mathcal{A_P} \cup H'_Y$. 
%If $Y$ is $\bot$, then return $X$, else repeat this Step.
%\end{enumerate}

%Let $P$ be a logic program with signature $\mathcal{A}_P$, 
%let $s$ be a preference statement defining preference relation $\succ_s$ over $\mathcal{A}\times\mathcal{A}$, 
%and $Q$ a preference program for $s$ with signature $\mathcal{A}_Q$ disjoint from $\mathcal{A}_P$.
%Define $P'$ as a copy of $P$ but with signature $\mathcal{A}_{P'}$ disjoint from $\mathcal{A}_P$ and $\mathcal{A}_Q$.
%The guess and check solutions of 
%$\langle P  \cup        \{ holds'(a) \leftarrow a \mid a \in \mathcal{A}_P \}, 
%         P' \cup Q \cup \{ holds(a)  \leftarrow a \mid a \in \mathcal{A}_{P'} \} \rangle$
%correspond to the $\succ_s$-preferred stable models of $P$. 

%\cite{eitpol06a} defined a G{\&}C framework as follows:
%\begin{definition}[G{\&}C \cite{eitpol06a}]
%Let $G$ and $C$ be two logic programs, and $X$ an interpretation of $G$.
%$X$ is a guess and check solution for $\langle G,C\rangle$ if
%$X$ is a stable model of $G$, and $X \cup C$ is unsatisfiable.
%\end{definition}


%To avoid these clash of predicate names, 
%we allow selecting some predicates of the guess program to be passed to the checker program.
%This is expressed via a guess directive
%\begin{lstlisting}
%#guess p/n.
%\end{lstlisting}
%where \texttt{p} is a predicate of arity \texttt{n} appearing in $G$.
%For example, for \asprin\ we could add \lstinline!#guess holds'/1.! and avoid use $P$ in both the guess and the check programs.

%The encoding avoids “guessing”
%    a level mapping to describe the formation of a counterexample, but directly denies models
%    for which there is no such construction.
%    \item
%    Notably, our meta-programs apply to (reified)
%    extended logic programs (Simons et al. 2002), possibly including choice rules and \#sum
%    constraints, and we are unaware of any existing meta-encoding of their answer sets, neither
%    as candidates nor as counterexamples refuting optimality

%%% Local Variables: 
%%% mode: latex
%%% TeX-master: "paper"
%%% End: 
