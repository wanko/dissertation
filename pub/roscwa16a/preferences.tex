%\newpage
\emph{Preferences over optimal models in \asprin.}
%
%Our approximation algorithms use a solution to the \emph{Most Distant Optimal Model} 
%problem that applies preferences over optimal models.
%
Formally, this extension of \asprin\ is defined as follows.
%
Let $P$ be a logic program over $\mathcal{A}$, 
and let $s$ and $t$ be two preference statements.
%
A stable model $X$ of $P$ is optimal wrt $s$ \emph{and then} $t$ if
it is optimal wrt $s$, 
and there is no optimal model $Y$ of $P$ wrt $s$ such that $Y \succ_t X$.
%
From the point of view of complexity theory, when $P$ is normal, 
finding a stable model optimal wrt $s$ and then $t$ is $F\Delta^p_3$-hard.
We prove this by reducing the problem of finding an optimal stable model of a 
disjunctive logic program with weight minimization (see~\cite{roscwa16b}). % \ref{sec:complexity}).
%
We note that finding a stable model of a normal logic program $P$ with preferences 
is in $F\Sigma^p_2$, given the translation to disjunctive logic programs using the \gc\ method.
%
Therefore, assuming $F\Sigma^p_2 \neq F\Delta^p_3$, 
we cannot find a polynomial translation to a normal program with preferences.
%This approach allows for solving the \emph{Most Distant Optimal Model} problem in a simple way.
%This approach allows for solving 

It turns out that the \emph{Most Distant Optimal Model} problem can be easily formulated and solved within this approach.
%
Given a logic program $P$ with a preference statement $s$, 
and a set $\mathcal{X}=\{ X_1, \ldots, X_m \}$ of optimal stable models of $P$, 
the most distant optimal models for this problem correspond to the stable models of the logic program
\lstinline[mathescape=true]!$P \cup \{\; $holds($a$,0) :- $a$. $\mid a \in \mathcal{A}\} \cup H_\mathcal{X}$!
that are optimal wrt $s$ and then \lm{distance}.
%
In \asprin, this is represented simply by adding to the resulting logic program 
the preference statements $s$ and \lm{distance}, 
along with the declarations `\lm{#optimize($s$).}' and `\lm{#reoptimize(distance).}'.

For computing optimal models of a logic program $P$ over $\mathcal{A}$ wrt preference statements $s$ and then $t$, 
we propose a variant of \asprin's iterative algorithm~\cite{brderosc15a}.
%
Let $\mathit{solveOpt}(P,s)$ be the \asprin\ procedure for computing one optimal model of $P$ wrt $s$, 
and let $\mathit{solveQuery}(P,s,q)$ be any of our algorithms for solving the query problem  
given a logic program $P$ with preference statement $s$ and query atom $q$.
%
The algorithm follows these steps:
\begin{enumerate}
\item
Call $\mathit{solveOpt}(P,s)$ and assign the result to $Y$. Return $Y$ if it is $\bot$.
\item
Assign $Y$ to $X$, and call $\mathit{solveQuery}(P \cup Q_t^* \cup R_\mathcal{A} \cup H_{X}',s,\mathtt{better(}t\mathtt{)})$ assigning the result to $Y$.
If $Y$ is $\bot$, return $X$, else repeat this step.
\end{enumerate}
where $Q_t^*$ is the result of deleting the constraint `\lm{:- not better($t$).}' 
from a preference program $Q_t$ for $t$.
%
The first step of the algorithm computes an optimal model of $P$ wrt $s$.
%
Then Step~2, like in \asprin's basic algorithm, searches iteratively for better models. 
%
Specifically, it searches for optimal models of $P$ wrt $s$ that are better than $X$ wrt $t$.
%
Note that by construction of $Q_t^*$, 
the stable models $Y$ of $P \cup Q_t^* \cup R_\mathcal{A} \cup H_{X}'$ 
are better than $X$ wrt $t$ iff \lm{better($t$) $\in Y$}. 
%
Then if $\mathit{solveQuery}$ returns a model $Y$, it contains \lm{better($t$)},
and therefore it is better than $X$ wrt $t$.
%
On the other hand, if $\mathit{solveQuery}$ returns $\bot$, there is no optimal model of $P$ wrt $s$ that is better than $X$ wrt $t$, 
and this implies that $X$ is an optimal model wrt $s$ and then $t$.

%Similarly to \texttt{\#optimize} statements, 
%this directive is translated into the fact \texttt{optimize(t)}, 
%which is added to the preference program for \texttt{t}.
%Internally, some modifications are needed to separate the preference programs for \texttt{s} and \texttt{t}
%(with facts \texttt{optimize(s)} and \texttt{optimize(t)}, respectively)
%but from a formal and a user perspective, the original definitions suffice.
%
%For computing a $\succ_{s,t}$-preferred we may use any of the algorithms of the previous section
%to solve queries over optimal models. 
%Let $P$ be a program over $\mathcal{A}$, $s$ be a preference statement, and $q \in \mathcal{A}$ be an atom, 
%then $\mathit{solveQuery}Opt(P,s,q)$ is a procedure that returns a
%$\succ_s$-preferred stable model of $P$ containing $q$ if it exists, and $\bot$ otherwise.
%Furthermore, let $\mathit{solveOpt}(P,s)$ be the basic \asprin\ procedure that returns a
%$\succ_s$-preferred stable model of $P$, if $P$ is satisfiable, and $\bot$ otherwise.
%
%The method starts computing any optimal model $Y$ of $P$ wrt $s$ calling $\mathit{solveOpt}(P,s)$.
%Then, it searches for an optimal model of $P$ wrt $s$ that is better than $Y$ wrt $t$.
%If it exists, it searches again for an optimal model of $P$ wrt $s$, better than the last computed wrt $t$.
%Whenever no such optimal model exists, the last optimal model computed is a solution.
%
%%%The search of the iterative step is posed as a query problem, 
%%%where the query holds if the model is better thatn the previous one wrt $t$.
%%%This is formalized as follows.
%%%Given a program $P$, define $P^q$ as the program
%%%\[
%%%(P \setminus \{ r\in P \mid head(r)=\emptyset \})
%%%\cup
%%%\{ bot \leftarrow body(r) \mid r\in P, head(r)=\emptyset \} \cup \{\ q \leftarrow \naf bot  \}
%%%\]
%%%where $bot$ and $q$ are new fresh atoms.
%%%% ----------------------------------------------------------------------
%%%\begin{proposition}
%%%If program $P$ is stratified, $P^q$ has a unique stable model $X$, 
%%%and $P$ is satisfiable iff $q \in X$.
%%%\end{proposition}
%%%
%%%Then we have the following proposition:
%%%% ----------------------------------------------------------------------
%%%\begin{proposition}\label{prop:preference:statements}
%%%Let $P$ be a program over $\mathcal{A}$, 
%%%let $s$ and $t$ be preference statements, 
%%%and let $O_s$ be a preference program for $s$.
%%%%
%%%%Then, we have that
%%%\begin{enumerate}
%%%\item
%%%If $X$ is a $\succ_s$-preferred stable model of $P$, 
%%%then $X$ is $\succ_{s,t}$-preferred
%%%iff
%%%\(
%%%\big(P\cup (O_t \cup R_{\mathcal{A}}  \cup \Holdsp{X})^q\big)
%%%\)
%%%has no $\succ_s$-preferred stable model containing $q$.
%%%\item
%%%If $Y$ is a $\succ_s$-preferred stable model of 
%%%\(
%%%\big(P\cup (O_t \cup R_{\mathcal{A}}  \cup \Holdsp{X})^q\big)
%%%%\big(P\cup E_{t_s}\cup F_s\cup C \cup R_{\mathcal{A}}  \cup \Holds{X}'\big)
%%%\)
%%%containing $q$ 
%%%for some $X\subseteq\mathcal{A}$,
%%%then $Y\cap\mathcal{A}$ is a $\succ_s$-preferred stable model of $P$ such that $(Y\cap\mathcal{A})\succ_t X$.
%%%\end{enumerate}
%%%\end{proposition}
%%%% ----------------------------------------------------------------------
%
%Putting all the pieces together, we have the following algorithm:
%%  ----------------------------------------------------------------------
%\begin{algorithm}[ht]\caption{$\mathit{\mathit{solveOpt}}(P,s,t)$\label{algo:solve:opt}}
%  \Input{A program $P$ over $\mathcal{A}$ and preference statements $s$ and $t$.}
%  \Output{A $\succ_{s,t}$-preferred stable model of $P$, if $P$ is satisfiable, and $\bot$ otherwise.}
%  \BlankLine
%  $Y \leftarrow \mathit{\mathit{solveOpt}}(P,s)$\;
%  \lIf{$Y = \bot$}{\Return $\bot$}
%  \Repeat{$Y = \bot$}{%
%    $X \leftarrow Y$\;
%    $Y  \leftarrow \mathit{\mathit{solveQuery}Opt}\big((P\cup (O_{t} \cup R_{\mathcal{A}} \cup \Holdsp{X} )^q ),s,q\big) \cap\mathcal{A}$\;
%  }
%  \Return $X$
%\end{algorithm}
% 
% ----------------------------------------------------------------------
%%%\begin{itemize}
%%%\item
%%%The Method:
%%%    \begin{enumerate}
%%%    \item
%%%    Find $\succ_s$-preferred model $X$ of $P$.
%%%    If $P$ is unsat, return unsat, else goto 2.
%%%    \item
%%%    TODO
%%%    %Find an optimal model $Y$ of $P$ optimizing $p1$ that is better than $X$ optimizing wrt $p2$.
%%%    %Assign $Y$ to $X$. 
%%%    %Repeat until we get UNSAT.
%%%    %The last model computed is a solution.
%%%    \end{enumerate}
%%%\end{itemize}
%
%%% Local Variables: 
%%% mode: latex
%%% TeX-master: "paper"
%%% End: 
