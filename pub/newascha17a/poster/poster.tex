  \PassOptionsToPackage{table}{xcolor}
  \documentclass[final]{beamer} % beamer 3.10: do NOT use option hyperref={pdfpagelabels=false} !
  %\documentclass[final,hyperref={pdfpagelabels=false}]{beamer} % beamer 3.07: get rid of beamer warnings
  \mode<presentation> {  %% check http://www-i6.informatik.rwth-aachen.de/~dreuw/latexbeamerposter.php for examples
  	\usetheme{ROPO}    %% you should define your own theme e.g. for big headlines using your own logos 
  }
  \usepackage[ngerman, english]{babel}
%  \usepackage[latin1]{inputenc}
  \usepackage[utf8]{inputenc}
  \usepackage{amsfonts}
  \usepackage{amsmath,amsthm, amssymb, latexsym}
  \usepackage{tikz}
  \usepackage{xcolor}\usepackage{calc}
  \definecolor{myblue}{HTML}{004A99}
  \usetikzlibrary{arrows.meta,chains,matrix,positioning,scopes,shapes.geometric,fit,calc}
  \usepackage{tcolorbox} 
  \usepackage{multirow}
  \usepackage{booktabs}
  \usepackage{float}
%  \usepackage[table]{xcolor}
  \usepackage{verbatim}
  %\usepackage{times}\usefonttheme{professionalfonts}  % times is obsolete
  \usefonttheme[onlymath]{serif}
  \boldmath
  \usepackage[orientation=portrait,size=a0,scale=1.3,debug]{beamerposter}                       % e.g. for DIN-A0 poster
  \newsavebox\CBox
  \def\textBF#1{\sbox\CBox{#1}\resizebox{\wd\CBox}{\ht\CBox}{\textbf{#1}}}
  
  \makeatletter
  \newcommand{\specificthanks}[1]{\@fnsymbol{#1}}% Inserts a specific \thanks symbol
  \renewcommand\@makefntext[1]{%
  	\@setpar{%
  		\@@par \@tempdima=\hsize
  		\advance\@tempdima by -0.4cm\relax
  		\parshape \@ne 0.4cm \@tempdima
  	}%
  	\par \parindent=\z@ \noindent
  	\hb@xt@ \z@{\hss \hb@xt@ 0.4cm\textsuperscript{\@thefnmark\hss}}%
  	#1%
  }
  \makeatother
  
  \setlength\heavyrulewidth{0.12cm}
  
  \title[Fancy Posters]{Enhancing Symbolic System Synthesis through ASPmT with Partial Assignment Evaluation}
  \author[]{Kai Neubauer\textsuperscript{\specificthanks{1}}, Philipp Wanko\textsuperscript{\specificthanks{2}}, Torsten Schaub\textsuperscript{\specificthanks{2}}, and Christian Haubelt\textsuperscript{\specificthanks{1}}}
  \institute[RostockPotsdam]{\textsuperscript{\specificthanks{1}}Applied Microelectronics and Computer Engineering, University of Rostock, Germany\\[0.5ex] \textsuperscript{\specificthanks{2}}Knowledge Processing and Information Systems, University of Potsdam, Germany}
  \date{Jul. 31th, 2007}
  
  \newlength\colsep
  \setlength\colsep{1cm}
  \newlength\colwidth
  \setlength\colwidth{0.5\textwidth-0.5\colsep}
  
  \renewcommand{\blacktriangleright}{\triangleright}
  
\begin{document}
  	\begin{frame}{} 
  		\vspace*{1.8cm}
  		\begin{columns}
  			\begin{column}{\colwidth}
  	           \begin{myblock}{Introduction}{A}
				\begin{itemize}
					\item[\color{HRO1}$\blacktriangleright$] Emerging industrial and consumer environments require efficient design techniques
					\item[\color{HRO1}$\blacktriangleright$] Abstraction raised to the electronic system level \cite{Gerstlauer2009}
					\item[\color{HRO1}$\blacktriangleright$] \emph{Answer Set Programming} (ASP) in combination with \emph{Quantifier Free Integer Difference Logic} (QF--IDL) has shown to be a promising approach~\cite{Biewer2014a}
					\item[\color{HRO1}$\blacktriangleright$] We propose a novel \emph{ASP modulo Theories} (ASPmT) approach that:
					    \begin{itemize}
					        \item Supports sophisticated system models
					        \item Tightly integrates QF--IDL solving 
					        \item Makes use of partial assigmnet checking
					        \item Offers a flexible declarative encoding encompassing binding, routing and scheduling
					    \end{itemize}
				\end{itemize}
				\begin{center}
					\vspace*{1cm}
					\scalebox{0.85}{
\begin{tikzpicture}[every text node part/.style={align=center},>=stealth,scale=2.7]
%\linespread{0.6}
\tikzstyle{block} = [draw=black,fill]
\tikzstyle{file} = [draw=black, line width = 0.12cm,fill=myblue,text=white]
\tikzstyle{arrow} = [->,line width=0.12cm,black]

\node [block,anchor=north west,minimum width=13cm,minimum height=8.8cm,fill=none,line width=0.12cm] (input) at (-2.8,-0.9) {};
\node[anchor=south east,font=\small] at (input.south east) {Input};
\draw[dashed,line width=0.12cm] (input.north) -- (input.south);
\draw[fill=myblue,line width=0.12cm] (-0.15,-1.4) node[above=0.5,anchor=west,font=\small,inner sep=0](encoding){Encoding} -- node[right,text=white,anchor=west,pos=0.3,font=\small]{Binding} node[right,text=white,anchor=west,pos=0.5,font=\small]{Routing} node[right,text=white,anchor=west,pos=0.7,font=\small]{Scheduling} (-0.15,-3.6) --  (1.65,-3.6) -- (1.65,-1.8) node (v2) {}--(1.25,-1.4) node (v1) {} -- cycle;
\draw[line width=0.12cm] (v1.center) -- (1.25,-1.8) -- (v2.center);
\draw[fill=myblue,line width=0.12cm] (-2.4,-1.4) node[above=0.5,anchor=west,font=\small,inner sep=0](inst){Specification} -- node[right,text=white,anchor=west,pos=0.3,font=\small]{Application} node[right,text=white,anchor=west,pos=0.5,font=\small]{Platform} node[right,text=white,anchor=west,pos=0.7,font=\small]{Constraints} (-2.4,-3.6) --  (-0.6,-3.6) -- (-0.6,-1.8) node (v2) {}--(-1,-1.4) node (v1) {} -- cycle;
\draw[line width=0.12cm] (v1.center) -- (-1,-1.8) -- (v2.center);
\draw[draw=black,fill=myblue,line width=0.12cm] (3,-1.4) -- node[](frameworkwest){} (3,-3.6) -- node[pos=0.5,text = white,font=\small,above=3.8] {\textbf{ASPmT Synthesis Framework}} (8.2,-3.6)--node[](frameworkeast){}(8.2,-1.4)--cycle;
%\node [fill=myblue, rotate=0,font=\small ] (framework) at (6,-1) {\textbf{ASPmT Synthesis Framework}};
\node [block, fill=myblue,text=white, rotate=0,font=\small,anchor=west,line width=0.12cm ] (implementation) at (9.2,-2.5) {Feasible Implementation};

\draw [arrow] (input) -- (frameworkwest.center);
\draw [arrow] (frameworkeast.center) -- (implementation);
\node[circle,draw,line width=0.12cm,font=\footnotesize,inner sep=2,fill=myblue!50,text=white,minimum width=2.74cm] (asp) at (4.4,-2.8) {ASP};
\node[circle,draw,line width=0.12cm,font=\footnotesize,inner sep=2,fill=myblue!50,text=white,minimum width=2.74cm] (idl) at (6.8,-2.8) {QF--IDL};
\draw [arrow](asp) .. controls (5,-2.1) and (6.2,-2.1) .. (idl);
\draw [arrow](idl) .. controls (6.2,-3.5) and (5,-3.5) .. (asp);
\end{tikzpicture}}
\vspace*{-1cm}
						\vspace*{0.0cm}
				\end{center}
  	           \end{myblock}
  			\end{column}
  			\begin{column}{\colwidth}
  	                \begin{myblock}{Specification Model}{A}
	  				\begin{itemize}
	  					\item[\color{HRO1}$\blacktriangleright$] Application Model
	  					\begin{itemize}
	  						\item Bipartite graph in task level granularity: Tasks $T$ and Messages $C$
	  						\item Characterized by a Period $P_i$ and a Deadline $D_i$
	  						\item Index delay $s:C\mapsto\mathbb{N}$ allows description of cyclic applications
	  					\end{itemize}
	  					\item[\color{HRO1}$\blacktriangleright$] Platform Model
	  					\begin{itemize}
	  						\item Directed graph containing computational resources $R_C$ connected by routers $R_R$
	  					\end{itemize}
	  					\item[\color{HRO1}$\blacktriangleright$] Specification Model
	  					\begin{itemize}
	  						\item Connection of Applications and Platform model by a set of mapping possibilities %$M\subseteq T\times R_C$
	  						\item Worst case execution times (WCET) associated to each mapping edge 
	  					\end{itemize}
	  				\end{itemize}
	  				\vspace*{0.0cm}
	  				\begin{center}
	  					%\begin{tikzpicture}[>=stealth]
%		\tikzstyle{axis} = [->, black, very thick]
%		\tikzstyle{link} = [<->, black!50, very thick]
%		\tikzstyle{task} = [anchor=west,shape=rectangle,thick,text=white,minimum height=0.5cm, draw=black, fill=myblue, align=center, rounded corners=2,inner sep=0]
%		\tikzstyle{difftask} = [shape=circle,thick,text=white,minimum height=0.5cm, draw=black, fill=myblue, align=center,inner sep=2]
%		\tikzstyle{router} = [diamond,draw,fill=black!30!yellow,inner sep=0]
%		\tikzstyle{resource} = [rectangle,minimum width=0.6cm,minimum height=0.4cm,draw,fill=black!50!green,text=white]
%		\tikzstyle{route} = [->,black,line width=1.5]
%		\node [router] (v1) at (-3.8,2.4) {\footnotesize$rsu_1$};
%		\node [router] (v2) at (-2.2,2.4) {\footnotesize$rsu_2$};
%		\node [router] (v4) at (-3.8,0.8) {\footnotesize$rsu_3$};
%		\node [router] (v5) at (-2.2,0.8) {\footnotesize$rsu_4$};
%		
%		\draw [link] (v1) edge (v2);
%		\draw [link] (v1) edge (v4);
%		\draw [link] (v2) edge (v5);
%		\draw [link] (v4) edge (v5);
%		
%		\node [resource] (v10) at (-4.6,3.2) {$r_1$};
%		\node [resource] (v11) at (-3,3.2) {$r_2$};
%		\node [resource] (v13) at (-4.6,1.6) {$r_3$};
%		\node [resource] (v14) at (-3,1.6) {$r_4$};
%		
%		\draw [link] (v10) edge (v1);
%		\draw [link] (v11) edge (v2);
%		\draw [link] (v13) edge (v4);
%		\draw [link] (v14) edge (v5);
%		
%		\node [difftask] (t1) at (-6.5,3) {$t_1$};
%		\node [difftask,fill=black!40!red,inner sep=0] (c1) at (-6.0,2) {\footnotesize$c_1$};
%		\node [difftask,fill=black!20!orange,inner sep=0] (c2) at (-7.0,2) {\footnotesize$c_2$};
%		\node [difftask] (t2) at (-6.5,1) {$t_2$};
%		
%		\draw [axis] (t1) edge (c1);
%		\draw [axis] (c1) edge (t2);
%		
%		\draw [axis,dashed] (t2) edge[bend right=14] node[midway,above=-0.4]{\tiny$2$}(v14);
%		\draw [axis,dashed] (t2) edge[bend left=14] node[midway,above=-0.4]{\tiny$4$}(v13);
%		\draw [axis,dashed] (t1) edge[bend left=10] node[midway,above=-0.4]{\tiny$5$}(v10);
%		\draw [axis,dashed] (t1) edge[bend right=14] node[midway,above=-0.4]{\tiny$7$}(v11);
%		
%		%\draw [route,black!40!red](v10) -- (-4.6,2.9) -- (-4.1,2.4) -- (-4.1,0.8) -- (-3.8,0.5) -- (-2.2,0.5)  --    (v14.south);
%		%\draw [route,black!20!orange](v14.east) -- (-1.9,0.8) -- (-1.9,2.4) -- (-2.2,2.7) -- (-3.8,2.7)  --    (v10.east);
%		\draw [axis] (t2) edge (c2);
%		\draw [axis] (c2) edge (t1);
%		\node[inner sep=0] at (-6.5,3.5) {\tiny $P_1=10, D_1=8$};
%		\node[] at (-6.5,0.4) {\tiny $s(c_1)=0, s(c_2)=1$};
%		
%		\node [difftask] (t3) at (-0.2,3) {$t_3$};
%		\node [difftask,fill=black!40!red,inner sep=0] (c3) at (0.3,2.0) {$c_3$};
%		\node [difftask,fill=black!40!red,inner sep=0] (c4) at (-0.7,2.0) {$c_4$};
%		\node [difftask] (t4) at (0.3,1.0) {$t_4$};
%		\node [difftask] (t6) at (-0.7,1.0) {$t_5$};
%		\draw [axis] (t3) edge (c4);
%		\draw [axis] (c4) edge (t6);
%		\draw [axis] (t3) edge (c3);
%		\draw [axis] (c3) edge (t4);
%		
%		
%		\draw [axis,dashed] (t3) edge[bend left=10] node[midway,above=-0.4]{\tiny$3$}(v11);
%		\draw [axis,dashed] (t3) edge[bend right=16] node[midway,above=-0.4]{\tiny$2$}(v10);
%		\draw [axis,dashed] (t6) edge[bend right=14] node[midway,above=-0.4]{\tiny$4$}(v14);
%		\draw [axis,dashed] (t4) edge[bend right=21] node[near start,above=-0.4]{\tiny$2$}(v13);
%		\node[inner sep=0] at (-0.1,3.5) {\tiny $P_2=15, D_2=15$};
%		\node[] at (-0.1,0.4) {\tiny $s(c_3)=0, s(c_4)=0$};
%		\node[] at (-6.5,2) {$A_1$};
%		\node at (-0.2,2) {$A_2$};
%	\end{tikzpicture}
\begin{tikzpicture}[>=stealth]
	\tikzstyle{axis} = [->, black, very thick]
	\tikzstyle{link} = [<->, black!50, very thick]
	\tikzstyle{task} = [anchor=west,shape=rectangle,thick,text=white,minimum height=0.5cm, draw=black, fill=myblue, align=center, rounded corners=2,inner sep=0]
	\tikzstyle{difftask} = [shape=circle,thick,text=white,minimum height=0.5cm, draw=black, fill=myblue, align=center,inner sep=2]
	\tikzstyle{router} = [diamond,draw,fill=black!30!yellow,inner sep=0]
	\tikzstyle{resource} = [rectangle,minimum width=0.6cm,minimum height=0.4cm,draw,fill=black!50!green,text=white]
	\tikzstyle{route} = [->,black,line width=1.5]
	\node [router] (v1) at (-3.8,2.4) {\footnotesize$rsu_1$};
	\node [router] (v2) at (-2.2,2.4) {\footnotesize$rsu_2$};
	\node [router] (v4) at (-3.8,0.8) {\footnotesize$rsu_3$};
	\node [router] (v5) at (-2.2,0.8) {\footnotesize$rsu_4$};
	
	\draw [link] (v1) edge (v2);
	\draw [link] (v1) edge (v4);
	\draw [link] (v2) edge (v5);
	\draw [link] (v4) edge (v5);
	
	\node [resource] (v10) at (-4.6,3.2) {$r_1$};
	\node [resource] (v11) at (-3,3.2) {$r_2$};
	\node [resource] (v13) at (-4.6,1.6) {$r_3$};
	\node [resource] (v14) at (-3,1.6) {$r_4$};
	
	\draw [link] (v10) edge (v1);
	\draw [link] (v11) edge (v2);
	\draw [link] (v13) edge (v4);
	\draw [link] (v14) edge (v5);
	
	\node [difftask] (t1) at (-0.3,3) {$t_5$};
	\node [difftask,fill=black!40!red,inner sep=0] (c1) at (0.2,2) {\footnotesize$c_4$};
	\node [difftask,fill=black!20!orange,inner sep=0] (c2) at (-0.8,2) {\footnotesize$c_5$};
	\node [difftask] (t2) at (-0.3,1) {$t_6$};
	
	\draw [axis] (t1) edge (c1);
	\draw [axis] (c1) edge (t2);
	
	\draw [axis,dashed] (t2) edge[bend right=22] node[midway,below]{\tiny$2$}(v13);
	\draw [axis,dashed] (t2) edge[bend left=14] node[midway,above]{\tiny$4$}(v14);
	\draw [axis,dashed] (t1) edge[bend left=10] node[midway,above]{\tiny$5$}(v10);
	\draw [axis,dashed] (t1) edge[bend right=14] node[midway,below]{\tiny$7$}(v11);
	
	%\draw [route,black!40!red](v10) -- (-4.6,2.9) -- (-4.1,2.4) -- (-4.1,0.8) -- (-3.8,0.5) -- (-2.2,0.5)  --    (v14.south);
	%\draw [route,black!20!orange](v14.east) -- (-1.9,0.8) -- (-1.9,2.4) -- (-2.2,2.7) -- (-3.8,2.7)  --    (v10.east);
	\draw [axis] (t2) edge (c2);
	\draw [axis] (c2) edge (t1);
	\node[inner sep=0] at (-0.3,3.5) {\tiny $P_2=10, D_2=8$};
	\node[] at (-0.3,0.5) {\tiny $s(c_5)=0, s(c_5)=1$};
	
	\node [difftask] (t3) at (-7.6,3) {$t_1$};
	\node [difftask,fill=black!40!red,inner sep=0] (c3) at (-6,2) {$c_3$};
	\node [difftask,fill=black!40!red,inner sep=0] (c4) at (-7.6,2) {$c_1$};
	\node [difftask,fill=black!40!red,inner sep=0] (c5) at (-6.8,1) {$c_2$};
	\node [difftask] (t4) at (-6,1) {$t_3$};
	\node [difftask] (t5) at (-6,3) {$t_4$};
	\node [difftask] (t6) at (-7.6,1) {$t_2$};
	\draw [axis] (t3) edge (c4);
	\draw [axis] (c4) edge (t6);
	\draw [axis] (t6) edge (c5);
	\draw [axis] (c5) edge (t4);
	\draw [axis] (t4) edge (c3);
	\draw [axis] (c3) edge (t5);
	
	
	\draw [axis,dashed] (t3) edge[bend left=15] node[very near end,above]{\tiny$2$}(v11);
	\draw [axis,dashed] (t3) edge[bend right=26] node[near end,below]{\tiny$3$}(v10);
	\draw [axis,dashed] (t6) edge[bend right=20] node[near end,right]{\tiny$2$}(v10);
	\draw [axis,dashed] (t4) edge[bend right=21] node[near start,above]{\tiny$2$}(v13);
	\draw [axis,dashed] (t5) edge[bend left=15] node[near start,below]{\tiny$1$}(v10);
	\node[inner sep=0] at (-6.9,3.5) {\tiny $P_1=7, D_1=12$};
	\node[] at (-6.8,0.5) {\tiny $s(c_1)=s(c_2)=s(c_3)=0$};
	\node[] at (-0.3,2) {$A_2$};
	\node at (-6.8,2) {$A_1$};
\end{tikzpicture}
	  						\vspace*{-0.0cm}
	  				\end{center}
	        \end{myblock}
	  		\end{column}
	  	\end{columns}
  	\vspace*{1.0cm}
		\begin{columns}[T]
			\begin{column}{1.4\colwidth}
	  			\begin{myblock}{ASPmT Synthesis Framework}{B}
	  				\begin{columns}[]
	  					\begin{column}{0.56\textwidth}
			  				\begin{itemize}
			  					\item[\color{HRO1}$\blacktriangleright$] Transformation from Specification into feasible Implementation including Binding, Routing, Schedule
			  					\begin{itemize}
%			  						\item ASP's support for reachability offers a scalable solution for binding and routing 
			  						\item Answer Set Programming (ASP) offers a scalable solution for binding and routing due to support of \emph{reachability}
			  						\item Domain specific heuristics utilized to steer solving
			  					\end{itemize}
								\item[\color{HRO1}$\blacktriangleright$] \textBF{ASP modulo Theories (ASPmT) Approach:} Tight integration of background theory into ASP Solver
								\begin{itemize}
									\item Detection of invalid schedules on \emph{partial} bindings and routings with \emph{Quantifier-free Integer Difference Logic} (QF--IDL)
									\item Exclusion of schedules where tasks overlap in subsequent iterations through an additional \emph{post propagator}
									\item One single encoding for binding, routing, and scheduling offers a succinct and elaboration tolerant formulation
								\end{itemize} 
			  				\end{itemize}
			  			\end{column}
			  			\begin{column}{.44\textwidth}
			  				\begin{center}
			  					\scalebox{0.85}{
\begin{tikzpicture}[every text node part/.style={align=center},>=stealth,scale=2.7]
%\linespread{0.6}
\tikzstyle{block} = [rounded corners=10,draw=black,fill]
\tikzstyle{arrow} = [->,line width=0.12cm,black]
\node [block, fill=black!30!yellow!05, rotate=0,font=\small ] (v1) at (1.5,3.8) {Synthesis Problem Instance};

\node[block,fit={(-3.4,3.1) (6.3,-3.25)},fill=blue!05] (syn) {};
%\draw[block,fill=blue!05] (-3.4, 3.25) rectangle (6.4,-3.4) ;

\node[inner sep=0.0cm,anchor=west,rotate=90] at (6.2,-3.2) {\textbf {ASP Solver (Clingo)}};
%\node[anchor=north west,inner sep=0.1cm] at (syn.north west) {\textbf {Answer Set Solver (Clingo)}};
\node [block, text width=5.5cm, fill=myblue, font=\footnotesize\color{white} ] (v7) at (-2,-0.9) {Decide binding or routing variable};
\draw [block, fill=blue!40](5.9,2.8) -- (5.9,-3.3) -- (-0.7,-3.3) -- (-0.7,0.0) -- (-3.1,0.0) -- (-3.1,2.8) -- (0.5,2.8) -- (0.6,2.8) -- cycle;
\node[anchor=west,inner sep=0.0cm,rotate=90] at (5.7,-3.2) {\textbf{\small {Background Theory}}};

\draw[block,fill=black!40!green!40]  (5,2.7) rectangle (-1.8,0.6);
\draw[block,fill=black!40!green!40]  (5,0.4) rectangle (2,-1.4);
\node[anchor=west,inner sep=0.0cm,align=left] at (3.8,0.8) {\footnotesize {\textbf{QF--IDL}}};
\node[anchor=west,inner sep=0.0cm,align=left] at (3.8,-1.2) {\footnotesize {\textbf{Periodic}}};
\node [draw, diamond, aspect=2,minimum width=8.1cm,minimum height=4.05cm, fill=myblue] (v2) at (0.0,1.7) {} ;


\node [block, text width=5cm, fill=myblue,text=white,font=\footnotesize] (v8) at (3.5,1.7) {(Partial) Schedule};

\node [block,  fill=black!30!yellow!05,rotate=0,font=\small] (v4) at (3.5,-4) {Feasible Implementation};
\node [block, text width=4.5cm,  fill=black!60!red!60,font=\footnotesize\color{white}] (v5) at (0.6,-0.9) {Restrict search space};
\draw [arrow] (v5) -- (v7);
\draw [arrow] (v1) edge (syn);
\draw [arrow](v7) -- node[near start, above right,rotate=90,font=\footnotesize]{partial\\ assignment}(-2,1.7) -- (v2);
%\draw [arrow](v7) -- node[midway, below,rotate=90]{\footnotesize assignment}node[midway, above,rotate=90]{\footnotesize partial}(-2,1.7) -- (v2);

\draw [arrow] (v2.center) edge node[near end, auto]{\footnotesize no}(v8);
\draw [arrow](v2.center) -- node[near end, right]{\footnotesize yes}(-0.0,0.5) -- (0.6,0.5) -- (v5);
\node [draw, diamond, aspect=2,minimum width=8.1cm,minimum height=4.05cm, fill=myblue] (v2a) at (v2.center) {} ;
\node[anchor=center,inner sep=0.2cm, text width=6.21cm,text=white,font=\footnotesize] at (v2.center) {Negative\\ cycle?};

\node [draw, diamond, aspect=2,minimum width=8.1cm,minimum height=4.05cm,text=white, fill=myblue] (v9) at (3.5,-2.5) {} ;
\node [draw, diamond, aspect=2,minimum width=8.1cm,minimum height=4.05cm,text=white, fill=myblue] (v19) at (3.5,-0.4) {} ;
\draw [arrow] (v8) edge (v19);
\draw [arrow](v19.center) -- (5.4,-0.4) node[very near end,below]{\footnotesize {\ yes}} -- (5.4,2.7)  -- (-0.0,2.7) -- (v2);
\draw [arrow] (v19.center) -- (3.5,-1.5) node[at end,left]{\footnotesize no}  -- (v9);
\node [anchor=center,draw, diamond, aspect=2,minimum width=8.1cm,minimum height=4.05cm,text=white, fill=myblue] (v19a) at (v19.center) {} ;
\node[anchor=center,inner sep=0.2cm, text width=6.21cm,text=white,font=\footnotesize] at (v19.center) {Periodic\\ overlapping?};
\draw [arrow] (v9.center) -- node[very near end,right]{\footnotesize yes}  (v4);
\draw [arrow](v9.center) -- (1.1,-2.5) -- (-2,-2.5) node[midway,above]{\footnotesize no}-- (-2,-2.1) -- (v7);
\node [draw, diamond, aspect=2,minimum width=8.1cm,minimum height=4.05cm,text=white, fill=myblue] (v9a) at (v9.center) {} ;
\node[anchor=center,inner sep=0.2cm, text width=6.21cm,text=white,font=\footnotesize] at (v9.center) {Complete\\ assignement?};



\end{tikzpicture}}
			  				\end{center}
	  					\end{column}
	  				\end{columns}
	  			\end{myblock}
	  		\end{column}
	  		\begin{column}{0.6\colwidth}
	  			\begin{myblock}{Spatial Binding and Routing}{B}
	  				\begin{itemize}
	  					\item[\color{HRO1}$\blacktriangleright$] Binding and Routing is derived from \cite{Biewer2014a}
	  					\begin{itemize}
	  					\item Selecting exactly one mapping possibility per task
	  					\item Routes for messages between tasks are subsequently decided recursively from receiver to sender
	  				\end{itemize}
	  					\begin{center}
	  						\begin{tikzpicture}[>=stealth,scale=2.7,cross/.style={path picture={ 
		\draw[black,line width=0.12cm]
		(path picture bounding box.south east) -- (path picture bounding box.north west) (path picture bounding box.south west) -- (path picture bounding box.north east);
	}},icon/.style={path picture={ 
	\draw[black,line width=0.12cm]
	($(path picture bounding box.south east)+(-0.2,0.2)$) -- ($(path picture bounding box.south east)+(-0.4,+0.2)$) -- ($(path picture bounding box.north west)+(0.4,-0.2)$) -- ($(path picture bounding box.north west)+(0.2,-0.2)$) ($(path picture bounding box.south west)+(0.2,0.2)$) -- ($(path picture bounding box.south west)+(0.4,0.2)$) -- ($(path picture bounding box.north east)+(-0.4,-0.2)$) -- ($(path picture bounding box.north east)+(-0.2,-0.2)$);
}}]
	\tikzstyle{axis} = [->, black, line width=0.12cm]
	\tikzstyle{link} = [<->, black!50, line width=0.12cm]
	\tikzstyle{task} = [anchor=west,shape=rectangle,thick,text=white,minimum height=0.5cm, draw=black, fill=myblue, align=center, rounded corners=2,inner sep=0]
	\tikzstyle{difftask} = [shape=circle,thick,text=white,minimum height=0.5cm, draw=black, fill=myblue, align=center,inner sep=2]
	\tikzstyle{router} = [icon,draw,fill=black!30!yellow,inner sep=0]
%	\tikzstyle{router} = [circle,cross,draw,fill=black!30!yellow,inner sep=0, line width=0.12cm]
%	\tikzstyle{router} = [diamond,draw,fill=black!30!yellow,inner sep=0]
	\tikzstyle{resource} = [rectangle,minimum width=0.6cm,minimum height=0.4cm,draw,fill=black!50!green,text=white]
	\tikzstyle{route} = [->,black,line width=1.5]
%	\node [router] (v1) at (-3.8,2.4) {\footnotesize$rsu_1$};
%	\node [router] (v2) at (-2.2,2.4) {\footnotesize$rsu_2$};
%	\node [router] (v4) at (-3.8,0.8) {\footnotesize$rsu_3$};
%	\node [router] (v5) at (-2.2,0.8) {\footnotesize$rsu_4$};
	\node [router,minimum width=1.5cm, minimum height=1cm] (v1) at (-3.8,2.4) {};
	\node [router,minimum width=1.5cm, minimum height=1cm] (v2) at (-2.2,2.4) {};
	\node [router,minimum width=1.5cm, minimum height=1cm] (v4) at (-3.8,0.8) {};
	\node [router,minimum width=1.5cm, minimum height=1cm] (v5) at (-2.2,0.8) {};
	
	\draw [link] (v1) edge (v2);
	\draw [link] (v1) edge (v4);
	\draw [link] (v2) edge (v5);
	\draw [link] (v4) edge (v5);
	
	\node [resource] (v10) at (-4.6,3.2) {$r_1$};
	\node [resource] (v11) at (-1.4,3.2) {$r_2$};
	\node [resource] (v13) at (-4.6,0) {$r_3$};
	\node [resource] (v14) at (-1.4,0) {$r_4$};
	
	\draw [link] (v10) edge (v1);
	\draw [link] (v11) edge (v2);
	\draw [link] (v13) edge (v4);
	\draw [link] (v14) edge (v5);
	
	
	
	\node [difftask] (t3) at (-5,3.5) {$t_1$};
	\node [difftask,fill=black!40!red,inner sep=0] (c3) at (-3.55,1.6) {\small$c_3$};
	\node [difftask,fill=black!40!red,inner sep=0] (c4) at (-1.95,1.6) {\small$c_4$};
	\node [difftask,fill=black!40!red,inner sep=0] (c5) at (-4.05,1.6) {\small$c_2$};
	\node [difftask,,fill=black!20!orange,inner sep=0] (c1) at (-2.45,1.6) {\small$c_5$};
	\node [difftask] (t4) at (-5,-0.3) {$t_3$};
	\node [difftask] (t5) at (-4.2,3.5) {$t_4$};
	\node [difftask] (t6) at (-5,2.9) {$t_2$};
	\node [difftask] (t1) at (-1,-0.3) {$t_6$};
	\node [difftask] (t2) at (-1.0,3.5) {$t_5$};

	\draw[axis,draw=black!40!red, dashed]  plot[smooth, tension=.7] coordinates {(v10.290) (-4.1,2.6) (c5) (-4.1,0.6) (v13.70)};
	\node [difftask,fill=black!40!red,inner sep=0] (c5a) at (c5.center) {\small$c_2$};
	\draw[axis,draw=black!40!red, dashed]  plot[smooth, tension=.7] coordinates {(v13.0) (-3.7,0.5) (c3) (-3.7,2.7) (v10.0)};
	\node [difftask,fill=black!40!red,inner sep=0] (c3a) at (c3) {\small$c_3$};
	
	\draw[axis,draw=black!40!red, dashed]  plot[smooth, tension=.7] coordinates {(v11.250) (-1.9,2.6) (c4) (-1.9,0.6)(v14.110)};
	\node [difftask,fill=black!40!red,inner sep=0] (c4a) at (c4.center) {\small$c_4$};
	\draw[axis,draw=black!20!orange, dashed]  plot[smooth, tension=.7] coordinates {(v14.west) (-2.3,0.5) (c1) (-2.3,2.7)(v11.west)};
	\node [difftask,,fill=black!20!orange,inner sep=0] (c1a) at (c1.center) {\small$c_5$};
\end{tikzpicture}
	  					\end{center}
	  				\end{itemize}
	  			\end{myblock}
	  		\end{column}
	  	\end{columns}
  	\vspace*{1.0cm}
		\begin{columns}[T]
  			\begin{column}{\colwidth}
  				\begin{myblock}{Scheduling}{S}
  					\begin{itemize}
  						\item[\color{HRO1}$\blacktriangleright$] Scheduling relayed into background theory QF--IDL
  						\item[\color{HRO1}$\blacktriangleright$] Constraints of form $x-y\leq k$ encode order of tasks
  						\begin{itemize}
  							\item Example: $\tau(t_1)-\tau(t_2)\leq -3$
  							\item Task $t_1$ has to be executed at least $3$ time units before $t_2$
  						\end{itemize}
  						\item[\color{HRO1}$\blacktriangleright$] Priority encoding based on partial order between tasks
  						\begin{itemize}
  							\item Prevent independent tasks to overlap
  							\item Decide order of tasks in subsequent iterations
  						\end{itemize}
  						\item[\color{HRO1}$\blacktriangleright$] Constraints form a directed Graph
  						\begin{itemize}
  							\item Shortest Path results in (negated) earliest possible start time
  							\item Negative cycle results in backtracking
  						\end{itemize}
  						\item[\color{HRO1}$\blacktriangleright$] Computation of a valid periodic schedule for a given binding of Application $A_1$:
  					\end{itemize}
  					\vspace*{-1.24cm}
  					\begin{center}
  						\begin{tikzpicture}[>=stealth,scale=2.7]
    \tikzstyle{axis} = [->, black, line width = 0.12cm]
    \tikzstyle{task} = [anchor=west,shape=rectangle,thick,text=white, draw=black, fill=myblue, align=center, rounded corners=2,inner sep=0, minimum height=1.35cm]
    \tikzstyle{difftask} = [shape=circle,thick,text=white,minimum height=0.5cm, draw=black, fill=myblue, align=center,inner sep=2]
    \draw[dashed, black!30] (-2.5,0) node[below, black]{\footnotesize 0};
    \foreach \i in {1,...,6}
    {
    	\draw[dashed, black!30, line width = 0.12cm] (\i*0.5-2.5,0) node[below, black]{\footnotesize\i}-- (\i*0.5-2.5,2.5);
    }
    \foreach \i in {8,...,10}
    {
    	\draw[dashed, black!30, line width=0.12cm] (\i*0.5-2.5,0) node[below, black]{\footnotesize\i}-- (\i*0.5-2.5,2.5);
    }
    	\draw[dashed, black!30, line width=0.12cm] (11*0.5-2.5,0) node[below, black]{}-- (11*0.5-2.5,2.5);
    	\draw[dashed, black!30, line width=0.12cm] (13*0.5-2.5,0) node[below, black]{}-- (13*0.5-2.5,2.5);
    \foreach \i in {14,...,15}
    {
    	\draw[dashed, black!30, line width=0.12cm] (\i*0.5-2.5,0) node[below, black]{\footnotesize\i}-- (\i*0.5-2.5,2.5);
    }
    
    \draw[black, line width = 0.12cm] (-2.5,0.5) --  (-2.7,0.5) node[left]{$r_1$};
    \draw[black, line width = 0.12cm] (-2.5,1.5) --  (-2.7,1.5) node[left]{$r_3$};
    \draw[dashed, black!40!green, line width = 0.12cm] (1,0) node[below, black]{\small P=7}-- (1,2.5);
    \draw[dashed, red, line width = 0.12cm] (3.5,0) node[below, black]{\small D=12}-- (3.5,2.5);
    \draw [axis] (-2.5,2.5) -- (-2.5,0) -- (5.5,0);
%    \node[task, minimum width=4.05cm] at (-2.5,0.5) {$t_1$};
%    \node[task, minimum width=2.7cm] at (-0.5,0.5) {$t_2$};
%    \node[task, minimum width=2.7cm] at (0.5,1.5) {$t_3$};
%    \node[task, minimum width=1.35cm,text width=1.35cm] at (2.5,0.5) {$t_4$};
    
    \node [difftask] (v1) at (-8.5,1) {$t_1$};
    \node [difftask] (v2) at (-7,1) {$t_2$};
    \node [difftask] (v3) at (-5.5,1) {$t_3$};
    \node [difftask] (v4) at (-4,1) {$t_4$};
    \draw [axis,black!40!red] (v1) edge node[above]{\footnotesize$-3$} (v2);
    \draw [axis,black!40!red] (v2) edge node[above]{\footnotesize$-2$}(v3);
    \draw [axis,black!40!red] (v3) edge node[above]{\footnotesize$-2$}(v4);
    \draw [axis,black!40!red] (v4) edge[bend left] node[below]{\footnotesize$-1+P=6$} (v1);
    \draw [axis] (v4) edge[bend right=45] node[above]{\footnotesize$-1+P=+6$} (v2);
    
    
%    \node[task, minimum width=4.05cm,fill=orange] at (1,0.5) {$t_1$};
%    \node[task, minimum width=2.7cm,fill=orange] at (3,0.5) {$t_2$};
%    \node[task, minimum width=2.7cm,fill=orange] at (4,1.5) {$t_3$};
%    \node[task, minimum width=1.35cm,fill=black!40!green,text width=1.35cm] at (-1,0.5) {$t_4$};
    
    \node[anchor=west] at (-9,2) {\footnotesize$t_4\succ t_2$};
    \node[anchor=west] at (-9,0) {\color{black!40!red}\footnotesize$t_4\succ t_1$};
    \node[anchor=north] at (5.45,-0.1) {\footnotesize{time}};
    
    \node[minimum width=1cm, fill=white] at (1.25,1.5) {\textsc{not schedulable}};
    \node[minimum width=1cm, fill=white] at (1.25,0.5) {\small negative cycle: $t_1\rightarrow t_2\rightarrow t_3\rightarrow t_4\rightarrow t_1$};
\end{tikzpicture}
  						\begin{tikzpicture}[>=stealth,scale=2.7]
    \tikzstyle{axis} = [->, black, line width = 0.12cm]
    \tikzstyle{task} = [anchor=west,shape=rectangle,thick,text=white, draw=black, fill=myblue, align=center, rounded corners=2,inner sep=0, minimum height=1.35cm]
    \tikzstyle{difftask} = [shape=circle,thick,text=white,minimum height=0.5cm, draw=black, fill=myblue, align=center,inner sep=2]
    \draw[dashed, black!30] (-2.5,0) node[below, black]{\footnotesize 0};
    \foreach \i in {1,...,6}
    {
    	\draw[dashed, black!30, line width = 0.12cm] (\i*0.5-2.5,0) node[below, black]{\footnotesize\i}-- (\i*0.5-2.5,2.5);
    }
    \foreach \i in {8,...,10}
    {
    	\draw[dashed, black!30, line width=0.12cm] (\i*0.5-2.5,0) node[below, black]{\footnotesize\i}-- (\i*0.5-2.5,2.5);
    }
    	\draw[dashed, black!30, line width=0.12cm] (11*0.5-2.5,0) node[below, black]{}-- (11*0.5-2.5,2.5);
    	\draw[dashed, black!30, line width=0.12cm] (13*0.5-2.5,0) node[below, black]{}-- (13*0.5-2.5,2.5);
    \foreach \i in {14,...,15}
    {
    	\draw[dashed, black!30, line width=0.12cm] (\i*0.5-2.5,0) node[below, black]{\footnotesize\i}-- (\i*0.5-2.5,2.5);
    }
    
    \draw[black, line width = 0.12cm] (-2.5,0.5) --  (-2.7,0.5) node[left]{$r_1$};
    \draw[black, line width = 0.12cm] (-2.5,1.5) --  (-2.7,1.5) node[left]{$r_3$};
    \draw[dashed, black!40!green, line width = 0.12cm] (1,0) node[below, black]{\small P=7}-- (1,2.5);
    \draw[dashed, red, line width = 0.12cm] (3.5,0) node[below, black]{\small D=12}-- (3.5,2.5);
    \draw [axis] (-2.5,2.5) -- (-2.5,0) -- (5.5,0);
    \node[task, minimum width=4.05cm] at (-2.5,0.5) {$t_1$};
    \node[task, minimum width=2.7cm] at (-1,0.5) {$t_2$};
    \node[task, minimum width=2.7cm] at (0,1.5) {$t_3$};
    \node[task, minimum width=1.35cm,text width=1.35cm] (v5) at (3.5,0.5) {$t_4$};
    
    \node [difftask] (v1) at (-8.5,1) {$t_1$};
    \node [difftask] (v2) at (-7,1) {$t_2$};
    \node [difftask] (v3) at (-5.5,1) {$t_3$};
    \node [difftask] (v4) at (-4,1) {$t_4$};
    \draw [axis,black!40!red] (v1) edge node[above,black!40!red]{\footnotesize$-3$} (v2);
    \draw [axis] (v2) edge node[above]{\footnotesize$-2$}(v3);
    \draw [axis] (v3) edge node[above]{\footnotesize$-2$}(v4);
    \draw [axis] (v1) edge[bend right] node[below]{\footnotesize$-3-P=-10$} (v4);
    \draw [axis,black!40!red] (v2) edge[bend left=45] node[above,black!40!red]{\footnotesize$-2-P=-9$} (v4);
    
    
    \node[task, minimum width=4.05cm,fill=orange] at (1,0.5) {$t_1$};
    \node[task, minimum width=2.7cm,fill=orange] at (2.5,0.5) {$t_2$};
    \node[task, minimum width=2.7cm,fill=orange] at (3.5,1.5) {$t_3$};
    \node[task, minimum width=1.35cm,fill=black!40!green,text width=1.35cm] at (0,0.5) {$t_4$};
    
    \node[anchor=west,black!40!red] at (-9,2) {\footnotesize$t_2\succ t_4$};
    \node[anchor=west] at (-9,0) {\footnotesize$t_1\succ t_4$};
    \node[anchor=north] at (5.45,-0.1) {\footnotesize{time}};
    
    \node[fill=white,draw=none,font=\small,text width=4.5cm,align=center] (v6) at (2.25,2) {\textsc{Deadline violated}};
    \draw[axis] (v6.south) -- (v5.west);
\end{tikzpicture}

  						\begin{tikzpicture}[>=stealth,scale=2.7]
    \tikzstyle{axis} = [->, black, line width = 0.12cm]
    \tikzstyle{task} = [anchor=west,shape=rectangle,thick,text=white, draw=black, fill=myblue, align=center, rounded corners=2,inner sep=0, minimum height=1.35cm]
    \tikzstyle{difftask} = [shape=circle,thick,text=white,minimum height=0.5cm, draw=black, fill=myblue, align=center,inner sep=2]
    \draw[dashed, black!30] (-2.5,0) node[below, black]{\footnotesize 0};
    \foreach \i in {1,...,6}
    {
    	\draw[dashed, black!30, line width = 0.12cm] (\i*0.5-2.5,0) node[below, black]{\footnotesize\i}-- (\i*0.5-2.5,2.5);
    }
    \foreach \i in {8,...,10}
    {
    	\draw[dashed, black!30, line width=0.12cm] (\i*0.5-2.5,0) node[below, black]{\footnotesize\i}-- (\i*0.5-2.5,2.5);
    }
    	\draw[dashed, black!30, line width=0.12cm] (11*0.5-2.5,0) node[below, black]{}-- (11*0.5-2.5,2.5);
    	\draw[dashed, black!30, line width=0.12cm] (13*0.5-2.5,0) node[below, black]{}-- (13*0.5-2.5,2.5);
    \foreach \i in {14,...,15}
    {
    	\draw[dashed, black!30, line width=0.12cm] (\i*0.5-2.5,0) node[below, black]{\footnotesize\i}-- (\i*0.5-2.5,2.5);
    }
    
    \draw[black, line width = 0.12cm] (-2.5,0.5) --  (-2.7,0.5) node[left]{$r_1$};
    \draw[black, line width = 0.12cm] (-2.5,1.5) --  (-2.7,1.5) node[left]{$r_3$};
    \draw[dashed, black!40!green, line width = 0.12cm] (1,0) node[below, black]{\small P=7}-- (1,2.5);
    \draw[dashed, red, line width = 0.12cm] (3.5,0) node[below, black]{\small D=12}-- (3.5,2.5);
    \draw [axis] (-2.5,2.5) -- (-2.5,0) -- (5.5,0);
    \node[task, minimum width=4.05cm] at (-2.5,0.5) {$t_1$};
    \node[task, minimum width=2.7cm] at (-0.5,0.5) {$t_2$};
    \node[task, minimum width=2.7cm] at (0.5,1.5) {$t_3$};
    \node[task, minimum width=1.35cm,text width=1.35cm] at (2.5,0.5) {$t_4$};
    
    \node [difftask] (v1) at (-8.5,1) {$t_1$};
    \node [difftask] (v2) at (-7,1) {$t_2$};
    \node [difftask] (v3) at (-5.5,1) {$t_3$};
    \node [difftask] (v4) at (-4,1) {$t_4$};
    \draw [axis] (v1) edge node[above]{\footnotesize$-3$} (v2);
    \draw [axis] (v2) edge node[above]{\footnotesize$-2$}(v3);
    \draw [axis] (v3) edge node[above]{\footnotesize$-2$}(v4);
    \draw [axis] (v1) edge[bend right] node[below]{\footnotesize$-3-P=-10$} (v4);
    \draw [axis,black!40!green] (v4) edge[bend right=45] node[above]{\footnotesize$-1+P=+6$} (v2);
    
    
    \node[task, minimum width=4.05cm,fill=orange] at (1,0.5) {$t_1$};
    \node[task, minimum width=2.7cm,fill=orange] at (3,0.5) {$t_2$};
    \node[task, minimum width=2.7cm,fill=orange] at (4,1.5) {$t_3$};
    \node[task, minimum width=1.35cm,fill=black!40!green,text width=1.35cm] at (-1,0.5) {$t_4$};
    
    \node[anchor=west] at (-9,2) {\color{black!40!green}\footnotesize$t_4\succ t_2$};
    \node[anchor=west] at (-9,0) {\footnotesize$t_1\succ t_4$};
    \node[anchor=north] at (5.45,-0.1) {\footnotesize{time}};
\end{tikzpicture}
  					\end{center}
  					\vspace*{-11mm}
  				\end{myblock}
  			\end{column}
  			\begin{column}{\colwidth}
  				\begin{tcolorbox}[size=minimal,equal height group=S,colback=tlg,colframe=tlg,beforeafter skip=0.0cm,toptitle=0mm,bottomtitle=0mm]
	  				\vspace*{-0.5cm}
	  				\begin{myblock}{Experimental Results}{E}
	  					\begin{itemize}
	  						\item[\color{HRO1}$\blacktriangleright$] Various specifications implemented onto regular 2D-mesh architecture
	  						\begin{itemize}
	  							\item ASP Solver \emph{clingo 5} \cite{gekakaosscwa16a}
	  							\item Number of necessary choices to find a solution decreased significantly
	  							\item Performance of background theory subject to improvements
	  						\end{itemize}
	  					\end{itemize}
	  					\begin{center}
		  					
		  					\begin{tabular}{lcc!{\vrule width 0.08cm}rrr!{\vrule width 0.08cm}rrr}
		  						\toprule
		  						\rowcolor{HRO1!50}[6pt][7pt]&  &  & \multicolumn{3}{c!{\vrule width 0.08cm}}{Partial Assignments} & \multicolumn{3}{c}{Full Assignments}  \\\noalign{\vskip-1pt} %\cmidrule(l){4-9} 
		  						\rowcolor{HRO1!50}[6pt][7pt]\hspace{1ex}$\lvert R\rvert$&$\lvert T\rvert$&$\lvert C\rvert$& \multicolumn{1}{c}{Choices}   & \multicolumn{2}{c!{\vrule width 0.08cm}}{Runtime}  & \multicolumn{1}{c}{Choices}   & \multicolumn{2}{c}{Runtime} \\ \noalign{\vskip-1pt}
		  						\rowcolor{HRO1!50}[6pt][7pt]&     &    &           &    \multicolumn{1}{c}{Overall}    &  \multicolumn{1}{c!{\vrule width 0.08cm}}{QF--IDL}  &  &   \multicolumn{1}{c}{Overall} &  \multicolumn{1}{c}{QF--IDL}       \\ %\midrule
		  						%$2\times2$& 28 & 32 & \textBF{3364} & \textBF{12.70 s} & 11.93 s  & 135692    &    14.69 s & 2.65 s \\ %7tpr_2x2_1apps
		  						$2\times2$& 20 & 44 & \textBF{11680} &  \textBF{87.17 s} & 85.19 s & 482870   &   173.08 s & 53.06 s  \\ %3tpr_3x3_1apps
		  						\rowcolor{HRO1!25}[6pt][7pt]$3\times3$& 18 & 18 & \textBF{181606} &  \textBF{399.37 s} & 367.56 s & 17445197  & 994.80 s & 187.35 s    \\ %3tpr_3x3_2apps
		  						$3\times3$& 45 & 50 & \textBF{47439}  & 388.64 s & 380.17 s &135546    &  \textBF{91.47 s}  &   10.02 s      \\ %5tpr_3x3_5apps
		  						\rowcolor{HRO1!25}[6pt][7pt]$4\times4$& 48 & 58 & \textBF{20851}&  355.95 s & 347.17 s &  287231    &  \textBF{187.56 s} & 29.82 s \\ %3tpr_4x4_1apps
		  						$4\times4^*$  & 48 & 54 & \textBF{235214} & \textBF{1345.67 s} & 1231.92 s & 18346301 & \multicolumn{2}{c}{\centering T/O\textsuperscript{**}} \\ \bottomrule %3tpr_4x4_3apps
		  						\rowcolor{white}\multicolumn{9}{l}{\footnotesize{\textsuperscript{*}unsatisfiable, \textsuperscript{**}timeout set to 1800 s}}
		  					\end{tabular}
		  					
	  					\end{center}
  					\vspace*{-0.7cm}
	  				\end{myblock}
	  			%	\vfill
	  				\begin{myblock}{Conclusion and Future work}{C}
	  					\begin{itemize}
	  						\item[\color{HRO1}$\blacktriangleright$] Approach that allows the description of deadline-constrained periodic application and heterogeneous hardware architectures
	  						\item[\color{HRO1}$\blacktriangleright$] Integration of QF--IDL into ASP Solver leads to a more efficient search 
	  						\item[\color{HRO1}$\blacktriangleright$] Extension towards design space exploration including multi-objective optimization in the future
	  					\end{itemize}
	  				\end{myblock}
	  				\begin{myblock}{References}{R}
	  					\vspace*{-1cm}
	  					\small
						\bibliographystyle{unsrt}
	  					\bibliography{library}
	  				\end{myblock}
  				\end{tcolorbox}
  			\end{column}
  		\end{columns}
  	\end{frame}
  \end{document}
