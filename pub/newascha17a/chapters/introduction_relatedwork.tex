\section{Introduction and Related Work}
With the ever growing demand for highly complex embedded systems in both industrial and consumer environments, efficient design techniques gain progressively more importance. 
To satisfy the requested productivity constraints, the abstraction of embedded systems design has been raised to the electronic system level. 
In system synthesis, a task-level behavioral description is transformed into a structural representation considering certain constraints such as available computational (CPUs, DSPs,  hardware accelerators) and communication (buses, routers) resources as well as extra-functional requirements \cite{Gerstlauer2009}. \par
To perform a holistic synthesis, resources have to be selected from an architectural template, 
computational tasks and messages have to be mapped onto computational resources and routed over the allocated communication infrastructure, respectively, and finally, 
a schedule has to be determined that complies with given timing constraints like latency or throughput requirements.\par
In order to cope with the ever increasing complexity of functional and non-functional requirements of embedded systems, 
symbolic techniques have traditionally been employed in embedded system synthesis since the early 2000s, e.g. \cite{Haubelt2003,Lukasiewycz2009,Lukasiewycz2012}. 
While the encoding of feasible mapping and routing decisions into Boolean formulas led to efficient synthesis frameworks by leveraging enhancements in state-of-the-art Boolean satisfiability (SAT) solvers, 
numerical and non-linear problems (such as scheduling) are not easily representable in Boolean logic. 
As a consequence, SMT-based techniques, i.e. the combination of Boolean logic (traditionally SAT) and variant background theories, have been developed in the domain of embedded systems synthesis \cite{Satish2007,Reimann2010,Andres2015,Biewer2015,Liu2011}. 
In one of the first works on SMT-based approaches, Reimann et al.\ \cite{Reimann2010} integrate a background theory solver to evaluate partial assignments of an underlying SAT-solver. 
In contrast to the work at hand, the authors do not consider scheduling decisions. 
In \cite{Reimann2011}, they extend their approach to constructing schedules though only consider complete assignments, i.e.\ mapping and routing is already completed.\par
The authors of \cite{Liu2011} present another SMT-based method for scheduling analysis in a background theory. 
Similar to the proposed approach in this paper, they utilize \emph{quantifier free integer difference logic} (QF--IDL) as background theory to identify valid schedules. 
The main advantages of QF--IDL are its decidability in polynomial time%\footnote{Note that the overall system synthesis problem is $\mathcal{NP}$--hard.} 
~\cite{Sebastiani2007} and the possibility to leverage the solving process to directly analyze and propagate observed conflicts. 
Yet, the work of \cite{Liu2011} does not support cyclic or periodic applications nor heterogeneous architectures whereas our approach considers a much more sophisticated system model allowing us to synthesize cyclic and periodic behavior.
%While the approach in \cite{Liu2011} only considers acyclic and aperiodic applications as well as homogeneous hardware architectures, our work provides a more sophisticated system model as it supports additionally cyclic and periodic applications that may be implemented on heterogeneous architectures. 
%While they also utilize QF--IDL, they do not support cyclic and periodic applications nor heterogeneous architectures. 
\par The work of Andres and Biewer et al. in \cite{Andres2015} and \cite{Biewer2015} is the most related to our approach. 
They suggest to replace the SAT-solver with an \emph{answer set programming} (ASP) solver as it has been shown to decide routing options more efficiently. 
To analyze timing constraints, they use separate ASP and QF--IDL-based SMT-solvers which communicate indicator variables through a shared text file. 
Thus, in this configuration, an evaluation of partial assignments is not supported.
%Furthermore, the applications considered in \cite{Andres2015} and \cite{Biewer2015} only contain simple linear dependencies and assume the deadline of applications to match their period. 
Furthermore, the system model considered in \cite{Andres2015} and \cite{Biewer2015} is very restrictive. It only allows the representation of simple linear applications where deadlines of tasks have to be smaller than their corresponding periods. 
In combination with a homogeneous hardware architecture and unconstrained mapping options, the authors are able to exploit symmetries in the synthesis problem and use advanced heuristics which simplifies the solving accordingly.

\textbf{Contributions:} In this paper, we first present a mathematical formulation that allows us to describe more sophisticated system models for streaming applications on heterogeneous hardware architectures compared to previous work. 
In particular, this includes deadline-constrained periodic and cyclic applications. Second, we integrate the background theory \emph{QF--IDL} directly into the state-of-the-art ASP-Solver \emph{clingo} \cite{gekakaosscwa16a} which results in a much tighter coupling between background and foreground theories. Third, individual propagators check extra-functional constraints early in the decision process by utilizing \emph{partial assignment checking}. Thus, large infeasible areas of design points can be excluded from the search space more efficiently.

\textbf{Paper organization: } 
In Sec.~\ref{sec:model}, a detailed presentation of the underlying specification model is given. 
The proposed synthesis framework including binding, routing, and scheduling encodings as well as first results are presented in Sec.~\ref{sec:synthesis}. 
Finally, Sec.~\ref{sec:conclusion} concludes this work and discusses future challenges. 
%\newpage