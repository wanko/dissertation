\section{Related Work}
\label{sec:relatedwork}
In order to cope with the ever increasing complexity of functional and non-functional requirements of embedded systems, symbolic techniques have been employed in embedded systems synthesis since the early 2000s, e.g. \cite{Haubelt2003,Lukasiewycz2009,Lukasiewycz2012}. While the encoding of feasible mapping and routing decisions into Boolean formulas led to efficient synthesis frameworks by leveraging enhancements in state-of-the-art SAT-solvers, numerical and non-linear problems (such as scheduling) are not easily representable in Boolean logic. However, other solving technologies, so-called background theories, are practical for such problems. This is why SMT-based techniques, i.e. the combination of Boolean logic (traditionally SAT) and variant background theories, have been developed in the domain of embedded systems synthesis \cite{Reimann2010,Andres2015,Biewer2015,Liu2011}. One of the first works on SMT-based approaches in systems synthesis has been proposed by Reimann et al. \cite{Reimann2010}. Here, the authors integrate a background theory solver to evaluate partial assignments of the underlying SAT-solver. In contrast to the work at hand, the authors only consider mapping and routing decisions. In \cite{Reimann2011}, they extend their approach to constructing schedules though only consider complete assignments in the background theory. Similar, the authors of \cite{Liu2011} present another SMT-based method for scheduling analysis in a background theory. While they also utilize QF--IDL, they do not support cyclic and periodic applications nor heterogeneous architectures. \par The work of Andres and Biewer et al. \cite{Andres2015,Biewer2015,Biewer2014,Biewer2014a} is the most related to our approach. They suggest to replace the SAT-solver with an ASP-solver as it has been shown to decide routing options more efficiently. To analyze timing constraints, they use separate ASP and QF--IDL-based SMT-solvers which communicate common variables through a shared text file. Thus, in this configuration, an evaluation of partial assignments is not supported. Furthermore, the applications considered in \cite{Andres2015} and \cite{Biewer2015} only contain simple linear dependencies and assume the deadline of applications to match their period. In this paper, we aim to integrate the background theory into the ASP-solver in order to achieve a much tighter coupling between both solvers and extend the support for modeling series-parallel and cyclic applications (cf. Sec.~\ref{sec:model}) as those are typical for current multimedia systems. Finally, we remove the restriction of matching deadlines and periods (i.e. applications may have larger deadlines than periods) in order to generalize the approach for a wider range of applications and allow functionally pipelined execution. 