\section{Specification Model}
\label{sec:model}
The emphasis of this work is placed on the system synthesis of embedded systems. 
%As characterized by the \emph{X-chart} approach \cite{Gerstlauer2009}, 
%a behavioral description is therefore transformed into a structural representation under consideration of given constraints. 
In this section, a mathematical description of the graph-based behavioral specification model consisting of deadline constrained periodic applications and a heterogeneous architecture template is presented. %\vspace*{-3mm}
%\subsection{Application}

\vspace*{.5mm}
\textbf{Application: }
The set of applications $A$ is composed of independent applications $A_i$. 
Each application $A_i$ is defined as a directed graph encoded by the quintuple $A_i=(V_{A}^i,E_{A}^i,P_i,D_i,s_i)$. 
Here, the vertices $V_A^i = T_i\cup C_i$ represent the union of computational tasks $T_i$ and communication messages $C_i$, 
the edges $E_A^i\subseteq T_i\times C_i \cup C_i\times T_i$ represent dependencies between tasks and messages. 
Each message $c\in C_i$ has exactly one predecessor task $pre(c)=t_x \text{ with } (t_x,c)\in E_A^i$ and one successor task $succ(c)=t_y \text{ with } (c,t_y)\in E_A^i$, 
i.e. interprocess communication is characterized in a point-to-point fashion.
A period $P_i\in\mathbb{N}$ determines the time after which the execution is repeated, 
an absolute deadline $D_i\in\mathbb{N}$ sets the maximum latency, 
and the function $s_i:C_i\mapsto\mathbb{N}$ assigns an index delay to each message. 
%The index delay $s_i(c)=n$ of a message $c\in C_i$ encodes a maximum communication delay of $n$ iterations. 
That is, the message $c$ sent by task $t_x=pre(c)$ in iteration $m$ must arrive at task $t_y=succ(c)$ at the latest in iteration $m+s_i$. 
Note that cyclic graphs are only valid if the sum of index delays along a path $p=\langle t_x,c_y,\ldots,c_z,t_x\rangle$ is greater than zero: 
$$\forall p\in\{\langle t_x,c_y,\ldots,c_z,t_x\rangle\mid (t_x, c_y), \ldots ,(c_z,t_x)\in E_A^i\}:\sum_{c_k\in p}s(c_k)> 0.$$ \par
In Fig.\ \!\ref{fig:specmodel}, two sample applications $A_1$ and $A_2$ are shown on the left and right, respectively: $$A_1=(\{t_1,t_2,t_3,t_4\}\cup\{c_1,c_2,c_3\},E_A^1,7,12,\{c_x\mapsto 0\mid x=1,2,3\})$$% and
$$A_2=(\{t_5,t_6\}\cup\{c_4,c_5\},E_A^2,10,8,\{c_4\mapsto 0,c_5\mapsto 1\}).$$
%Note that cyclic graphs are only valid if the sum of index delays along a path $p=\langle t_i,c_j,\ldots,c_k,t_i\rangle$ is greater than zero: 
%$$\forall p=\langle t_i,c_i,\ldots,c_j,t_i\rangle\mid t_i\in T,c_i,c_j\in C:\sum_{c_k\in p}s(c_k)> 0.$$ 
The cycle $\langle t_5,c_4,t_6,c_5,t_5\rangle$ in application $A_2$ is validated by the index delay $s(c_5)=1$, indicating that the message $c_5$ sent by task $t_6$ must arrive at task $t_5$ in the next iteration.

\vspace*{.5mm}
%\subsection{Architecture Template}
\textbf{Architecture Template: }
The architecture, or \emph{platform} template is similarly modeled as a directed graph $P=(V_P,E_P,D_P)$. 
The set of \mbox{vertices} $V_P=R_{t}\cup R_{rsu} $ is partitioned into tiles $R_{t}$, containing computational resources and router switching units (RSU) $R_{rsu}$. 
Similar to \cite{Biewer2015}, each RSU internally consists of independent crossbar switches and corresponding input buffers to allow concurrent routing of messages. 
That is, two (or more) messages may be routed simultaneously over one router if they have different destinations. 
A directed edge $l\in E_P\subseteq V_P\times V_P$ models a link between two resources. 
Finally, the routing delay $D_P\in\mathbb{N}$ determines the time for each message to get routed over one router. 
A (simplified) platform template is depicted in the center of Fig.~\ref{fig:specmodel}. 
It contains four tiles $R_t=\{r_1,\ldots,r_4\}$ and four RSUs $R_{rsu}=\{rsu_1,\ldots,rsu_4\}$. 
Each tile is connected to one RSU whereas the RSUs are arranged and interconnected in a regular mesh topology. 
Note that bidirectional arrows represent two separate links. 
For example, the connection between $r_1$ and $rsu_1$ represents the directed edges $l_1=(r_1,rsu_1)$ and $l_2=(rsu_1,r_1)$.

\vspace*{.5mm}
%\subsection{Specification Model}
\textbf{Specification Model: }
The specification connects the set of applications and the platform template. 
Formally, it is defined as the quadruple $S=(A,P,M,e)$. Here, $A,P$ and $M$ conform to a set of applications, a platform template and a set of mapping edges, respectively. 
A mapping edge $m=(t,r)\in M\subseteq T\times R_t$ indicates that task $t\in T$ may be executed on resource $r\in R_t$. 
The function $e:M\mapsto\mathbb{N}$ assigns a worst case execution time (WCET) to each mapping edge $m\in M$. 
Note that specifying more than one mapping edge $m$ per task with different WCETs permits the modeling of heterogeneous architectures.
In Fig.~\ref{fig:specmodel}, the mapping edges are displayed as dashed curved arrows. 
For example, the task $t_1$ may be executed on tile $r_1$ and $r_2$ with worst case execution times of $3$ and $2$, respectively.
\begin{figure}[tb]
	\centering
%	\resizebox{0.4\textwidth}{!}{%
		%\begin{tikzpicture}[>=stealth]
%		\tikzstyle{axis} = [->, black, very thick]
%		\tikzstyle{link} = [<->, black!50, very thick]
%		\tikzstyle{task} = [anchor=west,shape=rectangle,thick,text=white,minimum height=0.5cm, draw=black, fill=myblue, align=center, rounded corners=2,inner sep=0]
%		\tikzstyle{difftask} = [shape=circle,thick,text=white,minimum height=0.5cm, draw=black, fill=myblue, align=center,inner sep=2]
%		\tikzstyle{router} = [diamond,draw,fill=black!30!yellow,inner sep=0]
%		\tikzstyle{resource} = [rectangle,minimum width=0.6cm,minimum height=0.4cm,draw,fill=black!50!green,text=white]
%		\tikzstyle{route} = [->,black,line width=1.5]
%		\node [router] (v1) at (-3.8,2.4) {\footnotesize$rsu_1$};
%		\node [router] (v2) at (-2.2,2.4) {\footnotesize$rsu_2$};
%		\node [router] (v4) at (-3.8,0.8) {\footnotesize$rsu_3$};
%		\node [router] (v5) at (-2.2,0.8) {\footnotesize$rsu_4$};
%		
%		\draw [link] (v1) edge (v2);
%		\draw [link] (v1) edge (v4);
%		\draw [link] (v2) edge (v5);
%		\draw [link] (v4) edge (v5);
%		
%		\node [resource] (v10) at (-4.6,3.2) {$r_1$};
%		\node [resource] (v11) at (-3,3.2) {$r_2$};
%		\node [resource] (v13) at (-4.6,1.6) {$r_3$};
%		\node [resource] (v14) at (-3,1.6) {$r_4$};
%		
%		\draw [link] (v10) edge (v1);
%		\draw [link] (v11) edge (v2);
%		\draw [link] (v13) edge (v4);
%		\draw [link] (v14) edge (v5);
%		
%		\node [difftask] (t1) at (-6.5,3) {$t_1$};
%		\node [difftask,fill=black!40!red,inner sep=0] (c1) at (-6.0,2) {\footnotesize$c_1$};
%		\node [difftask,fill=black!20!orange,inner sep=0] (c2) at (-7.0,2) {\footnotesize$c_2$};
%		\node [difftask] (t2) at (-6.5,1) {$t_2$};
%		
%		\draw [axis] (t1) edge (c1);
%		\draw [axis] (c1) edge (t2);
%		
%		\draw [axis,dashed] (t2) edge[bend right=14] node[midway,above=-0.4]{\tiny$2$}(v14);
%		\draw [axis,dashed] (t2) edge[bend left=14] node[midway,above=-0.4]{\tiny$4$}(v13);
%		\draw [axis,dashed] (t1) edge[bend left=10] node[midway,above=-0.4]{\tiny$5$}(v10);
%		\draw [axis,dashed] (t1) edge[bend right=14] node[midway,above=-0.4]{\tiny$7$}(v11);
%		
%		%\draw [route,black!40!red](v10) -- (-4.6,2.9) -- (-4.1,2.4) -- (-4.1,0.8) -- (-3.8,0.5) -- (-2.2,0.5)  --    (v14.south);
%		%\draw [route,black!20!orange](v14.east) -- (-1.9,0.8) -- (-1.9,2.4) -- (-2.2,2.7) -- (-3.8,2.7)  --    (v10.east);
%		\draw [axis] (t2) edge (c2);
%		\draw [axis] (c2) edge (t1);
%		\node[inner sep=0] at (-6.5,3.5) {\tiny $P_1=10, D_1=8$};
%		\node[] at (-6.5,0.4) {\tiny $s(c_1)=0, s(c_2)=1$};
%		
%		\node [difftask] (t3) at (-0.2,3) {$t_3$};
%		\node [difftask,fill=black!40!red,inner sep=0] (c3) at (0.3,2.0) {$c_3$};
%		\node [difftask,fill=black!40!red,inner sep=0] (c4) at (-0.7,2.0) {$c_4$};
%		\node [difftask] (t4) at (0.3,1.0) {$t_4$};
%		\node [difftask] (t6) at (-0.7,1.0) {$t_5$};
%		\draw [axis] (t3) edge (c4);
%		\draw [axis] (c4) edge (t6);
%		\draw [axis] (t3) edge (c3);
%		\draw [axis] (c3) edge (t4);
%		
%		
%		\draw [axis,dashed] (t3) edge[bend left=10] node[midway,above=-0.4]{\tiny$3$}(v11);
%		\draw [axis,dashed] (t3) edge[bend right=16] node[midway,above=-0.4]{\tiny$2$}(v10);
%		\draw [axis,dashed] (t6) edge[bend right=14] node[midway,above=-0.4]{\tiny$4$}(v14);
%		\draw [axis,dashed] (t4) edge[bend right=21] node[near start,above=-0.4]{\tiny$2$}(v13);
%		\node[inner sep=0] at (-0.1,3.5) {\tiny $P_2=15, D_2=15$};
%		\node[] at (-0.1,0.4) {\tiny $s(c_3)=0, s(c_4)=0$};
%		\node[] at (-6.5,2) {$A_1$};
%		\node at (-0.2,2) {$A_2$};
%	\end{tikzpicture}
\begin{tikzpicture}[>=stealth]
	\tikzstyle{axis} = [->, black, very thick]
	\tikzstyle{link} = [<->, black!50, very thick]
	\tikzstyle{task} = [anchor=west,shape=rectangle,thick,text=white,minimum height=0.5cm, draw=black, fill=myblue, align=center, rounded corners=2,inner sep=0]
	\tikzstyle{difftask} = [shape=circle,thick,text=white,minimum height=0.5cm, draw=black, fill=myblue, align=center,inner sep=2]
	\tikzstyle{router} = [diamond,draw,fill=black!30!yellow,inner sep=0]
	\tikzstyle{resource} = [rectangle,minimum width=0.6cm,minimum height=0.4cm,draw,fill=black!50!green,text=white]
	\tikzstyle{route} = [->,black,line width=1.5]
	\node [router] (v1) at (-3.8,2.4) {\footnotesize$rsu_1$};
	\node [router] (v2) at (-2.2,2.4) {\footnotesize$rsu_2$};
	\node [router] (v4) at (-3.8,0.8) {\footnotesize$rsu_3$};
	\node [router] (v5) at (-2.2,0.8) {\footnotesize$rsu_4$};
	
	\draw [link] (v1) edge (v2);
	\draw [link] (v1) edge (v4);
	\draw [link] (v2) edge (v5);
	\draw [link] (v4) edge (v5);
	
	\node [resource] (v10) at (-4.6,3.2) {$r_1$};
	\node [resource] (v11) at (-3,3.2) {$r_2$};
	\node [resource] (v13) at (-4.6,1.6) {$r_3$};
	\node [resource] (v14) at (-3,1.6) {$r_4$};
	
	\draw [link] (v10) edge (v1);
	\draw [link] (v11) edge (v2);
	\draw [link] (v13) edge (v4);
	\draw [link] (v14) edge (v5);
	
	\node [difftask] (t1) at (-0.3,3) {$t_5$};
	\node [difftask,fill=black!40!red,inner sep=0] (c1) at (0.2,2) {\footnotesize$c_4$};
	\node [difftask,fill=black!20!orange,inner sep=0] (c2) at (-0.8,2) {\footnotesize$c_5$};
	\node [difftask] (t2) at (-0.3,1) {$t_6$};
	
	\draw [axis] (t1) edge (c1);
	\draw [axis] (c1) edge (t2);
	
	\draw [axis,dashed] (t2) edge[bend right=22] node[midway,below]{\tiny$2$}(v13);
	\draw [axis,dashed] (t2) edge[bend left=14] node[midway,above]{\tiny$4$}(v14);
	\draw [axis,dashed] (t1) edge[bend left=10] node[midway,above]{\tiny$5$}(v10);
	\draw [axis,dashed] (t1) edge[bend right=14] node[midway,below]{\tiny$7$}(v11);
	
	%\draw [route,black!40!red](v10) -- (-4.6,2.9) -- (-4.1,2.4) -- (-4.1,0.8) -- (-3.8,0.5) -- (-2.2,0.5)  --    (v14.south);
	%\draw [route,black!20!orange](v14.east) -- (-1.9,0.8) -- (-1.9,2.4) -- (-2.2,2.7) -- (-3.8,2.7)  --    (v10.east);
	\draw [axis] (t2) edge (c2);
	\draw [axis] (c2) edge (t1);
	\node[inner sep=0] at (-0.3,3.5) {\tiny $P_2=10, D_2=8$};
	\node[] at (-0.3,0.5) {\tiny $s(c_5)=0, s(c_5)=1$};
	
	\node [difftask] (t3) at (-7.6,3) {$t_1$};
	\node [difftask,fill=black!40!red,inner sep=0] (c3) at (-6,2) {$c_3$};
	\node [difftask,fill=black!40!red,inner sep=0] (c4) at (-7.6,2) {$c_1$};
	\node [difftask,fill=black!40!red,inner sep=0] (c5) at (-6.8,1) {$c_2$};
	\node [difftask] (t4) at (-6,1) {$t_3$};
	\node [difftask] (t5) at (-6,3) {$t_4$};
	\node [difftask] (t6) at (-7.6,1) {$t_2$};
	\draw [axis] (t3) edge (c4);
	\draw [axis] (c4) edge (t6);
	\draw [axis] (t6) edge (c5);
	\draw [axis] (c5) edge (t4);
	\draw [axis] (t4) edge (c3);
	\draw [axis] (c3) edge (t5);
	
	
	\draw [axis,dashed] (t3) edge[bend left=15] node[very near end,above]{\tiny$2$}(v11);
	\draw [axis,dashed] (t3) edge[bend right=26] node[near end,below]{\tiny$3$}(v10);
	\draw [axis,dashed] (t6) edge[bend right=20] node[near end,right]{\tiny$2$}(v10);
	\draw [axis,dashed] (t4) edge[bend right=21] node[near start,above]{\tiny$2$}(v13);
	\draw [axis,dashed] (t5) edge[bend left=15] node[near start,below]{\tiny$1$}(v10);
	\node[inner sep=0] at (-6.9,3.5) {\tiny $P_1=7, D_1=12$};
	\node[] at (-6.8,0.5) {\tiny $s(c_1)=s(c_2)=s(c_3)=0$};
	\node[] at (-0.3,2) {$A_2$};
	\node at (-6.8,2) {$A_1$};
\end{tikzpicture}
%	}
	\vspace*{-8mm}
	\caption{Example specification model containing two applications $A_1$ (left) and $A_2$ (right), an architecture template implementing an NoC with four computational resources and four RSUs (center), and nine mapping edges}
	\label{fig:specmodel}
	\vspace*{-6mm}
\end{figure}
