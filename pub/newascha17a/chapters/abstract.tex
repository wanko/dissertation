%\begin{abstract}
%	%\blindtext
%	The design of embedded systems is becoming continuously more complex such that hardware and software components cannot be examined independently anymore. Hence, the application of
%	efficient high-level design methods are crucial for competitive results regarding design time and compatibility. 
%	In this paper, we propose a system synthesis approach based on a combination of \emph{Boolean constraint solving} and \emph{background theories}. While \emph{answer set programming} is utilized to decide combinatorial variables (i.e. mapping and routing), the scheduling is outsourced to a tightly connected \emph{quantifier free integer difference logic} solver which is able to evaluate \emph{partial} assignments. Preliminary experimental results indicate that this approach finds infeasible design points much faster than the evaluation of complete assignments.
%\end{abstract}

%\begin{abstract}
%	The design of embedded systems is becoming continuously more complex such that the application of efficient high level design methods are crucial for competitive results regarding design time and performance. 
%	Recently, combined \emph{Answer Set Programming} (ASP) and \emph{Quantifier Free Integer Difference Logic} (QF--IDL) solving has been shown to be a promising approach in system synthesis. 
%	While ASP is utilized to decide combinatorial variables (i.e. mapping and routing), the scheduling is relayed to the QF--IDL solver in a background theory. 
%	However, the coupling of foreground and background theory is crucial for the performance of these approaches. 
%	In this paper, we present a novel approach which permits checking \emph{partial assignments} in the background theory. 
%	Preliminary experimental results indicate that this approach finds infeasible design points much faster than the evaluation of complete assignments.
%\end{abstract}

%\begin{abstract}
%	The design of embedded systems is becoming continuously more complex such that the application of efficient high level design methods are crucial for competitive results regarding design time and performance. 
%	Recently, combined \emph{Answer Set Programming} (ASP) and \emph{Quantifier Free Integer Difference Logic} (QF--IDL) solving has been shown to be a promising approach in system synthesis. 
%	While ASP is utilized to decide combinatorial variables (i.e. mapping and routing), the scheduling is relayed to the QF--IDL solver in a background theory. 
%	However, the coupling of foreground and background theory is crucial for the performance of these approaches. 
%	In this paper, we present a novel approach which combines the checking of \emph{partial assignments} in the background theory with a sophisticated system model description 
%	which allows us representing a wide variety of streaming applications. By utilizing individual propagators for variant extra-functional constraints, early exclusion of infeasible areas of design points is achieved.     
%%	Preliminary experimental results indicate that this approach finds infeasible design points much faster than the evaluation of complete assignments.
%\end{abstract}




\begin{abstract}
	The design of embedded systems is becoming continuously more complex such that efficient system-level design methods are becoming crucial. 
	Recently, combined \emph{Answer Set Programming} (ASP) and \emph{Quantifier Free Integer Difference Logic} (QF--IDL) solving has been shown to be a promising approach in system synthesis. 
	However, this approach still has several restrictions limiting its applicability. 
	In the paper at hand, we propose a novel ASP modulo Theories (\emph{ASPmT}) system synthesis approach, which (i) supports more sophisticated system models, 
	(ii) tightly integrates the QF--IDL solving into the ASP solving, and (iii) makes use of partial assignment checking. 
	As a result, more realistic systems are considered and an early exclusion of infeasible solutions improves the entire system synthesis. 
	%	Preliminary experimental results indicate that this approach finds infeasible design points much faster than the evaluation of complete assignments.
\end{abstract}