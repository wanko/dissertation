\section{Conclusion and Future Work}
\label{sec:conclusion}
In this paper, we presented a novel ASPmT-based approach towards symbolic system synthesis that supports a more expressive system model compared to previous work. Especially, the proposed approach allows for the description of deadline-constrained periodic applications as well as heterogeneous hardware architectures.
Furthermore, the tight integration of QF--IDL into the ASP solver allows us to evaluate partial assignments in order to prune the search space more efficiently. 
First experimental results showed the general applicability of our approach. 

In the future, we aim to improve the performance of used propagators for example by using lazy variable creation or incorporating domain-specific heuristics. 
Since our ASPmT framework encompasses all relevant information,
the model is easily adaptable and features may be added comfortably.
Namely, we will extend the framework towards design space exploration including multi-objective optimization techniques.
%\blindtext
%\begin{itemize}
% \item introduce system using ASP modulo theory with following advantages:
% 	\begin{itemize}
% 		\item more expressive synthesis model
% 		\item efficient solving due to detection of invalid schedules for partial bindings,routings
% 		\item periodic scheduling without introducing new tasks
% 		\item formulation encompassing binding, routing and scheduling
% 	\end{itemize}
% \item promising results for previous synthesis model
% \item currently not scalable for periodic schedules due to many unnecessary choices, alleviated by adding variables lazily in the future
% \item scales up to NxN with Y apps with Z comms
% \item future capabilities
% 	\begin{itemize}
% 		\item multi-objective optimization
% 		\item subset of representative optimal design points
% 	\end{itemize}
% \item future techniques
% 	\begin{itemize}
% 		\item domain-specific heuristic similar to \cite{Andres2015}
% 		\item additional propagators (acyclicity)
% 	\end{itemize}
%\end{itemize}
%\begin{table}[]
%	\centering
%	\caption{My caption}
%	\label{my-label}
%	\begin{tabular}{@{}llll@{}}
%		\toprule
%		\#Resources              & \#Tasks & \#Apps & Time     \\ \midrule
%		\multirow{2}{*}{$2\times2$} & 20      & 1      & 2.08s    \\
%		& 28      & 2      & 4.85s    \\\midrule
%		\multirow{4}{*}{$3\times3$} & 27      & 1      & 8.24s    \\
%		& 45      & 5      & 13.99s   \\
%		& 45      & 2      & 121.28s  \\
%		& 45      & 1      & TO       \\\midrule
%		\multirow{5}{*}{$4\times4$} & 48      & 3      & 191.871s \\
%		& 48      & 1      & 280.86s  \\
%		& 80      & 6      & 543s     \\
%		& 80      & 5      & 577.63s  \\
%		& 80      & 1      & TO       \\\midrule
%		$5\times5$                  & 75      & 5      & TO       \\ \bottomrule
%	\end{tabular}
%\end{table}
