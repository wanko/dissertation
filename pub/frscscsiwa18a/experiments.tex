%!TEX root = paper.tex
\section{System and Experiments}\label{sec:experiments}
\label{sec:sysandexp}

In this section, we introduce \fluto, our new system for hybrid metabolic network completion, and empirically evaluate its performance.
The system relies on the hybrid encoding described in Section~\ref{sec:approach}
along with the hybrid solving capacities of \clingo~\citep{gekakaosscwa16a} for implementing the combination of ASP and LP.
We use \clingo~5.2.0 %\footnote{To be precise, https://github.com/potassco/clingo/tree/wip commit 99a17f6}
incorporating as LP solvers either \cplex~12.7.0.0 or \lpsolve~5.5.2.5 via their respective \python\ interfaces.
We describe the details of the underlying solving techniques in a separate paper and focus below on application-specific aspects.

The output of \fluto\ consists of two parts.
First, the completion $R''$, given by instances of predicate \texttt{completion}, and
second, an assignment of floats to (metabolic flux variables $v_r$ for) all $r\in R\cup R''$.
In our example, we get
\begin{align*}%*
R''=\{\texttt{completion}(r_6), \texttt{completion}(r_8), \texttt{completion}(r_9)\}
\\
\text{ and } \{\SeedReaction=49999.5, r_9=49999.5, r_3=49999.5, r_2=49999.5, \\
 \ExportReaction=99999.0, r_6=49999.5, r_5=49999.5, r_4=49999.5\}.
\end{align*}%*

Variables assigned $0$ are omitted.
Note the flux value $r_8=0$ even though $r_8\in R''$.
This is to avoid the self-activation of cycle $C$, $D$ and $E$.
By choosing $r_8$, we ensure that the cycle has been externally initiated at some point
but activation of $r_8$ is not necessary at the current steady state.

We analyze
(i)   the impact of different system configurations
(ii)  the quality of \fluto's approach to metabolic network completion, and
(iii) compare the quality of \fluto's solutions with other approaches.
To have a realistic setting,
we use degradations of a functioning metabolic network of \textit{Escherichia coli}~\citep{Reed2003} comprising 1075 reactions.
The network was randomly degraded by 10, 20, 30 and 40 percent,
creating 10 networks for each degradation
by removing reactions until the target reactions were inactive according to \emph{Flux Variability Analysis}~\citep{Becker2007}.
90 target reactions with varied reactants were randomly chosen for each network, yielding 3600 problem instances in total ~\citep{Prigent2017}.
The reference network consists of reactions of the original metabolic network.

We ran each benchmark on a Xeon E5520 2.4 GHz processor under Linux limiting RAM to 20~GB.
At first,
we investigate two alternative optimization strategies for computing completions of minimum size.
The first one, \emph{branch-and-bound}~(\bb), iteratively produces solutions of better quality until the optimum is found and the other,
\emph{unsatisfiable core}~(\usc), relies on successively identifying and relaxing unsatisfiable cores until an optimal solution is obtained.
Note that we are not only interested in optimal solutions
but if unavailable also solutions activating target reactions without trivially restoring the whole reference network.
In \clingo, \bb\ naturally produces these solutions in contrast to \usc.
Therefore, we use \usc\ with stratification~\citep{anbole13a}, which provides at least some suboptimal solutions.

\subsection{System configurations}
\label{sec:sysconf}
\newcommand{\shade}[2]% #1 = cell, #2 = max
{%
    \cellcolor{black!\xinttheiexpr 80*#1/#2-30\relax}%
       {#1}%
}%


    \begin{table}[t]
        \begin{center}
            \begin{tabular}{r|*{5}{c}}
				\backslashbox{\core{}}{\prop{}} & 0 & 25 & 50 & 75 & 100\\\hline
0 & \shade{383.20}{600}(130) & \shade{388.51}{600}(134) & \shade{384.46}{600}(133) & \shade{388.45}{600}(137) & \shade{398.21}{600}(134)\\
25 & \shade{385.05}{600}(132) & \shade{385.95}{600}(133) & \shade{391.84}{600}(131) & \shade{382.29}{600}(137) & \shade{401.73}{600}(134)\\
50 & \shade{383.95}{600}(131) & \shade{377.51}{600}(123) & \shade{385.46}{600}(132) & \shade{391.05}{600}(137) & \shade{399.88}{600}(141)\\
75 & \shade{358.77}{600}(129) & \shade{360.54}{600}(127) & \shade{356.89}{600}(131) & \shade{390.69}{600}(137) & \shade{399.36}{600}(134)\\
100 & \shade{376.03}{600}(133) & \shade{370.75}{600}(132) & \shade{375.77}{600}(133) & \shade{389.77}{600}(139) & \shade{401.18}{600}(139)\\

            \end{tabular}
        \end{center}
        \caption{Comparision of propagation and core minimization heuristics for \bb. \label{tab:pchbb}}
    \end{table}


    \begin{table}[t]
        \begin{center}
            \begin{tabular}{r|*{5}{c}}
				\backslashbox{\core{}}{\prop{}} & 0 & 25 & 50 & 75 & 100\\\hline
0 & \shade{297.38}{600}(102) & \shade{296.39}{600}(102) & \shade{296.48}{600}(103) & \shade{299.14}{600}(105) & \shade{475.12}{600}(200)\\
25 & \shade{297.29}{600}(101) & \shade{293.69}{600}(100) & \shade{297.09}{600}(102) & \shade{293.43}{600}(101) & \shade{478.39}{600}(202)\\
50 & \shade{292.65}{600}(102) & \shade{296.43}{600}(102) & \shade{294.4}{600}(103) & \shade{295.48}{600}(102) & \shade{477.67}{600}(200)\\
75 & \shade{331.72}{600}(127) & \shade{336.34}{600}(129) & \shade{331.17}{600}(127) & \shade{294.17}{600}(103) & \shade{476.16}{600}(202)\\
100 & \shade{308.88}{600}(108) & \shade{309.47}{600}(107) & \shade{324.9}{600}(122) & \shade{489.97}{600}(214) & \shade{476.17}{600}(201)\\

            \end{tabular}
        \end{center}
        \caption{Comparision of propagation and core minimization heuristics for \usc. \label{tab:pchusc}}
    \end{table}

The configuration space of \fluto\ is huge.
In addition to its own parameters, the ones of \clingo\ and the respective LP solver amplify the number of options.
We thus concentrate on distinguished features revealing an impact in our experiments.

The first focus are two system options controlling the hybrid solving nature of \fluto.
First, \prop{$n$} controls the frequency of LP propagation:
the consistency of linear constraints is only checked if $n\%$ of atoms are decided.
Second,
the \fluto\ option \core{$n$} invokes the irreducible inconsistent set algorithm~\citep{ostsch12a} whenever $n\%$ of atoms are decided.
This algorithm extracts a minimal set of conflicting linear constraints for a given conflict.
Note that the second parameter depends on the first one,
since conflict analysis may only be invoked if the LP solver found an inconsistency.

The \default{} is to use \core{100}, \prop{0}, and use LP solver \cplex\footnote{We do not present results of \lpsolve\ since it produced inferior results.}.
This allows us to detect conflicts among the linear constraints as soon as possible and only perform expensive conflict analysis on the full assignment.

To get an overview, we conducted a preliminary experiment
using \bb\ and \usc\ with \fluto's default configuration on the 10, 20, and 30 percent degraded networks,
2700 instances in total,
limiting execution time to 20 minutes.
For our performance experiments,
we selected at random three networks with at least one instance
for which \bb\ and \usc\ could find the optimum in 100 to 600 seconds.
With the resulting 270 medium to hard instances,
we examined the cross product of values $n\in\{0,25,50,75,100\}$ for \core{n} and \prop{n}, respectively, limiting time to 600 seconds.

Table~\ref{tab:pchbb} and Table~\ref{tab:pchusc} display the results using \bb\ and \usc\ respectively.
The columns increase the value for \prop{n} and the rows for \core{n} in steps of 25,
i.e., LP propagation becomes less frequent from left to right,
and conflict minimization from top to bottom.
The first value in each cell is the average runtime in seconds and the value in brackets shows the number of timeouts.
The shade of the cells depends on the average runtime,
i.e., the darker the cell, the less performant the combination of propagation and conflict minimization heuristics.

Table~\ref{tab:pchbb} shows that propagation and conflict minimization heuristics have an overall small impact on the performance of \bb\ optimization.
Since \bb\ relies on iterating solutions and learns weaker constraints,
only pertaining to the best known bound, while optimizing,
the improvement step is less constraint compared to \usc.
Due to this, conflicts are more likely to appear later on in the optimization process allowing for less impact of frequent LP propagation and conflict minimization
Nevertheless, we see a slight performance improvement of propagating and conflict minimizing for every partial assignment (\prop{0}, \core{0})
compared to only on full assignments (\prop{100}, \core{100}).
To prove the optimum, the solver is still required to cover the whole search space.
For this purpose, early pruning and conflict minimization may be effective.
Furthermore, we see the best average runtime in the area \prop{0-50} at \core{75}.
That indicates a good tradeoff between the better quality conflicts which prune the search effectively
and the overhead of the costly conflict minimization.
There is no clear best configuration,
 but \prop{25} and \core{75} shows the best tradeoff between average runtime and number of timeouts.

\usc\ on the other hand (Table~\ref{tab:pchusc}), clearly benefits from early propagation and conflict minimization.
The area \prop{0-75} and \core{0-50} has the lowest average runtime and number of timeouts,
best among them \prop{25} and \core{25}
with the lowest timeouts and average runtime that is not significantly different from the best value.
\usc\ aims at quickly identifying unsatisfiable partial assignments
and learning structural constraints building upon each other,
which is enhanced by frequent conflict detection and minimization.
Disabling LP propagation on partial assignments with \usc\ leads to the overall worst performance
and we also see deterioration with \core{75} and \core{100} in the interval \prop{0-50}.
Overall, \usc\ is more effective than \bb\ for the instances and we see a benefit in early LP propagation and conflict minimization as well as in fine-tuning the heuristics at which point both are applied.

    \begin{table}[t]
        \begin{center}
            \begin{tabular}{r|cccccccccccc}
            	 & \multicolumn{2}{c}{\textsc{FR}}& \multicolumn{2}{c}{\textsc{JP}}& \multicolumn{2}{c}{\textsc{TW}}& \multicolumn{2}{c}{\textsc{TR}}& \multicolumn{2}{c}{\textsc{CR}}& \multicolumn{2}{c}{\textsc{HD}}\\
              & \T & \TO & \T & \TO & \T & \TO & \T & \TO & \T & \TO & \T & \TO\\\hline
             \bb  & 400.41 & 154 & 389.68 & 147 & \textbf{360.54} & 127 & 409.33 & 141 & 362.74 & \textbf{120} & 434.54 & 160\\
             \usc & 227.38 &  78 & 293.96 & 100 & 316.54 & 107 & 293.54 & 102 & \textbf{221.84} &  \textbf{74} & 297.32 & 104\\
            \end{tabular}
        \end{center}
        \caption{Comparision of \clingo's portfolio configurations for \bb\ and \usc. \label{tab:search}}
    \end{table}

Now, we focus on the portfolio configurations of \clingo.
Those configurations were crafted by experts to enhance the solving performance of problems with certain attributes.
\review{To examine their impact, we take the best result for \bb\ (\prop{25} and \core{75}) and \usc\ (\prop{25} and \core{25}), }
and employ the following \clingo\ options:
\begin{description}
	\item[\emph{\textsc{FR}}]
      Refers to {\clingo}'s configuration \emph{frumpy} that uses more conservative defaults.
    \item[\emph{\textsc{JP}}]
      Refers to {\clingo}'s configuration \emph{jumpy} that uses more aggressive defaults.
    \item[\emph{\textsc{TW}}]
      Refers to {\clingo}'s configuration \emph{tweety} that is geared toward typical ASP problems.
    \item[\emph{\textsc{TR}}]
      Refers to {\clingo}'s configuration \emph{trendy} that is geared toward industrial problems.
	\item[\emph{\textsc{CR}}]
      Refers to {\clingo}'s configuration \emph{crafty} that is geared towards crafted problems.
    \item[\emph{\textsc{HD}}]
      Refers to {\clingo}'s configuration \emph{handy} that is geared towards larger problems.
\end{description}
For more information on {\clingo}'s configurations, see ~\citep{gekakarosc15a}.

Table~\ref{tab:search} shows the average runtime in seconds (\T) and number of timeouts (\TO)
for all six configurations using \bb\ and \usc\ on the same 270 instances.
Even though \textsc{CR} has slightly higher average runtime for \bb\ compared to \textsc{TW},
it is the overall best configuration.
This configuration is geared towards problems with an inherent structure
compared to randomly generated benchmarks
which fits with the metabolic network completion problem at hand
since the data is taken from an existing bacteria.
Interestingly, \bb\ performs worse under more specific configurations
and favors moderate once like \textsc{TW} and \textsc{CR}.
This might be due to the changing nature of improvement steps as the optimization process goes on
from finding any random solutions to an unsatisfiability proof in the end.
\usc\ on the other hand, benefits from a more structural heuristics in \textsc{CR} and more conservative defaults in \textsc{FR}
which allow the solver to explore and collect conflicts instead of frequently restarting and forgetting.

\subsection{Solution quality}

\begin{table}[t]%
\newcommand{\mc}[3]{\multicolumn{#1}{#2}{#3}}
\centering
\begin{tabular}{r|rr|rr|rr||r} %||rrr
 & \mc{2}{c|}{\f(\bb)}  & \mc{2}{c|}{\f(\usc)} & \mc{2}{c}{\f(\bb+\usc)} & \f(\bb+\usc) \\%& \mc{3}{c}{\verified}\\
\degradation & \sols & \opts & \sols & \opts & \sols & \opts & \verified \\%& \f(\bb+\usc)\\\hline  & \m & \g
10\% (900) & \textbf{900} & \textbf{900} & 892 & 892 & 900 & 900 & 900 \\ %& \textbf{900} & 660 & 56
20\% (900) & \textbf{830} & 669 & 793 & \textbf{769} & 867 & 814 & 867 \\%& \textbf{867} \\ & 225 & 52
30\% (900) & \textbf{718} & 88 & 461 & \textbf{344} & 780 & 382 & 780 \\\hline%& \textbf{780} \\\hline & 61 & 0
all (2700) & \textbf{2448} & 1657 & 2146 & \textbf{2005} & 2547 & 2096 & 2547 \\%& \textbf{2547} & 946 & 108
\end{tabular}
\caption{Comparison of qualitative results.\label{tab:quality}}
\end{table}

\begin{table}[t]%
\newcommand{\mc}[3]{\multicolumn{#1}{#2}{#3}}
\centering
\begin{tabular}{r|rr||rrr}
 & \mc{2}{c||}{\f(\vbs)} & \mc{3}{c}{\verified}\\
\degradation & \sols & \opts & \f(\vbs)\\\hline
10\% (900) & 900 & 900 & 900\\
20\% (900) & 896 & 855 & 896\\
30\% (900) & 848 & 575 & 848\\
40\% (900) & 681 & 68  & 681\\\hline
all (3600) & 3325 & 2398 & 3325
\end{tabular}
\caption{Results using best system options.\label{tab:final}}
\end{table}

Now, we examine the quality of the solutions provided by \fluto.
Table~\ref{tab:quality} gives the number of solutions~(\sols) and optima~(\opts) obtained by \fluto{}~(\f) in its default setting
within 20 minutes
for \bb, \usc\ and the best of both~(\bb+\usc),
individually for each \textsc{degradation} and over\textbf{all}.
\review{The default setting for \fluto\ includes the default configurations for \clingo\ and \cplex.}
The data was obtained in our preliminary experiment using networks with 10, 20, and 30 percent degradation.
%
For 94.3\% of the instances \fluto(\bb+\usc) found a solution within the time limit and 82.3\% of them were optimal.
We observe that \bb\ provides overall more useful solutions but \usc\ acquires more optima,
which was to be expected by the nature of the optimization techniques.
Additionally, each technique finds solutions to problem instances where the other exceeds the time limit,
underlining the merit of using both in tandem.
%
Column~\verified\ shows the quality of solutions provided by \fluto.
Each obtained best solution was checked with \cobrapy~0.3.2~\citep{Ebrahim2013},
a renowned system implementing an FBA-based gold standard (for verification only).
All solutions found by \fluto\ could be verified by \cobrapy.
In detail, \fluto\ found a smallest set of reactions completing the draft network for 77.6\%,
a suboptimal solution for 16.7\%,
and no solution for 5.6\% of the problem instances.

Finally, we change the system configuration and examine how \fluto\ scales on harder instances.
To this end, we use the best configurations from Section~\ref{sec:sysconf},
\prop{25}, \core{75} and \textsc{CR} for \bb, and \prop{25}, \core{25} and \textsc{CR} for \usc,
and rerun the experiment on all 3600 instances.
The results are shown in (Table~\ref{tab:final}).
\f(\vbs) denotes the virtual best results, meaning for each problem instance the best known solution among the two configurations was verified.
For 20\% and 30\% degradation, we obtain additional 29 and 68 solutions and 41 and 193 optima, respectively.
Overall, we find solutions for 92.4\% out of the 3600 instances and 72.1\% of them are optimal.
The number of solutions decreases slightly and the number of optima more drastically with higher degradation.
The results show that \fluto\ is capable of finding correct completions for even highly degraded networks for most of the instances in reasonable time.

\subsection{Comparison to other approaches}

\begin{table}[t]%
\newcommand{\mc}[3]{\multicolumn{#1}{#2}{#3}}
\centering
\begin{tabular}{r|rrr|rrr}
                        & \mc{3}{c|}{\fluto} & \mc{3}{c}{\meneco} \\ %\cline{2-7}
                        & min & average & max & min & average & max \\ \hline
solutions per instance  & 1   & 2.24    & 12  & 1   & 1.88    & 6 \\
reactions per solution  & 1   & 6.66    & 9   & 1   & 6.24    & 9 \\ \hline
verified solutions     & \mc{3}{r|}{100\%} & \mc{3}{r}{73.39\%} \\
instances with only verified solutions & \mc{3}{r|}{100\%} & \mc{3}{r}{72.94\%} \\
instances without verified solutions   & \mc{3}{r|}{0\%}   & \mc{3}{r}{26.61\%} \\
instance with some verified solutions & \mc{3}{r|}{0\%}   & \mc{3}{r}{0.45\%} \\
\end{tabular}
\caption{Comparison of \fluto\ and \meneco\ solutions for 10 percent degraded networks.\label{tab:enumeration}}
\end{table}

\begin{table}[t]%
\newcommand{\mc}[3]{\multicolumn{#1}{#2}{#3}}
\centering
\begin{tabular}{r|r|r|r}
& \fluto & \meneco & \gapfill \\ \hline
verified union                                 & 100\% & 73.39\% & 6.20\% \\
verified union of verified solutions           & 100\% & 72.94\% & NA \\
verified union of unverified solutions         & 0\%   & 0.00\%  & NA\\
verified union of partially verified solutions & 0\%   & 0.45\%  & NA \\
\end{tabular}
\caption{Comparison of \fluto, \meneco\ and \gapfill\ unions for 10 percent degraded networks.\label{tab:union}}
\end{table}

We compare the quality of \fluto\ with \meneco~1.4.3~\citep{Prigent2017} and \gapfill%
\footnote{Update of 2011-09-23 see http://www.maranasgroup.com/software.htm }~\citep{SatishKumar2007}.
\footnote{The results for \meneco\ and \gapfill\ are taken from previous work~\citep{Prigent2017},
where they were run to completion with \emph{no} time limit.}
Both \meneco\ and \gapfill\ are systems for metabolic network completion.
While \meneco\ pursues the topological approach,
\gapfill\ applies the relaxed stoichiometric variant using Inequation~\eqref{eq:stoichiometric:equation:relaxed}.
We performed an enumeration of all minimal solutions to the completion problem
under the topological (\meneco), the relaxed stoichiometric (\gapfill),
and hybrid (\fluto) activation semantics for the 10 percent degraded networks of the benchmark set (900 instances to be completed).

First, we compare the quality of individual solutions of \fluto\ and \meneco.
\footnote{There was no data available for the individual solutions of \gapfill.}
Results are displayed in Table~\ref{tab:enumeration}.
The first two rows give the minimum, average and maximum number of solutions per instance,
and reactions per solution, respectively, for \fluto\ and \meneco.
While \fluto\ finds 19\% more solutions on average and twice as many maximum solutions per instance compared to \meneco,
the numbers of reactions in minimal solutions of both tools are similar.
The next four rows pertain to the solution quality as established by \cobrapy.
First, what percent solutions over all instances could be verified,
second, what percent of instances had verified solutions exclusively,
third, how many instances had no verified solutions at all,
and finally, percent of instances where only a portion of solutions could be verified.
All of \fluto's solutions could be verified,
compared to the 72.04\% of \meneco\ across all solutions and 72.94\% of instances that were correctly solved.
Interestingly, \meneco\ achieves hybrid activation in some but not all solutions for 0.45\% (4) of the instances.
\fluto\ does not only improve upon the quality of \meneco,
but also provides more solution per instances without increasing the number of relevant reactions significantly.

To empirically evaluate the properties established in Section~\ref{sec:union},
and be able to compare to \gapfill, for which only the union of reactions was available,
we examine the union of minimal solutions provided by all three systems
and present the results in Table~\ref{tab:union}.
The four rows show,
first, for what percent of instances the union of solutions could be verified,
second, how many instances had only verified solutions and their union was also verified,
third, the percentage of instances where the union of solutions displayed activation of the target reactions even though all individual solutions did not provide that,
and forth, instances where the solutions were partly verifiable and their union could also be verified.
While again 100\% of \fluto's solutions could be verified, only 73.3\% and 6.2\% are obtained for \meneco\ and \gapfill, respectively, for 10 percent degraded networks.
As reflected by the results, the ignorance of \meneco\ regarding stoichiometry leads to possibly unbalanced networks.
Still, the union of solutions provided a useful set of reactions in almost three quarters of the instances, showing merit in the topological approximation of the metabolic network completion problem.
On the other hand, the simplified view of \gapfill\ in terms of stoichiometry misguides the search for possible completions and eventually leads to unbalanced networks even in the union.
Moreover, \gapfill's ignorance of network topology results in self-activated cycles.
By exploiting both topology and stoichiometry,
\fluto\ avoids such cycles while still satisfying the stoichiometric activation criteria.
The results support the observations made in Section~\ref{sec:union}.
For both \fluto\ and \meneco\, all instances, for which the complete solution set could be verified,
the union is also verifiable,
as well as all unions for instances where \meneco\ established hybrid activation for a fraction of solutions.


%%% Local Variables:
%%% mode: latex
%%% TeX-master: "paper"
%%% End:
