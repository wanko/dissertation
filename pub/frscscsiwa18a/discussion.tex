
\section{Discussion}\label{sec:discussion} % Summary

We presented the first hybrid approach to metabolic network completion
by combining topological and stoichiometric constraints in a uniform setting.
To this end,
we elaborated a formal framework capturing different semantics for the activation of reactions.
Based upon these formal foundations, we developed a hybrid ASP encoding reconciling
disparate approaches to network completion.
The resulting system, \fluto, thus combines the advantages of both approaches and 
yields greatly superior results compared to purely quantitative or qualitative existing systems.
Our experiments show that \fluto\ scales to more highly degraded networks 
and produces useful solutions in reasonable time. %
In fact, all of \fluto's solutions passed the biological gold standard.
The exploitation of the network's topology guides the solver to more likely completion candidates,
and furthermore avoids self-activated cycles, as obtained in FBA-based approaches.
Also, unlike other systems, \fluto\ allows for establishing optimality and address the strict stoichiometric completion problem without approximation.

\fluto\ takes advantage of the hybrid reasoning capacities of the ASP system \clingo{}
for extending logic programs with linear constraints over reals.
This provides us with a practically relevant application scenario for evaluating this hybrid form of ASP.
To us, the most surprising empirical result was the observation that domain-specific heuristic allow for boosting unsatisfiable core based
optimization.
So far, such heuristics have only been known to improve satisfiability-oriented reasoning modes, and usually hampered unsatisfiability-oriented ones
(cf.~\citep{gekakarosc15a}).


%%% Local Variables: 
%%% mode: latex
%%% TeX-master: "paper"
%%% End: 
