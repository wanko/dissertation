
\section{Answer Set Programming with Linear Constraints}\label{sec:background}

For encoding our hybrid problem,
we rely upon the theory reasoning capacities of the ASP system \clingo\ that allows us to extend ASP with linear constraints over reals
(as addressed in Linear Programming).
We confine ourselves below to features relevant to our application and refer the interested reader for details to~\citep{gekakaosscwa16a}.

As usual, a \emph{logic program} consists of \emph{rules} of the form
\begin{lstlisting}[mathescape=true,numbers=none]
   a$_0$ :- a$_1$,...,a$_m$,not a$_{m+1}$,...,not a$_n$
\end{lstlisting}
where each \lstinline[mathescape=true]{a$_i$} is either
a \emph{(regular) atom} of form \lstinline[mathescape=true]{p(t$_1$,...,t$_k$)}
where all \lstinline[mathescape=true]{t$_i$} are terms
or
a \emph{linear constraint atom} of form%
\footnote{In \clingo, theory atoms are preceded by `\texttt{\&}'.}
`\lstinline[mathescape=true]@&sum{w$_1$*x$_1$;$\dots$;w$_l$*x$_l$} <= k@'
that stands for the linear constraint
\(
w_1\cdot x_1+\dots+w_l\cdot x_l\leq k
\).
All \lstinline[mathescape=true]{w$_i$} and \lstinline[mathescape=true]{k} are finite sequences of digits with at most one dot%
\footnote{In the input language of \clingo, such sequences must be quoted to avoid clashes.}
and represent real-valued coefficients $w_i$ and $k$.
Similarly all \lstinline[mathescape=true]{x$_i$} stand for the real-valued variables $x_i$.
%
As usual, \lstinline[mathescape=true]{not} denotes (default) \emph{negation}.
A rule is called a \emph{fact} if $n=0$.

Semantically, a logic program induces a set of \emph{stable models},
being distinguished models of the program determined by stable models semantics~\citep{gellif91a}.
%
Such a stable model $X$ is an \emph{LC-stable model} of a logic program $P$,%
\footnote{This corresponds to the definition of $T$-stable models using a \emph{strict} interpretation of theory atoms~\citep{gekakaosscwa16a},
  and letting $T$ be the theory of linear constraints over reals.}
if there is an assignment of reals to all real-valued variables occurring in $P$ that
(i)     satisfies all linear constraints associated with linear constraint atoms in $P$ being     in $X$
and
(ii) falsifies all linear constraints associated with linear constraint atoms in $P$ being not in $X$.
%
For instance, the (non-ground) logic program containing the fact
`\lstinline[mathescape=true]{a("1.5").}'
along with the rule
`\lstinline[mathescape=true]@&sum{R*x} <= 7 :- a(R).@'
has the stable model
\par
\lstinline[mathescape=true]@$\{$a("1.5")$,\;$&sum{"1.5"*x}<=7$\}$@.
\\
This model is LC-stable since there is an assignment,
e.g.\ $\{x\mapsto 4.2\}$,
that satisfies the associated linear constraint `$1.5*x\leq 7$'.
We regard the stable model along with a satisfying real-valued assignment as a solution to a logic program containing linear constraint atoms.
\review{For a more detailed introduction of ASP extended with linear constraints, illustrated with more complex examples, we refer the interested reader to~\citep{jakaosscscwa17a}.}

To ease the use of ASP in practice,
several extensions have been developed.
First of all, rules with variables are viewed as shorthands for the set of their ground instances.
Further language constructs include
\emph{conditional literals} and \emph{cardinality constraints} \citep{siniso02a}.
The former are of the form
\lstinline[mathescape=true]{a:b$_1$,...,b$_m$},
the latter can be written as
\lstinline[mathescape=true]+s{d$_1$;...;d$_n$}t+,
where \lstinline{a} and \lstinline[mathescape=true]{b$_i$} are possibly default-negated (regular) literals  % for $0\leq i\leq m$,
and each \lstinline[mathescape=true]{d$_j$} is a conditional literal; % for $1\leq i\leq n$;
\lstinline{s} and \lstinline{t} provide optional lower and upper bounds on the number of satisfied literals in the cardinality constraint.
We refer to \lstinline[mathescape=true]{b$_1$,...,b$_m$} as a \emph{condition}.
%
The practical value of both constructs becomes apparent when used with variables.
For instance, a conditional literal like
\lstinline[mathescape=true]{a(X):b(X)}
in a rule's antecedent expands to the conjunction of all instances of \lstinline{a(X)} for which the corresponding instance of \lstinline{b(X)} holds.
%
Similarly,
\lstinline[mathescape=true]+2{a(X):b(X)}4+
is true whenever at least two and at most four instances of \lstinline{a(X)} (subject to \lstinline{b(X)}) are true.
%
Finally, objective functions minimizing the sum of weights $w_i$ subject to condition $c_i$ are expressed as
\lstinline[mathescape=true]!#minimize{$w_1$:$c_1$;$\dots$;$w_n$:$c_n$}!.
% \lstinline[mathescape=true]!#minimize{$w_1$@$l_1$:$c_1$;$\dots$;$w_n$@$l_n$:$c_n$}!.
% Lexicographically ordered objective functions are (optionally) distinguished via levels indicated by $l_i$.

In the same way,
the syntax of linear constraints offers several convenience features.
As above,
elements in linear constraint atoms can be conditioned,
viz.\par
`\lstinline[mathescape=true]@&sum{w$_1$*x$_1$:c$_1$;...;w$_l$*x$_l$:c$_n$} <= k@'
\\
where each \lstinline[mathescape=true]{c$_i$} is a condition.
% Also, linear constraints can be formed with further relations, viz.\
% \texttt{>=},
% \texttt{<},
% \texttt{>},
% \texttt{=},
% and
% \texttt{!=}.
Moreover, the theory language for linear constraints offers a domain declaration for real variables,
`\lstinline[mathescape=true]@&dom{lb..ub} = x@'
expressing that all values of \texttt{x} must lie between \texttt{lb} and \texttt{ub}.
And finally the maximization (or minimization) of an objective function can be expressed with
\lstinline[mathescape=true]@&maximize{w$_1$*x$_1$:c$_1$;...;w$_l$*x$_l$:c$_n$}@
(by \texttt{minimize}).
The full theory grammar for linear constraints over reals is available at~\url{https://potassco.org}.

%%% Local Variables:
%%% mode: latex
%%% TeX-master: "paper"
%%% End:
