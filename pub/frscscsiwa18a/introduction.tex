%!TEX root = paper.tex

\section{Introduction}\label{sec:introduction}
\newcommand\blfootnote[1]{%
  \begingroup
  \renewcommand\thefootnote{}\footnote{#1}%
  \addtocounter{footnote}{-1}%
  \endgroup
}
\blfootnote{This is an extended version of a paper presented at LPNMR-17, invited as a rapid publication in TPLP. The authors acknowledge the assistance of the conference chairs Tomi Janhunen and Marco Balduccini.}
Among all biological processes occurring in a cell, metabolic networks are in charge of transforming
input nutrients into both energy and output nutrients necessary for the functioning of other cells.
In other words, they capture all chemical reactions occurring in an organism.
In biology,
such networks are crucial from a fundamental and technological point of view
to estimate and control the capability of organisms to produce certain products.
%
Metabolic networks of high quality exist for many model organisms.
In addition,
recent technological advances enable their semi-automatic generation for many less studied organisms, also described as non-model organisms.
However,
the resulting metabolic networks are usually of poor quality,
due to error-prone, genome-based construction processes and a lack of (human) resources.
As a consequence, they usually suffer from substantial incompleteness.
The common fix is to fill the gaps by completing a draft network by borrowing chemical pathways
from reference networks of well studied organisms until the augmented network provides the measured functionality.

In previous work~\citep{schthi09a}, we introduced a logical approach to \emph{metabolic network completion}
by drawing on the work in~\citep{haebhe05a}. % ebhahe04a,
We formulated the problem as a qualitative combinatorial (optimization) problem and solved it with Answer Set Programming (ASP~\citep{baral02a}).
%
The basic idea is that reactions apply only if all their reactants are available,
either as nutrients or provided by other metabolic reactions.
Starting from given nutrients, referred to as \emph{seeds},
this allows for extending a metabolic network by successively adding operable
reactions and their products.
The set of compounds in the resulting network is called the \emph{scope} of the
seeds and represents all compounds that can principally be synthesized from the seeds.
In metabolic network completion, we query a database of metabolic reactions
looking for (minimal) sets of reactions that can restore an observed bio-synthetic behavior.
This is usually expressed by requiring that certain \emph{target} compounds are in the scope of some given seeds.
%
For instance, in the follow-up work in~\citep{coevgeprscsith13a,prcodideetdaevthcabosito14a},
we successfully applied our ASP-based approach to the reconstruction of the metabolic network of the macro-algae \emph{Ectocarpus siliculosus},
using the collection of \review{reference networks Metacyc \citep{Caspi2016}.}

We evidenced in~\citep{Prigent2017}
that our ASP-based method partly restores the bio-synthetic capabilities of a large proportion of moderately degraded networks: it fails to restore the ones of both some moderately degraded and most of highly degraded metabolic networks.
%
The main reason for this is that our purely qualitative approach misses quantitative constraints
accounting for the law of mass conservation,
a major hypothesis about metabolic networks.
%
This law stipulates that
each internal metabolite of a network must balance its production rate with its consumption rate at the steady state of the system.
Such rates are given by the weighted sums of all reaction rates consuming or producing a metabolite, respectively.
This calculation is captured by the \emph{stoichiometry}\footnote{See also \url{https://en.wikipedia.org/wiki/Stoichiometry}.} of the involved reactions.
%
Hence,
the qualitative ASP-based approach fails to tell apart solution candidates with correct and incorrect stoichiometry
and therefore reports inaccurate results for some degraded networks.

We address this by proposing a hybrid approach to metabolic network completion that integrates our qualitative ASP approach
with quantitative techniques from
\emph{Flux Balance Analysis} (FBA\footnote{See also \url{https://en.wikipedia.org/wiki/Flux_balance_analysis}.}~\citep{marzom16a}), % orthpa10a,
the state-of-the-art quantitative approach for capturing reaction rates in metabolic networks.
%
We accomplish this by taking advantage of recently developed theory reasoning capacities for the ASP system \clingo~\citep{gekakaosscwa16a}.
More precisely,
we use an extension of \clingo\ with linear constraints over reals, as dealt with in Linear Programming (LP~\citep{dantzig63a}).
%
This extension provides us with an extended ASP modeling language as well as a generic interface to alternative LP solvers, viz.\ \cplex\ and \lpsolve,
for dealing with linear constraints.
%
We empirically evaluate our approach by means of the metabolic network of \emph{Escherichia coli}.
Our analysis shows that our novel approach yields superior results than obtainable from purely qualitative or quantitative approaches.
%
Moreover, our hybrid application provides a first evaluation of the theory extensions of the ASP system \clingo\
with linear constraints over reals in a non-trivial setting.
%
% In what follows,
% we presuppose some basic familiarity with ASP and concentrate below on concepts relevant to our application.

%%% Local Variables:
%%% mode: latex
%%% TeX-master: "paper"
%%% End:
