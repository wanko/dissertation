\section{Curriculum-based Course Timetabling}\label{sec:cb-ctt}

As mentioned, we focus on the curriculum-based course timetabling
(CB-CTT) problems used in the ITC-2007 competition.
The problem description of CB-CTT presented here is based on 
\citep{DBLP:journals/anor/BonuttiCGS12}.

The CB-CTT instance consists mainly of
\textit{curricula},
\textit{courses},
\textit{rooms},
\textit{days}, and
\textit{periods} per day.
A curriculum is a set of courses that shares common students.
We refer to a pair of day and period as \textit{timeslot}.
%
The CB-CTT problem is defined as the task of assigning all lectures
of each course into a weekly timetable, 
subject to a given set of hard and soft constraints.
%
Hard constraints must be strictly satisfied.
Soft constraints are not necessarily satisfied,
but the sum of their violations should be minimal.
%
A \textit{feasible solution} of the problem is an assignment
so that the hard constraints are satisfied.
The objective of the problem is to find a feasible solution with minimal penalty.
%
The CB-CTT problem has the following hard constraints.
\begin{list}{}{}
\item \bm{$H_1$}. \textbf{Lectures}: 
  All lectures of each course must be scheduled, 
  and they must be assigned to distinct timeslots.
\item \bm{$H_2$}. \textbf{Conflicts}: 
  Lectures of courses in the same curriculum or taught by the same
  teacher must be all scheduled in different timeslots.
\item \bm{$H_3$}. \textbf{RoomOccupancy}: 
  Two lectures cannot take place in the same room in the same timeslot.
\item \bm{$H_4$}. \textbf{Availability}: 
  If the teacher of the course is unavailable to teach that course
  at a given timeslot, then no lecture of the course can be scheduled at
  that timeslot.
\end{list}
The CB-CTT problem has the following soft constraints.
\begin{list}{}{}
\item\bm{$S_1$}. \textbf{RoomCapacity}: 
  For each lecture, the number of students that attend the course must
  be less than or equal the number of seats of all the rooms that host
  its lectures. 
  The penalty points, reflecting the number of students above the
  capacity, are imposed on each violation.
\item\bm{$S_2$}. \textbf{MinWorkingDays}: 
  The lectures of each course must be spread into a given minimum
  number of days. 
  The penalty points, reflecting the number of days below the minimum,
  are imposed on each violation.
\item\bm{$S_3$}. \textbf{IsolatedLectures}: 
  Lectures belonging to a curriculum should be adjacent to each other
  in consecutive timeslots. For a given curriculum we account
  for a violation every time there is one lecture not adjacent to any
  other lecture within the same day. 
  Each isolated lecture in a curriculum counts as 1 violation.
\item\bm{$S_4$}. \textbf{Windows}: 
  Lectures belonging to a curriculum should not have time windows
  (periods without teaching) between them. 
  For a given
  curriculum we account for a violation every time there is one
  window between two lectures within the same day. 
  The penalty points, reflecting the length in periods of time window,
  are imposed on each violation.
\item\bm{$S_5$}. \textbf{RoomStability}: 
  All lectures of a course should be given in the same room. 
  The penalty points, reflecting the number of distinct rooms but the first, 
  are imposed on each violation.
\item\bm{$S_6$}. \textbf{StudentMinMaxLoad}: 
  For each curriculum the number of daily lectures should be within a
  given range. 
  The penalty points, reflecting the number of lectures below the minimum or above the
  maximum, are imposed on each violation.
\item\bm{$S_7$}. \textbf{TravelDistance}: 
  Students should have the time to move from one building to another
  one between two lectures. For a given curriculum we account for a
  violation every time there is an \textit{instantaneous move}: 
  two lectures in rooms located in different building in two adjacent
  periods within the same day. 
  Each instantaneous move in a curriculum counts as 1 violation.
\item\bm{$S_8$}. \textbf{RoomSuitability}:
  Some rooms may be not suitable for a given course because of the
  absence of necessary equipment.
  Each lecture of a course in an unsuitable room counts as 1
  violation.
\item\bm{$S_9$}. \textbf{DoubleLectures}:
  Some courses require that lectures in the same day are grouped
  together (\textit{double lectures}). For a course that requires grouped
  lectures, every time there is more than one lecture in one day, 
  a lecture non-grouped to another is not allowed. 
  Two lectures are grouped if they are adjacent and in the same room. 
  Each non-grouped lecture counts as 1 violation.
\end{list}

%%%%%%%%%%%%%%%%%%%%%%%%%%%%%%%%%%%%%%%%%%%%
\begin{table}
\centering
\caption{Problem Formulations}
\label{table:problem_formulations}
%\renewcommand{\arraystretch}{0.9}
%\tabcolsep = 3mm
\begin{tabular}[t]{l|ccccc}\hline
Constraint & UD1 & UD2 & UD3 & UD4 & UD5\\\hline
$H_1$. Lectures &  
H &  H &  H &  H & H\\
$H_2$. Conflicts &  
H &  H &  H &  H & H\\
$H_3$. RoomOccupancy &  
H &  H &  H &  H & H\\
$H_4$. Availability &  
H &  H &  H &  H & H\\
$S_1$. RoomCapacity &  
1 & 1 & 1 & 1  & 1 \\
$S_2$. MinWorkingDays &  
5 &  5 & - & 1 & 5 \\
$S_3$. IsolatedLectures &  
1 & 2 & - & - & 1 \\
$S_4$. Windows &  
- & - & 4 & 1 & 2\\
$S_5$. RoomStability &  
- & 1 & - & - & -\\
$S_6$. StudentMinMaxLoad &  
- & - & 2 & 1 & 2\\
$S_7$. TravelDistance &  
- & - & - & - & 2\\
$S_8$. RoomSuitability &  
- & - & 3 & H & -\\
$S_9$. DoubleLectures &  
- & - & - & 1 & -\\\hline
\end{tabular}
\end{table}
%%%%%%%%%%%%%%%%%%%%%%%%%%%%%%%%%%%%%%%%%%%%

A \textit{formulation} is defined as a specific set of soft constraints
together with the weights associated with each of them.
%
The five formulations UD1--UD5 have been proposed so far.
UD1 is the most basic formulation among them~\citep{DBLP:conf/patat/GasperoS02}.
UD2 is a well known formulation used in the ITC-2007 competition~\citep{GasperoMS/ITC2007}.
UD3, UD4, and UD5 have been recently proposed
to capture more different scenarios~\citep{DBLP:journals/anor/BonuttiCGS12}.
These formulations focus on 
student load (UD3), 
double lectures (UD4), and
travel cost (UD5), respectively.
%
The weights of soft constraints in each formulation is shown in 
Table~\ref{table:problem_formulations}.
The symbol `H' stands for inclusion in a formulation as hard constraint.
The symbol `-' stands for exclusion from a formulation.

In this paper, we formulate the CB-CTT problem as a single-objective
combinatorial optimization problem whose objective function is to
minimize the weighted sum of penalty points in the same manner as
ITC-2007, 
as well as a multi-criteria optimization problem based on lexicographic ordering.
Furthermore, we consider a multi-objective course timetabling problem
combining CB-CTT and Minimal Perturbation Problem.

%%% Local Variables:
%%% mode: latex
%%% TeX-master: "paper"
%%% End:

