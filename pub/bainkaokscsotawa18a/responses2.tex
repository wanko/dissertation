\documentclass[a4j]{article}
\title{Responses to the reviewers}
\author{Mutsunori Banbara}
\date{December 13, 2017}
%
%\usepackage{bm}
%\usepackage{mathptmx}
\usepackage{amsmath,amssymb}
\usepackage{color}
%\usepackage{comments}
\usepackage{listings}
\usepackage{array}
\lstset{numbers=left,numberblanklines=false,basicstyle=\ttfamily\footnotesize,%
numberbychapter=false,columns=fullflexible,keepspaces=true}
%\usepackage{slashbox}
%\usepackage{natbib}
%\usepackage{hyperref}
%\usepackage{url}
%\usepackage{tikz}
%\usetikzlibrary{arrows,chains,positioning,automata,decorations,shapes,calc,matrix,fit,backgrounds} % CHECK!!
% - rules etc --------------------------------------------------------------------
\newcommand{\naf}[1]{\ensuremath{{\sim}{#1}}}
\newcommand{\poslits}[1]{\ensuremath{#1^+}}
\newcommand{\neglits}[1]{\ensuremath{#1^-}}
\newcommand{\body}[1]{\ensuremath{B(#1)}} % {\ensuremath{\mathit{body}(#1)}}
\newcommand{\pbody}[1]{\poslits{\body{#1}}} % {\ensuremath{\mathit{body}^+(#1)}}
\newcommand{\nbody}[1]{\neglits{\body{#1}}} % {\ensuremath{\mathit{body}^-(#1)}}
\newcommand{\head}[1]{\ensuremath{h(#1)}} % {\ensuremath{\mathit{head}(#1)}}

\newcommand{\Tsign}{\ensuremath{\mathbf{T}}}
\newcommand{\Fsign}{\ensuremath{\mathbf{F}}}
\newcommand{\Tlit}[1]{\ensuremath{\Tsign #1}}
\newcommand{\Flit}[1]{\ensuremath{\Fsign #1}}
\newcommand{\Ass}{\ensuremath{\mathbf{A}}}
\newcommand{\DL}{\ensuremath{\mathit{DL}}}

\newcommand{\code}[1]{{\ttfamily #1}}
\newcommand{\codeClass}[2]{\code{#2}}

% - systems ----------------------------------------------------------------------
%
\newcommand{\sysfont}{\textit}
\newcommand{\acthex}{\sysfont{acthex}}
\newcommand{\asparagus}{\sysfont{asparagus}}
\newcommand{\aspic}{\sysfont{aspic}}
\newcommand{\aspmt}{\sysfont{aspmt}}
\newcommand{\asprin}{\sysfont{asprin}}
\newcommand{\assat}{\sysfont{assat}}
\newcommand{\berkmin}{\sysfont{berkmin}}
\newcommand{\claspD}{\sysfont{claspD}}
\newcommand{\claspar}{\sysfont{claspar}}
\newcommand{\claspfolio}{\sysfont{claspfolio}}
\newcommand{\clasp}{\sysfont{clasp}}
\newcommand{\clingcon}{\sysfont{clingcon}}
\newcommand{\clingo}{\sysfont{clingo}}
\newcommand{\cmodels}{\sysfont{cmodels}}
\newcommand{\coala}{\sysfont{coala}}
\newcommand{\dingo}{\sysfont{dingo}}
\newcommand{\dflat}{\sysfont{dflat}}
\newcommand{\dlvhex}{\sysfont{dlvhex}}
\newcommand{\dlv}{\sysfont{dlv}}
\newcommand{\ezcsp}{\sysfont{ezcsp}}
\newcommand{\ftolp}{\sysfont{f2lp}}
\newcommand{\gecode}{\sysfont{gecode}}
\newcommand{\gidl}{\sysfont{gidl}\xspace}
\newcommand{\gnt}{\sysfont{gnt}}
\newcommand{\gringo}{\sysfont{gringo}}
\newcommand{\iclingo}{\sysfont{iclingo}}
\newcommand{\idp}{\sysfont{idp}}
\newcommand{\inca}{\sysfont{inca}}
\newcommand{\jdlv}{\sysfont{jdlv}}
\newcommand{\lparse}{\sysfont{lparse}}
\newcommand{\lptodiff}{\sysfont{lp2diff}}
\newcommand{\lptosat}{\sysfont{lp2sat}}
\newcommand{\lctocasp}{\sysfont{lc2casp}}
\newcommand{\mchaff}{\sysfont{mchaff}}
\newcommand{\metasp}{\sysfont{metasp}}
\newcommand{\mingo}{\sysfont{mingo}}
\newcommand{\minisat}{\sysfont{minisat}}
\newcommand{\nomorepp}{\sysfont{nomore++}}
\newcommand{\oclingo}{\sysfont{oclingo}}
\newcommand{\piclasp}{\sysfont{piclasp}}
\newcommand{\picosat}{\sysfont{picosat}}
\newcommand{\plasp}{\sysfont{plasp}}
\newcommand{\quontroller}{\sysfont{quontroller}}
\newcommand{\rosoclingo}{\sysfont{rosoclingo}}
\newcommand{\sag}{\sysfont{sag}}
\newcommand{\satz}{\sysfont{satz}}
\newcommand{\siege}{\sysfont{siege}}
\newcommand{\smodelscc}{\sysfont{smodels$_{\!cc}$}}
\newcommand{\smodelsr}{\sysfont{smodels}$_r$}
\newcommand{\smodels}{\sysfont{smodels}}
\newcommand{\unclasp}{\sysfont{unclasp}}
\newcommand{\wasp}{\sysfont{wasp}}
\newcommand{\zchaff}{\sysfont{zchaff}}
\newcommand{\zzz}{\sysfont{z3}}

\newcommand{\theory}{\emph{Theory}}
\newcommand{\hybrid}{\sysfont{Hybrid}}

\newcommand{\aspif}{\sysfont{aspif}}

\newcommand{\python}{Python}
\newcommand{\lua}{Lua}
\newcommand{\cpp}{C++}
\newcommand{\C}{C}
\newcommand{\java}{Java}
\newcommand{\haskell}{Haskell}

\newacro{ILP}{Integer Linear Programming}
\newacro{SAT}{Boolean Satisfiability}
\newacro{ASP}{Answer Set Programming}
\newacro{DSE}{Design Space Exploration}
\newacro{ASPmT}{\ac{ASP} modulo Theories}
\newacro{MOEA}{multi-objective evolutionary algorithm}
\newacro{MOOP}{multi-objective optimization problem}
\newacro{QF--IDL}{qunatifier-free integer difference logic}

% - hacks ----------------------------------------------------------------------

\newcommand{\neghspace}{\!\!\!\!\!}

%%% Local Variables:
%%% mode: latex
%%% TeX-master: "paper"
%%% End:

%%%%%%%%%%%%%%%%%%%%%%%%%%%%%%%%%%%%%%%%%%%%%%%%%%%%%%%%%%%%%%%%%%%%%%%%%%%%%%
\begin{document}
\maketitle
%%%%%%%%%%%%%%%%%%%%%%%%%%%%%%%%%%%%%%%%%%%%%%%%%%%%%%%%%%%%%%%%%%%%%%%%%%%%%%

\begin{quote}
\begin{list}{}{}
\item[PAPER ID:]
  ANOR-D-16-01064
\item[TITLE:]
  \textit{teaspoon}: Solving the Curriculum-Based Course Timetabling Problems with Answer Set Programming
\item[AUTHORS:]
  Mutsunori Banbara, 
  Katsumi Inoue,
  Benjamin Kaufmann,
  Tenda Okimoto,
  Torsten Schaub,
  Takehide Soh,
  Naoyuki Tamura, and
  Philipp Wanko
\end{list}
\end{quote}

%%%%%%%%%%%%%%%%%%%%%%%%%%%%%%%%%%%%%%%%%%%%%%%%%%%%%%%%%%%%%%%%%%%%%%%%%%%%%%
\section*{Response to review \#1}
%%%%%%%%%%%%%%%%%%%%%%%%%%%%%%%%%%%%%%%%%%%%%%%%%%%%%%%%%%%%%%%%%%%%%%%%%%%%%%
\begin{it}\color{blue}
Reviewer \#1: Thank you for all the revisions, I am very happy with
them. Now I see only few minor revisions which can be handled easily.

\begin{itemize}
\item 
p.2,par.1: Based on the sentence "In this paper, we focus on
curriculum-based course timetabling (CB-CTT; (Bettinelli et al,
2015)), one of the most studied course timetabling problems, as well
as post-enrollment course timetabling.", it may seem that your study
is also about the post-enrollment course timetabling. However, this is
not the case. Please reformulate.
\item 
p.5,par.-1: Based on the sentence "In this paper, we formulate the
CB-CTT problem as a single-objective combinatorial optimization
problem whose objective function is to minimize the weighted sum of
penalty points in the same manner as ITC-2007.", you seem to forgot
the multi-criteria problem.  Please reformulate.
\end{itemize}
\end{it}

Thank you for your comments. We reformulated the sentences.

\begin{it}\color{blue}
\begin{itemize}
\item 
p.10,listing 7: You are using the construct 
"\code{\{ assigned(C,D,P) : course(C,T,_,_,_,_) \} 1}" which was not
defined before.  I think that you could easily explain it on page 6
when defining "\code{s \{c1 ,...,cn\} t"}.
p.11,listing 8: Again it would be helpful to define semantics of 
"\code{N = \{ working_day(C,D) \}}" and "\code{2 \{ assigned(C,D,PPD) \}}".
\end{itemize}
\end{it}

We added some explanations about those constructs.

\begin{it}\color{blue}
\begin{itemize}
\item 
p.25,par.2: Why "new program"? It is there three times but I believe
that it should be replaced by "new problem".
\end{itemize}
\end{it}

You are right. We replaced ``new program''by ``new problem''.

\begin{it}\color{blue}
\begin{itemize}
\item 
Please check carefully references. There are several mistakes having
incorrect typesetting such as: Maxsat, udine, maxsat, csps.
\end{itemize}
\end{it}

We carefully checked the references and corrected the mistakes.

%%%%%%%%%%%%%%%%%%%%%%%%%%%%%%%%%%%%%%%%%%%%%%%%%%%%%%%%%%%%%%%%%%%%%%%%%%%%%%
\section*{Response to review \#3}
%%%%%%%%%%%%%%%%%%%%%%%%%%%%%%%%%%%%%%%%%%%%%%%%%%%%%%%%%%%%%%%%%%%%%%%%%%%%%%
\begin{it}\color{blue}
Reviewer \#3: The revisions have addressed my comments
satisfactorily. I'm happy to recommend accepting the revised paper.
\end{it}

Thank you very much!
%%%%%%%%%%%%%%%%%%%%%%%%%%%%%%%%%%%%%%%%%%%%%%%%%%%%%%%%%%%%%%%%%%%%%%%%%%%%%%
\end{document}

