\documentclass[a4j]{article}
\title{Responses to the reviewers}
\author{Mutsunori Banbara}
\date{December 13, 2017}
%
%\usepackage{bm}
%\usepackage{mathptmx}
\usepackage{amsmath,amssymb}
\usepackage{color}
%\usepackage{comments}
\usepackage{listings}
\usepackage{array}
\lstset{numbers=left,numberblanklines=false,basicstyle=\ttfamily\footnotesize,%
numberbychapter=false,columns=fullflexible,keepspaces=true}
%\usepackage{slashbox}
%\usepackage{natbib}
%\usepackage{hyperref}
%\usepackage{url}
%\usepackage{tikz}
%\usetikzlibrary{arrows,chains,positioning,automata,decorations,shapes,calc,matrix,fit,backgrounds} % CHECK!!
\newcommand{\gringo}{\textit{gringo}}
\newcommand{\clasp}{\textit{clasp}}
\newcommand{\clingo}{\textit{clingo}}
\newcommand{\asprin}{\textit{asprin}}
\newcommand{\asap}{\textit{teaspoon}}
\newcommand{\piclasp}{\textit{piclasp}}

\newcommand{\code}[1]{\lstinline[basicstyle=\ttfamily]{#1}}

\newcommand{\lw}[1]{\smash{\lower1.ex\hbox{#1}}}
\newcommand{\llw}[1]{\smash{\lower3.ex\hbox{#1}}}

%\newcommand{\dataCL}[5]{%
%  \code{#1} & #3 & #5 & #4
%}
%\newcommand{\dataCS}[5]{%
%  #3 & #5 & #4
%}

\newenvironment{tableC}{%
  \scriptsize
  \tabcolsep = 0.6mm
  \begin{tabular}[t]{l|rlr|rlr|rlr|rlr|rlr}\hline
    \multicolumn{1}{l|}{\llw{Instance}} &
    \multicolumn{3}{c|}{UD1} &
    \multicolumn{3}{c|}{UD2} &
    \multicolumn{3}{c|}{UD3} &
    \multicolumn{3}{c|}{UD4} &
    \multicolumn{3}{c}{UD5} \\
    & 
    \multicolumn{1}{c}{Best} & & \multicolumn{1}{c|}{\emph{tea-}} & 
    \multicolumn{1}{c}{Best} & & \multicolumn{1}{c|}{\emph{tea-}} & 
    \multicolumn{1}{c}{Best} & & \multicolumn{1}{c|}{\emph{tea-}} & 
    \multicolumn{1}{c}{Best} & & \multicolumn{1}{c|}{\emph{tea-}} & 
    \multicolumn{1}{c}{Best} & & \multicolumn{1}{c}{\emph{tea-}} \\
    & 
    known & & \emph{spoon} & 
    known & & \emph{spoon} & 
    known & & \emph{spoon} & 
    known & & \emph{spoon} & 
    known & & \emph{spoon} \\
    \hline
  }{%
    \hline
  \end{tabular}
}

\newenvironment{tableB}{%
  \scriptsize
  \tabcolsep = 0.7mm
%  \begin{tabular}[t]{|l|c|r|l|l|l|}\hline
  \begin{tabular}[t]{lcrlll}\hline
    Instance &
    Formulation &
    Time (sec.)\\
    \hline
  }{%
    \hline
  \end{tabular}
}
\newenvironment{tableL}{%
  \scriptsize
  \tabcolsep = 0.7mm
  \begin{tabular}[t]{l|rrrrrrrr|r}\hline
    \lw{Instance} &
    \lw{Time (sec.)} &
    \multicolumn{6}{c}{The best utility vector} &
    The sum of  &
    The best of basic\\
    &
    &
    $(S_1,$ & $S_4,$ & $S_2,$ & $S_7,$ & $S_6,$ & $S_3)$ &
    utility vector &
    and optimized \\
    \hline
  }{%
    \hline
  \end{tabular}
}

%%% Local Variables:
%%% mode: latex
%%% TeX-master: "paper"
%%% End:

%%%%%%%%%%%%%%%%%%%%%%%%%%%%%%%%%%%%%%%%%%%%%%%%%%%%%%%%%%%%%%%%%%%%%%%%%%%%%%
\begin{document}
\maketitle
%%%%%%%%%%%%%%%%%%%%%%%%%%%%%%%%%%%%%%%%%%%%%%%%%%%%%%%%%%%%%%%%%%%%%%%%%%%%%%

\begin{quote}
\begin{list}{}{}
\item[PAPER ID:]
  ANOR-D-16-01064
\item[TITLE:]
  \textit{teaspoon}: Solving the Curriculum-Based Course Timetabling Problems with Answer Set Programming
\item[AUTHORS:]
  Mutsunori Banbara, 
  Katsumi Inoue,
  Benjamin Kaufmann,
  Tenda Okimoto,
  Torsten Schaub,
  Takehide Soh,
  Naoyuki Tamura, and
  Philipp Wanko
\end{list}
\end{quote}

%%%%%%%%%%%%%%%%%%%%%%%%%%%%%%%%%%%%%%%%%%%%%%%%%%%%%%%%%%%%%%%%%%%%%%%%%%%%%%
\section*{Response to review \#1}
%%%%%%%%%%%%%%%%%%%%%%%%%%%%%%%%%%%%%%%%%%%%%%%%%%%%%%%%%%%%%%%%%%%%%%%%%%%%%%
\begin{it}\color{blue}
Reviewer \#1: Thank you for all the revisions, I am very happy with
them. Now I see only few minor revisions which can be handled easily.

\begin{itemize}
\item 
p.2,par.1: Based on the sentence "In this paper, we focus on
curriculum-based course timetabling (CB-CTT; (Bettinelli et al,
2015)), one of the most studied course timetabling problems, as well
as post-enrollment course timetabling.", it may seem that your study
is also about the post-enrollment course timetabling. However, this is
not the case. Please reformulate.
\item 
p.5,par.-1: Based on the sentence "In this paper, we formulate the
CB-CTT problem as a single-objective combinatorial optimization
problem whose objective function is to minimize the weighted sum of
penalty points in the same manner as ITC-2007.", you seem to forgot
the multi-criteria problem.  Please reformulate.
\end{itemize}
\end{it}

Thank you for your comments. We reformulated the sentences.

\begin{it}\color{blue}
\begin{itemize}
\item 
p.10,listing 7: You are using the construct 
"\code{\{ assigned(C,D,P) : course(C,T,_,_,_,_) \} 1}" which was not
defined before.  I think that you could easily explain it on page 6
when defining "\code{s \{c1 ,...,cn\} t"}.
p.11,listing 8: Again it would be helpful to define semantics of 
"\code{N = \{ working_day(C,D) \}}" and "\code{2 \{ assigned(C,D,PPD) \}}".
\end{itemize}
\end{it}

We added some explanations about those constructs.

\begin{it}\color{blue}
\begin{itemize}
\item 
p.25,par.2: Why "new program"? It is there three times but I believe
that it should be replaced by "new problem".
\end{itemize}
\end{it}

You are right. We replaced ``new program''by ``new problem''.

\begin{it}\color{blue}
\begin{itemize}
\item 
Please check carefully references. There are several mistakes having
incorrect typesetting such as: Maxsat, udine, maxsat, csps.
\end{itemize}
\end{it}

We carefully checked the references and corrected the mistakes.

%%%%%%%%%%%%%%%%%%%%%%%%%%%%%%%%%%%%%%%%%%%%%%%%%%%%%%%%%%%%%%%%%%%%%%%%%%%%%%
\section*{Response to review \#3}
%%%%%%%%%%%%%%%%%%%%%%%%%%%%%%%%%%%%%%%%%%%%%%%%%%%%%%%%%%%%%%%%%%%%%%%%%%%%%%
\begin{it}\color{blue}
Reviewer \#3: The revisions have addressed my comments
satisfactorily. I'm happy to recommend accepting the revised paper.
\end{it}

Thank you very much!
%%%%%%%%%%%%%%%%%%%%%%%%%%%%%%%%%%%%%%%%%%%%%%%%%%%%%%%%%%%%%%%%%%%%%%%%%%%%%%
\end{document}

