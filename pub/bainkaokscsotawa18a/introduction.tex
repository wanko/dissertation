\section{Introduction}\label{intro}

\textit{Educational timetabling}~\citep{%
DBLP:journals/eor/BurkeP02,%
Lewis2007:survey,%
DBLP:journals/air/Schaerf99}
is generally defined as the
task of assigning a number of events, such as lectures and examinations,
to a limited set of timeslots (and perhaps rooms), 
subject to a given set of hard and soft constraints.
Hard constraints must be strictly satisfied.
Soft constraints must not necessarily be satisfied but
the overall number of violations should be minimal.
%
The educational timetabling problems can be 
classified into three categories:
\textit{school timetabling},
\textit{examination timetabling}, and
\textit{course timetabling}.
In this paper, 
we focus on \textit{curriculum-based course timetabling} 
(CB-CTT;~\citep{Bettinelli2015}), 
one of the most studied course timetabling problems.
%as well as \textit{post-enrollment course timetabling}.

The CB-CTT problems have been used in the third track of 
the second international timetabling competition 
(ITC-2007;~\citep{%
  GasperoMS/ITC2007,%
  DBLP:journals/informs/McCollumSPMLPGQB10}).
%
A web portal\footnote{\texttt{http://tabu.diegm.uniud.it/ctt/}}
for CB-CTT has been actively maintained by the ITC-2007 organizers%
~\citep{DBLP:journals/anor/BonuttiCGS12}.
The web site provides necessary infrastructures for benchmarking
such as validators, data formats, problem instances,
solutions in different formulations (uploaded by researchers), 
and visualizers.
All problem instances on the web are based on
real data from various universities in Europe.
The best known bounds on the web have been obtained by 
state-of-the-art CB-CTT solving techniques
including the winner algorithm of ITC-2007:
metaheuristics-based algorithms~\citep{%
  DBLP:journals/heuristics/AbdullahTMM12,%
%  ClarkHL/PATAT2008,%
  DBLP:conf/patat/GasperoS02,%
  DBLP:journals/jmma/GasperoS06,%
  DBLP:journals/anor/Geiger12,%
  DBLP:journals/eor/LuH10},
Integer Programming~\citep{DBLP:journals/anor/LachL12},
hybrid methods~\citep{DBLP:journals/anor/Muller09},
SAT/MaxSAT~\citep{DBLP:journals/aicom/AchaN12},
and many others.

However, each method has strength and weakness.
Metaheuristics-based dedicated implementations can quickly find better
upper bounds, but cannot guarantee their optimality.
Although complete methods such as SAT can guarantee the optimality, 
it is costly to implement a dedicated encoder from the CB-CTT problems in SAT\@.
Integer Programming has been widely used for CB-CTT solving,
but in general it does not scale to large instances in complex formulations.
It is therefore particularly challenging to develop a universal
timetabling solver which can efficiently find optimal solutions 
as well as better bounds for a wide range of CB-CTT instances in
different formulations at present.

Answer Set Programming (ASP;~\citep{%
  baral03:cambridge,%
  DBLP:conf/iclp/GelfondL88,%
  DBLP:journals/amai/Niemela99})
is an approach to declarative problem solving.
Recent advances in ASP open up a successful direction to extend logic
programming to be both more expressive as well as more effective.
%
ASP provides a rich language and 
is well suited for modeling combinatorial (optimization) problems 
in Artificial Intelligence and Computer Science.
Recent remarkable improvements in the effectiveness of ASP systems
have encouraged researchers to use ASP for solving problems in diverse areas,
such as
automated planning,
constraint satisfaction,
model checking,
music composition,
robotics,
system biology,
etc~\citep{ergele16a}.
However, so far, little attention has been paid to using ASP for timetabling.

In this paper, we describe an ASP-based approach for solving the
CB-CTT problems and present the resulting {\asap} system.
The {\asap} system reads a CB-CTT instance 
of a standard input format~\citep{DBLP:journals/anor/BonuttiCGS12}
and converts it into ASP facts.
In turn, these facts are combined with a first-order encoding for
CB-CTT solving, which is subsequently solved by an off-the-shelf ASP
system, in our case {\clingo}.
% ~\footnote{%
% ASP system {\clingo} is a monolithic combination of the grounder {\gringo} with the solver {\clasp}.}.
Figure~\ref{fig:arch} shows the \asap\ architecture.
% ----------------------------------------------------------------------
\begin{figure}[t]
  \centering
%  \thicklines
  \setlength{\unitlength}{1.28pt}
  \small
  % \frame <- uncomment this to see the bounding box ;)
  {\begin{picture}(210,65)(4,-15)
    \put( 14, 20){\dashbox(26,24){\shortstack{CB-CTT\\Instance}}}
    \put( 50, 20){\framebox(40,24){\shortstack{Format\\Converter}}}
    \put(100, 20){\dashbox(20,24){\shortstack{ASP\\Facts}}}
    \put( 80,-10){\dashbox(40,24){\shortstack{ASP\\Encoding}}}
    \put(130, 20){\framebox(40,24){\clingo}}
    \put(180, 20){\dashbox(26,24){\shortstack{CB-CTT\\Solution}}}
    \put( 40, 32){\vector(1,0){10}}
    \put( 90, 32){\vector(1,0){10}}
    \put(120, 32){\vector(1,0){10}}
    \put(170, 32){\vector(1,0){10}}
    \put(120, +2){\line(1,0){4}}
    \put(124, +2){\line(0,1){30}}
  \end{picture}}
\caption{Architecture of \asap.}
\label{fig:arch}
\end{figure}
%%% Local Variables: 
%%% mode: latex
%%% TeX-master: "paper"
%%% End: 

% ----------------------------------------------------------------------

The high-level approach of ASP has obvious advantages.
First, the problems are solved by general-purpose ASP systems rather
than dedicated implementation.
Second,
the elaboration tolerance of ASP allows for easy maintenance and
modifications of encodings.
And finally,
it is easy to experiment with advanced techniques in ASP solving such
as core-guided optimization, domain heuristics, and 
portfolios of prefabricated expert configurations~\citep{gekakarosc15a}.
%
However, the question is whether the high-level approach of \asap\ matches the
performance of dedicated systems.
We empirically address this question by contrasting the performance of 
\asap\ with the best known bounds on the CB-CTT web portal
obtained by state-of-the-art CB-CTT solving techniques.

From the perspective of applying ASP to educational timetabling, 
an early work studied school timetabling with ASP~\citep{Faber98}.
Recently, we showed in previous work~\citep{basotainsc13a} that ASP's
modeling language is well-suited for course timetabling by providing
a compact encoding for CB-CTT solving.
However, at the same time, 
we observed that a simple branch-and-bound optimization strategy
is insufficient to decrease the upper bounds of large instances in complex formulations.
In this paper, we provide insights into how more advanced solving techniques can be used 
to overcome this practical issue.

The main contributions of this paper are as follows.
\begin{enumerate}
\item We present a basic ASP encoding for solving CB-CTT problems, 
  which is an enhancement of our previous encoding~\citep{basotainsc13a}. 
  This enhancement provides the ability to use advanced ASP solving
  techniques such as 
  core-guided optimization, 
  domain heuristics, 
  portfolios of prefabricated expert configurations,
  multi-criteria optimization based on lexicographic ordering, and
  multi-shot ASP solving~\citep{gekakarosc15a,gekaobsc15a}.
\item We extend the basic encoding in view of enhancing the
  scalability and flexibility of solving (multi-criteria) CB-CTT problems.
  The extended \asap\ encodings have the following features:
  \begin{itemize}
  \item A collection of optimized encodings for soft constraints
  \item Easy composition of different formulations
  \item Multi-criteria optimization based on lexicographic ordering
%  \item Reusing legacy timetables in multi-shot ASP solving
  \end{itemize}
\item Our empirical analysis considers all 61 instances in 5
  different formulations, which are publicly available from the CB-CTT
  portal ($61\times 5 = 305$ combinations in a total)~\footnote{As of July 20, 2017}.
  Overall, \asap\ managed to either improve or reproduce the best
  known bounds for 182 combinations (59.7\% in the total).
  In detail, \asap\ provided 54 better bounds, 
  16 new optima,
  and 128 same bounds, 
  35 of which were proven optimal for the first time.
  Furthermore,
  \asap\ was able to produce upper bounds for very large instances in
  the category \verb+erlangen+ with every formulation, and 24 of them
  were unsolvable before.
\item We also extend the {\asap} system to finding Pareto optimal solutions
  of multi-objective course timetabling and consider
  \textit{minimal perturbation problems}~\citep{%
    BartakMR03,MullerRB04,RudovaMM11,Phillips2016}
  by utilizing multi-shot ASP solving techniques~\citep{gekaobsc15a}.
\end{enumerate}
All in all, the proposed declarative approach represents a significant
contribution to the state-of-the-art for CB-CTT.

The rest of the paper is structured as follows.
Section~\ref{sec:cb-ctt} provides the problem description of CB-CTT\@.
Although we give a brief introduction to ASP and its basic language
constructs in Section~\ref{sec:asp}, 
we refer the reader to the literature~\citep{baral03:cambridge,gekakasc12a}
for a comprehensive treatment of ASP\@.
Section~\ref{sec:approach}
describes {\asap}'s fact format of CB-CTT instances and then 
presents a basic {\asap} encoding for solving CB-CTT problems.
Section~\ref{sec:ext} presents a variety of features of
extended {\asap} encodings for (multi-criteria) CB-CTT solving.
Section~\ref{system} provides a detailed empirical analysis of {\asap}
features and performance in contrast to 
the best known bounds obtained by state-of-the-art CB-CTT solving techniques.
Section~\ref{mpp} presents an extension of the {\asap} system to 
minimal perturbation problems in course timetabling. 
Finally, a conclusion is given in Section~\ref{conclusion}.

%%% Local Variables:
%%% mode: latex
%%% TeX-master: "paper"
%%% End:
