\documentclass[a4j]{article}
\title{Responses to the reviewers}
\author{Mutsunori Banbara}
\date{September 30, 2017}
%
%\usepackage{bm}
%\usepackage{mathptmx}
\usepackage{amsmath,amssymb}
\usepackage{color}
%\usepackage{comments}
\usepackage{listings}
\usepackage{array}
\lstset{numbers=left,numberblanklines=false,basicstyle=\ttfamily\footnotesize,%
numberbychapter=false,columns=fullflexible,keepspaces=true}
%\usepackage{slashbox}
%\usepackage{natbib}
%\usepackage{hyperref}
%\usepackage{url}
%\usepackage{tikz}
%\usetikzlibrary{arrows,chains,positioning,automata,decorations,shapes,calc,matrix,fit,backgrounds} % CHECK!!
\newcommand{\gringo}{\textit{gringo}}
\newcommand{\clasp}{\textit{clasp}}
\newcommand{\clingo}{\textit{clingo}}
\newcommand{\asprin}{\textit{asprin}}
\newcommand{\asap}{\textit{teaspoon}}
\newcommand{\piclasp}{\textit{piclasp}}

\newcommand{\code}[1]{\lstinline[basicstyle=\ttfamily]{#1}}

\newcommand{\lw}[1]{\smash{\lower1.ex\hbox{#1}}}
\newcommand{\llw}[1]{\smash{\lower3.ex\hbox{#1}}}

%\newcommand{\dataCL}[5]{%
%  \code{#1} & #3 & #5 & #4
%}
%\newcommand{\dataCS}[5]{%
%  #3 & #5 & #4
%}

\newenvironment{tableC}{%
  \scriptsize
  \tabcolsep = 0.6mm
  \begin{tabular}[t]{l|rlr|rlr|rlr|rlr|rlr}\hline
    \multicolumn{1}{l|}{\llw{Instance}} &
    \multicolumn{3}{c|}{UD1} &
    \multicolumn{3}{c|}{UD2} &
    \multicolumn{3}{c|}{UD3} &
    \multicolumn{3}{c|}{UD4} &
    \multicolumn{3}{c}{UD5} \\
    & 
    \multicolumn{1}{c}{Best} & & \multicolumn{1}{c|}{\emph{tea-}} & 
    \multicolumn{1}{c}{Best} & & \multicolumn{1}{c|}{\emph{tea-}} & 
    \multicolumn{1}{c}{Best} & & \multicolumn{1}{c|}{\emph{tea-}} & 
    \multicolumn{1}{c}{Best} & & \multicolumn{1}{c|}{\emph{tea-}} & 
    \multicolumn{1}{c}{Best} & & \multicolumn{1}{c}{\emph{tea-}} \\
    & 
    known & & \emph{spoon} & 
    known & & \emph{spoon} & 
    known & & \emph{spoon} & 
    known & & \emph{spoon} & 
    known & & \emph{spoon} \\
    \hline
  }{%
    \hline
  \end{tabular}
}

\newenvironment{tableB}{%
  \scriptsize
  \tabcolsep = 0.7mm
%  \begin{tabular}[t]{|l|c|r|l|l|l|}\hline
  \begin{tabular}[t]{lcrlll}\hline
    Instance &
    Formulation &
    Time (sec.)\\
    \hline
  }{%
    \hline
  \end{tabular}
}
\newenvironment{tableL}{%
  \scriptsize
  \tabcolsep = 0.7mm
  \begin{tabular}[t]{l|rrrrrrrr|r}\hline
    \lw{Instance} &
    \lw{Time (sec.)} &
    \multicolumn{6}{c}{The best utility vector} &
    The sum of  &
    The best of basic\\
    &
    &
    $(S_1,$ & $S_4,$ & $S_2,$ & $S_7,$ & $S_6,$ & $S_3)$ &
    utility vector &
    and optimized \\
    \hline
  }{%
    \hline
  \end{tabular}
}

%%% Local Variables:
%%% mode: latex
%%% TeX-master: "paper"
%%% End:

%%%%%%%%%%%%%%%%%%%%%%%%%%%%%%%%%%%%%%%%%%%%%%%%%%%%%%%%%%%%%%%%%%%%%%%%%%%%%%
\begin{document}
\maketitle
%%%%%%%%%%%%%%%%%%%%%%%%%%%%%%%%%%%%%%%%%%%%%%%%%%%%%%%%%%%%%%%%%%%%%%%%%%%%%%

\begin{quote}
\begin{list}{}{}
\item[PAPER ID:]
  ANOR-D-16-01064
\item[TITLE:]
  \textit{teaspoon}: Solving the Curriculum-Based Course Timetabling Problems with Answer Set Programming
\item[AUTHORS:]
  Mutsunori Banbara, 
  Katsumi Inoue,
  Benjamin Kaufmann,
  Tenda Okimoto,
  Torsten Schaub,
  Takehide Soh,
  Naoyuki Tamura, and
  Philipp Wanko
\end{list}
\end{quote}

\noindent
Dear anonymous reviewers,\\[1em]
We would like to thank the reviewers for their time and
valuable comments and suggestions to our paper.
Your efforts helped us to address several issues and to 
further improve the paper. 
In this response,
we first explain the major points of our revision and then
address the specific questions and comments (highlighted in blue)
from the reviewers.

\vskip 1em

\noindent Best regards,

\qquad Mutsunori Banbara

%%%%%%%%%%%%%%%%%%%%%%%%%%%%%%%%%%%%%%%%%%%%%%%%%%%%%%%%%%%%%%%%%%%%%%%%%%%%%%
\section*{The major points of our revision}
%%%%%%%%%%%%%%%%%%%%%%%%%%%%%%%%%%%%%%%%%%%%%%%%%%%%%%%%%%%%%%%%%%%%%%%%%%%%%%

We have carefully revised the paper in view of the reviewer comments.
Before addressing the specific questions and comments, 
we explain the major points of the revision.

\begin{itemize}
%%%%%%%%%%%%%%%%
\item \textbf{Point \#1.~Contributions:} To clarify the main contributions of this
  paper, we added the following in Introduction.\\[1em]
\begin{it}
The main contributions of this paper are as follows.
\begin{enumerate}
\item We present a basic ASP encoding for solving CB-CTT problems, 
  which is an enhancement of our previous encoding (Banbara et al, 2013).
  This enhancement provides the ability to use advanced ASP solving
  techniques such as 
  core-guided optimization, 
  domain heuristics, 
  portfolios of prefabricated expert configurations,
  multi-criteria optimization based on lexicographic ordering, and
  multi-shot ASP solving (Gebser et al, 2015a,b).
\item We extend the basic encoding in view of enhancing the
  scalability and flexibility of solving (multi-criteria) CB-CTT problems.
  The extended \asap\ encodings have the following features:
  \begin{itemize}
  \item A collection of optimized encodings for soft constraints
  \item Easy composition of different formulations
  \item Multi-criteria optimization based on lexicographic ordering
  \end{itemize}
\item Our empirical analysis considers all 61 instances in 5
  different formulations, which are publicly available from the CB-CTT
  portal as of July 20, 2017
  ($61\times 5 = 305$ combinations in a total).
  Overall, \asap\ managed to either improve or reproduce the best
  known bounds for 182 combinations (59.7\% in the total).
  In detail, \asap\ provided 54 better bounds, 
  16 new optima,
  and 128 same bounds, 
  35 of which were proven optimal for the first time.
  Furthermore,
  \asap\ was able to produce upper bounds for very large instances in
  the category \verb+erlangen+ with every formulation, and 24 of them
  were unsolvable before.
\item We also extend the {\asap} system to finding Pareto optimal solutions
  of multi-objective course timetabling and consider
  \textit{minimal perturbation problems}
 (Bart{\'{a}}k et al, 2004; M{\"{u}}ller et al, 2005; Rudov{\'{a}} et al, 2011; Phillips et al, 2016) 
  by utilizing multi-shot ASP solving techniques (Gebser et al, 2015b).
\end{enumerate}
All in all, the proposed declarative approach represents a significant
contribution to the state-of-the-art for CB-CTT.
\end{it}
%%%%%%%%%%%%%%%%
\item \textbf{Point \#2.~Experiments:}
To verify the effectiveness of the extensions presented in Section~5, 
we showed new experimental results in Section~6.3 for 
the optimized encodings as well as for 
multi-criteria optimization based on lexicographic ordering.
Consequently, 
we were able to obtain good performance as can be seen in 
Table~5 and Table~6.
%
Furthermore, to make our experimental results more solid, 
our empirical analysis included 
the newest \code{UUMCAS_A131} instance 
inserted to the CB-CTT portal in March 2017, and 
{\asap} was able to produce new upper bounds with every formulation,
and 4 of them were unsolvable before.
%
It is noted that the related tables such as 
Table~2 and Table~7 have been all updated
due to the addition of this new instance to the
benchmark set.

%%%%%%%%%%%%%%%%
\item \textbf{Point \#3.~Features:}
To clarify the main features of our proposed approach,
we added the following in Conclusion.\\[1em]
\begin{it}
The main features of our declarative approach are as follows:
\begin{list}{}{}
\item \textbf{Expressiveness}. 
The expressive power of ASP's modelling language enables us to compactly express
a wide variety of hard and soft constraints of CB-CTT as demonstrated
by a collection of {\asap} encodings. 
Given new constraints (e.g., Manhattan distance), 
all we have to do is adding ASP rules.

\item \textbf{Flexibility}. 
{\asap} provides flexible lexicographic optimization and easy
composition of different formulations, since any combination of
constraints can be represented as ASP facts. 
Consequently, it enables a timetable keeper to experiment with
different formulations as well as to optimize the obtained timetable 
with different priority levels of soft constraints, 
at a purely declarative level.

\item \textbf{Efficiency}. 
Our empirical analysis considers all instances in five different 
formulations, which are publicly available from the CB-CTT portal. 
% Many instances are based on real world data from several European
% universities. 
We have contrasted the performance of \asap\ with the best known
bounds obtained so far via more dedicated implementations.
\asap\ demonstrated that our declarative approach allows us to
compete with state-of-the-art CB-CTT solving techniques.

\item \textbf{Scalability}. 
The optimized encoding drastically reduces the grounding time compared
to the basic encoding.
Consequently, our declarative approach scales to large instances in
complex formulations, as demonstrated by the fact that
{\asap} was able to find upper bounds for very
large instances in the category \code{erlangen} with every
formulation, and 24 of them were unsolvable before. 

\item \textbf{Extensibility}. 
The high-level approach of ASP facilitates extensions and variations
of first-order encodings.
From this viewpoint, 
we extended the {\asap} system to solving multi-objective course
timetabling problem combining CB-CTT and minimal perturbation problems
with two criteria of optimality and stability.
\end{list}
\end{it}
\end{itemize}

%%%%%%%%%%%%%%%%%%%%%%%%%%%%%%%%%%%%%%%%%%%%%%%%%%%%%%%%%%%%%%%%%%%%%%%%%%%%%%
\section*{Response to review \#1}
%%%%%%%%%%%%%%%%%%%%%%%%%%%%%%%%%%%%%%%%%%%%%%%%%%%%%%%%%%%%%%%%%%%%%%%%%%%%%%
\begin{it}\color{blue}
Reviewer \#1: The paper aims to introduce general approach to solve
curriculum-based course timetabling (CCT) problems. The work builds on
earlier journal paper of the authors introducing application of answer
set programming (ASP) to course timetabling.  Current approach studies
various configurations of ASP solver to improve existing results,
optimizes encoding in ASP and shows how to apply the approach on
minimal perturbation version of the problem.

First part of the paper is devoted to the traditional description of
the curriculum-based timetabling. Next section shows application of
ASP to CCT as it was published in earlier journal version.

Consequent part is devoted to description of various configurations of
the ASP solver, their possible influence on actual results and
presentation of computational results in comparison with the results
at CTT web portal in November 2015. The most interesting results: the
approach was able to find new optimal solution for 14 problems (given
by instance and formulation) and improve bounds for 40 problems.
\end{it}

Thank you for your comments.

\begin{it}\color{blue}
Here I was rather surprised why results refers to November 2015 status
of the other results (footnote 10, page 14). This paper was submitted
at the end of 2016, so it would be better to compare against results
at about the middle of 2016.
\end{it}

As suggested, in Table~7, 
we empirically contrasted the performance of {\asap} with the best
known bounds on the CB-CTT web portal as of \textbf{July 20, 2017}.
We clarified this date in the footnote of Page 3 and 23.

\begin{it}\color{blue}
Section 6 presents extensions of encoding presented earlier, describes
possible combination of different formulation and describes support
for multi-criteria reasoning. It was not clear how presented
extensions are included in computational results. Are they included
anywhere? If some extensions are introduced, corresponding
computational results should verify their importance.
\end{it}

Indeed such questions are very important.
We verified the effectiveness of the extensions in the revision.
Please see \textbf{Point \#2} mentioned above.

\begin{it}\color{blue}
Study of minimal perturbation problems is presented at the end. This
part needs more explanations and justifications. 
\end{it}
\begin{enumerate}
\item 
  \begin{it}\color{blue}
  Results presented for extreme cases are important from the
  research point of view. From practical point of view, results where
  both criteria are reasonably good are much more important.  Given
  the results, it seems that the practical value is rather weak -
  improvement from extremal values (0 in stability) is obtained by
  subsequent increases of the complete stability by one (e.g., \code{DDS2}
  $(47,1)$, $(46,2)$, $(45,3)$).
  \end{it}

In our experiments, 
we first calculated the extreme Pareto optimal solutions regarding
stability and optimality giving both a time limit of 3 hours.
Afterwards, we used another 3 hours to enumerate additional Pareto
optimal solutions.
We showed that the proposed method can find a small subset
of Pareto optimal solutions for many real-world instances.
However, at the same time, 
we observed that the current implementation
is insufficient to find a promising subset that includes solutions
such that both criteria are reasonably well-balanced as you pointed out.
%
This shows a limitation of our approach at present.
To overcome this issue, there might be at least two approaches. 
One is spending more times, since our approach is complete, and we can
find more Pareto optimal solutions.
Another is extending our approach with an advanced technique in MODOP solving 
based on P-minimal model generation (Soh et al, 2017).
It would be promising because P-minimal models can be 
efficiently computed by ASP.
%
We wrote this discussion clearly in the last paragraph of Section~7.

\item 
  \begin{it}\color{blue}
  In addition, it is rather hard to understand results in Table 6:
  it would be nice to see the total number of variables which can be
  possibly changed 
  (How bad or good is 48 in $(0,48)$ for \code{DDS2}? Percentage of changes makes sense here.) 
  and the quality of the optimization criterion for the original
  problem 
  (How the optimization quality is changed from original to new problem?).
  \end{it}

As suggested, we displayed the total number of possible changes in Table~8.
In a scenario we used, the optimization criterion is changed from
an original (a problem instance in UD1) to a new problem (a problem
instance in UD2) due to the additional soft constraint $S_5$. 
Therefore, the optimization quality might be different between them depending on the
instance. For each instance, the difference between UD1 and UD2 can be
seen in Table~7.
%
In the case of the \code{DDS2} instance, 
we obtained five Pareto optimal solutions: 
$(0,48)$,
$(47,1)$,
$(46,2)$, 
$(45,3)$, and 
$(48,0)$.
In each solution, 
the first is a value representing its optimization quality 
(viz. the penalty of soft constraints), and 
the second is a value representing its stabilization quality
(viz. the change from the legacy solution).
Since the optimization quality of \code{DDS2} is 0 both in UD1 and UD2,
and the number of possible changes is 146,
the extreme solution $(0,48)$ has the minimal penalty 0
but 33\% of the legacy assignments are changed.
The extreme solution $(48,0)$ has the minimal change 0
but incurs higher penalty 48 than the optimal quality 0.
The others $(47,1)$, $(46,2)$, and $(45,3)$
are additional Pareto optimal solutions.
%
We clarified this in the revision as well as the case of the
\code{comp13} instance.

\item 
  \begin{it}\color{blue}
  Actually I am not sure if I understand your results well. You
  write ``the first is the value for optimality and the second for
  stability'' (p.21,l.31). However, the extremal values $(48,0)$ and
  $(0,48)$ for \code{DDS2} are rather surprising. Is it the ``optimization
  quality'' (first value) really the same as the number of changes
  (second value)? What do you mean precisely by the 
  ``value for optimality'' for ``stability''? I would expect that it
  is about the optimization quality and the number of changed
  variables. And what is the variable then (do you measure changes in
  the number of changed lectures or courses)?
  \end{it}

  In the case of \code{DDS2}, both has the same values (this happened
  by chance).
  In each solution, 
  the first value is the penalty of soft constraints representing its
  optimization quality.
  The second value is the change from the legacy solution 
  representing its stabilization quality.
  In detail, 
  the legacy solution is represented as a set of legacy assignments
  represented by \code{legacy(assigned(C,R,D,P))}) atoms.
  Each legacy assignment not in the obtained timetable
  counts as 1 violation.
  %
  We clarified this in the revision.
\end{enumerate}

\begin{it}\color{blue}
\noindent
Minor comments:
\end{it}
\begin{itemize}
\item 
  \begin{it}\color{blue}
  Listing 2, l.4: it is not clear from the description of
  integrity constraint (p.5,l.35) that conjunction of literals
  actually must not hold. And the same for Listing 7, l. 17 where the
  similar case is described on p.9,l.33 with the negative statement 
  ``\code{C} is \_not\_ assigned at a period \code{P} on a day \code{D} if
  \code{unavailability_constraint(C,D,P)} holds.''
  \end{it}
  
A rule is called an \emph{integrity constraint} 
if the \emph{head} of the rule (left of `\code{:-}') is empty.
\begin{quote}
\lstinline[basicstyle=\ttfamily\normalsize,mathescape=true]{:- a$_1$,...,a$_m$,not a$_{m+1}$,...,not a$_n$.}
\end{quote}
An integrity constraint is considered as a rule that filters solution candidates, 
meaning that the conjunction of literals in its body must not hold. 
We clarified this in the revision.

\item 
  \begin{it}\color{blue}
  Listing 7, l.9,10: it is not explained what \code{not \{a(X):b(X)\} 1}
  means.  Please add explanation about the ``at most one'' rule.
  \end{it}

The rule in Line 9 enforces that,
for every teacher \code{T}, day \code{D}, and period \code{P},
there is at most one course \code{C} taught by \code{T}
such that \code{assigned(C,D,P)} holds.
In detail, 
if \code{t(T)}, \code{d(D)}, and \code{ppd(P)} hold,
this integrity constraint tells us that
the at-most-one constraint represented by \\
`\verb+{ assigned(C,D,P) : course(C,T,_,_,_,_) } 1+'\\
must be true as well in order to prevent its body from being satisfied. 
In the similar way,
the rule in Line 10 enforces that,
for every curriculum \code{Cu}, day \code{D}, and period \code{P},
there is at most one course \code{C} that belongs to \code{Cu}
such that \code{assigned(C,D,P)} holds.
We clarified this in the revision.

\item 
  \begin{it}\color{blue}
  It is not clear why N-way configuration sorts particular
  configurations in the given ordering as given at Table 3. Can you
  describe reasons for reordering in contrast to the ordering given by
  mean rank in Table 2?
  \end{it}

In Table~3, the rows are ordered 
in descending order by the times the
configuration in the row was included in a best $k$-way configuration.
We clarified this in the revision.

\end{itemize}

%%%%%%%%%%%%%%%%%%%%%%%%%%%%%%%%%%%%%%%%%%%%%%%%%%%%%%%%%%%%%%%%%%%%%%%%%%%%%%
\section*{Response to review \#2}
%%%%%%%%%%%%%%%%%%%%%%%%%%%%%%%%%%%%%%%%%%%%%%%%%%%%%%%%%%%%%%%%%%%%%%%%%%%%%%

\begin{it}\color{blue}
Reviewer \#2: The manuscript presents a model referred to as ``teaspoon''
based on an Answer Set Programming approach to solve a
curriculum-based course timetabling problem and its corresponding
minimal perturbation problem.

In my opinion:
\begin{itemize}
\item this work is original.
\item the paper is appropriately organised so that it is easy to read.
\item the tables and listings are good and support the understanding of the research.
\item the proposed method is able to produce new best known solutions
  for a number of instances of a well-known difficult timetabling
  problem.
\end{itemize}
\end{it}

Thank you very much for your comments!

\begin{it}\color{blue}
The following minor corrections need to be made before publication:
\end{it}

\begin{itemize}
\item 
  \begin{it}\color{blue}
  Page 2, Lines 29-36. Cite a reference(s) to back up these statements.
  \end{it}

  Thank you for your suggestion.
  We added a citation of the following paper:

  \begin{it}%\small
    Erdem E, Gelfond M, Leone N (2016) Applications of ASP. AI Magazine 37(3):53-–68.        
  \end{it}

\item 
  \begin{it}\color{blue}
  Page 3, Lines 35-36. Provide a summary of the remaining sections
  (e.g. Section 2 describes the curriculum-based course timetabling
  problem. Section 2 presents ...).
  \end{it}

  As suggested, we added the following in the revision.

  \begin{it}
The rest of the paper is structured as follows.
Section~2 provides the problem description of CB-CTT\@.
Although we give a brief introduction to ASP and its basic language
constructs in Section~3,
we refer the reader to the literature
(Baral, 2003; Gebser et al, 2012) 
for a comprehensive treatment of ASP\@.
Section~4
describes {\asap}'s fact format of CB-CTT instances and then 
presents a basic {\asap} encoding for solving CB-CTT problems.
Section~5 presents a variety of features of
extended {\asap} encodings for (multi-criteria) CB-CTT solving.
Section~6 provides a detailed empirical analysis of {\asap}
features and performance in contrast to 
the best known bounds obtained by state-of-the-art CB-CTT solving techniques.
Section~7 presents an extension of the {\asap} system to 
minimal perturbation problems in course timetabling. 
Finally, a conclusion is given in Section~8.
  \end{it}


\item 
  \begin{it}\color{blue}
  Page 7, Section 4. The authors could briefly explain the
  reasoning of naming their proposed method as teaspoon?
  \end{it}

  Thank you for your suggestion!

{\asap}: \textbf{T}im\textbf{E}tabling with \textbf{A}nswer \textbf{S}et \textbf{P}r\textbf{O}grammi\textbf{N}g

We explained this in the footnote of Page 8.

\item 
  \begin{it}\color{blue}
  Page 18, Line 43. ``greater than and equal'' should be ``greater than or equal''?
  \end{it}

We did a change as suggested.

\end{itemize}

%%%%%%%%%%%%%%%%%%%%%%%%%%%%%%%%%%%%%%%%%%%%%%%%%%%%%%%%%%%%%%%%%%%%%%%%%%%%%%
\section*{Response to review \#3}
%%%%%%%%%%%%%%%%%%%%%%%%%%%%%%%%%%%%%%%%%%%%%%%%%%%%%%%%%%%%%%%%%%%%%%%%%%%%%%

\begin{it}\color{blue}
Reviewer \#3: The paper investigates a descriptive ASP approach for
course timetabling problems, and evaluates its effectiveness on
problem instances at the timetabling international competition. As a
less studied approach in timetabling research, ASP seems to be an
interesting new direction, thus new contributions are encouraged in
the area. A couple of issues however need to be addressed before the
paper can be considered again for publication at ANOR.

In the paper, it's not clear how ASP has been used for course
timetabling or related education timetabling problems in the existing
research. Due to the missing literature on relevant work and
especially the authors' previous work, it's thus difficult to see the
new contributions from this research. 
For example, are the
formulations in section 2 new? Is the model in section 3 the same or
similar to that's used in the authors' previous research? Is the
teaspoon system built upon an existing system? What's new and improved
in this work? The additional contributions to the literature, based on
previous work, should be clearly presented in the paper to clearly
demonstrates the significance of the research for a new publication.
\end{it}

Indeed such questions are very important.
For the main contributions of the paper, 
please see \textbf{Point \#1} mentioned above.

From the perspective of applying ASP to educational timetabling, 
there exists an early work that studied school timetabling with ASP
(Faber et al,1998). 
The problem description of CB-CTT presented in Section~2 is based on
(Bonutti et al, 2012).
The {\asap} system is built upon general-purpose ASP systems,
in our case {\clingo}. 
That is, {\asap} relies on high-level ASP encodings presented in
Section 4, 5, and 7 and delegates both the grounding and solving tasks
to {\clingo}.
We also clarified these points in the revision.

\begin{it}\color{blue}
Although scalabilty and flexibility have been discussed in the paper,
it's still not clear how the approach should be adapted for other
instance or problems, and how effective it is for different
problems. How much efforts are needed to define a new problem? Are new
models or new formulations needed if problems changed (or even just
slightly changed)? Without this information the scalability and
flexibility are not properly justified.
\end{it}

Thank you for your comments.
For the main features of our proposed approach,
please see \textbf{Point \#3} mentioned above.
From this view, 
our ASP-based approach can be applied to a wide range of timetabling
problems such as school timetabling, examination timetabling, and
post-enrollment course timetabling.

\begin{it}\color{blue}
As a descriptive approach, what are the main limitations compared to
the other related approaches? 
\end{it}

We have not found any drawback so far compared with other related
declarative approaches such as SAT, MaxSAT, and constraint programming
systems.

\begin{it}\color{blue}
Is the approach in the teaspoon system
applied for real world problems?
\end{it}

Yes, our approach can be applied for real world problems, since 
all problem instances we used in our experiments are
based on real data from various universities in Europe.

\begin{it}\color{blue}
The work on multi-criteria optimisation seems interesting and new, and
could be extended to demonstrate new contributions in this research.
\end{it}

Thank you very much for your comments!
We extended the work on multi-objective course timetabling in
Section~7 with more examples and explanations.


%%%%%%%%%%%%%%%%%%%%%%%%%%%%%%%%%%%%%%%%%%%%%%%%%%%%%%%%%%%%%%%%%%%%%%%%%%%%%%
\end{document}

%  LocalWords:  Mutsunori Banbara CTT et al multi Gebser scalability
%  LocalWords:  encodings erlangen ller Rudov UUMCAS ASP's modelling
%  LocalWords:  Extensibility optimality ANOR Katsumi Inoue Kaufmann
%  LocalWords:  Tenda Okimoto Torsten Schaub Takehide Soh Naoyuki CCT
%  LocalWords:  Tamura Philipp Wanko extremal MODOP UD basicstyle ppd
%  LocalWords:  mathescape descendingly organised Erdem Gelfond Leone
%  LocalWords:  Baral im nswer grammi Faber Bonutti scalabilty MaxSAT
%  LocalWords:  optimisation
