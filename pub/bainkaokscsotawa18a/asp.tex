\section{Answer Set Programming}\label{sec:asp}

Answer Set Programming (ASP;~\citep{%
  baral03:cambridge,%
  DBLP:conf/iclp/GelfondL88,%
  DBLP:journals/amai/Niemela99})
is a popular tool for declarative problem solving due to its attractive combination of 
a high-level modeling language with high-performance search engines.

In ASP, problems are described as logic programs, which are sets of rules of the form:
\begin{lstlisting}[basicstyle=\ttfamily\normalsize,mathescape=true,numbers=none]
   a$_0$ :- a$_1$,...,a$_m$,not a$_{m+1}$,...,not a$_n$.
\end{lstlisting}
where $0\leq m\leq n$ and 
each \lstinline[basicstyle=\ttfamily\normalsize,mathescape=true]{a$_i$}
is a propositional atom for ${0}\leq{i}\leq{n}$.
%
The connectives `\code{:-}' and `\code{,}' stand for `if' and `and', respectively. 
The connective `\code{not}' stands for \emph{default negation}.
Each rule is terminated by a period `\code{.}'.
A \emph{literal} is an atom \code{a} or 
\lstinline[basicstyle=\ttfamily\normalsize,mathescape=true]{not a}.
%
The intuitive meaning of the rule is that 
\lstinline[basicstyle=\ttfamily\normalsize,mathescape=true]{a$_0$}
must be true if
\lstinline[basicstyle=\ttfamily\normalsize,mathescape=true]{a$_1$}, 
$\ldots$ , 
\lstinline[basicstyle=\ttfamily\normalsize,mathescape=true]{a$_m$}
are true and if 
\lstinline[basicstyle=\ttfamily\normalsize,mathescape=true]{a$_{m+1}$}, 
$\ldots$ , 
\lstinline[basicstyle=\ttfamily\normalsize,mathescape=true]{a$_n$}
are false.
Semantically, a logic program induces a collection of so-called \emph{answer sets},
which are distinguished models of the program determined by answer sets semantics;
see \citep{DBLP:conf/iclp/GelfondL88} for details.

We call a rule a \emph{fact} if the \emph{body} of the rule 
(right of `\code{:-}') is empty, and we often skip `\code{:-}' when writing facts.
A rule is called an \emph{integrity constraint} 
if the \emph{head} of the rule (left of `\code{:-}') is empty.
\begin{quote}
\lstinline[basicstyle=\ttfamily\normalsize,mathescape=true]{a$_0$.} \\
\lstinline[basicstyle=\ttfamily\normalsize,mathescape=true]{:- a$_1$,...,a$_m$,not a$_{m+1}$,...,not a$_n$.}
\end{quote}
A fact is unconditionally true, i.e., it belongs to every answer set.
An integrity constraint is considered as a rule that filters solution candidates, 
meaning that the conjunction of literals in its body must not hold. 
%the literals in its body must not jointly be satisfied.

To facilitate the use of ASP in practice, 
several extensions have been developed. 
First of all, rules with first-order variables are viewed as shorthand for the set of their ground instances.
Further language constructs include
\emph{conditional literals} and \emph{cardinality constraints} \citep{DBLP:journals/amai/Niemela99}.
The former are of the form
\lstinline[basicstyle=\ttfamily\normalsize,mathescape=true]{a:b$_1$,...,b$_m$},
the latter can be written as
\lstinline[basicstyle=\ttfamily\normalsize,mathescape=true]+s {c$_1$,...,c$_n$} t+,
where \lstinline[basicstyle=\ttfamily\normalsize]{a} and \lstinline[basicstyle=\ttfamily\normalsize,mathescape=true]{b$_i$} are possibly default-negated literals  % for $0\leq i\leq m$,
and each \lstinline[basicstyle=\ttfamily\normalsize,mathescape=true]{c$_j$} is a conditional literal; % for $1\leq i\leq n$;
\lstinline[basicstyle=\ttfamily\normalsize]{s} and \lstinline[basicstyle=\ttfamily\normalsize]{t} provide lower and upper bounds on the number of satisfied literals in the cardinality constraint.
%
Note that either
\lstinline[basicstyle=\ttfamily\normalsize]{s} or
\lstinline[basicstyle=\ttfamily\normalsize]{t} 
can be omitted.
That is, 
\lstinline[basicstyle=\ttfamily\normalsize,mathescape=true]+s {c$_1$,...,c$_n$}+ 
and 
\lstinline[basicstyle=\ttfamily\normalsize,mathescape=true]+{c$_1$,...,c$_n$} t+ 
represent at-least-\lstinline[basicstyle=\ttfamily\normalsize]{s}
and at-most-\lstinline[basicstyle=\ttfamily\normalsize]{t}
constraints respectively.
%
The practical value of both constructs becomes apparent when used with variables.
For instance, a conditional literal like
\lstinline[basicstyle=\ttfamily\normalsize,mathescape=true]{a(X):b(X)}
in a rule's antecedent expands to the conjunction of all instances of \lstinline[basicstyle=\ttfamily\normalsize]{a(X)} for which the corresponding instance of \lstinline[basicstyle=\ttfamily\normalsize]{b(X)} holds.
%
Similarly,
\lstinline[basicstyle=\ttfamily\normalsize,mathescape=true]+2 {a(X):b(X)} 4+
is true whenever at least two and at most four instances of \lstinline[basicstyle=\ttfamily\normalsize]{a(X)} (subject to \lstinline[basicstyle=\ttfamily\normalsize]{b(X)}) are true.
%
A useful\footnote{Care must be taken whenever such expressions are evaluated during solving (rather than grounding).} shortcut are expressions of the form
\lstinline[basicstyle=\ttfamily\normalsize,mathescape=true]+N = {c$_1$,...,c$_n$}+
that binds \lstinline[basicstyle=\ttfamily\normalsize]{N} to the number of satisfied conditional literals
\lstinline[basicstyle=\ttfamily\normalsize,mathescape=true]{c$_j$}.
%
Finally, objective functions minimizing the sum of weights
\lstinline[basicstyle=\ttfamily\normalsize,mathescape=true]{w$_j$}
of conditional literals 
\lstinline[basicstyle=\ttfamily\normalsize,mathescape=true]{c$_j$} are expressed as
\lstinline[basicstyle=\ttfamily\normalsize,mathescape=true]+#minimize {w$_1$:c$_1$,$\dots$,w$_n$:c$_n$}+.
\footnote{Syntactically, each $\mathtt{w_j}$ can be an arbitrary term.
In fact, often tuples are used rather than singular weights to ensure a multi-set property;
in such a case the summation only applies to the first element of selected tuples.}

%%%%%%%%%%%%%%%%%%%%%%%%%%%%%%%%%%%%%%%%%
\begin{figure}[t]
  \centering
  \begin{tikzpicture}[
      x=1pt,y=1pt,
      >=stealth',
      vertex/.style={draw,minimum width=16,minimum height=16,inner sep=0,circle},
      scale = 0.9, transform shape
    ]
    \begin{scope}[every node/.style={vertex}]
      \node (1) at ( 80,75 ) {1};
      \node (2) at (200,75 ) {2};
      \node (3) at ( 80,145) {3};
      \node (4) at (140,25 ) {4};
      \node (5) at (200,145) {5};
      \node (6) at (140,110) {6};
    \end{scope}
    \path
      (1) edge [-]  (2) edge [-] (3) edge [-] (4)
      (2) edge [-] (4) edge [-]  (5) edge [-] (6)
      (3) edge [-,bend right=60,looseness=1.5]  (4) edge [-] (5) edge [-]  (6)
      (4) edge [-,bend right=60,looseness=1.5]  (5)
      (5) edge [-] (6)
      ;
  \end{tikzpicture}
\caption{Undirected graph having 6 nodes and 11 edges}
\label{fig:graph}
\end{figure}

%%% Local Variables:
%%% mode: latex
%%% TeX-master: "paper"
%%% End:

%
\lstinputlisting[float=t,caption={%
ASP facts representing the graph of Figure.~\ref{fig:graph} (\code{graph.lp})},%
captionpos=b,frame=single,label=code:graph.lp,%
numbers=none,%
breaklines=true,%
columns=fullflexible,keepspaces=true,%
basicstyle=\ttfamily\scriptsize]{code_graph_lp.tex}
%
\lstinputlisting[float=t,caption={%
ASP rules for graph coloring (\texttt{color.lp})},%
captionpos=b,frame=single,label=code:color.lp,%
%numbers=none,%
breaklines=true,%
columns=fullflexible,keepspaces=true,%
basicstyle=\ttfamily\scriptsize]{code_color_lp.tex}
%
\lstinputlisting[float=t,caption={Solving a graph coloring problem (\code{graph.lp} and \code{color.lp})},%
captionpos=b,frame=single,label=code:color.log,%
numbers=none,%
breaklines=true,%
columns=fullflexible,keepspaces=true,%
basicstyle=\ttfamily\scriptsize]{code_color_log.tex}
%%%%%%%%%%%%%%%%%%%%%%%%%%%%%%%%%%%%%%%%%

For solving a problem instance of a problem class in ASP, 
we encode the problem instance as a set of ASP facts and the problem
class as a set of ASP rules.
In turn, the facts are combined with the rules, 
and the result is
subsequently solved by an off-the-shelf ASP system that returns
an answer set representing a solution to the original problem.

As an example, let us consider a graph coloring problem.
The problem consists in finding
assignments of colors to nodes such that no two nodes connected by an
edge have the same color. 
A problem instance is given by a graph as in Figure~\ref{fig:graph}.
%
It is represented as facts of predicates
\code{node/1} and \code{edge/2} in Listing~\ref{code:graph.lp}.
The 3-colorability problem class is encoded in Listing~\ref{code:color.lp}.
%
Line 1 provides the available colors as facts.
Line 3 and 4 express the actual colorability problem.
The predicate \code{color(X,C)} is used to express that a node
\code{X} is colored with \code{C}.
%
The rule in Line 3 generates for each node \code{X} a set of candidate assignments
subject to the condition that there is exactly one color \code{C} such that \code{color(X,C)} holds.
In detail, the conditional literal
\code{color(X,C):col(C)}
in the cardinality constraint
expands to the conjunction of 
\code{color(X,r)}, \code{color(X,b)}, and \code{color(X,g)}, since
the facts \code{col(r)}, \code{col(b)}, and \code{col(g)} unconditionally hold.
That is, line 3 expresses that each node \code{X} must be colored with
exactly one color among red (\code{r}), blue (\code{b}), and green (\code{g}).
%
The integrity constraint in Line 4 expresses that all connected nodes
\code{X} and \code{Y} must not be colored with the same color \code{C},
since, as mentioned above, 
the conjunction of literals in its body must not hold. 
%
Line 6 is a directive, that is, a meta statement advising the ASP system to project answer sets
onto instances of predicate \code{color/2}.
%
An answer set computed by the ASP system {\clingo} is shown in
Listing~\ref{code:color.log}; 
it represents a coloring of node \code{1} with green,
\code{2} and \code{3} with blue, 
\code{4} with red, 
\code{5} with green, and 
\code{6} with red.

Modern ASP systems like {\clingo} first translate
user-defined logic programs (with variables) into 
equivalent ground (that is, variable-free) programs, 
and then compute the answer sets of the ground programs.
Particularly, the former task is called \emph{grounding}.
The foundations and algorithms underlying the technology of {\clingo}
are described in detail in \citep{gekakasc12a}.

%%% Local Variables:
%%% mode: latex
%%% TeX-master: "paper"
%%% End:
