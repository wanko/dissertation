
\subsection{Semantic underpinnings}
\label{sec:semantics}

Given the hands-on nature of this tutorial,
we only give an informal idea of the semantic principles underlying theory solving in ASP.

As mentioned in Section~\ref{sec:background}, a logic program induces a set of stable models.
%
To extend this concept to logic programs with theory expressions,
we follow the approach of lazy theory solving~\cite{baseseti09a}.
We abstract from the specific semantics of a theory by considering the theory atoms representing the underlying theory constraints.
The idea is that a regular stable model of a program over regular and theory atoms is only valid with respect to a theory,
if the constraints induced by the truth assignment to the theory atoms are satisfiable in the theory.

In the above example, 
this amounts to finding a numeric assignment to all theory variables satisfying all difference and linear constraints associated with theory atoms.
%
The ground program in~\ref{prg:grd:diff} has a single stable model consisting of all regular and theory atoms in lines~1-16.
%
Here, we easily find assignments satisfying the induced constraints,
e.g.\
\(
\texttt{start(1)}\mapsto 1
\),
\(
\texttt{end(1)}\mapsto 2
\),
\(
\texttt{start(2)}\mapsto 2
\), and
\(
\texttt{end(1)}\mapsto 3
\).

In fact,
there are alternative semantic options for capturing theory atoms, as detailed in~\cite{gekakaosscwa16a}.
%
First of all,
we may distinguish whether imposed constraints are only determined outside or additionally inside a logic program.
%
This leads to the distinction between \emph{defined} and \emph{external} theory atoms.%
\footnote{This distinction is analogous to that between head and input atoms,
  defined via rules or \lstinline{#external} directives \cite{gekakasc14b}, respectively.}
%
While external theory atoms must only be satisfied by the respective theory,
defined ones must additionally be derivable through rules in the program.
%
The second distinction concerns the interplay of ASP with theories.
More precisely, it is about the logical correspondence between theory atoms and theory constraints.
%
This leads us to the distinction between \emph{strict} and \emph{non-strict} theory atoms.
%
The strict correspondence requires
a constraint to be satisfied 
\textit{iff}
the associated theory atom is true.
%
A weaker since only implicative condition is imposed in the non-strict case.
Here, a constraint must hold 
\textit{only if}
the associated theory atom is true.
%
In other words, only non-strict theory atoms assigned true impose requirements, 
while constraints associated with falsified non-strict theory atoms are free to hold or not.
%
However, by contraposition, a violated constraint leads to a false non-strict theory atom.

%%% Local Variables:
%%% mode: latex
%%% TeX-master: "paper"
%%% End:
