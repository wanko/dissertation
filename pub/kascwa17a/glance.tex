
\subsection{A gentle introduction}
\label{sec:glance}

A key feature, distinguishing \clingo{} from its predecessors,
is the possibility to structure (non-ground) input rules into subprograms.
To this end,
a program can be partitioned into several subprograms by means of the directive \lstinline{#program};
it comes with a name and an optional list of parameters.
Once given in the input,
the directive gathers all rules up to the next such directive (or the end of file)
within a subprogram identified by the supplied name and parameter list.
%
As an example,
two subprograms \lstinline{base} and \lstinline{acid(k)} can be specified as follows:
%
\begin{lstlisting}[language=clingo]
a(1).
#program acid(k).
b(k).
c(X,k) :- a(X).
#program base.
a(2).
\end{lstlisting}
%
Note that \lstinline{base} is a dedicated subprogram (with an empty parameter list):
in addition to the rules in its scope,
it gathers all rules not preceded by any \lstinline{#program} directive.
Hence, in the above example, the \lstinline{base} subprogram includes the facts \lstinline{a(1)} and \lstinline{a(2)},
although, only the latter is in the actual scope of the directive in line~5.
% 
Without further control instructions (see below),
\clingo{} grounds and solves the \lstinline{base} subprogram only,
essentially, yielding the standard behavior of ASP systems.
The processing of other subprograms such as \lstinline{acid(k)} is subject to scripting control.

For customized control over grounding and solving,
a \lstinline{main} routine
(taking a control object representing the state of \clingo{} as argument)
can be supplied.
For illustration, let us consider two \python{} \lstinline{main} routines:
\footnote{The \lstinline{ground} routine takes a list of pairs as argument.
  Each such pair consists of a subprogram name (e.g.\ \lstinline{base} or \lstinline{acid}) and a list of actual parameters (e.g.\ \lstinline{[]} or \lstinline{[42]}).}
%
\begin{lstlisting}[firstnumber=7,language=clingo]
#script(python)
def main(prg):
    prg.ground([("base",[])])
    prg.solve()
#end.
\end{lstlisting}
%
While the above control program matches the default behavior of \clingo,
the one below ignores all rules in the \lstinline{base} program but rather
contains a \lstinline{ground} instruction for \lstinline{acid(k)} in line~8,
where the parameter~\lstinline{k} is to be instantiated with the term \lstinline{42}.
%
\begin{lstlisting}[firstnumber=7,language=clingo]
#script(python)
def main(prg):
    prg.ground([("acid",[42])])
    prg.solve()
#end.
\end{lstlisting}
%
Accordingly, the schematic fact \lstinline{b(k)} is turned into \lstinline{b(42)},
no ground rule is obtained from `\lstinline{c(X,k) :- a(X)}' due to lacking instances of \lstinline{a(X)},
and the \lstinline{solve} command in line~10 yields a stable model consisting of
\lstinline{b(42)} only.
Note that \lstinline{ground} instructions apply to the subprograms
given as arguments,
while \lstinline{solve} triggers reasoning w.r.t.\ all accumulated ground rules.

In order to accomplish more elaborate reasoning processes,
like those of \iclingo~\cite{gekakaosscth08a} and \oclingo~\cite{gegrkasc11a} or other customized ones,
it is indispensable to activate or deactivate ground rules on demand.
For instance, former initial or goal state conditions need to be
relaxed or completely replaced when modifying a planning problem, e.g.,
by extending its horizon.%
\footnote{The planning horizon is the maximum number of steps a planner takes into account when searching for a plan.}
%
While the two mentioned predecessors of \clingo{} relied on the \lstinline{#volatile} directive
to provide a rigid mechanism for the expiration of transient rules,
\clingo{} captures the respective functionalities and customizations
thereof in terms of the \lstinline{#external} directive.
This directive goes back to \lparse\ \cite{lparseManual} and was also
supported by \clingo's predecessors to exempt (input) atoms
from simplifications (and fixing them to false).
As detailed in the following,
the \lstinline{#external} directive of \clingo{} provides a generalization
that, in particular, allows for a flexible handling of yet undefined atoms.

For continuously assembling ground rules evolving at different stages of a
reasoning process, \lstinline{#external} directives declare atoms that may
still be defined by rules added later on.
%
In terms of module theory~\cite{oikjan06a},
such atoms correspond to inputs, which (unlike undefined output atoms) must not be simplified.
For declaring input atoms, 
\clingo{} supports schematic \lstinline{#external} directives that are instantiated along with
the rules of their respective subprograms.
To this end, a directive like
%
\begin{lstlisting}[numbers=none,language=clingo]
#external p(X,Y) : q(X,Z), r(Z,Y).
\end{lstlisting}
%
is treated similar to a rule
%
`\lstinline{p(X,Y) :- q(X,Z), r(Z,Y)}'
during grounding.
%
However, the head atoms of the resulting ground instances are merely collected as inputs,
whereas the ground rules as such are discarded.

Once grounded, the truth value of external atoms can be changed via the \clingo{} API
(until the atoms become defined by corresponding rules).
By default, the initial truth value of external atoms is set to false.
Then, for example, with \clingo's \python{} API,
\lstinline{assign_external(self,p(a,b),True)}%
\footnote{%
In order to construct atoms, symbolic terms, or function terms, respectively, the \clingo{} API function \lstinline{Function} has to be used.
Hence, the expression \lstinline{p(a,b)} actually stands for \lstinline{Function("p", [Function("a"), Function("b")])}.}
can be used to set the truth value of the external atom \lstinline{p(a,b)} to true.
Among others,
this can be used to activate and deactivate rules in logic programs.
For instance,
the integrity constraint `\lstinline{:- q(a,c), r(c,b), p(a,b)}' is ineffective whenever \lstinline{p(a,b)} is false.

A full specification of \clingo's \python{} API can be found at \url{https://potassco.org/clingo/python-api/current/clingo.html}.

%%% Local Variables:
%%% mode: latex
%%% TeX-master: "paper"
%%% End:
