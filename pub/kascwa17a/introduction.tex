
\section{Introduction}\label{sec:introduction}

Answer Set Programming (ASP~\cite{baral02a}) has established itself among the popular paradigms for Knowledge Representation and Reasoning (KRR),
in particular, when it comes to solving knowledge-intense combinatorial (optimization) problems.
ASP's unique combination of a simple yet rich modeling language with highly performant solving technology
has led to an increasing interest in ASP in academia as well as industry.
Another primary asset of ASP is its versatility, arguably elicited by its roots in KRR.
On the one hand,
ASP's first-order modeling language offers, for instance, cardinality and weight constraints
as well as means to express multi-objective optimization functions.
This allows ASP to readily express problems in neighboring fields such as
Satisfiability Testing (SAT~\cite{SATHandbook})
and
Pseudo-Boolean Solving (PB~\cite{rouman09a}),
as well as
Maximum Satisfiability Testing (MaxSAT~\cite{liman09a})
and even
more general constraint satisfaction problems possibly involving optimization.
On the other hand,
these constructs must be supported by the corresponding solvers,
leading to dedicated treatments of cardinality and weight constraints
along with sophisticated optimization algorithms.
Moreover, mere satisfiability testing is often insufficient for addressing KRR problems.
That is why ASP solvers offer additional reasoning modes involving enumerating, intersecting, or unioning solutions,
as well as combinations thereof, e.g., intersecting all optimal solutions.

In a sense,
the discussed versatility of modern ASP can be regarded as the result of hybridizing the original approach~\cite{gellif88b} in several ways.
So far, however, most hybridization was accomplished within the solvers and is thus inaccessible to the user.
For instance, the dedicated treatment of aggregates like cardinality and weight constraints is fully opaque.
The same applies to the control of successive solver calls happening during optimization.
Although a highly optimized implementation of such prominent concepts makes perfect sense,
the increasing range and resulting diversification of applications of ASP
calls for easy and generic means to enrich ASP with dedicated forms of reasoning.
This involves
the extension of ASP's solving capacities with means for handling constraints foreign to ASP
as well as
means for customizing solving processes to define complex forms of reasoning.
The former extension is usually called \emph{theory reasoning} (or \emph{theory solving}) and
the resulting conglomerate of ASP extensions is subsumed under the umbrella term \emph{ASP modulo theories}.
The other extension addresses the customization of ASP solving processes by \emph{multi-shot ASP solving},
providing operative solving processes that deal with continuously changing logic programs.

Let us motivate both techniques by means of two exemplary ASP extensions, aggregate constraints and optimization.
%
With this end in view,
keep in mind that ASP is a model, ground, and solve paradigm.
Hence such extensions are rarely limited to a single component
but often spread throughout the whole workflow.
This begins with the addition of new language constructs to the input language,
requiring in turn amendments to the grounder as well as
syntactic means for passing the ground constructs to a downstream system.
In case they are to be dealt with by an ASP solver,
it must be enabled to treat the specific input and incorporate corresponding solving capacities.
%
Finally,
each such extension is theory-specific and requires different means at all ends.

So first of all, consider what is needed to extend an ASP system like \clingo{} with a new type of aggregate constraint?
The first step consists in defining the syntax of the aggregate type.
Afterwards, the ASP grounder has to be extended to be able to parse and instantiate the corresponding constructs.
Then, there are two options, either the ground aggregates are translated into existing ASP language constructs (and we are done),%
\footnote{Alternatively, this could also be done before instantiation.}
or they are passed along to a downstream ASP solver.
The first alternative is also referred to as \emph{eager}, the latter as \emph{lazy theory solving}.
The next step in the lazy approach is to define an intermediate format (or data structure) to pass instances of
the aggregate constraints from the grounder to the solver,
not to forget respective extensions to the back- and front-ends of the two ASP components.
Now, that the solver can internalize the new constructs, it must be equipped with corresponding processing capacities.
They are usually referred to as \emph{theory propagators} and inserted into the solver's infrastructure for propagation.
When solving, the idea is to leave the Boolean solving machinery intact by associating with each theory constraint
an auxiliary Boolean variable.
During propagation, the truth values of the auxiliary variables are passed to the corresponding theory propagators
that then try to satisfy or falsify the respective theory constraints, respectively.
Finally, when an overall solution is found, the theory propagators are in charge of outputting their part (if applicable).
One can imagine that each such extension involves a quite intricate engineering effort
since it requires working with the ASP system's low level API.
%
\clingo{} allows us to overcome this problem by providing easy and generic means for adding theory solving capacities.
On the one side, it offers theory grammars for expressing theory languages whose expressions are seamlessly integrated in its grounding process.
On the other side, a simple interface consisting of four methods offers an easy integration of theory propagators into the solver,
either in \C, \cpp, \lua, or \python.

Let us now turn to (branch-and-bound-based) optimization and see what infrastructure is needed to extend a basic ASP solver.
In fact, for the setup, we face a similar situation as above and all steps from syntax definition to internalization
are analogous for capturing objective functions.
The first step in optimization is to find an initial solution.
If none exists, we are done.
Otherwise the system enters a simple loop.
The objective value of the previous solution is determined
and a constraint is added to the problem specification requiring that
a solution must have a strictly better objective value than the one just obtained.
Then, the solver is launched again to compute a better solution.
If none is found, the last solution is optimal.
Otherwise, the system re-enters the loop in order to find an even better solution.
This solving process faces a succession of solver invocations dealing with slightly changing problem specifications.
The direct way to implement this is to use a script that repeatedly calls an ASP solver after each problem expansion.
However, such an approach bears great redundancies due to repeated grounding and solving efforts from scratch.
%
Unlike this,
\clingo\ offers evolving grounding and solving processes.
Such processes lead to operative ASP systems that possess an internal state that can be manipulated by certain operations.
Such operations allow for adding, grounding, and solving logic programs as well as
setting truth values of (external) atoms.
The latter does not only provide a simple means for incorporating external input but
also for enabling or disabling parts of the current logic program.
These functionalities allow for dealing with changing logic programs in a seamless way.
As above, corresponding application programming interfaces (APIs) are available in \C, \cpp, \lua, or \python.

The remainder of this tutorial is structured as follows.
Section~\ref{sec:background} provides some formal underpinnings for the following sections
without any claim to completeness.
Rather we refer the reader to the literature for comprehensive introductions to ASP and its computing machinery,
among others~\cite{baral02a,lifschitz04a,eiiakr09a,gekakasc12a,gelkah14a}.
As a result,
this tutorial is not self-contained and rather aims at a hands-on introduction to using \clingo's API
for multi-shot and theory solving.
Both approaches are described in Section~\ref{sec:multi} and~\ref{sec:theory} by drawing on material from~\cite{gekakasc14b,gekaobsc15a}
and~\cite{gekakaosscwa16a}, respectively.
Section~\ref{sec:case} is dedicated to a case-study detailing how \clingo{} can be extended with difference constraints over integers,
or more precisely Quantifier-free Integer Difference Logic (QF-IDL).

%%% Local Variables:
%%% mode: latex
%%% TeX-master: "paper"
%%% End:
