\subsection{Algorithmic aspects}
\label{sec:algo}

The algorithmic approach to ASP solving modulo theories of \clingo, 
or more precisely that of its underlying ASP solver \clasp, 
follows the lazy approach to solving in Satisfiability Modulo Theories (SMT~\cite{baseseti09a}).
We give below an abstract overview that serves as light algorithmic underpinning for the description of \clingo's implementation
given in the next section.

As detailed in~\cite{gekasc09c}, a ground program~$P$ induces
\emph{completion} and \emph{loop nogoods}, called $\Delta_P$ or $\Lambda_P$, respectively,
that can be used for computing stable models of $P$.
%
Nogoods represent invalid partial assignments and can be thought of as negative Boolean constraints.
%
We represent (partial) assignments as consistent sets of literals.
%
An assignment is total if it contains either the positive or negative literal of each atom.
We say that a nogood is violated by an assignment if the former is contained in the latter;
a nogood is unit if all but one of its literals are in the assignment. 
%
Each total assignment not violating any nogood in $\Delta_P\cup\Lambda_P$
yields a regular stable model of~$P$, and such an assignment is called a solution
(for $\Delta_P\cup\Lambda_P$).
%
To accommodate theories,
we identify a theory $T$ with a set $\Delta_T$ of \emph{theory nogoods},%
\footnote{See \cite{gekakaosscwa16a} for different ways of associating theories with nogoods.}
and extend the concept of a solution in the straightforward way.

The nogoods in $\Delta_P\cup\Lambda_P\cup\Delta_T$ provide the
logical fundament for the Conflict-Driven Constraint Learning (CDCL) procedure
(cf.\ \cite{malyma09a,gekasc09c})
outlined in Figure~\ref{fig:cdcl}.
While the completion nogoods in~$\Delta_P$ are usually made explicit and subject to
unit propagation,%
\footnote{Unit propagation extends an assignment with literals complementary to the ones missing in unit nogoods.}
the loop nogoods in~$\Lambda_P$ as well as theory nogoods in~$\Delta_T$ are typically
handled by dedicated propagators and particular members are selectively recorded.

\begin{figure}[t]
  \newcounter{cddl}
  \setcounter{cddl}{8}
  \renewcommand{\thecddl}{\Alph{cddl}}
  \begin{tabbing}
   (XX)  \= XX \= XX \= XX \= \kill%
   \refstepcounter{cddl}(\thecddl)\label{fig:cdcl:init}
         \> \textit{initialize} \` // register theory propagators and initialize watches \\
         \> \textbf{loop} \\
         \> \> \textit{propagate} completion, loop, and recorded nogoods
               \` // deterministically assign literals \\
         \> \> \textbf{if} no conflict \textbf{then} \\
         \> \> \> \textbf{if} all variables assigned \textbf{then} \\
   \setcounter{cddl}{2}%
   \refstepcounter{cddl}(\thecddl)\label{fig:cdcl:check}
         \> \> \> \> \textbf{if} some $\delta\in\Delta_T$ is violated \textbf{then} record $\delta$
                     \` // theory propagators check $\Delta_T$ \\
         \> \> \> \> \textbf{else} \textbf{return} variable assignment
                   \`// theory-based stable model found \\
         \> \> \> \textbf{else} \\
   \setcounter{cddl}{15}%
   \refstepcounter{cddl}(\thecddl)\label{fig:cdcl:propagate}
         \> \> \> \> \textit{propagate} theories
                     \`// theory propagators may record theory nogoods from $\Delta_T$ \\
         \> \> \> \> \textbf{if} no nogood recorded \textbf{then} \textit{decide}
                     \`// non-deterministically assign some literal \\
         \> \> \textbf{else} \\
         \> \> \> \textbf{if} top-level conflict \textbf{then} \textbf{return} unsatisfiable \\
         \> \> \> \textbf{else} \\
         \> \> \> \> \textit{analyze} \`// resolve conflict and record a conflict constraint \\
   \setcounter{cddl}{20}%
   \refstepcounter{cddl}(\thecddl)\label{fig:cdcl:undo}
         \> \> \> \> \textit{backjump} \`// undo assignments until conflict constraint is unit
 \end{tabbing}
\caption{Basic algorithm for Conflict-Driven Constraint Learning (CDCL) modulo theories}
\label{fig:cdcl}
\end{figure}%
%
While a dedicated propagator for loop nogoods is built-in in systems like \clingo,
those for theories are provided via the interface \textbf{Propagator} in Figure~\ref{fig:interface}.
To utilize custom propagators,
the algorithm in Figure~\ref{fig:cdcl} includes an \emph{initialization} step in line~(\ref{fig:cdcl:init}).
In addition to the ``registration'' of a propagator for a theory
as an extension of the basic CDCL procedure,
common   tasks performed in this step include setting up internal data structures and
so-called watches for (a subset of) the theory atoms,
so that the propagator will be invoked (only) when some watched literal gets assigned.

As usual,
the main CDCL loop starts with unit propagation on completion and loop nogoods,
the latter handled by the respective built-in propagator, as well as any nogoods
already recorded.
If this results in a non-total assignment without conflict,
theory propagators for which some of their watched literals have been assigned
are invoked in line~(\ref{fig:cdcl:propagate}).
A propagator for a theory~$T$ can then inspect the current assignment,
update its data structures accordingly, and most importantly,
perform \emph{theory propagation} determining theory nogoods $\delta\in\Delta_T$ to record.
Usually, any such nogood~$\delta$ is unit in order to trigger a conflict or unit propagation,
although this is not a necessary condition.
The interplay of unit and theory propagation continues until a conflict or
total assignment arises,
or no (further) watched literals of theory propagators get assigned by unit propagation.
In the latter case, some non-deterministic decision is made to extend the partial
assignment at hand and then to proceed with unit and theory propagation.

If no conflict arises and an assignment is total,
in line~(\ref{fig:cdcl:check}), theory propagators are called, one by one,
for a final \emph{check}.
The idea is that, e.g., a ``lazy'' propagator for a theory~$T$
that does not exhaustively test violations of its theory nogoods by partial assignments
can make sure that the assignment is indeed a solution for~$\Delta_T$, or record some
violated nogood(s) from $\Delta_T$ otherwise.
Even in case theory propagation on partial assignments is exhaustive and a final
check is not needed to detect conflicts,
the information that search led to a total assignment can be useful in practice, e.g.,
to store values for integer variables like
\lstinline{start(1)},
\lstinline{start(2)},
\lstinline{end(1)}, and
\lstinline{end(2)} in Listing~\ref{prg:grd:diff}
that witness the existence of a solution for $T$.

Finally, in case of a conflict, i.e., some completion or recorded nogood is violated by
the current assignment,
provided that some non-deterministic decision is involved in the conflict,
a new conflict constraint is recorded and utilized to guide backjumping
in line~(\ref{fig:cdcl:undo}), as usual with CDCL.
In a similar fashion as the assignment of watched literals serves as
trigger for theory propagation, theory propagators are informed when they become
unassigned upon backjumping.
This allows the propagators to \emph{undo} earlier operations, e.g., internal data structures can be
reset to return   to a state taken prior to the assignment of watches.

In summary, the basic CDCL procedure is extended in four places to account for
custom propagators: initialization, propagation of (partial) assignments,
final check of total assignments, and undo steps upon backjumping.
% The next section presents the corresponding interface of \clingo.

%%% Local Variables:
%%% mode: latex
%%% TeX-master: "paper"
%%% End:
