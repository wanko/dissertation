
\subsection{Branch-and-bound-based optimization}
\label{sec:optimization}

We illustrate \clingo's multi-shot solving machinery in this as well as the next section via a simple Towers of Hanoi puzzle.
The complete source code of this example is available at \url{https://github.com/potassco/clingo/tree/master/examples/clingo/opt}.
Our example consists of three pegs and four disks of different size; it is shown in Figure~\ref{fig:hanoi}.
The goal is to move all disks from the left peg to the right one.
Only the topmost disk of a peg can be moved at a time.
% ------------------------------------------------------------
\begin{figure}[ht]
\centering
	\begin{tikzpicture}
		% peg a 0.0 - 3.0
		\draw (0.5,0)   rectangle node {1} (2.5,0.5);
		\draw (0.7,0.5) rectangle node {2} (2.3,1);
		\draw (0.9,1)   rectangle node {3} (2.1,1.5);
		\draw (1.1,1.5) rectangle node {4} (1.9,2);
		\draw (1.4,2)   rectangle          (1.6,2.1);
		\draw [-] (1.5,2.3) node [above] {\textbf{a}};
		% peg b 3.0 - 6.0
		\draw (4.4,0.0)   rectangle          (4.6,2.1);
		\draw [-] (4.5,2.3) node [above] {\textbf{b}};
		% peg c 6.0 - 9.0
		\draw [dashed] (6.5,0)   rectangle node {1} (8.5,0.5);
		\draw [dashed] (6.7,0.5) rectangle node {2} (8.3,1);
		\draw [dashed] (6.9,1)   rectangle node {3} (8.1,1.5);
		\draw [dashed] (7.1,1.5) rectangle node {4} (7.9,2);
		\draw (7.4,2.0) rectangle          (7.6,2.1);
		\draw [-] (7.5,2.3) node [above] {\textbf{c}};
	\end{tikzpicture}
\caption{Towers of Hanoi: initial and goal situation}
\label{fig:hanoi}
\end{figure}%
%%% Local Variables: 
%%% mode: latex
%%% TeX-master: "../book"
%%% End: 

% ------------------------------------------------------------
Furthermore,
a disk cannot be moved to a peg already containing a disk of smaller size.
Although there is an efficient algorithm to solve our simple puzzle,
we do not exploit it and below merely specify conditions for
sequences of moves being solutions.
%
More generally,
the Towers of Hanoi puzzle is a typical planning problem,
in which the aim is to find a plan, that is, a sequence of actions, that leads from an initial state to a state satisfying a goal.
% Each action application describes a state transition.

To illustrate how multi-shot solving can be used for realizing branch-and-bound-based optimization,
we consider the problem of finding the shortest plan solving our puzzle within a given horizon.
%
To this end,
we adapt the Towers of Hanoi encoding from~\cite{gekakasc12a} in Listing~\ref{fig:toh:opt:enc}.
% --------------------------------------------------------------------------------------------------------------------------------------------
\lstinputlisting[float=tb,literate={\%\%}{}{0},escapeinside={\#(}{\#)},label={fig:toh:opt:enc},language=clingo,caption={Bounded towers of hanoi encoding (tohB.lp)}]{example/opt/tohB.lp}
% --------------------------------------------------------------------------------------------------------------------------------------------
Here, the length of the horizon is given by parameter \texttt{n}.
%
The problem instance in Listing~\ref{fig:toh:opt:ins} together with line~\ref{fig:toh:opt:enc:init} in Listing~\ref{fig:toh:opt:enc} gives the initial
configuration of disks in Figure~\ref{fig:hanoi}.
% --------------------------------------------------------------------------------------------------------------------------------------------
\lstinputlisting[float=tb,literate={\%\%}{}{0},escapeinside={\#(}{\#)},label={fig:toh:opt:ins},language=clingo,caption={Towers of hanoi instance (tohI.lp)}]{example/opt/tohI.lp}
% --------------------------------------------------------------------------------------------------------------------------------------------
%
Similarly,
the goal is checked in lines~\ref{fig:toh:opt:enc:goal:begin}--\ref{fig:toh:opt:enc:goal:end} of Listing~\ref{fig:toh:opt:enc} 
(by drawing on the problem instance in Listing~\ref{fig:toh:opt:ins}).
Because the overall objective is to solve the problem in the minimum number of steps within a given bound,
it is successively tested in line~\ref{fig:toh:opt:enc:goal:begin}.
Once the goal is established, it persists in the following steps.
This allows us to read off whether the goal was reached at the planning horizon (in line~\ref{fig:toh:opt:enc:goal:end}).
%
The state transition function along with state constraints are described in lines~\ref{fig:toh:opt:enc:trans:begin}--\ref{fig:toh:opt:enc:trans:end}.
Since the encoding of the Towers of Hanoi problem is fairly standard, we refer the interested reader to~\cite{gekakasc12a}
and devote ourselves in the sequel to implementing branch-and-bound-based minimization.
In view of this, note that line~\ref{fig:toh:opt:enc:move} ensures that moves are only permitted if the goal is not yet achieved in the previous state.
This ensures that the following states do not change anymore and allows for expressing the optimization function in line~\ref{fig:toh:opt:enc:minimize} as: minimize the number of states where the goal is not reached.

% --------------------------------------------------------------------------------------------------------------------------------------------
\lstinputlisting[float=tb,literate={\%\%}{}{0},escapeinside={\#(}{\#)},basicstyle={\ttfamily\small},label={prg:opt:main},caption={Branch and bound optimization (opt.lp)},language=clingo]{example/opt/opt.lp}
% --------------------------------------------------------------------------------------------------------------------------------------------

Listing~\ref{prg:opt:main} contains a logic program for bounding the next solution and the actual optimization algorithm.
%
The logic program expects a bound \lstinline{b} as parameter and adds an integrity constraint in line~\ref{prg:opt:main:sum} 
ensuring that the next stable model yields a better bound than the given one.
%
The minimization algorithm starts by grounding the \lstinline{base} program in line~\ref{prg:opt:main:ground}
before it enters the loop in lines~\ref{prg:opt:loop:begin}--\ref{prg:opt:loop:end}.
This loop implements the branch-and-bound-based search for the minimum
by searching for stable models while updating the bound until the problem is unsatisfiable.
%
Note the use of the $\code{with}$ clause in line~\ref{prg:opt:main:solve} that is used to acquire and release a solve handle.
With it,
the nested loop in lines~\ref{prg:opt:main:models:begin}--\ref{prg:opt:main:models:end} iterates over the found stable models.
%
If there is a stable model, lines~\ref{prg:opt:main:atoms:begin}--\ref{prg:opt:main:atoms:end} iterate over the atoms of the stable model
while summing up the current bound by extracting the weight of atoms over predicates \lstinline[mathescape=true]{_minimize/$n$} with $n>0$.%
\footnote{In our case, $n=2$ would be sufficient.}
%
We check that the first argument of the atom is an integer and ignore atoms where this is not the case; 
just as is the case of the \lstinline{#sum} aggregate in line~\ref{prg:opt:main:sum}.
%
The loop over the stable models is exited in line~\ref{prg:opt:main:models:end}.
%
Note that this bypasses the $\code{else}$ clause in line~\ref{prg:opt:main:finish:begin}
and the algorithm continues in line~\ref{prg:opt:main:print_bound} 
with printing the bound and adding an integrity constraint in line~\ref{prg:opt:main:ground_bound}
making sure that the next stable model is strictly better than the current one.
%
Furthermore, note that grounding happens after the $\code{with}$ clause because it must not interfere with an active search for stable models.
%
Finally,
if the program becomes unsatisfiable, the branch and bound loop in lines~\ref{prg:opt:loop:begin}--\ref{prg:opt:loop:end} is exhausted.
Hence, control continues in the $\code{else}$ clause in lines~\ref{prg:opt:main:finish:begin}--\ref{prg:opt:main:finish:end} printing that the previously found stable model (if any) is the optimal solution and exiting the outermost while loop in line~\ref{prg:opt:main:finish:end} terminating the algorithm.

When running the augmented logic program in Listing~\ref{fig:toh:opt:enc}, \ref{fig:toh:opt:ins}, and~\ref{prg:opt:main} with a horizon of $17$,
the solver finds plans of length 17, 16, and 15 and shows that no plan of length 14 exists.
This is reflected by \clingo's output indicating 4 solver calls and 3 found stable models:
%
% --------------------------------------------------------------------------------------------------------------------------------------------
\lstinputlisting[numbers=none,escapechar=!,basicstyle=\ttfamily\small]{example/opt/opt.txt}
% --------------------------------------------------------------------------------------------------------------------------------------------

Last but not least,
note that the implemented above functionality is equivalent to using \clingo's inbuilt optimization mode by replacing line~\ref{fig:toh:opt:enc:minimize} in Listing~\ref{fig:toh:opt:enc} with
\begin{lstlisting}[language=clingo,firstnumber=23]
#minimize { 1,T : ngoal(T) }.
\end{lstlisting}

%%% Local Variables:
%%% mode: latex
%%% TeX-master: "paper"
%%% End:
