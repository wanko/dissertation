\subsection{Input language}
\label{sec:language}

This section introduces the theory-related features of \clingo's input language.
All of them are situated in the underlying grounder \gringo{} and can thus
also be used independently of \clingo.
%
We start with a detailed description of \gringo's generic means for defining theories
and complement this in Appendix~\ref{sec:aspif} with an overview of the corresponding intermediate language.

Our generic approach to theory specification rests upon two languages:
the one defining theory languages and the theory language itself.
Both borrow elements from the underlying ASP language,
foremost an aggregate-like syntax for formulating variable length expressions.
To illustrate this, consider Listing~\ref{prg:lc},
where a logic program is extended by constructs for handling difference and linear constraints.
While the former are binary constraints of the form%
\footnote{For simplicity, we consider normalized difference constraints rather than general ones of form $x_1-x_2\circ k$.}
$x_1-x_2\leq k$, % where $x,y$ are integer variables and $k$ is an integer,
the latter have a variable size and are of form
\(
a_1x_1+\dots+a_nx_n\circ k
\),
where $x_i$ are integer variables, $a_i$ and $k$ are integers, and $\circ\in\{\leq,\geq,<,>,=\}$
for $1\leq i\leq n$.
Note that solving difference constraints is polynomial, while solving linear equations (over integers) is NP-complete.
The theory language for expressing both types of constraints is defined in lines~\ref{prg:lc:begin-theory}--\ref{prg:lc:end-theory} and preceded by the directive \lstinline{#theory}.
The elements of the resulting theory language are preceded by \lstinline{&} and used as regular atoms in the logic program in lines~\ref{prg:lc:begin-program}--\ref{prg:lc:end-program}.
% ------------------------------------------------------------
\lstinputlisting[language=clingo,morekeywords={n,m},float=t,escapeinside={\%(}{\%)},basicstyle=\ttfamily\small,label=prg:lc,caption=Logic program enhanced with difference and linear constraints (\texttt{lc.lp})]{code/diff.lp}
% ------------------------------------------------------------

To be more precise,
a \emph{theory definition} has the form
\begin{lstlisting}[numbers=none,mathescape=t,language=clingo]
#theory $T$ {$D_1$;$\dots$;$D_n$}.
\end{lstlisting}
where $T$ is the theory name and each $D_i$ is a definition for a theory term or a theory atom for $1\leq i\leq n$.
The language induced by a theory definition is the set of all theory atoms constructible from its theory atom definitions.

A \emph{theory atom definition} has form
\begin{lstlisting}[numbers=none,mathescape=t]
&$p$/$k$ : $t$,$o$ $\quad\textrm{ or }\quad$ &$p$/$k$ : $t$,{$\diamond_1$,$\dots$,$\diamond_m$},$t'$,$o$
\end{lstlisting}
where $p$ is a predicate name and $k$ its arity,
$t,t'$ are names of theory term definitions,
each $\diamond_i$ is a theory operator for $m\geq 1$,
and
\(
o\in\{\code{head},\code{body},\code{any}, \code{directive}\}
\)
determines where theory atoms may occur in a rule.
%
Examples of theory atom definitions are given in lines~\ref{lst:ex1-begin-def-atom}--\ref{lst:ex1-end-def-atom} of Listing~\ref{prg:lc}.
%%
The language of a theory atom definition as above contains all \emph{theory atoms} of form
\begin{lstlisting}[numbers=none,mathescape=t]
&$a$ {${C}_1$:${L}_1$;$\dots$;${C}_n$:${L}_n$}$\quad\textrm{ or }\quad$ &$a$ {${C}_1$:${L}_1$;$\dots$;${C}_n$:${L}_n$} $\diamond$ $c$
\end{lstlisting}
where $a$ is an atom over predicate $p$ of arity $k$,
each ${C}_i$ is a tuple of theory terms in the language for~$t$,
$c$ is a theory term in the language for~$t'$,
$\diamond$ is a theory operator among $\{ \diamond_1, \dots, \diamond_m \}$,
and each ${L}_i$ is a regular condition (i.e., a tuple of regular literals)
for $1\leq i\leq n$.
% The last part `${}\diamond c$' is optional.
Whether the last part `${}\diamond c$' is included depends on the form of a theory atom definition.
%
Further, observe that theory atoms with occurrence type \code{any} can be used both in the head and body of a rule;
with occurrence types \code{head} and \code{body}, their usage can be restricted to rule heads and bodies only.
Occurrence type \code{directive} is similar to type \code{head} but additionally requires that the rule body must be completely evaluated during grounding.
Five occurrences of theory atoms can be found in lines~\ref{prg:lc:begin-use-theory}--\ref{prg:lc:end-use-theory} of Listing~\ref{prg:lc}.

A \emph{theory term definition} has form
\begin{lstlisting}[numbers=none,mathescape=t]
$t$ {$D_1$;$\dots$;$D_n$}
\end{lstlisting}
where $t$ is a name for the defined terms and each $D_i$ is a theory operator definition
for $1\leq i\leq n$.
A respective definition specifies the language of all theory terms
that can be constructed via % admissible term constructors and
its operators. % , as described next.
%
Examples of theory term definitions are given in lines~\ref{prg:lc:begin-def-term}--\ref{prg:lc:end-def-term} of Listing~\ref{prg:lc}.
%%
Each resulting \emph{theory term} is one of the following:
\par\medskip\noindent
\begin{minipage}{0.5\linewidth}
  \begin{itemize}
  \item a constant term: \ $c$
  \item a variable term: \ $v$
  \item a binary theory term: \  $t_1 \diamond t_2$
  \item a unary theory term: \  ${}\diamond t_1$
  \end{itemize}
\end{minipage}
\begin{minipage}{0.5\linewidth}
\begin{itemize}
  \item a function theory term: \  $f(t_1,\dots,t_k)$
  \item a tuple theory term: \  $(t_1,\dots,t_l,)$
  \item a set theory term: \  $\{t_1,\dots,t_l\}$
  \item a list theory term: \  $[t_1,\dots,t_l]$
\end{itemize}
\end{minipage}
\par\medskip\noindent
where each $t_i$ is a theory term,
$\diamond$ is a theory operator defined by some $D_i$,
$c$ and $f$ are symbolic constants,
$v$ is a first-order variable,
$k\geq 1$, and
$l\geq 0$.
(The trailing comma in tuple theory terms is optional if $l \neq 1$.)
% Furthermore,
Parentheses can be used to specify operator precedence.

A \emph{theory operator definition} has form
\begin{lstlisting}[numbers=none,mathescape=t]
$\diamond$ : $p$,unary $\quad\textrm{ or }\quad$ $\diamond$ : $p$,binary,$a$
\end{lstlisting}
where $\diamond$ is a unary or binary theory operator with precedence $p\geq 0$
(determining implicit parentheses). % , respectively.
Binary theory operators are additionally characterized by an associativity
$a\in\{\text{\lstinline{right}},\linebreak[1]\text{\lstinline{left}}\}$.
%
As an example,
consider lines~\ref{prg:lc:op-add}--\ref{prg:lc:op-mul} of Listing~\ref{prg:lc},
where the \lstinline{left} associative \lstinline{binary} operators~\code{+} and~\code{*} are defined with precedence~\lstinline{2} and~\lstinline{1}.
Hence, parentheses in terms like `\code{(X+(2*Y))+Z}' can be omitted.
%
In total, lines~\ref{prg:lc:begin-def-term}--\ref{prg:lc:end-def-term} of Listing~\ref{prg:lc}
include nine theory operator definitions.
%%
Specific \emph{theory operators} can be assembled (written consecutively without spaces) from the symbols
`\lstinline{!}',
`\lstinline{<}',
`\lstinline{=}',
`\lstinline{>}',
`\lstinline{+}',
`\lstinline{-}',
`\lstinline{*}',
`\lstinline{/}',
`\lstinline{\}',
`\lstinline{?}',
`\lstinline{&}',
%`\texttt{\&}',
`\lstinline{|}',
`\lstinline{.}',
`\lstinline{:}',
`\lstinline{;}',
`\lstinline{~}', and
`\lstinline{^}'.
%
For instance,
in line~\ref{prg:lc:op-dot} of Listing~\ref{prg:lc}, the operator `\lstinline{..}' is defined as the concatenation of two periods.
%
% Note that
The tokens
`\lstinline{.}',
`\lstinline{:}',
`\lstinline{;}', and
`\lstinline{:-}'
must be combined with other symbols
due to their dedicated usage.
Instead, one may write
`\lstinline{..}',
`\lstinline{::}',
`\lstinline{;;}',
`\lstinline{::-}',
etc.
% Each operator is either unary or binary.
% To resolve ambiguities,
% a binary theory operator is either left or right associative.
% Furthermore, each operator has a precedence to allow for omitting parentheses in some situations.

While theory terms are formed similar to regular ones,
theory atoms rely upon an aggregate-like construction for forming variable-length theory expressions.
In this way, standard grounding techniques can be used for gathering theory terms.
(However, the actual atom \lstinline{&a} within a theory atom comprises regular terms only.)
%
The treatment of theory terms still % and atoms % is different
differs from their regular counterparts
% This is due to the fact
in that the grounder skips simplifications like, e.g., arithmetic evaluation.
% completely ignores the theory's semantics and thus performs no theory-specific simplifications.
This can be nicely seen on the different results in Listing~\ref{prg:grd:diff} of grounding terms formed with
the regular and theory-specific variants of operator `\lstinline{..}'.
Observe that
the fact \lstinline{task(1..n)} in line~\ref{prg:lc:rule-task} of Listing~\ref{prg:lc} results in \lstinline{n} ground facts,
viz.\ \lstinline{task(1)} and \lstinline{task(2)} because of \lstinline{n=2}.
Unlike this, the theory expression \lstinline{1..m} stays structurally intact and is only transformed into \lstinline{1..1000} in view of \lstinline{m=1000}.
That is, the grounder does not evaluate the theory term \lstinline{1..1000} and
leaves its interpretation to a downstream theory solver.
%
% ------------------------------------------------------------
\lstinputlisting[language=clingo,float=t,basicstyle=\ttfamily\small,label=prg:grd:diff,caption={Human-readable result of grounding Listing~\ref{prg:lc} via `\texttt{gringo --text lc.lp}'}]{code/diff.grd}
% ------------------------------------------------------------
%
A similar situation is encountered when comparing the treatment of the regular term `\lstinline{200*T}' in line~\ref{prg:lc:rule-duration} of Listing~\ref{prg:lc} to the
theory term `\lstinline{end(T)-start(T)}' in line~\ref{prg:lc:rule-diff}.
While each instance of `\lstinline{200*T}' is evaluated during grounding,
instances of the theory term `\lstinline{end(T)-start(T)}' are left intact in lines~11 and~12 of Listing~\ref{prg:grd:diff}.
In fact, if `\lstinline{200*T}' had been a theory term as well,
it would have resulted in the unevaluated instances  `\lstinline{200*1}' and `\lstinline{200*2}'.
% This is because the grounder ignores the meaning of the theory operator `\lstinline{*}'.

%%% Local Variables:
%%% mode: latex
%%% TeX-master: "paper"
%%% End:
