
% * Title of the tutorial
% * Subarea within AI/Keywords: Please give a few keywords of different granularity.
%   For example a tutorial on AlphaGo should write: Game Playing, Machine Learning, Deep-Learning, AlphaGo.
% * Name, affiliation, and email addresses of presenters
% * Suggested duration: Half day (3 hours, 30 minutes, plus 30-minute break) or quarter day (1 hour, 45 minutes).
% * Goal of the tutorial (1-2 paragraphs): Who is the target audience? What will the audience walk away with? What makes the topic innovative?
% * Tutorial outline (1 paragraph): A short abstract of the tutorial content.
% * History: List of previous venues and approximate audience sizes, if the same or a similar tutorial has been given elsewhere.
% * Estimated number of participants: Please estimate the audience size (large audience is not always preferable).
% * Prerequisite knowledge (1 paragraph): What knowledge is assumed of the target audience.
% * Content (as many paragraphs as needed): Detailed outline of the tutorial.
% * (Optional) Supplementary materials: List of supplemental materials augmented with samples, such as past tutorial slides and survey articles,
%   whenever possible. Be as complete as possible.
% * CVs of presenters: A short CV (1-2 pages) for each presenter that includes their name, affiliation, current position (e.g., Ph.D. student,
%   Postdoc, Professor, Researcher, etc.), mailing address, phone number, and email address. The CVs should also describe their background in the
%   tutorial area, including a list of relevant publications and/or presentations; any available examples of work in the area (ideally, a published
%   tutorial-level article or presentation materials on the subject); evidence of teaching experience (courses taught or references); and evidence of
%   scholarship in AI or computer science.

\documentclass{article}
\usepackage[english]{babel}
\usepackage[utf8]{inputenc}
\usepackage[colorlinks=false,pagebackref=true]{hyperref}
\usepackage{url}
\newcommand{\gringo}{\textit{gringo}}
\newcommand{\clasp}{\textit{clasp}}
\newcommand{\clingo}{\textit{clingo}}
\newcommand{\asprin}{\textit{asprin}}
\newcommand{\asap}{\textit{teaspoon}}
\newcommand{\piclasp}{\textit{piclasp}}

\newcommand{\code}[1]{\lstinline[basicstyle=\ttfamily]{#1}}

\newcommand{\lw}[1]{\smash{\lower1.ex\hbox{#1}}}
\newcommand{\llw}[1]{\smash{\lower3.ex\hbox{#1}}}

%\newcommand{\dataCL}[5]{%
%  \code{#1} & #3 & #5 & #4
%}
%\newcommand{\dataCS}[5]{%
%  #3 & #5 & #4
%}

\newenvironment{tableC}{%
  \scriptsize
  \tabcolsep = 0.6mm
  \begin{tabular}[t]{l|rlr|rlr|rlr|rlr|rlr}\hline
    \multicolumn{1}{l|}{\llw{Instance}} &
    \multicolumn{3}{c|}{UD1} &
    \multicolumn{3}{c|}{UD2} &
    \multicolumn{3}{c|}{UD3} &
    \multicolumn{3}{c|}{UD4} &
    \multicolumn{3}{c}{UD5} \\
    & 
    \multicolumn{1}{c}{Best} & & \multicolumn{1}{c|}{\emph{tea-}} & 
    \multicolumn{1}{c}{Best} & & \multicolumn{1}{c|}{\emph{tea-}} & 
    \multicolumn{1}{c}{Best} & & \multicolumn{1}{c|}{\emph{tea-}} & 
    \multicolumn{1}{c}{Best} & & \multicolumn{1}{c|}{\emph{tea-}} & 
    \multicolumn{1}{c}{Best} & & \multicolumn{1}{c}{\emph{tea-}} \\
    & 
    known & & \emph{spoon} & 
    known & & \emph{spoon} & 
    known & & \emph{spoon} & 
    known & & \emph{spoon} & 
    known & & \emph{spoon} \\
    \hline
  }{%
    \hline
  \end{tabular}
}

\newenvironment{tableB}{%
  \scriptsize
  \tabcolsep = 0.7mm
%  \begin{tabular}[t]{|l|c|r|l|l|l|}\hline
  \begin{tabular}[t]{lcrlll}\hline
    Instance &
    Formulation &
    Time (sec.)\\
    \hline
  }{%
    \hline
  \end{tabular}
}
\newenvironment{tableL}{%
  \scriptsize
  \tabcolsep = 0.7mm
  \begin{tabular}[t]{l|rrrrrrrr|r}\hline
    \lw{Instance} &
    \lw{Time (sec.)} &
    \multicolumn{6}{c}{The best utility vector} &
    The sum of  &
    The best of basic\\
    &
    &
    $(S_1,$ & $S_4,$ & $S_2,$ & $S_7,$ & $S_6,$ & $S_3)$ &
    utility vector &
    and optimized \\
    \hline
  }{%
    \hline
  \end{tabular}
}

%%% Local Variables:
%%% mode: latex
%%% TeX-master: "paper"
%%% End:

\begin{document}

\section*{Answer Set Engineering}

Javier Romero$^2$ \
Roland Kaminski$^2$ \
Torsten Schaub$^{1,2}$ \
Philipp Wanko$^2$

\medskip
\noindent
$^1$Inria Rennes, France \
$^2$University of Potsdam, Germany

\medskip
\noindent
\texttt{\{javier,kaminski,torsten,wanko\}@cs.uni-potsdam.de}

\subsection*{Subarea within AI/Keywords}

Knowledge representation and reasoning,
Answer Set Programming,
Combinatorial optimization problems

\subsection*{Suggested duration}

Half day (3 hours, 30 minutes, plus 30-minute break)

\subsection*{Goal of the tutorial}

Answer Set Programming (ASP) has become an established paradigm for Knowledge Representation and Reasoning,
in particular, when it comes to solving knowledge-intense combinatorial (optimization) problems.
ASP's unique pairing of a simple yet rich modeling language with highly performant solving technology
has led to an increasing interest in ASP systems in academia as well as industry.
To further boost this development and make ASP fit for real world applications it is indispensable to possess means
for an easy integration into software environments and for adding complementary forms of reasoning.
These means are furnished by \emph{multi-shot} and \emph{theory reasoning} in modern ASP systems.
The participant will learn
(i) how to use multi-shot solving
to integrate ASP systems into software environments and
to deal with solver objects to model complex reasoning modes (see below)
and
(ii) how to use theory reasoning
to integrate dedicated reasoners into ASP systems and
to extend the modeling language to express corresponding constraints,
or in a nutshell, how to exploit modern ASP technology for solving complex reasoning problems.

\subsection*{Tutorial outline}

In this tutorial, we describe how both issues are addressed in the ASP system CLINGO.
After a gentle introduction to ASP, we outline features of CLINGO's application programming interface (API) that are essential for multi-shot ASP solving,
a technique for dealing with continuously changing logic programs.
This is illustrated by realizing two exemplary reasoning modes, namely branch-and-bound-based optimization and incremental ASP solving.
As a final example we show how to use multi-shot solving for realizing the round-based board game of ricochet robots.
We then switch to the design of the API for integrating complementary forms of reasoning and detail this in an extensive case study
dealing with the integration of difference constraints.
We show how the syntax of these constraints is added to the modeling language and seamlessly merged into the grounding process.
We then develop in detail a corresponding theory propagator for difference constraints and present how it is integrated into CLINGO's solving process.
Finally, we discuss aspects of the quite sophisticated use of both techniques in the implementation of the ASPRIN system,
providing a generic framework for qualitative and quantitative optimization with CLINGO.

\subsection*{History}

Over the years, Torsten Schaub gave the following tutorials with various members of his research group.

\begin{itemize}

\item Tutorial (4h)
  \emph{Answer Set Programming}.
  \begin{itemize}
  \item Institute of Computer Science and Applied Mathematics,
    \UNIof{ Bern},
    \SWITZERLAND,
    \FEB\ 2008.
  \item Third International Compulog/ALP Summer School on Logic Programming and Computational Logic.
    New Mexico State University,
    Las Cruces, \USA,
    \JUL\ 2008.
  \item Annual Logic Summer School,
    Australian National University,
    Canberra, \AUSTRALIA.
    \DEC\ 2009.
  \item First Workshop of the DFG Research Unit on Hybrid Reasoning for Intelligent Systems,
    RWTH Aachen University, Aachen\xGERMANY.
    \NOV\ 2012.
  \item School of Science and Technology,
    Orebro University, \SWEDEN.
    \NOV\ 2012.
  \end{itemize}

\item Tutorial (8h)
  \emph{Answer Set Solving in Practice}.
  \begin{itemize}
  \item Twenty-second International Joint Conference on Artificial Intelligence (IJCAI'11),
    Barcelona, \SPAIN,
    \JUL\ 2011.
  \item SIEMENS AG,
    \VIENNA, \AUSTRIA,
    \SEP\ 2011.
  \end{itemize}

\item Tutorial (3h)
  \emph{Modeling and Solving in Answer Set Programming}.
  Third International SAT/SMT Summer School,
  Helsinki, \FINLAND.
  \JUL\ 2013.

\item Tutorial (4h)
  \emph{Answer Set Solving in Practice}.
  \begin{itemize}
  \item Twenty-seventh Conference on Artificial Intelligence (AAAI'13),
    Bellevue, Washington, \USA,
    \JUL\ 2013.
  \item Twenty-third International Joint Conference on Artificial Intelligence (IJCAI'13),
    Beijing, \CHINA,
    \AUG\ 2013.
  \end{itemize}

\item Tutorial (6h)
  \emph{Answer Set Solving in Practice}.
  \begin{itemize}
  \item Summer School on Verification Technology, Systems, and Applications (VTSA'13),
    Nancy, \FRANCE,
    \SEP\ 2013.
  \item Second NICTA International Summer School on Optimisation,
    Kioloa, \AUSTRALIA,
    \JAN\ 2014.
  \item ICCL Summer School,
    Dresden\xGERMANY,
    \SEP\ 2015.
  \item Advanced Courses in Artificial Intelligence (ACAI’15; Summer School of EurAI),
    Lille, \FRANCE,
    \OCT\ 2015.
  \end{itemize}

\item Tutorial (4h)
  \emph{Answer set programming for systems biology}.
  Workshop on Integrative-Omics.
  Pucón, Chile,
  \DEC\ 2013.

\item Tutorial (8h)
  \emph{Answer set programming  and its application to robotics}.
  Winter PhD School on AI and Robotics.
  Orebro, \SWEDEN,
  \DEC\ 2014.

\item Tutorial (8h)
  \emph{Answer Set Solving in Practice: Advanced techniques}.
  Twenty-fourth International Joint Conference on Artificial Intelligence (IJCAI'15),
  Buenos Aires, \ARGENTINA,
  \JUL\ 2015.

\item Tutorial (3h)
  \emph{Towards embedded Answer Set Solving}.
  Eleventh Reasoning Web Summer School (RW'15),
  Berlin\xGERMANY,
  \AUG\ 2015.

\item Tutorial (1h)
  \emph{Towards embedded Answer Set Solving}.
  Twenty-first International Conference on Principles and Practice of Constraint Programming (CP'15),
  Cork, \IRELAND,
  \SEP\ 2015.

\item Tutorial (3h)
  \emph{ASP foundations and applications}.
  Autumn School on
  Logic programming for verification and constraint solving,
  New York, \USA,
  \OCT\ 2016.

\item Tutorial (3h)
  \emph{Hybrid Answer Set Solving with Clingo}.
  Thirteenth Reasoning Web Summer School (RW'17),
  London, \UK,
  \JUL\ 2017.
\end{itemize}

\subsection*{Estimated number of participants} 20-30 participants

\subsection*{Prerequisite knowledge} Basic notions of ASP and Python are advantageous but not imperative

\subsection*{Content}

\begin{enumerate}
\item Introduction to Answer Set Programming
  \begin{enumerate}
  \item Motivation
  \item Modeling
  \item Grounding
  \item Solving
  \end{enumerate}
\item Multi-shot ASP solving
  \begin{enumerate}
  \item Language
  \item API
  \item State changes
  \item Examples
  \item Case study
  \end{enumerate}
\item Theory reasoning in ASP
  \begin{enumerate}
  \item Theory Grammar
  \item Propagator interface
  \item Case study
  \end{enumerate}
\item Case study: Qualitative and quantitative optimization
\end{enumerate}

\subsection*{Supplementary materials}

In order of relevance:

\begin{enumerate}
\item%{kascwa17a}
  R.~Kaminski, T.~Schaub, P.~Wanko.
  A Tutorial on Hybrid Answer Set Solving with clingo.
  In {\em Proceedings of the Thirteenth International Summer School of the Reasoning Web (RW'17)},
  167-203, Springer, 2017.

\item%{gekakasc12a}
  M.~Gebser, R.~Kaminski, B.~Kaufmann, T.~Schaub.
  \emph{Multi-shot {ASP} solving with clingo}.
  \emph{Theory and Practice of Logic Programming}.
  \APPEARS{ 2018}.

\item % {gekakaosscwa16a}
  M.~Gebser, R.~Kaminski, B.~Kaufmann, et~al.
  Theory solving made easy with clingo~5.
  In {\em Technical Communications of the Thirty-second International Conference on Logic Programming (ICLP'16)},
  52(2):1--2:15, OASIcs, 2016.

\item%{brderosc15a}
  G.~Brewka, J.~Delgrande, J.~Romero, and T.~Schaub.
  asprin: Customizing answer set preferences without a headache.
  In {\em Proceedings of the Twenty-Ninth National Conference on Artificial Intelligence (AAAI'15)}.
  AAAI Press, 2015.

\item%{gekakasc12a}
  M.~Gebser, R.~Kaminski, B.~Kaufmann, T.~Schaub.
  \emph{Answer Set Solving in Practice}.
  Morgan and Claypool Publishers, 2012, 238~pages.

\end{enumerate}

\newpage
\subsection*{CVs of presenters}

Note that proposers R.~Kaminski, T.~Schaub, and P.~Wanko were conducting the predecessor of this tutorial at RW'17
(cf.\ Publication 1.\ above).
J.~Romero and T.~Schaub run since several years a 60h MSc course on ASP at the University of Potsdam.

\paragraph{Roland Kaminski}
%
is a research engineer at the University of Potsdam, Germany.
%
His scientific interests are focused on the implementation of systems for Answer Set Programming (ASP).
%
He is the main developer of the widely used ASP system clingo.
Recently, his focus is the improvement of CLINGO's application programming interface,
providing the basis for systems extending core ASP solving, like for example, ASPRIN and CLINGO[DL].
%
He has co-authored more than 40 publications, among them a text book on Answer Set Solving.

\paragraph{Javier Romero}
%
is a researcher at the University of Potsdam, Germany.
He received a MSc.\ in Computer Science from the University of Corunna in 2006, and in Philosophy from the University of Santiago de Compostela in 2008, both in Spain.
His research interests are knowledge representation and logic programming, and his research focuses in extensions of ASP for declarative heuristics and
preference reasoning. He is a member of the open source project potassco.org developed at Potsdam, and the main developer of ASPRIN.
%
He has co-authored several international conference publications in top venues (\mbox{IJCAI}, AAAI, KR\ldots),
he has co-supervised multiple M.Sc.\ theses, and has taught university courses for several years.

\paragraph{Torsten Schaub}
%
received his diploma and dissertation in informatics in 1990 and 1992, respectively, from the Technical University of Darmstadt, Germany, and his
habilitation in informatics in 1995 from the University of Rennes I, France. From 1990 to 1993 he was a research assistant at the Technical University
at Darmstadt. From 1993 to 1995, he was a research associate at IRISA/INRIA at Rennes. In 1995 he became University Professor at the University of
Angers. Since 1997, he is University Professor for knowledge processing and information systems at the University of Potsdam. In 1999, he became
Adjunct Professor at the School of Computing Science at Simon Fraser University, Canada; and since 2006 he is also an Adjunct Professor in the
Institute for Integrated and Intelligent Systems at Griffith University, Australia. Since 2014, Torsten Schaub holds an Inria International Chair at
Inria Rennes. Torsten Schaub has become a fellow of the European Association for Artificial Intelligence EurAI in 2012. In 2014 he was elected
President of the Association of Logic Programming. He served as program (co-)chair of LPNMR'09, ICLP'10, and ECAI'14. The research interests of
Torsten Schaub range from the theoretic foundations to the practical implementation of reasoning from incomplete, inconsistent, and evolving
information. His current research focus lies on ASP and materializes at potassco.org, the home of the open source project Potassco
bundling software for ASP developed at the University of Potsdam.

\paragraph{Philipp Wanko}
%
is a doctoral student and researcher since 2015 at the University of Potsdam, Germany,
where he received a  MSc.\ in Computer Science the same year.
His interests are focused on hybrid reasoning and multi-objective optimization using ASP.
Both fields are of particular interest to his main field of application Design Space Exploration,
where implementations of embedded systems featuring a complex network-like hardware structure are automatically synthesized.
To tackle ASP's shortcomings to effectively encode timing constraints featured in this application,
he developed the system CLINGO[DL] that combines ASP with difference constraints.
Furthermore, he extended CLINGO[DL] with the capacity for multi-objective optimization,
in particular enumerating Pareto-optimal (non-dominated) solutions.
The results were published in the renowned system synthesis conference DATE'17 and DATE'18,
as well as, presented to the logic programming community in LPMNR'17 and ICLP'16.

\end{document}

%%% Local Variables:
%%% mode: latex
%%% TeX-master: t
%%% End:
