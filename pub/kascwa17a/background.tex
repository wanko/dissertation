
\section{Answer Set Programming}\label{sec:background}

As usual, a {logic program} consists of {rules} of the form
\begin{lstlisting}[mathescape,numbers=none]
   a$_1$;...;a$_m$ :- a$_{m+1}$,...,a$_n$,not a$_{n+1}$,...,not a$_o$
\end{lstlisting}
where each \lstinline[mathescape]{a$_i$} is
an {atom} of form \lstinline[mathescape]{p(t$_1$,...,t$_k$)}
and all \lstinline[mathescape]{t$_i$} are terms,
composed of function symbols and variables.
%
Atoms \lstinline[mathescape]{a$_1$} to \lstinline[mathescape]{a$_m$} are often called head atoms,
while \lstinline[mathescape]{a$_{m+1}$} to \lstinline[mathescape]{a$_n$}
and \lstinline[mathescape]{not a$_{n+1}$} to \lstinline[mathescape]{not a$_o$}
are also referred to as positive and negative body literals, respectively.
%
An expression is said to be {ground}, if it contains no variables.
%
As usual, \lstinline[mathescape]{not} denotes (default) {negation}.
A rule is called a {fact} if $m=o=1$, 
normal if $m=1$, and 
an integrity constraint if $m=0$.
%
Semantically, a logic program induces a set of {stable models},
being distinguished models of the program determined by the stable models semantics;
see \cite{gellif91a} for details.

To ease the use of ASP in practice, 
several extensions have been developed. 
First of all, rules with variables are viewed as shorthands for the set of their ground instances.
Further language constructs include
{conditional literals} and {cardinality constraints} \cite{siniso02a}.
The former are of the form
\lstinline[mathescape]{a:b$_1$,...,b$_m$},
the latter can be written as\footnote{More elaborate forms of aggregates can be obtained by explicitly using function (eg.\ \texttt{\#count}) and
  relation symbols (eg.\ \texttt{<=}).}
\lstinline[mathescape]+s{d$_1$;...;d$_n$}t+,
where \lstinline{a} and \lstinline[mathescape]{b$_i$} are possibly default-negated (regular) literals  % for $0\leq i\leq m$,
and each \lstinline[mathescape]{d$_j$} is a conditional literal; % for $1\leq i\leq n$;
\lstinline{s} and \lstinline{t} provide optional lower and upper bounds on the number of satisfied literals in the cardinality constraint.
We refer to \lstinline[mathescape]{b$_1$,...,b$_m$} as a {condition}.
%
The practical value of both constructs becomes apparent when used with variables.
For instance, a conditional literal like
\lstinline[mathescape]{a(X):b(X)}
in a rule's antecedent expands to the conjunction of all instances of \lstinline{a(X)} for which the corresponding instance of \lstinline{b(X)} holds.
%
Similarly,
\lstinline[mathescape]+2{a(X):b(X)}4+
is true whenever at least two and at most four instances of \lstinline{a(X)} (subject to \lstinline{b(X)}) are true.
%
Finally, objective functions minimizing the sum of a set of weighted tuples $(w_i,\boldsymbol{t}_i)$ subject to condition $c_i$ are expressed as
\lstinline[mathescape]!#minimize{$w_1$@$l_1$,$\boldsymbol{t}_1$:$c_1$;$\dots$;$w_n$@$l_n$,$\boldsymbol{t}_n$:$c_n$}!.
Lexicographically ordered objective functions are (optionally) distinguished via levels indicated by $l_i$.
An omitted level defaults to 0.

As an example, consider the rule in Line~9 of Listing~\ref{fig:toh:opt:enc}:
% --------------------------------------------------------------------------------------------------------------------------------------------
% \lstinputlisting[numbers=none,firstline=9,lastline=9,literate={\%\%}{}{0},escapeinside={\#(}{\#)},language=clingo]{example/opt/tohB.lp}%
\begin{lstlisting}[numbers=none,basicstyle=\ttfamily\small]
1 { move(D,P,T) : disk(D), peg(P) } 1 :- ngoal(T-1), T<=n.
\end{lstlisting}
% --------------------------------------------------------------------------------------------------------------------------------------------
This rule has a single head atom consisting of a cardinality constraint;
it comprises all instances of \lstinline{move(D,P,T)} where \lstinline{T} is fixed by the two body literals and \lstinline{D} and \lstinline{P} vary
over all instantiations of predicates \lstinline{disk} and \lstinline{peg}, respectively.
Given 3 pegs and 4 disks as in Listing~\ref{fig:toh:opt:ins}, this results in 12 instances of \lstinline{move(D,P,T)} for each valid replacement of
\lstinline{T}, among which exactly one must be chosen according to the above rule.

Full details on the input language of \clingo{} along with various examples can be found in~\cite{PotasscoUserGuide}.

%%% Local Variables: 
%%% mode: latex
%%% TeX-master: "paper"
%%% End: 
