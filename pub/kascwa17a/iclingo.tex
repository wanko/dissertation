
\subsection{Incremental ASP solving}
\label{sec:iclingo}

As mentioned, \clingo{} fully supersedes its special-purpose predecessor \iclingo{} aiming at incremental ASP solving.
To illustrate this,
we give below in Listing~\ref{fig:iclingo:python} a \python\ implementation of \iclingo's control loop,
corresponding to the one shipped with \clingo.\footnote{Alternatively, this can be invoked by `\lstinline{#include <incmode>.}'.}%
${}^,$\footnote{The \python\ as well as a \lua\ implementation can be found in \texttt{examples/clingo/iclingo} in the \clingo{} distribution.}
%
Roughly speaking,
\iclingo\ offers a step-oriented, incremental approach to ASP that avoids redundancies by gradually processing the extensions to a problem
rather than repeatedly re-processing the entire extended problem (as in iterative deepening search).
%
To this end, a program is partitioned into a
base part, describing static knowledge independent of the step parameter~\lstinline{t},
a cumulative part, capturing knowledge accumulating with increasing~\lstinline{t},
and
a volatile part specific for each value of~\lstinline{t}.
%
% These parts are delineated in \iclingo\ by the special-purpose directives \lstinline{#base}, `\lstinline{#cumulative t}', and `\lstinline{#volatile t}'.
%
In \clingo, all three parts are captured by \lstinline{#program} declarations
along with \lstinline{#external} atoms for handling volatile rules.
More precisely,
the implementation in Listing~\ref{fig:iclingo:python} relies upon subprograms named \lstinline{base}, \lstinline{step}, and \lstinline{check}
along with external atoms of form \lstinline{query(t)}.%
\footnote{These names have no general, predefined meaning; their meaning emerges from their usage in the associated script (see below).}

We illustrate this approach by adapting the Towers of Hanoi encoding from Listing\ref{fig:toh:opt:enc} in Section~\ref{sec:optimization}
to an incremental version in Listing~\ref{fig:toh:enc}.
% --------------------------------------------------------------------------------------------------------------------------------------------
\lstinputlisting[float=tb,literate={\%\%}{}{0},escapeinside={\#(}{\#)},language=clingo,caption={Towers of hanoi incremental encoding (\lstinline{tohE.lp})},label={fig:toh:enc},linerange={3-24}]{example/opt/tohE.lp}
% --------------------------------------------------------------------------------------------------------------------------------------------
To this end,
we arrange the original encoding in program parts \lstinline{base}, \lstinline{check(t)}, and \lstinline{step(t)}, 
use \lstinline{t} instead of \lstinline{T} as time parameter, and 
simplify checking the goal.
%
Checking the goal is easier here because the iterative deepening approach guarantees a shortest plan and, hence, does not require additional minimization.

At first, we observe that
the problem instance in Listing~\ref{fig:toh:opt:ins} as well as line~\ref{fig:toh:enc:static} in Listing~\ref{fig:toh:enc} 
constitute static knowledge and thus belong to the \lstinline{base} program.
%
More interestingly, the query is expressed in line~\ref{fig:toh:enc:goal} of Listing~\ref{fig:toh:enc}.
Its volatility is realized by making it subject to the truth assignment to the external atom \lstinline{query(t)}.
For convenience,
this atom is predefined in line~\ref{fig:iclingo:python:query} in Listing~\ref{fig:iclingo:python} as part of the \lstinline{check} program (cf.~line~\ref{fig:iclingo:python:check}).
Hence, subprogram \lstinline{check} consists of a user- and predefined part.
%
Finally,
the transition function along with state constraints are described in the subprogram \lstinline{step} in lines~\ref{fig:toh:enc:step:begin}--\ref{fig:toh:enc:step:end}.

The idea is now to control the successive grounding and solving of the program parts in Listing~\ref{fig:toh:opt:ins} and~\ref{fig:toh:enc}
by the \python\ script in Listing~\ref{fig:iclingo:python}.
%
% --------------------------------------------------------------------------------------------------------------------------------------------
\lstinputlisting[float=tbp,literate={\%\%}{}{0},escapeinside={\#(}{\#)},language=clingo,basicstyle=\ttfamily\small,caption={\python\ script implementing \iclingo\ functionality in \clingo\ (\lstinline{inc.lp})},label=fig:iclingo:python]{example/opt/inc.lp}
% --------------------------------------------------------------------------------------------------------------------------------------------
%
Lines~\ref{fig:iclingo:python:const:begin}--\ref{fig:iclingo:python:const:end} fix the values of the constants \lstinline{imin}, \lstinline{imax}, and \lstinline{istop}.
In fact, the setting in line~\ref{fig:iclingo:python:imin} and~\ref{fig:iclingo:python:istop} relieves us from adding
`\lstinline{-c imin=0 -c istop="SAT"}'
when calling \clingo.
All three constants mimic command line options in \iclingo.
\lstinline{imin} and \lstinline{imax} prescribe a least and largest number of iterations, respectively;
\lstinline{istop} gives a termination criterion.
The initial values of variables \lstinline{step} and \lstinline{ret} are set in line~\ref{fig:iclingo:python:vars}.
The value of \lstinline{step} is used to instantiate the parametrized subprograms
and \lstinline{ret} comprises the solving result.
Together, the previous five variables control the loop in lines~\ref{fig:iclingo:python:loop:begin}--\ref{fig:iclingo:python:loop:end}.

The subprograms grounded at each iteration are accumulated in the list \lstinline{parts}.
Each of its entries is a pair consisting of a subprogram name along with its list of actual parameters.
In the very first iteration, the subprograms \lstinline{base} and \lstinline{check(0)} are grounded.
Note that this involves the declaration of the external atom \lstinline{query(0)} and the assignment of its default value false.
The latter is changed in line~\ref{fig:iclingo:python:assign:query} to true in order to activate the actual query.
The \lstinline{solve} call in line~\ref{fig:iclingo:python:solve} then amounts to checking whether the goal situation is already satisfied in the initial state.
As well, the value of \lstinline{step} is incremented to \lstinline{1}.

As long as the termination condition remains unfulfilled,
each following iteration takes the respective value of variable \lstinline{step}
to replace the parameter in subprograms \lstinline{step} and \lstinline{check}
during grounding.
In addition,
the current external atom \lstinline{query(t)} is set to true,
while the previous one is permanently set to false.
This disables the corresponding instance of the integrity constraint in line~\ref{fig:toh:enc:goal} of Listing~\ref{fig:toh:enc} before it is replaced in the next iteration.
In this way,
the query condition only applies to the current horizon.

An interesting feature is given in line~\ref{fig:iclingo:python:cleanup}.
As its name suggests, this function cleans up domains used during grounding.
That is, whenever the truth value of an atom is ultimately determined by the solver,
it is communicated to the grounder where it can be used for simplifications.

The result of each call to \lstinline{solve} is printed by \clingo.
In our example, the solver is called 16 times before a plan of length 15 is found:
%
% --------------------------------------------------------------------------------------------------------------------------------------------
\lstinputlisting[numbers=none,mathescape=true,numbers=none,basicstyle=\ttfamily\small]{example/opt/inc.txt}
% --------------------------------------------------------------------------------------------------------------------------------------------

%%% Local Variables:
%%% mode: latex
%%% TeX-master: "paper"
%%% End:
