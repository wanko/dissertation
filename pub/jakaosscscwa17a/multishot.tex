\section{Multi-Shot ASP Solving with Linear Constraints}
\label{sec:multishot}
Multi-shot solving~\cite{gekakasc14b} is about solving continuously changing logic programs in an operative way.
This can be controlled via reactive procedures that loop on solving while reacting, for instance, to outside changes or previous solving results.
These reactions may entail the addition or retraction of rules that the operative approach can accommodate by leaving the unaffected program parts
intact within the solver.
This avoids re-grounding and benefits from heuristic scores and nogoods learned over time.
%
In fact, 
evolving logic programs with linear constraints can be extremely useful in dynamic applications, for example, to %:
add new resources in a planning domain,
or to set the value of an observed variable measured using sensors.
The abstraction from actual constraints to constraint atoms
allows us to easily extend multi-shot solving to lc-programs.

To illustrate how seamlessly our systems \clingod{dl} and \clingod{lp} support multi-shot solving,
we apply the exemplary \python\ script, shipped with \clingo\ to illustrate incremental solving,
to model the spoiling Yale shooting scenario~\cite{caotpo00a}.
%
Multi-shot solving in \clingo\ relies on two directives (cf.~\cite{gekakasc14b}),
the \texttt{\#program} directive for regrouping rules
and
the \texttt{\#external} directive for declaring atoms
as being external to the program at hand.
The truth value of such external atoms can be set via \clingo's API\@.
The aforementioned \python\ script loops over increasing integers until a stop criterion is met.
It presupposes three groups of rules declared via \texttt{\#program} directives.
At step 0 the programs named \texttt{base} and \texttt{check(n)} are grounded and solved for $\texttt{n}=0$.
Then, in turn programs \texttt{check(n)} and \texttt{step(n)} are added for $\texttt{n}>0$, grounded, and the resulting overall program solved.
% Other names and components are definable by appropriate changes to the script.
% Stop criteria can be the satisfiability or unsatisfiability of the respective program at each iteration.
In addition, at each step $\texttt{n}$ an external atom \texttt{query(n)} is introduced;
it is set to true for the current iteration $\texttt{n}$ and false for all previous instances with smaller integers than $\texttt{n}$.
We refer the reader to~\cite{gekakasc14b} for further details on the \python{} part.
Notably, for dealing with lc-programs,
we can use the exemplary \python\ script as is---once the respective propagator is registered with the solver.

In the spoiled Yale shooting scenario~\cite{caotpo00a},
we have a gun and two actions, viz.\ load and shoot.
If we load, the gun becomes loaded.
If we shoot, it kills the turkey, if the gun was loaded for no more than 35 minutes.
Otherwise, the gun powder is spoiled.
We model this planning problem in \ASPm{lc}.
% --------------------------------------------------------------------------------
\lstinputlisting[caption={Spoiled Yale shooting instance},float=ht,label=encoding:yalebase,basicstyle=\ttfamily\footnotesize]{encodings/base.lps}
% --------------------------------------------------------------------------------
We start by including the incremental \python\ program,
the grammar, and the propagator for linear constraints in the first line of Listing~\ref{encoding:yalebase}.%
\footnote{For uniformity, we use semi-colons '\texttt{;}' rather than '\texttt{,}' for separating body elements.}
This listing is the base program.
All actions and their durations are introduced in Lines~4 and~5.
At the initial situation, the gun is unloaded (Line~6).
Line~7 and~8 initialize integer variables \texttt{at(0)} and \texttt{armed(0)} with 0 (see below).
% --------------------------------------------------------------------------------
\lstinputlisting[caption={Spoiled Yale shooting scenario},float=ht,label=encoding:yale,basicstyle=\ttfamily\footnotesize]{encodings/yale.lps}
% --------------------------------------------------------------------------------
Listing~\ref{encoding:yale} gives the dynamic part of the problem;
it is grounded for each step \texttt{n}.
Line~2 enforces that exactly one action is done per step.
The exact times at which each step takes place is captured by the integer variables \texttt{at(n)}.
The difference between two consecutive time steps is the duration
of the respective action (Line~3).
%
The next three lines make the fluents inertial, viz.\
the gun stays loaded/unloaded if it was loaded/unloaded before,
and the turkey remains dead.
%
Lines~9 and~10 use the integer variable \texttt{armed(n)}
to describe for how long the weapon has been loaded at step \texttt{n}.
Whenever it is unloaded, \texttt{armed(n)} is 0,
otherwise it is increased by the duration of the last action.
%
The following four lines (12--15) encode the conditions and effects of the actions.
When we load the gun, it becomes loaded; when we shoot, it becomes unloaded.
If we shoot and the gun was loaded for no longer than 35 minutes (and thus the gun powder is unspoiled),
the turkey is dead.
The last line ensures that we cannot shoot if the gun is not loaded.
Together with the initial situation and the actions from Listing~\ref{encoding:yalebase}
this encodes the spoiled Yale shooting problem,
and any solution represents an executable plan.
% --------------------------------------------------------------------------------
\lstinputlisting[caption={Query for the spoiled Yale Shooting Scenario.},float=ht,label=encoding:yalequery,basicstyle=\ttfamily\footnotesize]{encodings/query.lps}
% --------------------------------------------------------------------------------
Listing~\ref{encoding:yalequery} adds a query to our problem.
In Line~2 we require that the turkey is dead at step \texttt{n}.
As this constraint is subject to the external atom \texttt{query(n)},
it is only active at solving step \texttt{n}.
The next line ensures that we kill the turkey within 100 minutes.
And as an additional constraint,
we added some preparation time such that we are not allowed 
to shoot in the first 35 minutes.
%
It is possible to solve this problem within three steps.
There exist two solutions at this time point,
one of them containing
\texttt{unloaded(0)}, \texttt{do(wait,1)}, \texttt{unloaded(1)},
\texttt{do(load,2)}, \texttt{loaded(2)},
\texttt{do(shoot,3)}, \texttt{unloaded(3)}, \texttt{dead(3)}.
That is, we simply wait before loading and shooting.
The second solution loads the gun instead of waiting,
thus loading the gun twice before shooting.

%%% Local Variables:
%%% mode: latex
%%% TeX-master: "paper"
%%% End:
