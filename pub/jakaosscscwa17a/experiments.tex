
\section{Experimental analysis}\label{sec:experiments}

We begin with an empirical analysis of our \clingo{} derivatives in different settings.
We investigate, 
first,  different types of lc-atoms, viz.\ \lcatype{defined}{non-strict} versus \lcatype{external}{strict},
second, different levels of theory interfaces, \python\ or \cpp, for \clingod{dl},
and, third, different levels of integration, namely, dedicated implementations versus off-the-shelf solver.
Finally, we contrast the performance of our systems with other systems for \ASPm{lc}.

We ran each benchmark on a Xeon E5520 2.4 GHz processor under Linux limiting RAM to 20~GB and execution time to 1800s.
For \clingod{dl} and \clingod{lp}, we use \clingo~5.2.0.
Furthermore, we use \clingcon~3.2.0,
\dingo~v.2011-09-23,
\mingo~v.2012-09-30,
\ezsmt~1.0.0, 
and \ezcsp~1.7.9 for our experiments.
We upgraded \dingo\ and \mingo{} to use recent versions of their back-end solvers.
Hence, in our experiments,
the LP-based systems \clingod{lp} and \mingo{} use \cplex~12.7.0.0
and 
the SMT-based systems \dingo{} and \ezsmt{} use \zzz~4.4.2.
The benchmark set consists of 165 instances, 
among which 110 can be encoded using difference constraints~(\dl)
and 55 require linear constraints with more than two variables~(\lc).
In detail, the \dl\ set consists of
38 instances of two-dimensional strip packing~(\tsp)~\cite{sointabana10a},
and 72 instances of flow shop~(\fs), job shop~(\js), and open shop~(\os) problems~\cite{taillard93a},
selecting three instances for each job and machine at random.
Since not all systems support optimization over variable values,
we bounded the instances with 1.2 times the best known bound and solved the resulting decision problem.
The \lc\ instance set includes
20 instances of incremental scheduling~(\is),
15 instances of reverse folding~(\rf), and
20 instances of weighted sequence~(\ws).
Encodings have been adopted from ~\cite{liesus16a}
in combination with the instances from the ASP competition.%
\footnote{We refrained from using the other three benchmark classes from this source because the available instances were too easy in view of producing informative results.}
Our empirical evaluation focuses on available systems sharing comparable encodings.
This was not the case for \aspartame, \aspmttosmt, \inca, and \dlvhex[\textsc{cp}].
The first two systems have a proper and thus different input language and encoding philosophy,
\inca\ produced incorrect results (cf.~\cite{bakaossc16a} for details),
and \dlvhex[\textsc{cp}] is no longer maintained.

Table~\ref{tab:clingolc} compares \clingod{dl} and \clingod{lp} with different 
encoding techniques, 
types of theory atoms, and 
programming language hosting the theory interface
by measuring average time~(\T) and timeouts~(\TO).
% --------------------------------------------------------------------------------
\begin{table}%
\newcommand{\mc}[3]{\multicolumn{#1}{#2}{#3}}%
\caption{Comparison of \clingo\ derivatives \clingod{dl} and \clingod{lp}\label{tab:clingolc}}%
\centering%
\begin{tabular}{@{}r@{}@{}r@{}@{}r@{}@{}r@{\ }r@{}@{}r@{\ }r@{}@{}r@{\ }r@{}@{}r@{\ }r@{}@{}r@{\ }r@{}@{}r@{\ }}
\toprule
       &        & \mc{2}{@{\,}c@{\,}}{\clingolc{dl}{dns}{py}} 
                & \mc{2}{@{\,}c@{\,}}{\clingolc{dl}{es}{py}} 
                & \mc{2}{@{\,}c@{\,}}{\clingolc{lp}{dns}{py}} 
                & \mc{2}{@{\,}c@{\,}}{\clingolc{lp}{es}{py}} 
                & \mc{2}{@{\,}c@{\,}}{\clingolc{dl}{dns}{cpp}} 
                & \mc{2}{@{\,}c@{\,}}{\clingolc{dl}{es}{cpp}}  \\
\class & \#inst &   \T &       \TO  &   \T &       \TO  &   \T & \TO &   \T & \TO &           \T &         \TO &         \T &       \TO  \\
\midrule
\tsp   &     38 &  344 &         6  &  484 &         9  & 1346 &  23 & 1753 &  36 & \textbf{148} &  \textbf{3} &       342  &         7  \\
\fs    &     35 &  678 &        11  & 1541 &        27  & 1221 &  21 & 1800 &  35 & \textbf{465} &  \textbf{5} &      1349  &        26  \\
\js    &     24 & 1261 &        15  & 1229 &        14  & 1800 &  24 & 1800 &  24 & \textbf{534} &  \textbf{4} &       678  &         7  \\
\os    &     13 &    8 & \textbf{0} &   17 & \textbf{0} &  963 &   6 & 1532 &  10 &   \textbf{0} &  \textbf{0} & \textbf{0} & \textbf{0} \\
\midrule
\dl    &    110 &  611 &        32  &  928 &        50  & 1360 &  74 & 1752 & 105 & \textbf{316} & \textbf{12} &       695  &        40  \\
\bottomrule
\end{tabular}%
\end{table}%
%
%%% Local Variables:
%%% mode: latex
%%% TeX-master: "../paper"
%%% End:

% --------------------------------------------------------------------------------
Each column consists of one combination of form \emph{system}/\emph{atom}/\emph{language}, 
where
\emph{system} is either \textsc{dl} or \textsc{lp} for \clingod{dl} and \clingod{lp},
\emph{atom} either \atomtype{dns} or \atomtype{es} for \lcatype{defined}{non-strict} and \lcatype{external}{strict} lc-atoms,
and
\emph{language} either \textsc{py} or \textsc{cpp} for \python\ and \cpp, respectively.
To compare \clingod{dl} and \clingod{lp}, we restrict the set of benchmarks to \dl.
We observe that \atomtype{dns} performs better than \atomtype{es} in all settings.
Under lc-stable model semantics, defined lc-atoms are more tightly constrained.
External lc-atoms, on the other hand, induce an implicit choice leading to a
larger search space and might introduce duplicate solutions with different assignments.
Furthermore, strict lc-atoms double the amount of implications 
that have to be considered by the propagator.
As expected, the \cpp\ variant of \clingod{dl} outperforms its \python\ counterpart,
even though the performance gain does not reach an order of magnitude.

Table~\ref{tab:systems} compares different systems dealing with \ASPm{lc}
by average time~(\T) and timeouts~(\TO).
%
% --------------------------------------------------------------------------------
\begin{table}%
\newcommand{\mc}[3]{\multicolumn{#1}{#2}{#3}}%
\caption{Comparison of different systems for ASP with linear constraints\label{tab:systems}}%
\centering%
\begin{tabular}{@{}r@{\!}@{\!}r@{\!}
  @{\ }r@{\hspace{-5pt}}@{\hspace{-5pt}}r@{\,}
  @{\ }r@{\hspace{-5pt}}@{\hspace{-5pt}}r@{\,}
  @{\ }r@{\hspace{-1pt}}@{\hspace{-1pt}}r@{\,}
  @{\ }r@{\hspace{-1pt}}@{\hspace{-1pt}}r@{\,}
  @{\ }r@{\hspace{-1pt}}@{\hspace{-1pt}}r@{\,}
  @{\ }r@{\hspace{-1pt}}@{\hspace{-1pt}}r@{\,}
  @{\ }r@{\hspace{-1pt}}@{\hspace{-1pt}}r@{}}
\toprule
       &        & \mc{2}{@{}c@{}}{\clingolc{dl}{dns}{cpp}} 
                & \mc{2}{@{}c@{}}{\clingolc{lp}{dns}{py}} 
                & \mc{2}{@{}c@{}}{\clingcon} 
                & \mc{2}{@{}c@{}}{\dingo} 
                & \mc{2}{@{}c@{}}{\mingo} 
                & \mc{2}{@{}c@{}}{\ezsmt} 
                & \mc{2}{@{}c@{}}{\ezcsp}                  \\
\class & \#inst &          \T  &        \TO  &   \T & \TO & \T & \TO & \T & \TO & \T & \TO & \T & \TO & \T & \TO\\
\midrule
\tsp   &    38  &         148  &          3  & 1346 & 23 & \textbf{3}    & \textbf{0}  & 403  & 7  & 292  & 5  & 318  & 6  & 1800 & 38\\
\fs    &    35  & \textbf{465} &  \textbf{5} & 1221 & 21 & 1022 & 19 & 1047 & 20 & 1040 & 16 & 1667 & 32 & 735  & 9\\
\js    &    24  &         534  &          4  & 1800 & 24 & \textbf{277}  & \textbf{3}  & 1258 & 15 & 1423 & 18 & 1315 & 15 & 1800 & 24 \\
\os    &    13  &   \textbf{0} &  \textbf{0} &  963 & 6  & 1    & \textbf{0}  & 4    & \textbf{0}  & 76   & \textbf{0}  & 24   & \textbf{0}  & 1044 & 7 \\
\midrule
\dl    &    110 & \textbf{316} & \textbf{12} & 1360 & 74 & 387  & 22 & 765  & 42 & 743  & 39 & 930  & 52 & 1372 & 78 \\
\midrule
\is    &    20  &         \na  &         \na & 1800 & 20 & \textbf{582}  & \textbf{5}  & \na & \na & 649  & 7  & 648  & 7  & 1620 & 18\\
\rf    &    15  &         \na  &         \na & 1680 & 14 & \textbf{21}   & \textbf{0}  & \na & \na & 542 & 1 & 121  & \textbf{0}  & 1013 & 7\\
\ws    &    20  &         \na  &         \na & 1800 & 20 & 27   & \textbf{0}  & \na  & \na & 90   & \textbf{0}  & \textbf{12}   & \textbf{0}  & 1800 & 20 \\
\midrule
\lc    &    55  &         \na  &         \na & 1767 & 54 & \textbf{227}  & \textbf{5}  & \na  & \na & 416  & 8 & 273  & 7  & 1520 & 45\\
\midrule
all    &    165 &         \na  &         \na & 1564 & 128 & \textbf{307}  & \textbf{27} & \na  & \na & 580  & 47 & 602  & 59 & 1446 & 123\\
\bottomrule
\end{tabular}
\end{table}
%
%%% Local Variables:
%%% mode: latex
%%% TeX-master: "../paper"
%%% End:

% --------------------------------------------------------------------------------
%
Only the best configurations from Table~\ref{tab:clingolc} were selected for comparison.
All systems were tested using their default configurations.
For \dl{}, \clingolc{dl}{dns}{cpp} performs best overall,
even though \clingcon\ is better for \tsp\ and \js.
The class \fs\ generates the most difference constraints among all benchmark classes,
making it less suited for translation-based approaches, like \dingo, \mingo, and \ezsmt,
and producing overhead for more involved propagation as in \clingcon.
By default, \ezcsp\ performs the theory consistency check on full answer sets,
and by doing so avoids handling vast amounts of constraints during search
and therefore performs comparatively well on \fs.
For the other classes though, this generate and test approach is less effective.
Regarding \lc\ and overall results, \clingcon\ clearly dominates the competition,
followed by the two translation-based approaches \mingo\ and \ezsmt\ 
with underlying state-of-the-art solvers \cplex\ and \zzz, respectively.
\clingolc{lp}{dns}{py} falls behind,
since it is a straightforward \python\ implementation
and uses an exponential consistency check.
In addition, distinct features of \clingod{lp} like real-valued variables and optimization
as well as dynamic conditions are not supported by other systems 
and thus not included in the benchmark set.

  
%%% Local Variables: 
%%% mode: latex
%%% TeX-master: "paper"
%%% End: 
