\textbf{Big picture.}
%
Finally, let us relate our systems with others extending ASP with linear constraints.
%
The first category, referred to as translation-based approaches,
includes systems such as \ezsmt{}~\cite{liesus16a}, \dingo{}~\cite{jalini11a}, \aspmttosmt{}~\cite{barlee14b}, and \mingo{}~\cite{lijani12a}.
The first three translate both ASP and constraints into SAT Modulo Theories (SMT;~\cite{baseseti09a});
\dingo{} is restricted to difference constraints.
Unlike this,
\mingo's target formalism is Mixed Integer Linear Programming (MILP).
Furthermore, \aspartame~\cite{bageinospescsotawe15a} translates \ASPm{lc} (over integers) back to ASP by using the order encoding.
An advantage of translation-based approaches is that
once the input program is translated,
only a solver for the target formalism is needed.
In this way, they benefit from the features and performance of the respective target systems.
A drawback is the translation itself
since it may result in large propositional representations or weak propagation strength.
%
The second category extends the standard Conflict Driven Nogood Learning (CDNL;~\cite{gekasc09c}) machinery of ASP solvers with constraint propagators.
This allows for propagating both Boolean and linear constraints during search.
The latter are thus continuously checked for consistency and even new constraints may get derived.
For instance, the \clingo-based system \dlvhex[\textsc{cp}]~\cite{roeireri15a} uses \gecode, while \ezcsp{} uses a \prolog{} constraint solver
for consistency checking.
Unlike this,
\inca~\cite{drewal10a} extends a previous \clingo{} version with a customized lazy propagator generating constraints according to the order encoding.
This approach allows for deriving new constraints such as bounds of integer variables.

The \clingo{} derivatives \clingod{dl} and \clingod{lp} belong to the second category of systems, just like \clingcon~3.
%
% --------------------------------------------------------------------------------
\begin{table}[ht]
\caption{Comparing related applications}
\label{tab:related}
\center\small
\begin{threeparttable}
\begin{tabular}{@{\!\!}l@{\!\!}@{}c@{}@{}c@{}@{\!}c@{\!}@{\!}c@{\!}@{}c@{}@{\!}c@{\!}@{\!}c@{\!}@{\!}c@{\!}@{\!}c@{\!}@{\!}c@{\!}@{\!}c@{\!}}
\toprule
 & \clingo{}     & \clingo{}     & \clingcon{} & \aspartame{} & \inca{} & \ezcsp{} & \ezsmt{} & \mingo{} & \dingo{} & \sysfont{aspmt} & \dlvhex{}\\
 & [\textsc{dl}] & [\textsc{lp}] &             &              &         &          &          &          &          & \sysfont{2smt}  & [\textsc{cp}]\\\midrule
translation  & \xmark{} & \xmark{} & \cmark{}\tnote{1} & \cmark{} & \xmark{} & \xmark{} & \cmark{} & \cmark{} & \cmark{} & \cmark{} & \xmark{} \\
explicit     & \xmark{} & \xmark{} & \cmark{}\tnote{2} & \cmark{} & \cmark{} & \xmark{} & \xmark{} & \xmark{} & \xmark{} & \xmark{} & \xmark{} \\
non-linear   & \xmark{}\tnote{3} & \xmark{} & \cmark{}\tnote{4} & \cmark{}\tnote{4} & \cmark{} & \cmark{} & \cmark{} & \xmark{} & \xmark{}\tnote{3} & \cmark{} & \cmark{} \\
real numbers & \xmark{} & \cmark{} & \xmark{} & \xmark{} & \xmark{} & \cmark{} & \cmark{} & \cmark{}\tnote{5} & \xmark{} & \cmark{} & \xmark{}\\
optimization & \xmark{} & \cmark{}\tnote{6} & \cmark{} & \cmark{} & \cmark{} & \xmark{} & \xmark{} & \xmark{} & \xmark{} & \xmark{} & \cmark{} \\
non-tight    & \cmark{} & \cmark{} & \cmark{} & \cmark{} & \cmark{} & \cmark{} & \xmark{} & \cmark{} & \cmark{} & \xmark{} & \cmark{} 
\end{tabular}
\begin{tablenotes}\footnotesize
\begin{minipage}{0.5\textwidth}
\item[1] Allows for partial problem translations
\item[2] Lazily created
\item[3] Only difference constraints
\end{minipage}%
\begin{minipage}{0.45\textwidth}
\item[4] Translation of distinct into linear constraints
\item[5] Only for variables
\item[6] Optimization relative to stable models
\end{minipage}
\end{tablenotes}
\end{threeparttable}
\end{table}
%%% Local Variables:
%%% mode: latex
%%% TeX-master: "../paper"
%%% End:

% --------------------------------------------------------------------------------
%
Table~\ref{tab:related} summarizes important similarities and differences of the aforementioned systems.
The first row tells us whether a system relies on a translation to SMT, MILP, or ASP\@.
The second one indicates whether an approach uses some form of explicit variable representation.
This is the case when using an encoding and usually results in a large number of propositional atoms to represent variables with large domains.
Half of the systems are able to handle constraints over reals while the other half is restricted to integers.
Note that for a system of inequalities, a solution over reals can be found much easier than one over integers.
For all systems, real numbers are implemented as floating point numbers.
Due to this, round-off errors cannot completely be avoided.
Note that since computers are finite precision machines, 
the imprecision of floating point computations is common to any computer systems
and/or languages~\cite{goldberg91a}. 
\cplex\ uses numerically stable methods to perform its
linear algebra so that round-off errors usually do not cause problems.%
\footnote{See \emph{Numeric difficulties} at \url{https://www.ibm.com/support/knowledgecenter/SSSA5P_12.7.0/ilog.odms.studio.help/pdf/usrcplex.pdf}}
With ``non-linear'' we distinguish systems handling global or non-linear constraints,
and ``non-tight'' indicates whether a system can deal with recursive programs.
Finally, the table lists all systems that are able to optimize an objective function over integer and/or real variables.

%%% Local Variables:
%%% mode: latex
%%% TeX-master: "paper"
%%% End:
