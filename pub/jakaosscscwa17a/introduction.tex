
\section{Introduction}\label{sec:introduction}

Answer Set Programming (ASP;~\cite{baral02a}) has become an established paradigm for knowledge representation and reasoning,
in particular, when it comes to solving knowledge-intense combinatorial (optimization) problems.
Despite its versatility,
however, ASP falls short in handling non-Boolean constraints, such as linear constraints over unlimited integers or reals.
This shortcoming was broadly addressed in the recent \clingo~5 series~\cite{gekakaosscwa16a} by
providing generic means for incorporating theory reasoning.
They span from theory grammars for seamlessly extending \clingo's input language with theory expressions to 
a simple interface for integrating theory propagators into \clingo's solver component.

We instantiate this framework with different forms of linear constraints and elaborate upon its formal properties.
Given this, we discuss the respective implementations,
and present techniques for using these constraints in a reactive context.
In more detail,
we introduce extensions to \clingo{} with
difference and linear constraints over integers and reals, respectively, 
and realize them in complementary ways.
For handling difference constraints,
we provide customized implementations of well-established algorithms in \python{} and \cpp,
while we use \clingo's \python{} API to connect to off-the-shelf linear programming solvers, viz.\ \cplex{} and \lpsolve,
to deal with linear constraints.
In both settings, we support integer as well as real valued variables.
For a complement, we also consider \clingcon, a derivative of \clingo, 
integrating constraint propagators for handling linear constraints over integers at a low-level.
While this fine integration must be done at compile-time, the aforementioned \python{} extensions are added at run-time.
Our empirical analysis complements the study in~\cite{liesus16a} 
with experimental results on our new systems \clingod{dl} and \clingod{lp}.
Finally, we provide a comparison of different semantic options for integrating theories into ASP
and a systematic overview of the various features of state-of-the-art ASP systems handling linear constraints.

%%% Local Variables: 
%%% mode: latex
%%% TeX-master: "paper"
%%% End: 
