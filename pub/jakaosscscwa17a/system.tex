
\section{\clingo{} derivatives and related systems}\label{sec:system}

In this section, we give an overview of systems extending ASP with linear constraints.
We start with our own systems \clingod{dl} and \clingod{lp}
both relying upon \clingo's interface for theory propagators.
We also include \clingcon, since it is based on a much lower level API
using the internal functions of \clingo{} (and \clasp) in \cpp.
While \clingcon{} implements a highly sophisticated system using advanced preprocessing and solving techniques,
the \python{} variants of \clingod{dl} and \clingod{lp} provide easily modifiable and maintainable propagators for difference and linear constraints,
respectively.
This carries over to the \cpp{} variant of \clingod{dl} since the \cpp{} and \python{} API share the same functionality. 
% --------------------------------------------------------------------------------
\begin{table}%
\caption{Feature comparison}%
\label{tab:features}%
\center%
\begin{threeparttable}
\begin{tabular}{@{}l@{}@{}c@{}@{}c@{}@{\;}c@{}@{}c@{}@{}c@{}@{}c@{}@{}c@{}@{}c@{}@{}c@{}}
\toprule
             & \python{} &   \cpp{} &            strict & non-strict & external &           defined &    n-ary &             reals &      optimization\\
\midrule
\clingod{dl} &  \cmark{} & \cmark{} & \cmark{}\tnote{1} &   \cmark{} & \cmark{} & \cmark{}          & \xmark{} & \cmark{}\tnote{2} & \cmark{}\tnote{3}\\
\clingod{lp} &  \cmark{} & \xmark{} & \cmark{}          &   \cmark{} & \cmark{} & \cmark{}          & \cmark{} & \cmark{}          & \cmark{}\tnote{4}\\
\clingcon{}  &  \xmark{} & \cmark{} & \cmark{}          &   \xmark{} & \cmark{} & \xmark{}\tnote{5} & \cmark{} & \xmark{}          & \cmark{}         \\
%\bottomrule
\end{tabular}%
\begin{tablenotes}\footnotesize
  \begin{minipage}{0.375\textwidth}
  \item[1] Only with \python{} API
  \item[2] Only for non-strict lc-atoms
  \item[3] Needs an additional plugin
  \end{minipage}%
  \begin{minipage}{0.6\textwidth}
  \item[4] Optimization is relative to stable models
  \item[5] Theory atoms in rule heads are shifted into negative body
  \item[] \
  \end{minipage}
\end{tablenotes}%
\end{threeparttable}%
\end{table}%
%
%%% Local Variables:
%%% mode: latex
%%% TeX-master: "../paper"
%%% End:

% --------------------------------------------------------------------------------
Table~\ref{tab:features} shows a comparative list of features for these systems.
The two flexible \clingo{} derivatives support all four combinations
of strict/non-strict and defined/external lc-atom types,
whereas \clingcon{} has a fixed one.
Also the bandwidth of supported constraints is different.
While \clingod{dl} only supports difference constraints,
the other two support n-ary linear constraints.
Notably, \clingod{dl} and \clingod{lp} support (approximations of) real numbers (see below).
Moreover, all three \clingo{} derivatives allow for optimizing objective functions over
numeric variables (in addition to optimization in ASP).
% All systems are freely available at \url{https://potassco.org/{clingoDL,clingoLP,clingcon}}.

\textbf{\clingod{dl}}
extends \clingo{} with difference constraints of the form $x-y\leq k$,
where $x$ and $y$ are integer (or real) variables and $k$ is an integer (real) constant.
%
Despite the restriction to two variables, 
they allow for naturally encoding timing related problems, as e.g., in scheduling, % or timetabling,
and are solvable in polynomial time.
%Syntactically, a difference constraint $x-y\leq k$ is represented by a difference constraint atom of the form `\lstinline[mathescape]@&diff{x-y}<=k@'.
%
\clingod{dl} uses \clingo{}'s theory interface to realize a stateful propagator
that checks during search whether the current set of implied difference constraints
is satisfiable~\cite{cotmal06a}.
%
To this end,
it makes use of the stateful nature of the theory interface
that allows 
for incrementally updating internal states
and thus for backtracking to previous states without having to rebuild the internal representation.
% \clingod{dl} is available using either the \python\ or \cpp\ interface at \url{https://github.com/potassco/clingoDL}.
%
By default, all difference constraint atoms are considered to be non-strict.
In this case, it is only necessary to keep track of lc-atoms that are assigned true
since only then the constraint is required to hold.
In the strict case,
false assignments to difference constraint atoms are considered as well.
This is done by adding $y-x\leq-k-1$ 
whenever `\lstinline[mathescape]@&diff{x-y}<=k@' is assigned false.
%
As a side-product of the satisfiability check,
an integer (real) assignment for all variables is obtained
and ultimately printed for all lc-stable models.
Usually, several or even an infinite number of assignments exist. % if the set of difference constraints is satisfiable.
The returned assignment is the one with the lowest sum of the absolute values of all variables.
For instance, in terms of scheduling problems, this amounts to scheduling each job as soon as possible.
%%%Besides binary constraints, it is possible to encode unary constraints
%%%by using the constant `\lstinline[mathescape]@0@', 
%%%where `\lstinline[mathescape]@0@' is treated as a dedicated variable being always $0$, 
%%%eg $x\leq10$ is encoded as `\lstinline[mathescape]@&diff{x-0} <= 10@'.\comment{maybe remove mathescape or use it in the constraint?}

\textbf{\clingod{lp}}
fully covers the extension of \ASPm{lc} described in Section~\ref{sec:background}.
This \clingo{} derivative accepts lc-atoms containing integer and real variables possibly subject to dynamic conditions.
That is, \clingod{lp} extends ASP with constraints as dealt with in Linear Programming (LP;~\cite{dantzig63a})
as well as according objective functions for optimization. 
In \clingod{lp}, the latter are subject to dynamic conditions and thus depend on the respective Boolean assignment
(as in regular ASP optimization).
%
As above,
the theory interface of \clingo{} is used to integrate a stateful propagator that checks during search 
the satisfiability of the current set of linear constraints.
Here, however, this is done with a generic interface to dedicated LP solvers,
currently supporting \cplex\ and \lpsolve.
(Note that both LP solvers do an exponential consistency check.) % use the simplex algorithm. \comment{P: ref to simplex maybe} 
%
The \python\ interfaces of \cplex\ and \lpsolve\ natively support relations $=$, $\geq$, and $\leq$. 
We add support for $<$, $>$, and $\neq$.
To this end, %similar to \cplex\ linearization, %\footnote{See~\url{https://www.ibm.com/support/knowledgecenter/SS9UKU\_12.4.0/com.ibm.cplex.zos.help/Parameters/topics/EpLin.html} for detailed information.} 
we translate $<$ and $>$ into $\leq$ and $\geq$ by subtracting or adding an $\varepsilon$ to the right-hand-side of a linear constraint, respectively.%
\footnote{This $\varepsilon$ can be configured using the command line and defaults to $10^{-3}$ (as in \cplex).}
Furthermore, $\neq$ is treated as a disjunction of $<$ and $>$.
%
By default, \clingod{lp} treats lc-atoms in a non-strict manner.
Thus only linear constraints represented by true lc-atoms are considered.
When treating them strictly, false lc-atoms are handled using the complementary relation.
In this case, 
the corresponding linear constraint is derived by using the complementary relation.
%
Notably, \clingod{lp} offers dynamic conditions in lc-atoms.
This allows for linear constraints of variable length even during search.
All conditions have to be decided before such a constraint is included in the consistency check.
Furthermore, \clingod{lp} updates its internal state incrementally 
but rebuilds the linear constraint system after backtracking to avoid accumulating rounding errors. 
%
%If the LP solver establishes inconsistency, a nogood is added  only consisting of literals associated with current linear constraints.
%Naively, all assigned lc-atoms and conditions may be gathered as reason for the inconsistency.
%But often only a subset of linear constraints is responsible for a conflict.  
%To this end, we identify a core conflict using the Irreducible Inconsistent Set (IIS) algorithm \cite{ostsch12a,loon81}.
Also, it uses an Irreducible Inconsistent Set algorithm~\cite{loon81}
for extracting minimal sets of conflicting constraints to support conflict learning in the ASP solver.
On the one hand, this extraction is expensive, on the other hand, such core conflicts may significantly reduce the search space. 
To control this trade-off, \clingod{lp} only enables this feature after a certain percentage of lc-atoms and conditions is assigned (by default 20\%).
Similarly,
frequent theory consistency checks are expensive and a conflict is less likely to be found within a small assignment;
accordingly, an analogous percentage based threshold allows for controlling their invocation (default 0\%). 
%
% \clingod{lp} is available at \url{https://github.com/potassco/clingoLP}.

\textbf{\clingcon} series~3 offers a \clingo-based ASP system with handcrafted propagators for constraints over integers~\cite{bakaossc16a};
it is implemented in \cpp{} and features a strict, external semantics.
Sophisticated preprocessing techniques are supported
and non-linear constraints such as the global \emph{distinct} constraint are translated into linear ones.
Integer variables are represented using the order encoding~\cite{crabak94a},
and customized propagators using state-of-the-art lazy nogood and variable generation are employed.
The propagators do not only ensure bound consistency on the variables
but also derive new bounds.
%This means that new constraints are derived.
Furthermore,
multi-objective optimization on integer variables is supported.
In contrast to \clingod{lp},
conditions on integer variables must be static.
% The system is available at \url{https://potassco.org/clingcon}.

Our systems are available at \url{https://potassco.org/labs/{clingoDL,clingoLP}}
and \url{https://potassco.org/clingcon}.

%%% Local Variables: 
%%% mode: latex
%%% TeX-master: "paper"
%%% End: 
