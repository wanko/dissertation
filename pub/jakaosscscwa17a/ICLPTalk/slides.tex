\documentclass[t,fleqn]{beamer}

\usepackage[english]{babel}
\usepackage[utf8]{inputenc}

\usepackage{listings}
\lstset{basicstyle=\ttfamily,escapechar=\%,escapeinside={\#(}{\#)}}
\usepackage{lstlinebgrd}

\makeatletter
%%%%%%%%%%%%%%%%%%%%%%%%%%%%%%%%%%%%%%%%%%%%%%%%%%%%%%%%%%%%%%%%%%%%%%%%%%%%%%
%
% \btIfInRange{number}{range list}{TRUE}{FALSE}
%
% Test in int number <number> is element of a (comma separated) list of ranges
% (such as: {1,3-5,7,10-12,14}) and processes <TRUE> or <FALSE> respectively

\newcount\bt@rangea
\newcount\bt@rangeb

\newcommand\btIfInRange[2]{%
    \global\let\bt@inrange\@secondoftwo%
    \edef\bt@rangelist{#2}%
    \foreach \range in \bt@rangelist {%
        \afterassignment\bt@getrangeb%
        \bt@rangea=0\range\relax%
        \pgfmathtruncatemacro\result{ ( #1 >= \bt@rangea) && (#1 <= \bt@rangeb) }%
        \ifnum\result=1\relax%
            \breakforeach%
            \global\let\bt@inrange\@firstoftwo%
        \fi%
    }%
    \bt@inrange%
}
\newcommand\bt@getrangeb{%
    \@ifnextchar\relax%
        {\bt@rangeb=\bt@rangea}%
        {\@getrangeb}%
}
\def\@getrangeb-#1\relax{%
    \ifx\relax#1\relax%
        \bt@rangeb=100000%   \maxdimen is too large for pgfmath
    \else%
        \bt@rangeb=#1\relax%
    \fi%
}

%%%%%%%%%%%%%%%%%%%%%%%%%%%%%%%%%%%%%%%%%%%%%%%%%%%%%%%%%%%%%%%%%%%%%%%%%%%%%%
%
% \btLstHL<overlay spec>{range list}
%
% TODO BUG: \btLstHL commands can not yet be accumulated if more than one overlay spec match.
% 
\newcommand<>{\btLstHL}[1]{%
  \only#2{\btIfInRange{\value{lstnumber}}{#1}{\color{blue!80}\def\lst@linebgrdcmd{\color@block}}{\def\lst@linebgrdcmd####1####2####3{}}}%
}%
\makeatother

%%% Local Variables:
%%% mode: latex
%%% TeX-master: "slides"
%%% End:


\usepackage{tikz}
\usetikzlibrary{arrows,chains,positioning,automata,decorations,shapes,calc,matrix,fit,backgrounds} % CHECK!!
\usepackage{pgfpages}
%\usepackage{tikz-uml}
\usepackage[boxed,vlined]{algorithm2e}
\SetNlSty{bfseries}{\color{black}}{} %% make linenumbers black, as a sometimes highlight code in different colors
\newlength{\commentWidth}
\setlength{\commentWidth}{5cm}
\newcommand{\atcp}[1]{\tcp*[r]{\makebox[\commentWidth]{#1\hfill}}}

\lstnewenvironment{Examplecode}{}{}

\usepackage{threeparttable}

\title[Clingo goes Linear Constraints]{Clingo goes Linear Constraints over Reals and Integers}
\author[Six dudes]{%
  Tomi~Janhunen
  \and 
  Roland~Kaminski
  \and
  Max~Ostrowski
  \and 
  Torsten~Schaub
  \and
  Sebastian~Schellhorn
  \and
  Philipp~Wanko}

\institute[]{%
  Aalto~University
  \and
  INRIA~Rennes 
  \and
  University~of~Potsdam\\\bigskip\bigskip\includegraphics[scale=0.07]{potassco_logo_lightblue}}
\date{}

\usetheme{Pittsburgh}
\useinnertheme{rectangles}
\mode<beamer>{\usecolortheme{albatross}}
\mode<beamer>{\setbeamercolor{alerted text}{fg=yellow}}
\mode<handout>{\usecolortheme{seagull}}
\mode<handout>{\setbeamercolor{alerted text}{fg=blue}}
\useoutertheme{infolines}
\setbeamertemplate{navigation symbols}{\vspace{-5pt}} % to lower the logo
\setbeamercovered{dynamic}
\setbeamertemplate{bibliography item}[text]

\AtBeginSection{
  \begin{frame}{Outline}
    \bigskip
    \vfill
    \tableofcontents[currentsection]
  \end{frame}
}
\AtBeginSubsection{
  \begin{frame}{Outline}
    \vfill
    \tableofcontents[currentsubsection]
  \end{frame}
}

\newcommand{\backupbegin}{
   \newcounter{framenumbervorappendix}
   \setcounter{framenumbervorappendix}{\value{framenumber}}
}
\newcommand{\backupend}{
   \addtocounter{framenumbervorappendix}{-\value{framenumber}}
   \addtocounter{framenumber}{\value{framenumbervorappendix}} 
}


\usepackage{pifont}
\newcommand{\itarrow}{{\Pisymbol{pzd}{229}}}

\newcommand{\gringo}{\textit{gringo}}
\newcommand{\clasp}{\textit{clasp}}
\newcommand{\clingo}{\textit{clingo}}
\newcommand{\asprin}{\textit{asprin}}
\newcommand{\asap}{\textit{teaspoon}}
\newcommand{\piclasp}{\textit{piclasp}}

\newcommand{\code}[1]{\lstinline[basicstyle=\ttfamily]{#1}}

\newcommand{\lw}[1]{\smash{\lower1.ex\hbox{#1}}}
\newcommand{\llw}[1]{\smash{\lower3.ex\hbox{#1}}}

%\newcommand{\dataCL}[5]{%
%  \code{#1} & #3 & #5 & #4
%}
%\newcommand{\dataCS}[5]{%
%  #3 & #5 & #4
%}

\newenvironment{tableC}{%
  \scriptsize
  \tabcolsep = 0.6mm
  \begin{tabular}[t]{l|rlr|rlr|rlr|rlr|rlr}\hline
    \multicolumn{1}{l|}{\llw{Instance}} &
    \multicolumn{3}{c|}{UD1} &
    \multicolumn{3}{c|}{UD2} &
    \multicolumn{3}{c|}{UD3} &
    \multicolumn{3}{c|}{UD4} &
    \multicolumn{3}{c}{UD5} \\
    & 
    \multicolumn{1}{c}{Best} & & \multicolumn{1}{c|}{\emph{tea-}} & 
    \multicolumn{1}{c}{Best} & & \multicolumn{1}{c|}{\emph{tea-}} & 
    \multicolumn{1}{c}{Best} & & \multicolumn{1}{c|}{\emph{tea-}} & 
    \multicolumn{1}{c}{Best} & & \multicolumn{1}{c|}{\emph{tea-}} & 
    \multicolumn{1}{c}{Best} & & \multicolumn{1}{c}{\emph{tea-}} \\
    & 
    known & & \emph{spoon} & 
    known & & \emph{spoon} & 
    known & & \emph{spoon} & 
    known & & \emph{spoon} & 
    known & & \emph{spoon} \\
    \hline
  }{%
    \hline
  \end{tabular}
}

\newenvironment{tableB}{%
  \scriptsize
  \tabcolsep = 0.7mm
%  \begin{tabular}[t]{|l|c|r|l|l|l|}\hline
  \begin{tabular}[t]{lcrlll}\hline
    Instance &
    Formulation &
    Time (sec.)\\
    \hline
  }{%
    \hline
  \end{tabular}
}
\newenvironment{tableL}{%
  \scriptsize
  \tabcolsep = 0.7mm
  \begin{tabular}[t]{l|rrrrrrrr|r}\hline
    \lw{Instance} &
    \lw{Time (sec.)} &
    \multicolumn{6}{c}{The best utility vector} &
    The sum of  &
    The best of basic\\
    &
    &
    $(S_1,$ & $S_4,$ & $S_2,$ & $S_7,$ & $S_6,$ & $S_3)$ &
    utility vector &
    and optimized \\
    \hline
  }{%
    \hline
  \end{tabular}
}

%%% Local Variables:
%%% mode: latex
%%% TeX-master: "paper"
%%% End:

\newcommand{\cmark}{\ding{51}}%
\newcommand{\xmark}{\ding{55}}%

\begin{document}
% ----------------------------------------------------------------------
\frame{\titlepage}
\pgfdeclareimage[height=0.5cm]{potasscologo}{potassco_logo_blue}
\logo{\pgfuseimage{potasscologo}}
% ----------------------------------------------------------------------
\begin{frame}{Outline}
  \bigskip
  \vfill
  \tableofcontents[hideallsubsections]
\end{frame}
% ----------------------------------------------------------------------
\section{Motivation}
% ----------------------------------------------------------------------
\begin{frame}{Motivation}
  \bigskip
  \begin{itemize}
  \item Extend the range of applicability of ASP by combining it\\
    with linear constraints
    \uncover<2->{while preserving}
    \smallskip
    \begin{itemize}\normalsize
    \item<2-> declarative nature of ASP
    \smallskip
    \item<2-> performance of CDCL
    \end{itemize}
    \bigskip
  \item<3-> Builds on ``Theory Solving made easy with Clingo~5'' (ICLP'16) 
    \medskip
  \item<4-> Discuss 3 systems: \clingo[DL], \clingo[LP], \clingcon\uncover<5>{ (aka \clingo[CP])}
    \begin{itemize}
    \item<6-> semantic issues
    \item<6-> computaional impacts
    \end{itemize}
  \end{itemize}
\end{frame}
% ----------------------------------------------------------------------
\begin{frame}[c,fragile,shrink=20]{Example: Job Shop}
\smallskip
\begin{lstlisting}
% et(T,M,Time) --- execution time for a task
%  s(T,M,S)    --- order of machines for a task

 task(T)    :- et(T,_,_).
 machine(M) :- et(_,M,_).

 seq((T,M),(T,M'),Time)  :- s(T,M,S), s(T,M',S+1),
                            et(T,M,Time).

{seq((T,M),(T',M),Time)} :- s(T,M,S), s(T',M,S'), T<T',
                            et(T,M,Time), et(T',M,Time').

 seq((T',M),(T,M),Time') :- s(T,M,S), s(T',M,S'), T<T',
                            et(T,M,Time), et(T',M,Time'),
                            not seq((T,M),(T',M),Time).

#(\textcolor<2->{red}{start(T,M) + Time <= start(T',M')} :- seq((T,M),(T',M'),Time).
\end{lstlisting}
\end{frame}
% ----------------------------------------------------------------------
\section{Semantics}
% ----------------------------------------------------------------------
\begin{frame}[fragile]{Logic Programs}
  \begin{itemize}
  \item A logic program with linear constraints over $\mathcal{A}$ and $\mathcal{L}$ is a set of rules
    \smallskip
\begin{lstlisting}[mathescape,numbers=none]
   a$_1$;...;a$_m$ :- a$_{m+1}$,...,a$_n$,not$\,$a$_{n+1}$,...,not$\,$a$_o$
\end{lstlisting}
    \smallskip
  \item [] where each \lstinline[mathescape]{a$_i$} is either
    \begin{itemize}\normalsize
    \item  a regular atom in $\mathcal{A}$ or
    \item an {lc-atom} in $\mathcal{L}$ of form
      `\lstinline[mathescape]@&sum{c$_1$*v$_1$;$\dots$;c$_l$*v$_l$}<=b@'
    \end{itemize}
  \end{itemize}
\end{frame}
% ----------------------------------------------------------------------
\begin{frame}[fragile]{LC-atoms}
  \begin{itemize}
  \item<1-> External Atoms
    \begin{itemize}
    \item $\mathcal{L}\setminus\head{P}$ \uncover<2->{ --- only dependent on theory}
    \end{itemize}
  \item<1-> Defined Atoms
    \begin{itemize}
    \item $\mathcal{L}\cap\head{P}$ \uncover<3->{ --- need to be deriveable by rules}
    \end{itemize}
    \medskip
  \item<4-> Strict Atoms
    \begin{itemize}
    \item $\mathcal{L}^\leftrightarrow$ \uncover<5->{ --- atom is true \emph{iff}\/  constraint satisfied}
    \end{itemize}
  \item<4-> Non-strict Atoms
    \begin{itemize}
    \item $\mathcal{L}^\rightarrow$ \uncover<6->{ --- atom is true \emph{only if}\/ constraint satisfied}
    \end{itemize}
    \medskip
  \item<7-> Sensible combinations for linear constraints
    \begin{itemize}
    \item Defined-Nonstrict
    \item External-Strict
    \end{itemize}
    See paper for details
  \end{itemize}
\end{frame}
% ----------------------------------------------------------------------
\begin{frame}[fragile]{LC-solution}
  \begin{itemize}
  \item A set
    \(
    \mathcal{S}\subseteq\mathcal{L}
    \)
    of lc-atoms
    is a \emph{linear constraint solution}
    \smallskip
  \item []
    if there is an assignment of reals (or integers) to all real (integer) valued variables represented in $\mathcal{L}$ that
    \medskip
    \begin{itemize}\normalsize
    \item satisfies all linear constraints associated with\par strict and non-strict lc-atoms in $\mathcal{S}$ and
    \medskip
    \item falsifies all linear constraints associated with\par strict                lc-atoms in $\mathcal{L}^\leftrightarrow\setminus\mathcal{S}$.
    \end{itemize}
  \end{itemize}
\end{frame}
% ----------------------------------------------------------------------
\begin{frame}[fragile]{LC-stable model}
  \begin{itemize}
  \item 
    A set $X\subseteq\mathcal{A}\cup\mathcal{L}$ of atoms
    is an \emph{lc-stable model} of an lc-program~$P$
    \smallskip
  \item []
    if there is some lc-solution $\mathcal{S}\subseteq\mathcal{L}$ such that
    $X$ is a (regular) stable model of the logic program
    \pause
    \smallskip
\newcommand{\code}[1]{\text{\texttt{#1}}}
\begin{align*}
P\pause
&{}\cup
\{\code{a.}       \mid \code{a}\in (\mathcal{L}^\leftrightarrow\setminus\head{P})\cap \mathcal{S}\}\\
&{}\cup
\{\code{:- not a.}\mid \code{a}\in (\mathcal{L}^\leftrightarrow\cap\head{P})\cap \mathcal{S}\}
\\\pause
&{}\cup
\{\code{\{a\}.}   \mid \code{a}\in (\mathcal{L}^\rightarrow\setminus\head{P})\cap \mathcal{S}\}\\
&{}\cup
\{\code{:- a.}    \mid \code{a}\in (\mathcal{L}\cap\head{P})\setminus \mathcal{S}\}\rlap{.}
\end{align*}
\smallskip
\item<5-> See paper for formal elaboration
\end{itemize}
\end{frame}
% ----------------------------------------------------------------------
\begin{frame}[fragile,containsverbatim]{Examples}
\vspace{-\baselineskip}%
\begin{columns}[t]
\begin{column}[t]{0.48\linewidth}
%\leavevmode
\begin{block}{Defined-Nonstrict}
\texttt{\{a\}.}\newline
\texttt{\&sum\{x\} > 7 :- a.}
\end{block}
\end{column}
\begin{column}[t]{0.48\linewidth}
%\leavevmode
\begin{block}{External-Strict}
\visible<3->{
\texttt{\{a\}.}\newline
\texttt{:- a, not \&sum\{x\} > 7.}}
\end{block}
\end{column}
\end{columns}
%
\begin{columns}
\begin{column}[t]{0.48\linewidth}
%\leavevmode
\begin{block}{Solutions\hfill\small ($\mathtt{x}\in\{1,\dots,10\}$)}
\visible<2->{%
\begin{tabular}{l|ll}
\texttt{a} & & \\
\texttt{\&sum\{x\}>7} & &\\
\texttt{x=8..10} & \texttt{x=1..10} &
\end{tabular}}
%
\end{block}
\end{column}
\begin{column}[t]{0.48\linewidth}
%\leavevmode
\begin{block}{Solutions\hfill\small ($\mathtt{x}\in\{1,\dots,10\}$)}
\visible<4->{
 \begin{tabular}{l|l|l}
\texttt{a} & & \\ 
\texttt{\&sum\{x\}>7} & \texttt{\&sum\{x\}>7} & \\
\texttt{x=8..10} & \texttt{x=8..10} & \texttt{x=1..7}
\end{tabular}}
\end{block}
\end{column}
\end{columns}
\vspace{\baselineskip}
\begin{columns}<5->
\begin{column}[t]{0.48\linewidth}
Allows for compact representation

%Eg.\
\clingo[DL],
\clingo[LP]

\end{column}
\begin{column}[t]{0.48\linewidth}
Allow for constraints in the body

%Eg.\
\clingo[DL],
\clingo[LP],
\clingcon, 
\dlvhex[CP],
\inca, 
\dingo, 
\mingo
\end{column}
\end{columns}
\smallskip
\visible<6->{\structure{External-Nonstrict}: \ezcsp{}, \ezsmt}

\end{frame}
% ----------------------------------------------------------------------
\section{Theory Solving with clingo}
\begin{frame}[c]{ASP solving process\only<2->{ \alert<2>{modulo theories}}}
  \begin{center}
    \small
    \setlength{\unitlength}{.75pt}
    \begin{picture}(420,200)(-210,-35)
      \put(-200,110){{\framebox(80,40){Problem}}}
      \put(-200,-20){\alert<2->{\framebox(80,40){\shortstack{\textbf<2->{Logic}\\\textbf<2->{Program}}}}}
      \put(-80,-15){\alert<2->{\framebox(60,30){\textbf<2->{Grounder}}}}
      \put(  20,-15){\alert<2->{\framebox(60,30){\textbf<2->{Solver}}}}
      \put( 120,-20){\alert<2->{\framebox(80,40){\shortstack{\textbf<2->{Stable}\\\textbf<2->{Models}}}}}
      \put( 120,110){\framebox(80,40){Solution}}
      \put(-120,0){\vector(1,0){40}}
      \put( -20,0){\vector(1,0){40}}
      \put(  80,0){\vector(1,0){40}}
      \put(-160,110){\vector(0,-1){90}}
      \put( 160, 20){\vector(0, 1){90}}
      \put(-217, 62.5){{Modeling}}
      \put( 165, 62.5){{Interpreting}}
      \alt<3->{\put(   0,-42.5){{\makebox(0,0){Solving}}}}% MULTI-SHOT
              {\put(   0,-32.5){{\makebox(0,0){Solving}}}}
      % \only<7>{
      % \put(-160,-50){\vector(0, 1){30}}
      % \put( 160,-20){\line(0,-1){30}}
      % \put(-160,-50){\line(1,0){320}}
      % \put(   0,-65){\alert<7>{\makebox(0,0){\alert{Elaborating}}}}
      % }
      \only<3->{\put(-90,-30){\alert<3>{\framebox(180,60){}}}%
                \put( 20,0){\vector(-1,0){40}}}
    \end{picture}
  \end{center}
\end{frame}
% ----------------------------------------------------------------------
\begin{frame}[c]{\clingo's approach}
  \begin{center}
  \thicklines\small
  \setlength{\unitlength}{1.25pt}
    \begin{picture}(280,60)
    \put(  2, 20){\dashbox(35,30){\small\shortstack{{T-ASP}\\Program}}}
    \put( 50, 10){\framebox(170,50){}}
    \put( 65, 20){\framebox(60,30){\normalsize\gringo\ \qquad}}
    \put(145, 20){\framebox(60,30){\normalsize\clasp\ \qquad}}      
    \put(100, 25){\framebox(20,10){\small{\alert<2>{\textbf<2>{T}}}}}
    \put(180, 25){\framebox(20,10){\small{\alert<3>{\textbf<3>{T}}}}}
    \put(235, 20){\dashbox(35,30){\small\shortstack{{T-}ASP\\Solution}}}
%    \put(120, 35){\vector(1,0){30}}
    \put( 37, 35){\vector(1,0){13}}
    \put(220, 35){\vector(1,0){15}}
    \put(  2, 55){\dashbox(35,30){\small\shortstack{{Theory \alert<2>{\textbf<2>{T}}}\\Grammar}}}
    \put( 37, 72){\line(1,0){3}}
    \put( 40, 72){\line(0,-1){37}}    
    \put(  2,-15){\dashbox(35,30){\small\shortstack{Theory \alert<3>{\textbf<3>{T}}\\\footnotesize{Propagator}}}}
    \put( 37,  2){\line(1,0){3}}
    \put( 40,  2){\line(0,1){33}}
  \end{picture}
  \end{center}
\end{frame}
% ----------------------------------------------------------------------
\subsection{Theory Grammar}
% ----------------------------------------------------------------------
\begin{frame}[fragile]{\clingo[DL]}
  \small
\begin{lstlisting}
 #theory dl {
     constant  { - : 1, unary};
     diff_term { - : 1, binary, left};

     &diff/0 : diff_term, {<=}, constant, any
 }.    
\end{lstlisting}

  \only<2>{\normalsize{\bigskip\structure{Example} \ `$(x_1-x_2)\leq k$' is represented as `\lstinline{&diff \{ x(1)-x(2) \} <= k}'}}%
  \begin{onlyenv}<3>
\begin{lstlisting}
 #const n=2. 
 task(1..n).  

 duration(T,200*T) :- task(T). 

 &diff{end(T)-beg(T)} <= D :- duration(T,D). 
\end{lstlisting}
  \end{onlyenv}
  \begin{onlyenv}<4>
\begin{lstlisting}

 task(1).  task(2).   

 duration(1,200).  duration(2,400).

 &diff{end(1)-beg(1)}<=200.  &diff{end(2)-beg(2)}<=400.
\end{lstlisting}
  \end{onlyenv}
\end{frame}
% ----------------------------------------------------------------------
\begin{frame}[fragile,shrink=1]{clingcon}
\begin{lstlisting}
#theory csp {
    linear_term {                      show_term {            
      + : 5, unary;                      / : 1, binary, left  
      - : 5, unary;                    };                     
      * : 4, binary, left;             
      + : 3, binary, left;             
      - : 3, binary, left              minimize_term {        
    };                                   + : 5, unary;        
                                         - : 5, unary;        
    dom_term {                           * : 4, binary, left; 
      + : 5, unary;                      + : 3, binary, left; 
      - : 5, unary;                      - : 3, binary, left; 
      .. : 1, binary, left               @ : 0, binary, left  
    };                                 };                     

    &dom/0 : dom_term, {=}, linear_term, any;
    &sum/0 : linear_term, {<=,=,>=,<,>,!=}, linear_term, any;
    &distinct/0 : linear_term, any;
    &show/0 : show_term, directive;
    &minimize/0 : minimize_term, directive
}.    
\end{lstlisting}
\end{frame}
% ----------------------------------------------------------------------
\begin{frame}[fragile,shrink=20]{send+more=money}
\begin{lstlisting}
#include <csp>.

letter(s;e;n;d;m;o;r;y).
&dom {0..9} = X :- letter(X).

&sum {           1000*s;  100*e;  10*n;  1*d;
                 1000*m;  100*o;  10*r;  1*e;
      -10000*m; -1000*o; -100*n; -10*e; -1*y} = 0.
&sum {m} != 0.

&distinct {X : letter(X)}.

&show {X : letter(X)}.
\end{lstlisting}
\end{frame}
% ----------------------------------------------------------------------
\begin{frame}[fragile,shrink=20]{\clingo[LP]}
\begin{lstlisting}
#theory lp {
    lin_term {                     bounds{               
      - : 2, unary;                  - : 4, unary;         
      * : 1, binary, left;           * : 3, binary, left;  
      + : 0, binary, left;           / : 2, binary, left;  
      - : 0, binary, left            + : 1, binary, left;  
    };                               - : 1, binary, left;  
                                    .. : 0, binary, left  
                                   };

    &sum/0   : lin_term, {<=,>=,>,<,=,!=}, bounds, any;
    &minimize/0 : lin_term, head;
    &maximize/0 : lin_term, head;
    &dom/0 : bounds, {=}, lin_term, head
}.
\end{lstlisting}
\end{frame}
% ----------------------------------------------------------------------
\begin{frame}[fragile,shrink=20]{Example}
\begin{lstlisting}
#include "linprog.lp".

node(1).    in(1,"3.14",e1).                  out(1,3,e2).
node(2).    in(2,3,e2).       in(2,"4.2",e3). out(2,5,e4).

target(e4).

&dom{0..100}=E :-  in(_,_,E).
&dom{0..100}=E :- out(_,_,E).

&sum{ IW*IE   :  in(N,IW,IE) ;
     -OW*OE   : out(N,OW,OE) } = 0 :- node(N).

&maximize{ T : target(T) }.
\end{lstlisting}
\end{frame}
% ----------------------------------------------------------------------
\subsection{Theory Propagators}
% ----------------------------------------------------------------------
\begin{frame}[c]{Architecture of \textit{clasp}}
\setbeamercovered{transparent}
\newcommand{\rec}[4]{
  \node[#1,draw=none] #2 (#3tmp) { #4 };
  \draw[#1,fill=bg]
    ($(#3tmp.north west) + (5pt,0)$) --
    ($(#3tmp.north east) + (-5pt,0)$) --
    ($(#3tmp.north east) + (0,-5pt)$) --
    ($(#3tmp.south east) + (0,5pt)$) --
    ($(#3tmp.south east) + (-5pt,0)$) --
    ($(#3tmp.south west) + (5pt,0)$) --
    ($(#3tmp.south west) + (0,5pt)$) --
    ($(#3tmp.north west) + (0,-5pt)$) --
    cycle
    ;
  \node[#1,draw=none] #2 (#3) { #4 };
}
\newcommand{\funnywave}[2]
{
    \draw[#1]
      ($(#2.south west)+(0,-5pt)$) -- (#2.north west) -- (#2.north east) -- (#2.south east)
      ($(#2.south west)+(0,-5pt)$) edge [bend right=25] ($(#2.south)+(0,-5pt)$)
      ($(#2.south)+(0,-5pt)$) edge [bend left=15] (#2.south east)
    ;
}
%\IfFileExists{figures/pgf-new.txt}{
\newcommand{\MYTRSTYLE}{false}
% }{
% \newcommand{\MYTRSTYLE}{true}
% }
  % Note: for slides
  \colorlet{alertt}{yellow}%{red}
  \colorlet{bg}{background canvas.bg}
%  \colorlet{bg}{white}
  \begin{center}
    \begin{tikzpicture}[
        >=stealth',
        scale=0.77,
        transform shape,
        parlabel/.style={draw,minimum width=1.5cm,minimum height=0.5cm,inner sep=0},
        pp/.style={draw},
        pc/.style={draw},
        sc/.style={draw},
        co/.style={draw},
        so/.style={draw},
        prop/.style={draw},
        enum/.style={draw},
        parallel/.style={draw},
        pcontext/.style={draw},
%        no_transform/.style={transform shape=false}
        no_transform/.style={transform shape=\MYTRSTYLE}
      ]
      %
%	  \alt<2>{\tikzset{parallel/.style={draw=alertt,color=alertt}}}{\tikzset{parallel/.style={draw,opacity=0.4}}}


      % overwrite styles in overlay 5
      % \only<2>{\tikzset{pp/.style={draw=alertt,color=alertt}}}
      % \only<3>{\tikzset{sc/.style={draw=alertt,color=alertt}}}
      % \only<4>{\tikzset{so/.style={draw=alertt,color=alertt}}}
      \only<2>{\tikzset{prop/.style={draw=alertt,color=alertt}}}
%      \only<6>{\tikzset{enum/.style={draw=alertt,color=alertt}} \tikzset{pcontext/.style={draw=alertt,color=alertt}}}
%      \only<7>{\tikzset{parallel/.style={draw=alertt,color=alertt}}}



        \begin{scope}[so]
          % Solver
          \node[draw,no_transform, fill=bg,rounded corners,inner sep=0,fit={(0.12,0.12)(11.12,4.12)}] (solver stacked) {};
          \node[draw,no_transform,fill=bg,rounded corners,inner sep=0,fit={(0.06,0.06)(11.06,4.06)}] {};
          \node[draw,no_transform, fill=bg,rounded corners,inner sep=0,fit={(0,0)(11,4)}] (solver) {};
          \node[anchor=north west] at (solver.north west) {Solver 1\dots{}n};
          \rec{minimum height=1cm}{at (1.2,2.1)}{heuristic}{\shortstack{Decision\\Heuristic}}
          \rec{minimum height=1cm}{at (3.5,3.2)}{resolution}{\shortstack{Conflict\\Resolution}}
          \node[draw] at (3.5,1) (assignment){\shortstack{Assignment\\Atoms/Bodies}};
          \node[draw] at (7,3.2) (recorded nogoods){Recorded Nogoods};
          \begin{scope}[prop]
          \node[draw,no_transform, rounded corners,inner sep=0,fit={(5.5,0.3)(10.7,2.3)}] (propagation) {};
          \node[anchor=north west] at (propagation.north west) {Propagation};
          \rec{minimum height=1cm}{at (6.7,1.15)}{unit propagation}{\shortstack{Unit\\Propagation}}
          \begin{scope}[pc]\rec{minimum height=1cm}{at (9.5,1.21)}{post propagation stacked}{\shortstack{Post\\Propagation}}\end{scope}
          \rec{minimum height=1cm}{at (9.44,1.15)}{post propagation}{\shortstack{Post\\Propagation}}
          \end{scope}
          \draw [<-] (heuristic.north) |- (resolution.west);
          \draw [->] (heuristic.south) |- (assignment.west);
          \draw
        %(heuristic.north) edge [<-] (resolution.west)
        %(heuristic.south) edge [->] (assignment.west)
        (assignment) edge [->] (resolution)
        (resolution) edge [<->] (recorded nogoods)
        (recorded nogoods) edge [<->] (recorded nogoods |- propagation.north)
        (assignment) edge [<->] (propagation.west |- assignment)
        (unit propagation) edge [<->] (post propagation)
        ;
        \end{scope}
        \begin{scope}[co]
          % Coordination
          \coordinate (co) at (0,4.65);
          %\uncover<0>{
          \node[draw,rounded corners,no_transform, fit={(co)($(co)+(11.12,4.75)$)},inner sep=0] (coordination) {};
          \node[anchor=north west] at (coordination.north west) {Coordination};
          %}
          % Shared context
          \begin{scope}[sc]
            \coordinate (sc) at ($(co)+(0.2,0.2)$);
            \node[draw,no_transform, fill=bg,fit={(sc)($(sc)+(4,4)$)},inner sep=0] (shared context) {};
            \node[anchor=north west] at (shared context.north west) {SharedContext};
            \node[draw] at ($(sc)+(2,3)$) (vars) {\shortstack{Propositional\\Variables}};
            \node[draw] at ($(sc)+(1,1.9)$) (atoms) {Atoms};
            \node[draw] at ($(sc)+(3,1.9)$) (bodies) {Bodies};

            \node[draw] at ($(sc)+(2,1.2)$) {Static Nogoods};
            \node[draw] at ($(sc)+(2,0.5)$) (graph) {Short Nogoods};
            \draw
              (vars) edge[->] (atoms) edge[->] (bodies)
              (atoms) edge[<->] (bodies)
              ;
          \end{scope}
          % Parallel solve
          \begin{scope}[parallel]
            \coordinate (ps) at ($(co)+(6.92,0.2)$);
            \begin{scope}[pcontext]
            \node[draw,no_transform, fill=bg,fit={(ps)($(ps)+(4,4)$)},inner sep=0] (parsolve) {};
            \node[anchor=north west] at (parsolve.north west) {ParallelContext};

            \node[parlabel] at ($(ps)+(0.95,3.2)$) (threads) {Threads};
            \draw ($(threads.south east) + (0.1,0)$) rectangle +(0.5,0.5) rectangle +(1,0) rectangle +(1.5,0.5) rectangle +(2,0);
            \node at ($(threads.south east) + (0.35,0.25)$) {S$_1$};
            \node at ($(threads.south east) + (0.85,0.25)$) {S$_2$};
            \node at ($(threads.south east) + (1.4,0.25)$) {\dots};
            \node at ($(threads.south east) + (1.85,0.25)$) {S$_n$};

            \node[parlabel] at ($(ps)+(0.95,2.5)$) (counter) {Counter};
            \draw ($(counter.south east) + (0.1,0)$) rectangle +(0.5,0.5) rectangle +(1,0) rectangle +(1.5,0.5) rectangle +(2,0);
            \node at ($(counter.south east) + (0.35,0.25)$) {T};
            \node at ($(counter.south east) + (0.85,0.25)$) {W};
            \node at ($(counter.south east) + (1.4,0.25)$) {\dots};
            \node at ($(counter.south east) + (1.85,0.25)$) {S};

            \node[parlabel] at ($(ps)+(0.95,1.8)$) (queue) {Queue};
            \draw ($(queue.south east) + (0.1,0)$) rectangle +(0.5,0.5) rectangle +(1,0) rectangle +(1.5,0.5) rectangle +(2,0);
            \node at ($(queue.south east) + (0.35,0.25)$) {P$_1$};
            \node at ($(queue.south east) + (0.85,0.25)$) {P$_2$};
            \node at ($(queue.south east) + (1.4,0.25)$) {\dots};
            \node at ($(queue.south east) + (1.85,0.25)$) {P$_n$};

            \node [draw,no_transform, fit={($(ps)+(0.6,0.3)$)($(ps)+(3.4,1.2)$)}] (shared nogoods) {};
            \node at ($(ps)+(2,1)$) {Shared Nogoods};
            % tikz also has a for loop and a draw grid :)
            \draw
              ($(ps)+(0.8,0.3)$)
                rectangle +(0.2,0.2) rectangle +(0.4,0.0) rectangle +(0.6,0.2) rectangle +(0.8,0.0)
                rectangle +(1.0,0.2) rectangle +(1.2,0.0) rectangle +(1.4,0.2) rectangle +(1.6,0.0)
                rectangle +(1.8,0.2) rectangle +(2.0,0.0) rectangle +(2.2,0.2) rectangle +(2.4,0.0)
                rectangle +(2.2,0.4) rectangle +(2.0,0.2) rectangle +(1.8,0.4) rectangle +(1.6,0.2)
                rectangle +(1.4,0.4) rectangle +(1.2,0.2) rectangle +(1.0,0.4) rectangle +(0.8,0.2)
                rectangle +(0.6,0.4) rectangle +(0.4,0.2) rectangle +(0.2,0.4) rectangle +(0.0,0.2)
              ;
            \end{scope}
            \end{scope}
          % In-between
          \begin{scope}[enum]\rec{minimum height=1cm}{at ($(coordination |- threads)$)}{enum}{Enumerator}\end{scope}
          \begin{scope}[parallel]
            \rec{minimum height=1cm}{at ($(coordination |- graph)$)}{distributor}{\shortstack{Nogood\\Distributor}}
            \draw
              (enum) edge[<->] (threads)
              (distributor.east) edge[->] (shared nogoods.west |- graph)
              (distributor) edge[->] (graph)
              ($(distributor.south)+(0.4,0)$) edge[<-] ($(distributor |- recorded nogoods.north)+(0.4,0)$)
              ;
          \end{scope}
        \end{scope} % co
        \draw
          (post propagation stacked) edge [parallel, <->] (post propagation stacked |- parsolve.south)
          (shared context.south) edge[->] (shared context |- solver stacked.north)
          ;
        \begin{scope}[pp]
          % Logic program
          \coordinate (in) at ($(-2.05,3.5)$);
          \node[pp,draw=none,minimum height=1.3cm] at (in) (lp) { \shortstack{Logic\\Program} }; \funnywave{pp}{lp}

          % Preprocessing
          \coordinate (pp) at ($(-4.7,7.7)$);
          \node[draw,rounded corners,no_transform, fill=bg,fit={($(pp)+(1.25,-2.5)$)($(pp)+(4.1,1)$)},inner sep=0] (preprocessing) {};
          \node[anchor=north west] at (preprocessing.north west) {Preprocessing};
%          \rec{draw,minimum height=1cm}{at ($(pp)+(2.65,-0.15)$)}{prg builder}{\shortstack{Program\\Builder}}
%          \rec{draw,minimum height=1cm}{at ($(pp)+(2.71,-1.74)$)}{preprocessor stacked}{Preprocessor}
%          \rec{draw,minimum height=1cm}{at ($(pp)+(2.65,-1.8)$)}{preprocessor}{Preprocessor}
          \rec{draw,minimum height=1cm}{at ($(pp)+(2.65,-1.8)$)}{prg builder}{\shortstack{Program\\Builder}}
          \rec{draw,minimum height=1cm}{at ($(pp)+(2.71,-0.15)$)}{preprocessor stacked}{Preprocessor}
          \rec{draw,minimum height=1cm}{at ($(pp)+(2.65,-0.21)$)}{preprocessor}{Preprocessor}

          \draw
            (lp) edge[->] (lp |- preprocessing.south)
%            (prg builder) edge[<->] (preprocessor stacked)
            (prg builder) edge[<->] (preprocessor)
            ;
        \end{scope}
        \draw
          (preprocessing) edge[->] (shared context.west |- preprocessing)
          ;
%       \draw[help lines] (0,0) grid (11,12);

    \end{tikzpicture}
  \end{center}
%%% Local Variables: 
%%% mode: latex
%%% TeX-master: "slides"
%%% End: 

\end{frame}
% ----------------------------------------------------------------------
\begin{frame}[fragile,shrink=1]{The \textit{dot} propagator}
\begin{lstlisting}[linebackgroundcolor={%
      \btLstHL<1>{26-28}%
      \btLstHL<2>{7-10}%
      \btLstHL<3>{12-17}%
      \btLstHL<4>{19-23}%
    }]
#script (python)

import sys
import time

class Propagator:
    def init(self, init):
        self.sleep = .1
        for atom in init.symbolic_atoms:
            init.add_watch(atom.literal)

    def propagate(self, ctl, changes):
        for l in changes:
            sys.stdout.write(".")
            sys.stdout.flush()
            time.sleep(self.sleep)
        return True

    def undo(self, solver_id, assign, undo):
        for l in undo:
            sys.stdout.write("\b \b")
            sys.stdout.flush()
            time.sleep(self.sleep)

def main(prg):
    prg.register_propagator(Propagator())
    prg.ground([("base", [])])
    prg.solve()
    sys.stdout.write("\n")

#end.
\end{lstlisting}
\end{frame}
% ----------------------------------------------------------------------
\begin{frame}{Our propagators}
  \smallskip
  \begin{itemize}
  \item \clingo[DL]
    \begin{itemize}
    \item Difference constraints over reals and integers
    \item Stateless and stateful propagator (Python, C++)
    \item Design space exploration 
    \end{itemize}
    \smallskip
  \item \clingo[LP]
    \begin{itemize}
    \item Linear constraints over reals and integers
    \item Interface to \textit{cplex}, \textit{lpsolve} (Python)
    \item Hybrid metabolic network completion
    \end{itemize}
    \smallskip
  \item \clingcon-3
    \begin{itemize}
    \item Linear constraints over integers
    \item Stateful propagator (C++)
    \item Planning and scheduling
    \end{itemize}
  \item See paper for empirical analysis (incl.\ \dingo, \mingo, \ezcsp, \ezsmt)
  \end{itemize}
\end{frame}
% ----------------------------------------------------------------------
\section{Multi-Shot Theory Solving}
% ----------------------------------------------------------------------
\begin{frame}[fragile,shrink=20]{Spoiled Yale Shooting Problem}
\begin{lstlisting}[linebackgroundcolor={%
      \btLstHL<1>{1-1}%
      \btLstHL<2>{3-9}%
      \btLstHL<3>{11-12}%
      \btLstHL<4>{14-23}%
    }]
#include "incmode_lc.lp".

#program base.

  action(load).       action(shoot).      action(wait).
duration(load,25).  duration(shoot,5).  duration(wait,36).

&sum {    at(0) } = 0.                  unloaded(0).
&sum { armed(0) } = 0.

#program step(n).
[...]

#program check(n).

:- not dead(n), query(n).
:- not &sum { at(n) } <= 100, query(n).
:- do(shoot,n), not &sum { at(n) } > 35.
\end{lstlisting}
\end{frame}
% ----------------------------------------------------------------------
\begin{frame}[fragile,shrink=20]{Spoiled Yale Shooting Problem}
\begin{lstlisting}[linebackgroundcolor={%
      \btLstHL<1>{1-1}%
      \btLstHL<2>{3-4}%
      \btLstHL<3>{6-8}%
      \btLstHL<4>{10-12}%
      \btLstHL<5>{13-16}%
    }]
#program step(n).

 { do(X,n) : action(X) } = 1.
&sum { at(n); -at(n-1) } = D :- do(X,n), duration(X,D).

loaded(n)   :-   loaded(n-1), not unloaded(n).
unloaded(n) :- unloaded(n-1), not   loaded(n).
dead(n)     :-     dead(n-1).

&sum { armed(n)              } = 0 :- unloaded(n-1).
&sum { armed(n); -armed(n-1) } = D :- do(X,n), duration(X,D),
                                      loaded(n-1).
loaded(n)   :- do(load,n).
unloaded(n) :- do(shoot,n).
dead(n)     :- do(shoot,n), &sum { armed(n) } <= 35.
            :- do(shoot,n), unloaded(n-1).
\end{lstlisting}
\end{frame}
% ----------------------------------------------------------------------
\section{Summary}
% ----------------------------------------------------------------------
\begin{frame}{Summary}
  \begin{itemize}
  \item  ASP with Linear Constraints
    \smallskip
    \begin{itemize}
    \item Theory language via ASP-oriented grammars
    \item Theory reasoning via propagators (or translation)
    \item Theory APIs (Lua, Python, C++)
    \item Multi-Shot Solving
    \end{itemize}
    \medskip
  \item <2-> \clingo~5 \ = \ ASP $+$ \alert<3-4>{Control} $+$ \alert<3-4>{Theories}
    \smallskip
    \begin{itemize}\normalsize
    \item <2-> \alert<4>{Control} and \alert<4>{Theories} are \alert<4>{door openers} for ASP\par as regards embedded and complex applications
      \smallskip
    \item <2-> Open infrastructure for customized hybridization of ASP
    \end{itemize}
    \bigskip
  \item <3-> Have fun at\ \alert{\url{potassco.org}}~!
  \end{itemize}
\end{frame}
\end{document}

%%% Local Variables: 
%%% mode: latex
%%% TeX-master: t
%%% TeX-PDF-mode: t 
%%% End: 
