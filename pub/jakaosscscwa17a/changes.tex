
\section{Changes in view of reviewers comments}
\label{sec:changes}

We are grateful for the helpful comments by the reviewers.
We describe below how we addressed their comments

\begin{itemize}
\item 
We added a paragraph on the usage of strict/non-strict and external/defined lc-atoms, as raised by reviewers 1 and 3,
at the end of Section~\ref{sec:background}:

\begin{quote}
Finally, let us comment on the usability of both types of lc-atoms.
%
Their \lcatype{external}{strict} interpretation allows for deriving information from the respective theory.
This generates some overhead since the corresponding propagators have to deal with
two relations between lc-atoms and their associated constraints.
%
This approach is advantageous in our planning example in Section~\ref{sec:multishot}, 
where \lcatype{external}{strict} lc-atoms allow us 
to naturally express goal conditions as integrity constraints.
Conversely, we face the following difficulties.
First, defined lc-atoms must also occur in some rule head, which is rarely the case with goal conditions.
Second, non-strict lc-atoms may be false although the actual constraint is satisfied.
%
On the other hand, 
in the \lcatype{defined}{non-strict} setting,
the stable model semantics delineates the effective set of constraints that needs to be satisfied.
False lc-atoms are considered as unknown and can therefore be disregarded by the corresponding propagators.
We draw on this in our scheduling encodings where it halves the number of constraints and helps with faster propagation via the program's completion.
The impact of this is investigated in Section~\ref{sec:experiments}.
%
As a rule of thumb,
the choice between both settings depends on who should be in charge of delineating the set of constraints in focus.
If this is the theory propagator, an \lcatype{external}{strict} setting is preferable,
since the strict correspondence induces the relevant lc-atoms without any interference with derivable lc-atoms.
If this is the actual ASP system, a \lcatype{defined}{non-strict} setting is favorable,
in which derivable lc-atoms delineate the set of constraints checked by the constraint propagator.
\end{quote}
\item Reviewer 1 raised the question \texttt{why clingcon could only use certain solvers}.

This is a misunderstanding. 
Indeed, \clingcon{} series 1 and 2 comprise the off-the-shelf CP solver \gecode.
Unlike this, \clingcon{} 3 relies on constraint propagators implemented by us.

We tried to make this more explicit by rewriting the introductory sentence to clingcon in Section~\ref{sec:system}:
\begin{quote}
\textbf{\clingcon} series~3 offers a \clingo-based ASP system with handcrafted propagators for constraints over integers~\cite{bakaossc16a}.  
\end{quote}

\item Reviewer 1 asked for {\tt more explanation of the benchmark programs and expe\-riments}.

Although due to lacking space,
we refrained from adding the actual encodings to the paper,
we make all encodings and instances available online at
\url{https://potassco.org/labs/{clingoDL,clingoLP}}.

\item Reviewer 2 remarks that
{\tt from a KR perspective, it would be better to avoid a special treatment of fluent dead} \
to formalize that {\tt "the turkey stays dead"} on page 7.

We agree that this would be a more uniform encoding.

However, we would like to leave it as is, since 
(1) fluent \textit{dead} persists once satisfied, and
(2) it avoids the introduction of another fluent \textit{alive}
(and thus keeps things smaller).
\item 
We addressed the specific suggestions by Reviewer 2 and changed the indicated phrases accordingly. 
\end{itemize}
%%% Local Variables:
%%% mode: latex
%%% TeX-master: "paper"
%%% End:
