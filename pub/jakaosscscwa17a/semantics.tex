
In what follows,
we elaborate upon the formal relationships among the different types of lc-atoms.
To this end,
we distinguish four homogeneous settings, in which all lc-atoms are either
\lcatype{defined}{strict},
\lcatype{defined}{non-strict},
\lcatype{external}{strict}, or
\lcatype{external}{non-strict}, respectively.
We use the following notation.
For an lc-program $P$ over $\mathcal{A}\cup\mathcal{L}$ and an lc-solution $\mathcal{S}\subseteq\mathcal{L}$,
we define \PS{P}{S} as the extension of program $P$ given in (\ref{eq:stable:strict}) and (\ref{eq:stable:non-strict}).
Also,
we let \smset{P} denote the set of (regular) stable models of program $P$ over $\mathcal{A}\cup\mathcal{L}$,
and
\(
\smsetlc{P}=
\bigcup_{\mathcal{S}\subseteq\mathcal{L}\text{ lc-solution}}
\smset{\PS{P}{S}}
\)
its set of lc-stable models.
Note that the respective semantic setting is determined by the type of lc-atoms in $\mathcal{L}$.
In fact, two syntactically equivalent lc-programs may yield different lc-models in different settings.
%
This is made precise in the following proposition.
% --------------------------------------------------------------------------------
\begin{theorem}\label{thm:settings}
  Let $P$ be an lc-program over $\mathcal{A}\cup\mathcal{L}$
  and $P'$   an lc-program over $\mathcal{A}\cup\mathcal{L}'$
  such that $P=P'$.
  \begin{enumerate}
  \item \label{th:dsr} % X_d subseteq X_r
    If
    \(
    \mathcal{L}=\mathcal{L}\cap\head{P}
    \),
    then
    \(
    \smsetlc{P}\subseteq\smset{P}
    \)
  \item \label{th:rsen} % X_r subseteq X_en
    If
    \(
    \mathcal{L}=\mathcal{L}^\rightarrow\setminus\head{P}
    \),
    then
    \(
    \smset{P}\subseteq\smsetlc{P}
    \)
  \item \label{th:ssn} % X_s subseteq X_n
    If
    \(
    \mathcal{L}'=\mathcal{L}'^\rightarrow
    \),
    then
    \(
    \smsetlc{P}\subseteq\smsetlc{P'}
    \)
  \end{enumerate}
\end{theorem}
% --------------------------------------------------------------------------------
%
Note that $P=P'$ also makes $\mathcal{L}$ and $\mathcal{L}'$ syntactically equivalent,
although they may represent different types of lc-atoms.
%
The above results draw on the observation that if all atoms in $\mathcal{L}'$ are non-strict, then
\(
\{\mathcal{S}\subseteq\mathcal{L} \mid \mathcal{S}\text{ is an lc-solution}\}
\subseteq
\{\mathcal{S}\subseteq\mathcal{L}'^\rightarrow \mid \mathcal{S}\text{ is an lc-solution}\}
\).
This is because the former set of lc-solutions need to satisfy at least condition~(i)
while the latter must only satisfy~(i).
%
Note that Proposition~\ref{thm:settings} does not just apply to \ASPm{lc} but to ASP modulo arbitrary theories.

In more detail,
Proposition~\ref{thm:settings}.\ref{th:dsr} expresses that each      lc-stable model is also a  regular stable model
in a setting involving defined lc-atoms only.
%
Conversely,
Proposition~\ref{thm:settings}.\ref{th:rsen} expresses that each regular stable model is also an      lc-stable model
in the \lcatype{external}{non-strict} setting.
%
Proposition~\ref{thm:settings}.\ref{th:ssn} portrays that handling lc-atoms in a strict or non-strict way
may lead to fewer (or equal) lc-stable models than treating them just in a non-strict way.

In contrast to the observations of Proposition~\ref{thm:settings},
the following proposition tells us that regular and lc-stable models are in general incomparable in the \lcatype{external}{strict} setting.
%
\begin{theorem}\label{thm:settings2} % X_es not subseteq X_r   and   X_r not subseteq X_es
    There exist lc-programs $P$ over $\mathcal{A}\cup\mathcal{L}$ with
    \(
    \mathcal{L}=\mathcal{L}^\leftrightarrow\setminus\head{P}
    \),
    so that
    \(
    \smset{P}\not\subseteq\smsetlc{P}
    \)
    or
    \(
    \smsetlc{P}\not\subseteq\smset{P}
    \).
\end{theorem}
% --------------------------------------------------------------------------------
This results from the fact that
the treatment of strict lc-atoms may prune regular stable models
and, on the other hand, the pure external evaluation of lc-atoms may induce additional stable models.

Now that we have explored the formal correspondence among the alternative settings,
let us discuss their appropriateness for \ASPm{lc}.
To this end, let us consider two examples.

We first asses the two defined settings. Modifying our above example, let $P_1$ be
\begin{lstlisting}[numbers=none,mathescape]
{a("1.5")}.  &sum{"1.5"*x}<=7 :- a("1.5").  &sum{x}<"4.5".
\end{lstlisting}
This logic program has two regular stable models
$X_1=\{$\lstinline@ &sum{x}<"4.5" @$\}$ and
$X_2=\{$\lstinline@ a("1.5"),@ \lstinline@&sum{"1.5"*x}<=7, &sum{x}<"4.5" @$\}$.

Let us first consider the \lcatype{defined}{strict} case,
in which the lc-atoms \lstinline@&sum{"1.5"*x}<=7@ and \lstinline@&sum{x}<"4.5"@
belong to $\mathcal{L}^\leftrightarrow\cap\head{P}$.
Then, $\mathcal{S}_a=\emptyset$ is an lc-solution, since both $1.5*x\leq7$ and $x<4.5$ can be falsified.
However,
the resulting program $\PS{P_1}{\mathcal{S}_\mathit{a}}$ contains rules
`\lstinline[mathescape]@&sum{x}<"4.5".@' and `\lstinline[mathescape]@:- &sum{x}<"4.5".@'
and thus admits no regular stable model.
%
The same result is obtained for
$\mathcal{S}_b=\{$\lstinline@ &sum{"1.5"*x}<=7 @$\}$.
Unlike this,
$\mathcal{S}_c=\{$\lstinline@ &sum{x}<"4.5" @$\}$
is no lc-solution
although it appears to support $X_1$ as an lc-model.
In a strict setting, an \textit{iff} correspondence is imposed between lc-atoms and their associated linear constraints.
This excludes $\mathcal{S}_c$ as an lc-solution,
since there is no real-valued assignment satisfying $x<4.5$ while falsifying $1.5*x\leq7$.
%
This situation is caused by the non-derivability of lc-atom \lstinline@&sum{"1.5"*x}<=7@,
which is in turn falsified by the stable models semantics.
The strict interpretation of the lc-atom then requires the falsification of $1.5*x\leq7$.
%
Finally, $\mathcal{S}_d=\{$\lstinline@ &sum{x}<"4.5", &sum{"1.5"*x}<=7 @$\}$ is another lc-solution.
Given that
\(
\PS{P_1}{\mathcal{S}_\mathit{d}}=P_1\;\cup\;$\lstinline[mathescape]@$\{$ :- not &sum{"1.5"*x}<=7.  :- not &sum{x}<"4.5". @$\}
\)
has the regular stable model $X_2$, we establish that $X_2$ is the only lc-stable model of $P_1$.

This example illustrates that strict lc-atoms impose a rather strong connection to their associated constraints in a defined setting.
%
Hence, let us consider next the above example in a \lcatype{defined}{non-strict} setting,
requiring merely an \textit{only if} condition between constraints and their lc-atoms.
Now, $\mathcal{S}_c=\{$\lstinline@ &sum{x}<"4.5" @$\}$ is an lc-solution since $1.5*x\leq7$
must not be falsified.
Accordingly, the regular stable model $X_1$ of
\(
\PS{P_1}{\mathcal{S}_\mathit{c}}=P_1\;\cup\;$\lstinline[mathescape]@$\{$ :- &sum{"1.5"*x}<=7. @$\}
\)
attests that $X_1$ is also an lc-stable model of $P_1$.
The other lc-solutions yield the same results as above.

Next, let us analyze the two external settings.
For this,
let the lc-program $P_2$ be
\begin{lstlisting}[numbers=none,mathescape]
:- not &sum{x}<"4.5".  a("1.5") :- &sum{"1.5"*x}<=7.
\end{lstlisting}
admitting no regular stable models, due to the included integrity constraint.

First, we examine the \lcatype{external}{non-strict} setting.
In this case,
each combination of the lc-atoms \lstinline@&sum{"1.5"*x}<=7@ and \lstinline@&sum{x}<"4.5"@
in $\mathcal{L}^\rightarrow\setminus\head{P}$ results in an lc-solution.
However,
the existence of lc-stable models depends upon the presence of lc-atom \lstinline@&sum{x}<"4.5"@.
Lc-models are obtained if it is included, otherwise the integrity constraint in $P_2$ denies them.
%
The lc-solution $\mathcal{S}_a=\{$\lstinline@ &sum{x}<"4.5" @$\}$ results in the identical lc-stable model.
Note that all underlying assignments must satisfy $x<4.5$ and hence $1.5*x\leq7$.
However, the non-strict nature of \lstinline@&sum{"1.5"*x}<=7@ leaves its truth value open.
Thus, stable model semantics may set it to false and \lstinline@a("1.5")@ is not obtained
although the actual constraint $1.5*x\leq7$ in the rule body in $P_2$ is satisfied.
%
Similarly,
the lc-solution $\mathcal{S}_b=\{$\lstinline@ &sum{x}<"4.5", &sum{"1.5"*x}<=7 @$\}$
induces the same counter-intuitive lc-model $\{$\lstinline@ &sum{x}<"4.5" @$\}$
along with a second, arguably more intuitive lc-model
\(
\{$\lstinline[mathescape]@ a("1.5"), &sum{"1.5"*x}<=7, &sum{x}<"4.5" @$\}
\).

The previous discussion has revealed that non-strict lc-atoms may ignore information induced by the theory in an external setting.
This lack is compensated in an \lcatype{external}{strict} setting by the above condition (ii)
and the resulting assertion of lc-atoms representing satisfied constraints in (\ref{eq:stable:strict}).
Accordingly,
\(
\{$\lstinline[mathescape]@ a("1.5"), &sum{"1.5"*x}<=7, &sum{x}<"4.5" @$\}
\)
is the only lc-stable model of $P_2$.
By interpreting both lc-atoms in a strict manner,
the inclusion of \lstinline[mathescape]@&sum{x}<"4.5"@ entails that of
\lstinline@&sum{"1.5"*x}<=7@ as well.
Hence, the singleton $\{$\lstinline@ &sum{x}<"4.5" @$\}$ cannot be an lc-model of $P_2$
in a \lcatype{external}{strict} setting.

The previous discussion has shown that certain semantic combinations are more appropriate for treating linear constraints than others.
This may be different for other theories.
We have seen that a \lcatype{defined}{strict} interpretation of lc-atoms may be overly strong,
since the non-derivability of lc-atoms may severely restrict real-valued assignments.
%
Conversely, the \lcatype{external}{non-strict} treatment of lc-atoms may be too weak,
since it admits real-valued variable assignments satisfying constraints that are not
reflected in the corresponding lc-stable models.
%
As a consequence, we focus in what follows on the \lcatype{external}{strict} and \lcatype{defined}{non-strict} settings for lc-atoms.

Finally, let us comment on the usability of both types of lc-atoms.
%
Their \lcatype{external}{strict} interpretation allows for deriving information from the respective theory.
This generates some overhead since the corresponding propagators have to deal with
two relations between lc-atoms and their associated constraints.
%
This approach is advantageous in our planning example in Section~\ref{sec:multishot}, 
where \lcatype{external}{strict} lc-atoms allow us 
to naturally express goal conditions as integrity constraints.
Conversely, we face the following difficulties.
First, defined lc-atoms must also occur in some rule head, which is rarely the case with goal conditions.
Second, non-strict lc-atoms may be false although the actual constraint is satisfied.
%
On the other hand, 
in the \lcatype{defined}{non-strict} setting,
the stable model semantics delineates the effective set of constraints that needs to be satisfied.
False lc-atoms are considered as unknown and can therefore be disregarded by the corresponding propagators.
We draw on this in our scheduling encodings where it halves the number of constraints and helps with faster propagation via the program's completion.
The impact of this is investigated in Section~\ref{sec:experiments}.
%
As a rule of thumb,
the choice between both settings depends on who should be in charge of delineating the set of constraints in focus.
If this is the theory propagator, an \lcatype{external}{strict} setting is preferable,
since the strict correspondence induces the relevant lc-atoms without any interference with derivable lc-atoms.
If this is the actual ASP system, a \lcatype{defined}{non-strict} setting is favorable,
in which derivable lc-atoms delineate the set of constraints checked by the constraint propagator.
%
%%% Local Variables:
%%% mode: latex
%%% TeX-master: "paper"
%%% End:
