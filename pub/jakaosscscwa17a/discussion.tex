\section{Summary}\label{sec:summary}

We presented several truly hybrid ASP systems incorporating difference and linear constraints.
Previous approaches addressed this by resorting to translations into foreign solving paradigms like MILP or SMT.
This difference is analogous to the one between genuine ASP solvers like \clasp{} and \wasp{}
and earlier ones like \assat\ and \cmodels\ that translate ASP to SAT.
%
The resulting systems \clingod{dl} and \clingod{lp} comprise several complementary aspects.
For instance, \clingod{dl} relies upon customized propagators, one variant using a \python{} API, the other a \cpp{} API.
This is similar to the approach of \inca{} and \clingcon~3 for Constraint ASP.
Unlike this, \clingod{lp} builds upon the \python{} API to incorporate off-the-shelf LP solvers for propagation, optionally \cplex{} or \lpsolve.
This is similar to the approach of \dlvhex[\textsc{cp}] and \clingcon~2 integrating \gecode{} for constraint processing.
Both \clingod{dl} and \clingod{lp} allow for dealing with integer as well as real variables.
The former admits two, the latter an arbitrary number of such variables per linear constraint.
%
This is complemented by \clingcon~3 adding constraint processing to \clingo{} by using a low level API.

We accomplished this by instantiating the generic framework of ASP modulo theories described in~\cite{gekakaosscwa16a}.
%
We defined lc-stable models and elaborated upon different types of lc-atoms,
ultimately settling on the combinations \lcatype{defined}{non-strict} and \lcatype{external}{strict}
for \clingod{dl} and \clingod{lp}.%
\footnote{This is our recommendation in view of our analysis in Section~\ref{sec:background};
  both systems actually support all four combinations of strict/non-strict and defined/external lc-atoms.}
%
Our underlying formal analysis on the interaction of strict- and definedness has actually a much broader impact
given that other systems follow similar principles.
Although we lack a deeper analysis,
\inca{} and \dlvhex[\textsc{cp}] appear to adhere to an \lcatype{external}{strict} treatment of constraint atoms,
just as our previous systems \clingcon, \dingo, and \mingo, 
while \ezsmt{} and \ezcsp{} seem to follow an \lcatype{external}{non-strict} approach.
Moreover, the results in Proposition~\ref{thm:settings} are of a general nature and apply well beyond ASP systems dealing with linear constraints.

We provided a conceptual and empirical comparison of \clingod{dl} and \clingod{lp} with related systems 
for dealing with different forms of linear constraints in ASP.
%
Our experiments focused on,
first, examining different types of lc-atoms and APIs for both \clingo{} derivatives,
and, second, comparing them with related systems.
In the first case, \clingod{dl} using \lcatype{defined}{non-strict} lc-atoms along with the \cpp\ API 
yields the best results,
and in the second one, 
the aforementioned \clingod{dl} configuration outperforms the other systems for the set of benchmarks only involving difference constraints,
and \clingcon\ has an edge over all other systems regarding the set of benchmarks featuring arbitrary (integer-based) linear constraints.

Finally, % Last but not least,
we showed how easily our machinery can be applied to online reasoning scenarios
by using \clingo's multi-shot and theory reasoning capabilities in tandem.
% and illustrated this by encoding the spoiled Yale shooting.

%%% Local Variables: 
%%% mode: latex
%%% TeX-master: "paper"
%%% End: 
