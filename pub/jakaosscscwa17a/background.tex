
\section{Answer Set Programming with Linear Constraints}\label{sec:background}

Our paper centers upon the theory reasoning capabilities of \clingo{} that allow us to extend ASP with linear constraints,
also referred to as \ASPm{lc}.
We focus below on the corresponding syntactic and semantic features, 
and refer the reader to the literature for an introduction to the basics of ASP. 

We consider (disjunctive) \emph{logic programs with linear constraints}, for short%
\footnote{We keep using the prefix `\emph{lc-}' throughout as a shorthand for concepts related to linear constraints.}
\emph{lc-programs},
over sets $\mathcal{A}$ and $\mathcal{L}$ of ground \emph{regular} and \emph{linear constraint atoms},
respectively.
%
An expression is said to be \emph{ground}, if it contains no ASP variables.
%
Accordingly, such programs consist of \emph{rules} of the form
\begin{lstlisting}[mathescape,numbers=none]
   a$_1$;...;a$_m$ :- a$_{m+1}$,...,a$_n$,not a$_{n+1}$,...,not a$_o$
\end{lstlisting}
where each \lstinline[mathescape]{a$_i$} is either 
a {regular atom} in $\mathcal{A}$ of form \lstinline[mathescape]{p(t$_1$,...,t$_k$)}
such that all \lstinline[mathescape]{t$_i$} are ground terms
or 
an {lc-atom} in $\mathcal{L}$ of form%
\footnote{In \clingo, theory atoms are preceded by `\texttt{\&}'.}
`\lstinline[mathescape]@&sum{a$_1$*x$_1$;$\dots$;a$_l$*x$_l$}<=k@' 
that stands for the linear constraint
\(
a_1\cdot x_1+\dots+a_l\cdot x_l\leq k
\).
All \lstinline[mathescape]{a$_i$} and \lstinline[mathescape]{k} are finite sequences of digits with at most one dot%
\footnote{In the input language of \clingo, sequences containing dots must be quoted to avoid clashes.} 
and represent real-valued coefficients $a_i$ and $k$.
Similarly all \lstinline[mathescape]{x$_i$} stand for the real (or integer) valued variables $x_i$.
%
As usual, \lstinline[mathescape]{not} denotes (default) \emph{negation}.
A rule is called a \emph{fact} if ${m,o}=1$,
\emph{normal} if $m=1$, and 
an \emph{integrity constraint} if $m=0$.
%
A linear constraint of form
\(
x_1-x_2\leq k
\)
is called a \emph{difference constraint},
and represented as
`\lstinline[mathescape]@&sum{x$_1$; -1*x$_2$}<=k@'
(or `\lstinline[mathescape]@&diff{x$_1$-x$_2$}<=k@' in pure difference logic settings). 

To ease the use of ASP in practice, 
several extensions have been developed. 
First of all, rules with ASP variables are viewed as shorthands for the set of their ground instances.
Further language constructs include
\emph{conditional literals} and \emph{cardinality constraints} \cite{siniso02a}.
The former are of the form
\lstinline[mathescape]{a:b$_1$,...,b$_m$},
the latter can be written as
\lstinline[mathescape]+s{d$_1$;...;d$_n$}t+,
where \lstinline{a} and \lstinline[mathescape]{b$_i$} are possibly default-negated (regular) literals  % for $0\leq i\leq m$,
and each \lstinline[mathescape]{d$_j$} is a conditional literal; % for $1\leq i\leq n$;
\lstinline{s} and \lstinline{t} provide optional lower and upper bounds on the number of satisfied literals in the cardinality constraint.
We refer to \lstinline[mathescape]{b$_1$,...,b$_m$} as a \emph{condition},
and call it \textit{static} if it is evaluated during grounding, otherwise it is called \textit{dynamic}.
%
The practical value of such constructs becomes apparent when used with ASP variables. 
For instance, a conditional literal like
\lstinline[mathescape]{a(X):b(X)}
in a rule's antecedent expands to the conjunction of all instances of \lstinline{a(X)} for which the corresponding instance of \lstinline{b(X)} holds.
%
Similarly,
\lstinline[mathescape]+2{a(X):b(X)}4+
is true whenever at least two and at most four instances of \lstinline{a(X)} (subject to \lstinline{b(X)}) are true.
%
% Finally, objective functions minimizing the sum of weights $w_i$ subject to condition $c_i$ are expressed as
% \lstinline[mathescape]!#minimize{$w_1$:$c_1$;$\dots$;$w_n$:$c_n$}!.
% \lstinline[mathescape]!#minimize{$w_1$@$l_1$:$c_1$;$\dots$;$w_n$@$l_n$:$c_n$}!.
% Lexicographically ordered objective functions are (optionally) distinguished via levels indicated by $l_i$.

Likewise,
\clingo's syntax of linear constraints offers several convenience features.
As above,
elements in linear constraint atoms can be conditioned (and use ASP variables),
viz.\
`\lstinline[mathescape]@&sum{a$_1$*x$_1$:c$_1$;...;a$_l$*x$_l$:c$_n$}<=k@'
where each \lstinline[mathescape]{c$_i$} is a condition.
As above, the usage of ASP variables allows for forming arbitrarily long expressions
(cf.\ Listing~\ref{encoding:yale}).
That is, by using static or dynamic conditions,
we may formulate linear constraints that are determined relative to a problem instance during grounding
and even dynamically during solving, respectively. 
Also, linear constraints can be formed with further relations, viz.\
\texttt{>=},
\texttt{<},
\texttt{>},
\texttt{=},
and
\texttt{!=}.
Moreover, the theory language for linear constraints offers a domain declaration for real variables,
`\lstinline[mathescape]@&dom{lb..ub}=x@'
expressing that all values of \texttt{x} must lie between \texttt{lb} and \texttt{ub}, inclusive.
And finally the maximization (or minimization) of an objective function can be expressed with
\lstinline[mathescape]@&maximize{a$_1$*x$_1$:c$_1$;...;a$_l$*x$_l$:c$_n$}@
(or by \texttt{minimize}).
The full theory grammar for linear constraints over reals is available at~\url{https://potassco.org/clingo/examples}.

Semantically, a logic program induces a set of \emph{stable models},
being distinguished models of the program determined by the stable models semantics~\cite{gellif91a}.
%
To extend this concept to logic programs with linear constraints,
we follow the approach of lazy theory solving~\cite{baseseti09a}.
We abstract from the specific semantics of a theory by considering the lc-atoms representing the underlying linear constraints.
The idea is that a regular stable model $X$ of a program over $\mathcal{A}\cup\mathcal{L}$ is only valid wrt the theory,
if the constraints induced by the truth assignment to the lc-atoms in $\mathcal{L}$ are satisfiable in the theory.
%%
In our setting, 
this amounts to finding an assignment of reals (or integers) to all numeric variables that 
satisfies a set of linear constraints induced by $X\cap\mathcal{L}$.
%
Although this can be done in several ways, as detailed below,
let us illustrate this by a simple example.
The (non-ground) logic program containing the fact 
`\lstinline[mathescape]{a("1.5").}'
along with the rule
`\lstinline[mathescape]@&sum{R*x}<=7 :- a(R).@'
has the regular stable model \lstinline[mathescape]@$\{$a("1.5")$,\;$&sum{"1.5"*x}<=7$\}$@.
Here, we easily find an assignment, e.g.\ $\{x\mapsto 4.2\}$,
that satisfies the only associated linear constraint `$1.5*x\leq 7$'.

In what follows, we make this precise by instantiating 
the general framework of logic programs with theories in~\cite{gekakaosscwa16a}
to the case of linear constraints over reals and integers (and so difference constraints).
Also, we focus on one theory at a time.
%
Thereby,
our emphasis lies on the elaboration of alternative semantic options for stable models with linear constraints,
which pave the way for different implementation techniques discussed in Section~\ref{sec:system}.

We use the following notation.
Given a rule $r$ as above,
we call $\{\mathtt{a}_1,\dots,\mathtt{a}_m\}$ % and $\{\mathtt{a}_{m+1},\dots,\mathtt{a}_n,\mathtt{not}\,\mathtt{a}_{n+1},\dots,\mathtt{not}\,\mathtt{a}_o\}$
its \emph{head} % and \emph{body} 
and denote it % them
by \head{r}. % and \body{r}.
Furthermore, we define
\(
\head{P}=\bigcup_{r\in P}\head{r}
\).
% we let \head{P} stand for all head atoms in program $P$.

First of all,
we may distinguish whether linear constraints are only determined outside or additionally inside a program.
%
Accordingly,
we partition $\mathcal{L}$ into
\emph{defined} and \emph{external} lc-atoms,
namely $\mathcal{L}\cap\head{P}$ and $\mathcal{L}\setminus\head{P}$, 
respectively.\footnote{This distinction is analogous to that between head and input atoms,
  defined via rules or \lstinline{#external} directives \cite{gekakasc14b}, respectively.}
%
While external lc-atoms must only be satisfied by the respective theory,
defined ones must additionally be derivable through rules in the program.
%
The second distinction is about the logical correspondence between theory atoms and theory constraints.
%
To this end,
we partition $\mathcal{L}$ into
\emph{strict} and \emph{non-strict} lc-atoms,
denoted by $\mathcal{L}^\leftrightarrow$ and $\mathcal{L}^\rightarrow$, respectively.
%
The strict correspondence requires
a linear constraint to be satisfied 
\textit{iff}
the associated lc-atom in $\mathcal{L}^\leftrightarrow$ is true.
%
A weaker condition is imposed in the non-strict case.
Here, a linear constraint must hold 
\textit{only if}
the associated lc-atom in $\mathcal{L}^\rightarrow$ is true.
%
Thus, only lc-atoms in $\mathcal{L}^\rightarrow$ assigned true impose requirements, 
while constraints associated with falsified lc-atoms in $\mathcal{L}^\rightarrow$ are free to hold or not.
%
However, by contraposition, a violated constraint leads to a false lc-atom.

Different combinations of such correspondences are possible,
and we may even treat some constraints differently than others.
In view of this, we next provide an extended definition of stable models that accommodates all above correspondences.
%
Following~\cite{gekakaosscwa16a},
we accomplish this by mapping the semantics of lc-programs back to regular stable models.
%
To this end,
we abstract from the actual constraints and identify a solution with a set of linear constraint atoms.
%
More precisely,
we call
\(
\mathcal{S}\subseteq\mathcal{L}
\)
a \emph{linear constraint solution},
if there is an assignment of reals (or integers) to all real (integer) valued variables represented in $\mathcal{L}$ that 
\begin{enumerate}
 \item[(i)]  \label{en:lcsol1} satisfies all linear constraints associated with strict and non-strict lc-atoms in $\mathcal{S}$ and
 \item[(ii)] \label{en:lcsol2} falsifies all linear constraints associated with strict lc-atoms in $\mathcal{L}^\leftrightarrow\setminus\mathcal{S}$.
\end{enumerate}

Then, we define a set $X\subseteq\mathcal{A}\cup\mathcal{L}$
as an \emph{lc-stable model} of an lc-program~$P$,
if
there is some lc-solution $\mathcal{S}\subseteq\mathcal{L}$ such that
$X$ is a (regular) stable model of the logic program
%
\newcommand{\code}[1]{\text{\texttt{#1}}}
%
\begin{align}
\hspace{-10pt}
P
&{}\cup
\{\code{a.}       \mid \code{a}\in (\mathcal{L}^\leftrightarrow\setminus\head{P})\cap \mathcal{S}\}
\cup
\{\code{:- not a.}\mid \code{a}\in (\mathcal{L}^\leftrightarrow\cap\head{P})\cap \mathcal{S}\}
\label{eq:stable:strict}
\\
 &{}\cup
\{\code{\{a\}.}   \mid \code{a}\in (\mathcal{L}^\rightarrow\setminus\head{P})\cap \mathcal{S}\}
\cup
\{\code{:- a.}    \mid \code{a}\in (\mathcal{L}\cap\head{P})\setminus \mathcal{S}\}\rlap{.}
\label{eq:stable:non-strict}
\end{align}
%
The rules added to~$P$ % in~\eqref{eq:stable:strict} and~\eqref{eq:stable:non-strict}
express conditions aligning the lc-atoms in $X\cap\mathcal{L}$ with a corresponding lc-solution~$\mathcal{S}$.
%
To begin with, 
the set of facts on the left in~\eqref{eq:stable:strict} makes sure that 
all lc-atoms in~$\mathcal{S}$ that are external and strict % , viz.\ in $\mathcal{L}^\leftrightarrow\setminus\head{P}$,
also belong to~$X$.
%
Unlike this, the corresponding set of choice rules in~\eqref{eq:stable:non-strict}
merely says that non-strict external lc-atoms from~$\mathcal{S}$ may be included in~$X$ or not.
%
The integrity constraints in~\eqref{eq:stable:strict} and~\eqref{eq:stable:non-strict}
take care of defined lc-atoms, viz.\ the ones in~$\head{P}$.
%
The set in~\eqref{eq:stable:strict} again focuses on strict lc-atoms and
stipulates that the ones from~$\mathcal{S}$ are included in~$X$ as well.
%
Finally, for both strict and non-strict defined lc-atoms,
the integrity constraints in~\eqref{eq:stable:non-strict} assert the falsity
of atoms that are not in~$\mathcal{S}$.

%%% Local Variables: 
%%% mode: latex
%%% TeX-master: "paper"
%%% End: 
