
\section{Introduction}\label{sec:introduction}

The \clingo\ system, along with its grounding and solving components \gringo\ and \clasp, % respectively,
is  nowadays among the most widely used tools for Answer Set Programming (ASP; \cite{lifschitz08b}).
This does not only apply to end-users, but more and more to system developers who build upon \clingo's
infrastructure for developing their own systems.
Among them, we find (alphabetically)
\clasp\emph{-nk}~\cite{eiererfi13a},
\clingcon~\cite{ostsch12a},
\dflat~\cite{abblchduhewo14a},
\dingo~\cite{jalini11a},
\dlvhex~\cite{eifikrre12a},
\inca~\cite{drewal12a}, and
\mingo~\cite{lijani12a}.
None of these systems can use \clingo\ or its components without workarounds or even involved modifications to realize the desired functionality.
%
Moreover, since ASP is a model, ground, and solve paradigm, such modifications are rarely limited to a single component
but often spread throughout the whole workflow.
This begins with the addition of new language constructs to the input language,
requiring in turn amendments to the grounder as well as
syntactic means for passing the ground constructs to a downstream system.
In case they are to be dealt with by an ASP solver,
it must be enabled to treat the specific input and incorporate corresponding solving capacities.
%
Finally,
each such extension is application-specific and requires different means at all ends.

We address this challenge with the new \clingo\ series~5 and its components.
%
This is accomplished by introducing generic interfaces that allow for accommodating extensions to ASP at the salient stages of its workflow.
%
To begin with,
we extend \clingo's grounder component \gringo\ with means for specifying simple theory grammars in which new theories can be represented.
As theories are expressed using constructs close to ASP's basic modeling language,
the existing grounding machinery takes care of instantiating them.
%
This also involves a new intermediate ASP format that allows for passing the enriched information from grounders to solvers in a transparent way.
(Since this format is mainly for settings with stand-alone grounders and solvers,
 and thus outside the scope of \clingo,
 we delegate details to the Appendix in \ref{sec:aspif}.) % ~\cite{gekakaosscwa16b}.)
%%
For a complement,
\clingo~5 provides several interfaces for reasoning with theory expressions.
%
On the one hand,
the existing \lua\ and \python\ APIs are extended by high-level interfaces for
augmenting propagation in \clasp\ with so-called \emph{theory propagators}.
Several such propagators can be registered with \clingo,
each implementing an interface of four basic methods.
Our design is driven by the objective to provide means for rapid prototyping of dedicated reasoning procedures while enabling effective implementations.
To this end,
the interface supports, for instance, stateful theory propagators as well as multi-threading in the underlying solver.
%
On the other hand,
the functionality of the aforementioned extended APIs is now also offered via a \C\ interface.
This is motivated by the wide availability of foreign function interfaces for \C, which enable the import of \clingo\ % functionalities into
in % many other
programming languages like \java\ or \haskell.
A first application of this is the integration of \clingo~5 into SWI-Prolog.\footnote{\url{https://github.com/JanWielemaker/clingo}}

% For brevity, we concentrate below on the extensions to the \python\ interface, whose functionality is mirrored by the corresponding \lua\ and \C\ APIs.

%%% Local Variables:
%%% mode: latex
%%% TeX-master: "paper"
%%% End:
