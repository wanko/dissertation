\section{Implementation Aspects}\com{B: feel free to remove}
\begin{itemize}
\item algorithmic framework
  \begin{itemize}
    \item solving techniques similar to SMT approach
    \item Tight integration between CDCL-based Boolean solver (clasp) and theory propagation
    \item Invoke theory propagation on partial assignments
    \begin{itemize}
      \item can extend assignment and/or backtrack immediately on (theory) conflict
      \item but also possibly many expensive but useless theory propagations
    \end{itemize}
    \item Theory propagator invoked after BCP but only if at least one relevant atom was assigned
    \item (Theory) propagation only via explicit clause (no support for lazy reasons)
    \item No support (in python) for on-demand splitting yet (i.e. adding new variables)
    \item Termination in case of deletion subject to increasing limit!
    \item Multiple theory propagators possible
    \begin{itemize}
     \item No combination (i.e. Nelson-Oppen) and/or splitting on interface equalities
     \item Propagators interact only with main solver
     \item Model only if all theory propagators agree
    \end{itemize}

  \end{itemize}
\item challenges
\end{itemize}

\blockcomment{
\begin{lstlisting}
def DPLL(T1,..,Tn)
  while True
    // unit propagation, unfounded set propagation,
    // and other "simple" PPs
    conflict = bcp()
    changed  = True
    while conflict == None and changed = True
      for each t in T1...Tn:
        while conflict == None
          clause = t.propagate(all_assigned())
          if clause == None: break
          if unsat(clause)
            conflict = clause
            break
          if asserting(clause)
            // all but one literal false
            level = assertion_level(clause)
            if level < decision_level(): backtrack(level)
            changed = true
          learn(clause)
          conflict = bcp()
    if conflict != None:
      level, clause = analyze(conflict)
      if level < 0: return UNSAT
      backtrack(level)
      learn(clause)
    elif all_assigned(): return SAT
    decide()
\end{lstlisting}
\begin{itemize}
 \item Algorithm starts with boolean constraint propagation (BCP).
       This may include unfounded set propagation, acyclicity checking, and/or
       other propagators that only extend the current decision level
 \item If BCP terminates without conflict, theory propagation considers theory propagators one by one.
 \item Each theory propagator is called in a loop until it has reached a fixpoint wrt BCP
 \item Theory propagation is repeated until fixpoint (or conflict)
\end{itemize}}

\begin{itemize}
  \item On the clasp side, more elaborate propagation mechanisms can be added via so called \emph{Post propagators}
        and theory propagators are implemented as such.
  \item clasp distinguishes two categories of post propagators:
  \begin{itemize}
    \item \textit{simple} post propagators are deterministic and either \textit{conflict-complete} or \textit{implication-complete}.
    As such, they only extend the current decision level. Furthermore, they have associated priorities and
    are propagated after unit propagation in priority-order. \\
    Unfounded-set checking is a typical example.
    \item \textit{complex} post propagators on the other hand may or may not be deterministic and may
    add implications and/or conflicts on decision levels smaller than the current.
    Furthermore, they are propagated in fifo-order and only after simple post propagators.\\
    Given that theory propagators can add arbitrary clauses via the add\_clause() function,
    they are currently always implemented as complex propagators in clasp.
  \end{itemize}
  \item Propagating a post propagator $P_i$ amounts to computing a fixpoint wrt to unit propagation and all previous
        post propagators. To this end, clasp provides a dedicated propagateUntil() function that
        first does unit propagation and then calls post propagators up to but not including $P_i$.

  \item When integrating a new clause $C$ of size $N$ during propagation, clasp distinguishes two general cases.
  \item First, if $C$ is \emph{safe} it is added without requiring further action. Clause $C$ is considered safe
        if either at most $N-2$ literals are false or if some true literal $x\in~C$
        was assigned prior to the last assigned false literal $y\in~C$.
  \item Otherwise, $C$ is conflicting or unit-resulting and is integrated on the proper decision level $X$,
        the last level on which some false literal $y\in~C$ was assigned.
  \item If $X$ is smaller than the current decision level, clasp first cancels propagation on the current decision level and backtracks to decision level $X$.
  \item After possible backtracking, $C$ now either contains exactly one unassigned literal $x\in~C$ or is conflicting.
  \item In the former case, $x$ is assigned true and propagation continues on decision level $X$.
        In the latter case, propagation is terminated and a conflict analysis on $C$ is started.

  \item Regarding the implementation of PropagateControl:
  \begin{itemize}
    \item The propagate() function calls the previously mentioned propagateUntil() function to ensure
          that we do not recursively call into the propagate() function of a user-supplied propagator.
    \item The add\_clause() function returns true only if the new clause is \emph{safe}. Otherwise,
          the function returns false and the clause is added only after we returned from the
          user-supplied propagator. This way, we avoid calling back into undo() during an active propagate()
          or check() call.
  \end{itemize}

\end{itemize}

%%% Local Variables: 
%%% mode: latex
%%% TeX-master: "paper"
%%% End: 
