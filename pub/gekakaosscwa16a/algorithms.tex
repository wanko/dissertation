
\section{Algorithmic Characterization}
\label{sec:algo}

As detailed in \cite{gekasc09c}, a ground program~$P$ induces
\emph{completion} and \emph{loop nogoods},
% formally defined
given by % as
\(
  \Delta_P = \Delta_{\body{P}} \cup
             \Delta_{\mathcal{A}\cup(\mathcal{T}\cap\head{P})}
\)
or
\(
  \Lambda_P = \mbox{$\bigcup_{\emptyset\subset U\subseteq \mathcal{A}\cup(\mathcal{T}\cap\head{P})}$}
              \{ \lambda(a,U) \mid a\in U \}
\),
respectively,
where $\body{P}$ is the set of rule bodies occurring in~$P$.
Note that both sets of nogoods are restricted by regular atoms and defined theory atoms,
while external theory atoms can a priori be assigned freely,
although any occurrences in rule bodies are subject to evaluation
via respective nogoods in~$\Delta_{\body{P}}$.
% A \emph{literal} is of the form $\Tlit{v}$ or $\Flit{v}$ for
% $v\in(\mathcal{A}\cup\mathcal{T})\cup\body{P}$.
A (partial) \emph{assignment}~$\Ass$ is a consistent set
of \emph{(signed) literals} of the form $\Tlit{v}$ or $\Flit{v}$ for
$v\in(\mathcal{A}\cup\mathcal{T})\cup\body{P}$,
i.e., $\{\Tlit{v},\Flit{v}\}\nsubseteq\Ass$
for all $v\in(\mathcal{A}\cup\mathcal{T})\cup\body{P}$;
$\Ass$ is \emph{total} if $\{\Tlit{v},\Flit{v}\}\cap\Ass\neq\emptyset$
for all $v\in(\mathcal{A}\cup\mathcal{T})\cup\body{P}$.
We say that some nogood $\delta$ is \emph{violated} by $\Ass$ if $\delta\subseteq\Ass$.
%
When $\mathbb{T}=\emptyset$, so that $\mathcal{T}=\emptyset$ as well,
each total assignment~$\Ass$ that does not violate any nogood $\delta\in\Delta_P\cup\Lambda_P$
yields a regular stable model of~$P$, and such an assignment~$\Ass$ is called a \emph{solution}
(for $\Delta_P\cup\Lambda_P$).

We now extend the concept of a solution to $\mathbb{T}$-stable models.
To this end,
we follow the idea of external propagators in \cite{drewal12a}
and identify a theory $T\in\mathbb{T}$ with a set
$\Delta_T\subseteq 2^{\{\Tlit{a}\mid a\in\mathcal{T}^T\}\cup\{\Flit{a}\mid a\in\mathcal{T}^T_e\}}$
of \emph{theory nogoods} such that,
given a total assignment $\Ass$,
we have that $\delta\subseteq\Ass$ for some $\delta\in\Delta_T$
iff
there is no $T$-solution $\mathcal{S}^T$ such that
$\{a\in\mathcal{T}^T\mid\Tlit{a}\in\nolinebreak\Ass\}\subseteq\mathcal{S}^T$
and
$\{a\in\mathcal{T}^T_e\mid\Flit{a}\in\nolinebreak\Ass\}\cap\mathcal{S}^T=\emptyset$.
%
That is, the nogoods in~$\Delta_T$ must reject $\Ass$ iff no $T$-solution
(i)
includes all theory atoms in $\mathcal{T}^T$ that are assigned to true by $\Ass$
and
(ii)
excludes all strict theory atoms in $\mathcal{T}^T_e$ assigned to false by $\Ass$.
This semantic condition establishes a (one-to-one) correspondence between
$\mathbb{T}$-stable models of~$P$ and solutions for $(\Delta_P\cup\Lambda_P)\cup\bigcup_{T\in\mathbb{T}}\Delta_T$.
%
\begin{proposition}\label{prop:sound}
Let $P$ be a % ground
program, % over $\mathcal{A}\cup\mathcal{T}$,
$\mathbb{T}$ be
a collection of theories,
% such that $\mathcal{T}\subseteq\bigcup_{T\in\mathbb{T}}\mathcal{T}^T$,
$X\subseteq\mathcal{A}\cup\mathcal{T}$, and
$B\subseteq\body{P}$ % be
the set of rule bodies that hold w.r.t.~$X$.
Then, we have that $X$ is a $\mathbb{T}$-stable model of~$P$ iff
$\Ass=\{\Tlit{v}\mid v\in X\cup B\}\cup\{\Flit{v}\mid v\in ((\mathcal{A}\cup\mathcal{T})\setminus X)\cup(\body{P}\setminus B)\}$
is a solution for
$(\Delta_P\cup\Lambda_P)\cup\bigcup_{T\in\mathbb{T}}\Delta_T$.
\end{proposition}
%
% A formal elaboration can be found in~\cite{gekakaosscwa16b}.

The nogoods in $(\Delta_P\cup\Lambda_P)\cup\bigcup_{T\in\mathbb{T}}\Delta_T$ provide the
logical fundament for the Conflict-Driven Constraint Learning (CDCL) procedure
(cf.\ \cite{malyma09a,gekasc09c})
outlined in Figure~\ref{fig:cdcl}.
While the completion nogoods in~$\Delta_P$ are usually made explicit and subject to
unit propagation,
the loop nogoods in~$\Lambda_P$ as well as theory nogoods in~$\Delta_T$ are typically
handled by dedicated propagators and particular members are selectively recorded, i.e.,
when a respective propagator identifies some nogood~$\delta$ such that
$|\delta\setminus\Ass|\leq 1$
(and $(\{\Tlit{v}\mid\Flit{v}\in\nolinebreak\delta\}\cup\linebreak[1]\{\Flit{v}\mid\Tlit{v}\in\delta\})\cap\Ass=\emptyset$), and we say that such a nogood is \emph{unit}.
In fact, a unit nogood~$\delta$ yields either a conflict, if $\delta$ is violated by $\Ass$,
or otherwise a literal to be assigned by unit propagation.

\begin{figure}[t]
  \newcounter{cddl}
  \setcounter{cddl}{8}
  \renewcommand{\thecddl}{\Alph{cddl}}
  \begin{tabbing}
   (XX)  \= XX \= XX \= XX \= \kill%
   \refstepcounter{cddl}(\thecddl)\label{fig:cdcl:init}
         \> \textit{initialize} \` // register theory propagators and initialize watches \\
         \> \textbf{loop} \\
         \> \> \textit{propagate} completion, loop, and recorded nogoods
               \` // deterministically assign literals \\
         \> \> \textbf{if} no conflict \textbf{then} \\
         \> \> \> \textbf{if} all variables assigned \textbf{then} \\
   \setcounter{cddl}{2}%
   \refstepcounter{cddl}(\thecddl)\label{fig:cdcl:check}
         \> \> \> \> \textbf{if} some $\delta\in\Delta_T$ is violated for   $T\in\mathbb{T}$ \textbf{then} record $\delta$
                     \` // theory propagators check $\Delta_T$ \\
         \> \> \> \> \textbf{else} \textbf{return} variable assignment
                   \`// $\mathbb{T}$-stable model found \\
         \> \> \> \textbf{else} \\
   \setcounter{cddl}{15}%
   \refstepcounter{cddl}(\thecddl)\label{fig:cdcl:propagate}
         \> \> \> \> \textit{propagate} theories $T\in\mathbb{T}$
                     \`// theory propagators may record theory nogoods from $\Delta_T$ \\
         \> \> \> \> \textbf{if} no nogood recorded \textbf{then} \textit{decide}
                     \`// non-deterministically assign some literal \\
         \> \> \textbf{else} \\
         \> \> \> \textbf{if} top-level conflict \textbf{then} \textbf{return} unsatisfiable \\
         \> \> \> \textbf{else} \\
         \> \> \> \> \textit{analyze} \`// resolve conflict and record a conflict constraint \\
   \setcounter{cddl}{20}%
   \refstepcounter{cddl}(\thecddl)\label{fig:cdcl:undo}
         \> \> \> \> \textit{backjump} \`// undo assignments until conflict constraint is unit
 \end{tabbing}
\caption{Basic algorithm for Conflict-Driven % Boolean
         Constraint Learning (CDCL)
         modulo theories}
\label{fig:cdcl}
\end{figure}%
%
While the dedicated propagator for loop nogoods is built-in in systems like \clingo~5,
those for theories are provided via the interface detailed in Section~\ref{sec:system}.
To utilize custom propagators,
Figure~\ref{fig:cdcl} includes an \emph{initialization} step in Line~(\ref{fig:cdcl:init}).
In addition to the ``registration'' of a propagator for a theory~$T$
as an extension of the basic CDCL procedure,
common   tasks performed in this step include setting up internal data structures and
so-called watches for (a subset of) the theory atoms in $\mathcal{T}^T$,
so that the propagator will be invoked (only) when some watched literal gets assigned.

The main CDCL loop starts with unit propagation on completion and loop nogoods,
the latter handled by the respective built-in propagator, as well as any nogoods
already recorded.
If this results in a non-total assignment without conflict,
theory propagators for which some of their watched literals have been assigned
are invoked in Line~(\ref{fig:cdcl:propagate}).
A propagator for a theory~$T$ can then inspect the current assignment,
update its data structures accordingly, and most importantly,
perform \emph{theory propagation} determining theory nogoods $\delta\in\Delta_T$ to record.
Usually, any such nogood~$\delta$ is unit in order to trigger a conflict or unit propagation,
although this is not a necessary condition.
The interplay of unit and theory propagation continues until a conflict or
total assignment arises,
or no (further) watched literals of theory propagators get assigned by unit propagation.
In the latter case, some non-deterministic decision is made to extend the partial
assignment at hand and then to proceed with unit and theory propagation.

If no conflict arises and an assignment~$\Ass$ is total,
in Line~(\ref{fig:cdcl:check}), theory propagators are called, one by one,
for a final \emph{check} of~$\Ass$.
The idea is that, e.g., a ``lazy'' propagator for a theory~$T$
that does not exhaustively test violations of its theory nogoods by partial assignments
can make sure that $\Ass$ is indeed a solution for~$\Delta_T$, or record some
violated nogood(s) from $\Delta_T$ otherwise.
Even in case theory propagation on partial assignments is exhaustive and a final
check is not needed to detect conflicts,
the information that search led to a total assignment can be useful in practice, e.g.,
to store % theory-specific information such as
values for integer variables like
\lstinline{start(1)},
\lstinline{start(2)},
\lstinline{end(1)}, and
\lstinline{end(2)} in Listing~\ref{prg:grd:diff}
that witness the existence of a $T$-solution.

Finally, in case of a conflict, i.e., some completion or recorded nogood is violated by
the current assignment,
provided that some non-deterministic decision is involved in the conflict,
a new conflict constraint is recorded and utilized to guide backjumping
in Line~(\ref{fig:cdcl:undo}), as usual with CDCL.
In a similar fashion as the assignment of watched literals serves as
trigger for theory propagation, theory propagators are informed when they become
unassigned upon backjumping.
This allows them to \emph{undo} earlier operations, e.g., internal data structures can be
reset to return   to a state taken prior to the assignment of watches.

In summary, the basic CDCL procedure is extended in four places to account for
custom propagators: initialization, propagation of (partial) assignments,
final check of total assignments, and undo steps upon backjumping.
% The next section presents
% the corresponding interface of \clingo~5.

%%% Local Variables:
%%% mode: latex
%%% TeX-master: "paper"
%%% End:
