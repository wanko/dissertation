
\section{Input Language}\label{sec:language}

This section introduces the novel features of \clingo~5's input language.
All of them are situated in the underlying grounder \gringo~5 and can thus
also be used independently of \clingo.
%
We start with a detailed description of \gringo~5's generic means for defining theories
and afterwards summarize further % the remaining 
new features.

Our generic approach to theory specification rests upon two languages:
the one defining theory languages and the theory language itself.
Both borrow elements from the underlying ASP language,
foremost an aggregate-like syntax for formulating variable length expressions.
To illustrate this, consider Listing~\ref{prg:diff},
where a logic program is extended by constructs for handling difference and linear constraints.
While the former are binary constraints of the form $x_1-x_2\leq k$, % where $x,y$ are integer variables and $k$ is an integer,
the latter have a variable size and are of form
\(
a_1x_1+\dots+a_nx_n\circ k
\),
where $x_i$ are integer variables, $a_i$ and $k$ are integers, and $\circ\in\{\leq,\geq,<,>,=\}$
for $1\leq i\leq n$.%
\footnote{For simplicity, we consider normalized difference constraints rather than general ones of form $x_1-x_2\circ k$.}
Note that solving difference constraints is polynomial, while solving linear equations (over integers) is NP-hard.
The theory language for expressing both types of constraints is defined in Lines~\ref{prg:diff:begin-theory}--\ref{prg:diff:end-theory} and preceded by the directive \lstinline{#theory}.
The elements of the resulting theory language are preceded by \lstinline{&} and used as regular atoms in the logic program in Lines~\ref{prg:diff:begin-program}--\ref{prg:diff:end-program}.
% ------------------------------------------------------------
\lstinputlisting[language=clingo,morekeywords={n,m},float=t,escapeinside={\%(}{\%)},basicstyle=\ttfamily\small,label=prg:diff,caption=Logic program enhanced with difference and linear constraints (\texttt{diff.lp})]{code/diff.lp}
% ------------------------------------------------------------

To be more precise,
a \emph{theory definition} has the form
\begin{lstlisting}[numbers=none,mathescape=t]
#theory $T$ {$D_1$;$\dots$;$D_n$}.
\end{lstlisting}
where $T$ is the theory name and each $D_i$ is a definition for a theory term or a theory atom for $1\leq i\leq n$.
The language induced by a theory definition is the set of all theory atoms constructible from its theory atom definitions.

A \emph{theory atom definition} has form
\begin{lstlisting}[numbers=none,mathescape=t]
&$p$/$k$ : $t$,$o$ $\quad\textrm{ or }\quad$ &$p$/$k$ : $t$,{$\diamond_1$,$\dots$,$\diamond_m$},$t'$,$o$
\end{lstlisting}
where $p$ is a predicate name and $k$ its arity,
$t,t'$ are names of theory term definitions,
each $\diamond_i$ is a theory operator for $m\geq 1$,
and
\(
o\in\{\mathtt{head},\mathtt{body},\mathtt{any}, \mathtt{directive}\}
\)
determines where theory atoms may occur in a rule.
%
Examples of theory atom definitions are given in Lines~\ref{lst:ex1-begin-def-atom}--\ref{lst:ex1-end-def-atom} of Listing~\ref{prg:diff}.
%%
The language of a theory atom definition as above contains all \emph{theory atoms} of form
\begin{lstlisting}[numbers=none,mathescape=t]
&$a$ {${C}_1$:${L}_1$;$\dots$;${C}_n$:${L}_n$}$\quad\textrm{ or }\quad$ &$a$ {${C}_1$:${L}_1$;$\dots$;${C}_n$:${L}_n$} $\diamond$ $c$
\end{lstlisting}
where $a$ is an atom over predicate $p$ of arity $k$,
each ${C}_i$ is a tuple of theory terms in the language for~$t$,
$c$ is a theory term in the language for~$t'$,
$\diamond$ is a theory operator among $\{ \diamond_1, \dots, \diamond_m \}$,
and each ${L}_i$ is a regular condition (i.e., a tuple of regular literals)
for $1\leq i\leq n$.
% The last part `${}\diamond c$' is optional.
Whether the last part `${}\diamond c$' is included depends on the form of a theory atom definition.
%
Five occurrences of theory atoms can be found in Lines~\ref{prg:diff:begin-use-theory}--\ref{prg:diff:end-use-theory} of Listing~\ref{prg:diff}.

A \emph{theory term definition} has form
\begin{lstlisting}[numbers=none,mathescape=t]
$t$ {$D_1$;$\dots$;$D_n$}
\end{lstlisting}
where $t$ is a name for the defined terms and each $D_i$ is a theory operator definition 
for $1\leq i\leq n$.
A respective definition specifies the language of all theory terms
that can be constructed via % admissible term constructors and
its operators. % , as described next.
%
Examples of theory term definitions are given in Lines~\ref{prg:diff:begin-def-term}--\ref{prg:diff:end-def-term} of Listing~\ref{prg:diff}.
%%
Each resulting \emph{theory term} is one of the following:
\par\medskip\noindent
\begin{minipage}{0.5\linewidth}
  \begin{itemize}
  \item a constant term: \ $c$
  \item a variable term: \ $v$
  \item a binary theory term: \  $t_1 \diamond t_2$
  \item a unary theory term: \  ${}\diamond t_1$
  \end{itemize}
\end{minipage}
\begin{minipage}{0.5\linewidth}
\begin{itemize}
  \item a function theory term: \  $f(t_1,\dots,t_k)$
  \item a tuple theory term: \  $(t_1,\dots,t_l,)$
  \item a set theory term: \  $\{t_1,\dots,t_l\}$
  \item a list theory term: \  $[t_1,\dots,t_l]$
\end{itemize}
\end{minipage}
\par\medskip\noindent
where each $t_i$ is a theory term,
$\diamond$ is a theory operator defined by some $D_i$,
$c$ and $f$ are symbolic constants,
$v$ is a first-order variable,
$k\geq 1$, and
$l\geq 0$.
(The trailing comma in tuple theory terms is optional if $l \neq 1$.)
% Furthermore,
Parentheses can be used to specify operator precedence.

A \emph{theory operator definition} has form
\begin{lstlisting}[numbers=none,mathescape=t]
$\diamond$ : $p$,unary $\quad\textrm{ or }\quad$ $\diamond$ : $p$,binary,$a$
\end{lstlisting}
where $\diamond$ is a unary or binary theory operator with precedence $p\geq 0$
(determining implicit parentheses). % , respectively.
Binary theory operators are additionally characterized by an associativity
$a\in\{\text{\lstinline{right}},\linebreak[1]\text{\lstinline{left}}\}$.
%
As an example,
consider Line~\ref{prg:diff:op-mul} of Listing~\ref{prg:diff},
where the \lstinline{binary} operator~\lstinline{*} is defined with precedence~\lstinline{1} and \lstinline{left} associativity.
%
In total, Lines~\ref{prg:diff:begin-def-term}--\ref{prg:diff:end-def-term} of Listing~\ref{prg:diff}
include nine theory operator definitions.
%%
Particular \emph{theory operators} can be assembled (written consecutively without spaces) from the symbols
`\lstinline{!}',
`\lstinline{<}',
`\lstinline{=}',
`\lstinline{>}',
`\lstinline{+}',
`\lstinline{-}',
`\lstinline{*}',
`\lstinline{/}',
`\lstinline{\}',
`\lstinline{?}',
`\lstinline{&}',
%`\texttt{\&}',
`\lstinline{|}',
`\lstinline{.}',
`\lstinline{:}',
`\lstinline{;}',
`\lstinline{~}', and
`\lstinline{^}'.
%
For instance,
in Line~\ref{prg:diff:op-dot} of Listing~\ref{prg:diff}, the operator `\lstinline{..}' is defined as the concatenation of two periods.
%
% Note that
The tokens
`\lstinline{.}',
`\lstinline{:}',
`\lstinline{;}', and
`\lstinline{:-}'
must be combined with other symbols
due to their dedicated usage.
Instead, one may write
`\lstinline{..}',
`\lstinline{::}',
`\lstinline{;;}',
`\lstinline{::-}',
etc.
% Each operator is either unary or binary.
% To resolve ambiguities,
% a binary theory operator is either left or right associative.
% Furthermore, each operator has a precedence to allow for omitting parentheses in some situations.

While theory terms are formed similar to regular ones,
theory atoms rely upon an aggregate-like construction for forming variable-length theory expressions.
In this way, standard grounding techniques can be used for gathering theory terms.
(However, the actual atom within a theory atom comprises regular terms only.)
%
The treatment of theory terms still % and atoms % is different
differs from their regular counterparts
% This is due to the fact 
in that the grounder skips simplifications like, e.g., arithmetic evaluation.
% completely ignores the theory's semantics and thus performs no theory-specific simplifications.
This can be nicely seen on the different results in Listing~\ref{prg:grd:diff} of grounding terms formed with
the regular and theory-specific variants of operator `\lstinline{..}'.
Observe that
the     fact \lstinline{task(1..n)} in Line~\ref{prg:diff:rule-task} of Listing~\ref{prg:diff} results in \lstinline{n} ground     facts,
viz.\ \lstinline{task(1)} and \lstinline{task(2)} because of \lstinline{n=2}.
Unlike this, the theory expression \lstinline{1..m} stays structurally intact and is only transformed into \lstinline{1..1000} in view of \lstinline{m=1000}.
That is, the grounder does not evaluate the theory term \lstinline{1..1000} and
leaves its interpretation to a downstream theory solver.
%
% ------------------------------------------------------------
\lstinputlisting[language=clingo,float=t,basicstyle=\ttfamily\small,label=prg:grd:diff,caption={Human-readable result of grounding Listing~\ref{prg:diff} via `\texttt{gringo --text diff.lp}'}]{code/diff.grd}
% ------------------------------------------------------------
%
A similar situation is encountered when comparing the treatment of the regular term `\lstinline{200*T}' in Line~\ref{prg:diff:rule-duration} of Listing~\ref{prg:diff} to the
theory term `\lstinline{end(T)-start(T)}' in Line~\ref{prg:diff:rule-diff}.
While each instance of `\lstinline{200*T}' is evaluated during grounding,
instances of the theory term  are left in Line~6 of Listing~\ref{prg:grd:diff}.
In fact, if `\lstinline{200*T}' had been a theory term as well,
it would have resulted in the unevaluated instances  `\lstinline{200*1}' and `\lstinline{200*2}'. % , respectively.
% This is because the grounder ignores the meaning of the theory operator `\lstinline{*}'.

The remainder of this section is dedicated to other language extensions of \gringo~5 aiming
at a disentanglement of the various uses of \lstinline{#show} directives
(and their induced symbol table).
Such directives were beforehand used for
controlling the output of stable models,
delineating the scope of reasoning modes (e.g., intersection, union, projection, etc.),
and for
passing special-purpose information to downstream systems.
For instance,
theory and heuristic information was passed to \clasp\ via dedicated predicates like \lstinline{_edge} and \lstinline{_heuristic}.
This entanglement brought about several shortcomings.
% For one, the output of reasoning modes had to coincide with the atoms subject to the reasoning mode.
In fact, passing information via a symbol table did not only scramble the output, but also provoked overhead in grounding and filtering ``artificial'' symbolic information.

Now, in \gringo~5, the sole purpose of \lstinline{#show} is to furnish an output directive.
There are three different kinds of such statements:
%
\begin{lstlisting}[numbers=none,mathescape=t]
#show.     #show $p$/$n$.      #show $t$ : $l_1$,$\dots$,$l_n$.
\end{lstlisting}
%
The first form hides all atoms, and the second all except those over predicates~$p/n$ indicated by \lstinline{#show} statements.
The third form does not hide any atoms and can be used to output arbitrary terms~$t$,
whenever the literals $l_1,\dots,l_n$ in the condition after~`\lstinline{:}' hold.
This is particularly useful in meta-programming,
e.g., `\lstinline{#show A : holds(A).}' can be used to map back reified atoms.

Atoms used in reasoning modes are indicated by \lstinline{#project} directives, having two forms:
%
\begin{lstlisting}[numbers=none,mathescape=t]
#project $p$/$n$.     #project $a$ : $l_1$,$\dots$,$l_n$.
\end{lstlisting}
%
Here, $p$ is a predicate name with arity $n$, $a$ is an atom, and $l_1,\dots,l_n$ are % regular
literals.
While the first form declares all atoms over predicate~$p/n$ as % being
subject to projection,
the second includes instances of~$a$
obtained via grounding, as detailed in \cite{gekakasc14b} for
\lstinline{#external} directives.

The last two new directives of interest abolish the need for % using
the special-purpose predicates \lstinline{_edge} and \lstinline{_heuristic},
previously used in conjunction with the ASP solver \clasp:
\begin{lstlisting}[numbers=none,mathescape=t]
#edge ($u$,$v$) : $l_1$,$\dots$,$l_n$.
#heuristic $a$ : $l_1$,$\dots$,$l_n$. [$k$@$p$,$m$]
\end{lstlisting}
As above, $a$ is an atom, and $l_1,\dots,l_n$ are literals.
Moreover,
$u,v,k,p,m$ are terms, % is a condition;
where 
`\lstinline[mathescape=t]{($u$,$v$)}' stands for an
edge from $u$ to $v$ in an acyclicity extension \cite{bogejakasc15a}.
Integer values for $k$ and $p$ along with
\lstinline{init}, \lstinline{factor}, \lstinline{level}, \lstinline{sign}, \lstinline{true}, or \lstinline{false} for~$m$
determine a heuristic modifier \cite{gekaotroscwa13a}.
Finally, note that
zero is taken as default priority when
the optional `\lstinline[mathescape=t]{@$p$}' part 
in `\lstinline[mathescape=t]{[$k$@$p$,$m$]}',
resembling the syntax of ranks for weak constraints \cite{aspcore2},
is omitted.
% $k$ is a term evaluating to an integer and $m$ is a heuristic modifier among
% \lstinline{init}, \lstinline{factor}, \lstinline{level}, \lstinline{sign}, \lstinline{true}, or \lstinline{false}, respectively.
% A positive integer priority $p$ is optional (and zero by default).
% Note that the usage of the term \lstinline[mathescape=t]{[$k$@$p$,$m$]} is borrowed from weak constraints~\cite{aspcore2}.
% Also, observe that arcs \lstinline[mathescape=t]{($u$,$v$)} and heuristically modified atoms $a$ can still be defined (recursively) via arbitrary subprograms,
% but now without any need to show them in a resulting stable models.
% %
% We refer the reader to \cite{bogejakasc15a} and \cite{gekaotroscwa13a} for details on the underlying concepts.
%%% Local Variables:
%%% mode: latex
%%% TeX-master: "paper"
%%% End:
