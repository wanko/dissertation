
\section{Discussion}\label{sec:discussion}

The \clingo~5 system provides a comprehensive infrastructure for enhancing ASP with theory reasoning.
This ranges from
generic means for expressing theories along with their support by \gringo,
over a theory-aware intermediate format,
to simple yet powerful interfaces in \C, \lua, and \python.
In each case, a propagator can    specify (up to) four basic functions to customize its integration into \clasp's propagation,
where an arbitrary number of (independent) theory propagators can be be incorporated.
% Our approach leaves room for various alternatives.
Logically, ASP encodings may build upon defined or external theory atoms,
and their associated conditions may be strict or non-strict.
In practice,
\clingo~5 supports stateless and stateful theory propagators, which can be controlled in a fine-grained way.
%
For instance, propagators are thread-sensitive, watches can be set to symbolic as well as theory literals,
and the scope and lifetime of nogoods stemming from theory propagation can be configured.

A first step toward a more flexible ASP infrastructure
was done with \clingo~4 \cite{gekakasc14b} by introducing \lua\ and \python\ APIs for multi-shot solving.
Although this allows for fine-grained control of complex ASP reasoning processes,
the functionality provided no access to \clasp's propagation and was restricted to inspecting (total) stable models.
%
The extended framework for theory propagation relative to partial assignments (cf.\ Figure~\ref{fig:cdcl})
follows the canonical approach of SMT \cite{baseseti09a}.
% The \aspmt\ system \cite{barlee14b} extends ASP with theory reasoning by translating it to SMT.
%
While \dlvhex\ implicitly provides access to \clasp's propagation,
this is done on the more abstract level of higher-order logic programs.
Also, \dlvhex\ as well as many other systems, such as \clingcon\ or \inca,
implement specialized propagation via \clasp's internal interfaces,
whose usage is more involved and subject to change with each release.
%
Although the new high-level interfaces may not yet fully cover all desired features,
they provide a first step toward easing the development of such dedicated systems and
putting them on a more stable basis.
%
Currently,
\clingo~5's infrastructure is already used as a basis for
\clingcon~3 \cite{bakakaossc16a},
% lazy \dflat~\cite{blkascwo16a},
\lctocasp\ \cite{cakaossc16a},
and
its integration with SWI-Prolog.
%
Finally, we believe that the extended grounding capacities along with the intermediate format supplemented in Appendix~\ref{sec:aspif}  %~\cite{gekakaosscwa16b}
will also be beneficial for non-native approaches and ease the overall development of ASP-oriented solvers.
This applies to systems like \dingo, \mingo, and \aspmt\ \cite{barlee14b},
the latter implementing ASP with theory reasoning by translation to SMT,
which so far had to resort to specific input formats and meta-programming to bypass the grounder.

%%% Local Variables:
%%% mode: latex
%%% TeX-master: "paper"
%%% End:
