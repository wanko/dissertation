
\section{Logical Characterization}
\label{sec:foundations}

The semantics of logic programs modulo theories rests upon
ground programs~$P$ over two disjoint alphabets, $\mathcal{A}$ and $\mathcal{T}$,
consisting of regular and \emph{theory atoms}. % , respectively.
Accordingly, $P$ is a set of rules~$r$ of the form
$h \leftarrow a_1,\dots,a_m,\naf{a_{m+1}},\dots,\naf{a_n}$,
where the head $h$ is constant~$\bot$, $a_0$ or $\{a_0\}$
for an atom $a_0\in\mathcal{A}\cup\mathcal{T}$, and
$\{a_1,\dots,a_n\}\subseteq\mathcal{A}\cup\mathcal{T}$.
If $h=\bot$, $r$ is called an \emph{integrity constraint},
a \emph{normal rule} if $h=a_0$, or
a \emph{choice rule} if $h=\{a_0\}$;
as usual, we skip $\bot$ when writing integrity constraints.
We let $\head{r}=\emptyset$ for an integrity constraint~$r$,
$\head{r}=\{a_0\}$ for a normal or choice rule~$r$,
and define $\head{P}=\bigcup_{r\in P}\head{r}$ as the \emph{head atoms} of~$P$.
In analogy to inputs atoms from
\lstinline{#external} directives \cite{gekakasc14b},
we partition $\mathcal{T}$ into
\emph{defined} theory atoms $\mathcal{T}\cap\head{P}$ and
\emph{external} theory atoms $\mathcal{T}\setminus\head{P}$.

Given a collection $\mathbb{T}$ of theories,
we associate each $T\in\mathbb{T}$ with a \emph{scope}
$\mathcal{T}^T$ of atoms
relevant to~$T$, and let
$\mathcal{T}=\bigcup_{T\in\mathbb{T}}\mathcal{T}^T$
be the corresponding set of theory atoms.
% In view of
Reconsidering the input language in Section~\ref{sec:language},
a natural choice for $\mathcal{T}^T$ consists % of the set
of all (ground) atoms declared within a
\lstinline{#theory} directive for~$T$.
% \comment{BK: Those must be disjoint, right? M: They are syntactically different and therefore disjoint. The idea is that any atom handled by a theory propagator belongs to $\mathcal{T}$, no matter its syntax.}
However, as we see in Section~\ref{sec:system},
a scope
% does not necessarily have to be limited to atoms in a dedicated theory language, i.e,
% those preceded by \lstinline{&},
% but rather the set $\mathcal{T}$ of theory atoms
may in general include atoms written in
regular as well as extended syntax (the latter preceded by `\lstinline{&}') in the input language.
% The way logic programs extended by theories are specified still guarantees
% that $\mathcal{T}=\bigcup_{T\in\mathbb{T}}\mathcal{T}^T$,
% so that each theory atom belongs to the scope of at least one theory in~$\mathbb{T}$.

In order to reflect different forms of theory propagation,
we further consider a partition of the scope $\mathcal{T}^T$ of a theory~$T$ into % two subsets,
\emph{strict} theory atoms $\mathcal{T}^T_e$ and
\emph{non-strict} theory atoms $\mathcal{T}^T_i$
such that
$\mathcal{T}^T_e\cap\mathcal{T}^T_i=\emptyset$ and
$\mathcal{T}^T_e\cup\mathcal{T}^T_i=\mathcal{T}^T$.
The strict theory atoms in~$\mathcal{T}^T_e$ resemble equivalences as expressed by the constraint atoms
of \clingcon\ \cite{ostsch12a}, which must be assigned to true iff their associated
constraints hold.
This is complemented by viewing the non-strict theory atoms in~$\mathcal{T}^T_i$ as
implications similar to the constraint statements of \ezcsp\ \cite{balduccini09a},
where only statements assigned to true impose requirements, while constraints associated with
false ones are free to hold or not.
Given the distinction of respective kinds of theory atoms,
a combined theory~$T$ may integrate constraints according to the semantics of \clingcon\ and \ezcsp,
e.g., indicated by dedicated predicates or arguments thereof in $T$'s theory language.

We now turn to mapping the semantics of logic programs modulo theories back to regular stable models.
% \cite{siniso02a}.
In the abstract sense, we call any $\mathcal{S}^T\subseteq\mathcal{T}^T$ a \emph{$T$-solution}
if $T$ is consistent with the conditions expressed by elements of~$\mathcal{S}^T$ as well as the complements of
conditions associated with the false strict theory atoms in $\mathcal{T}^T_e\setminus \mathcal{S}^T$.%
\footnote{%
  Although we omit formal details, atoms in
  Satisfiability Modulo Theories (SMT; \cite{baseseti09a})
  belong to first-order predicates interpreted in a theory~$T$,
  and the ones that hold in some model of~$T$ provide a $T$-solution $\mathcal{S}^T\subseteq\mathcal{T}^T$.}
Generalizing this concept to a collection $\mathbb{T}$ of theories,
we say that $\mathcal{S}\subseteq\mathcal{T}$ is a \emph{$\mathbb{T}$-solution} if
$\mathcal{S}\cap\mathcal{T}^T$ is a $T$-solution for each $T\in\mathbb{T}$.
Then, we define a set $X\subseteq\mathcal{A}\cup\mathcal{T}$ of (regular and theory) atoms
as a \emph{$\mathbb{T}$-stable model} of a ground program~$P$ if
% $X\cap\mathcal{T}$ is a $\mathbb{T}$-solution and
there is some $\mathbb{T}$-solution $\mathcal{S}$ such that
$X$ is a (regular) stable model of the program
%
\begin{align}%{1}
% \begin{equation}
\neghspace
P
% &{}\cup
\cup
\{{a\leftarrow}       \mid T\in\mathbb{T},a\in (\mathcal{T}^T_e\setminus\head{P})\cap \mathcal{S}\}
\cup
\{ {\leftarrow\naf{a}}\mid T\in\mathbb{T},a\in (\mathcal{T}^T_e\cap\head{P})\cap \mathcal{S}\}
\label{eq:stable:strict}
\\
\neghspace
% &{}\cup
{}\cup
\{{\{a\}\leftarrow}   \mid T\in\mathbb{T},a\in (\mathcal{T}^T_i\setminus\head{P})\cap \mathcal{S}\}
\cup
\{ {\leftarrow{a}}    \mid T\in\mathbb{T},a\in (\mathcal{T}^T\cap\head{P})\setminus \mathcal{S}\}\rlap{.}
\label{eq:stable:non-strict}
\end{align}
% \label{eq:stable}
% \end{equation}
%
That is, the rules added to~$P$ in~\eqref{eq:stable:strict} and~\eqref{eq:stable:non-strict}
express conditions aligning $X\cap\mathcal{T}$ with an underlying $\mathbb{T}$-solution~$\mathcal{S}$.
First, the facts in~\eqref{eq:stable:strict} make sure that external theory atoms
that are strict, i.e., included in $\mathcal{T}^T_e\setminus\head{P}$ for some $T\in\mathbb{T}$,
and hold in~$\mathcal{S}$ belong to~$X$ as well.
Unlike this, the corresponding set of choice rules in~\eqref{eq:stable:non-strict}
merely says that non-strict external theory atoms from~$\mathcal{S}$ may be included in~$X$,
thus not insisting on a perfect match between non-strict theory atoms and elements of~$\mathcal{S}$.
Moreover, the integrity constraints in~\eqref{eq:stable:strict} and~\eqref{eq:stable:non-strict}
take care of defined theory atoms belonging to~$\head{P}$.
The respective set in~\eqref{eq:stable:strict} again focuses on strict theory atoms and
stipulates the ones from~$\mathcal{S}$ to be included in~$X$ as well.
In addition, for both strict and non-strict defined theory atoms,
the integrity constraints in~\eqref{eq:stable:non-strict} assert the falsity
of atoms that do not hold in~$\mathcal{S}$.

For example, consider a program $P=\{a\leftarrow b,\naf{c}\}$ subject to
some theory~$T$ with the strict and non-strict theory atoms
$\mathcal{T}^T_e=\{a,b\}$ and $\mathcal{T}^T_i=\{c\}$,
and let $\mathcal{S}=\{a,b,c\}$ be a $T$-solution.
Then, the extended program for $\mathcal{S}$ is
$P\cup\{{b\leftarrow{}}; {\{c\}\leftarrow{}};  \leftarrow\naf{a}\}$,
whose (only) regular stable model $X=\{a,b\}$ is a $\{T\}$-stable model of~$P$.
Note that $\mathcal{S}$ assigns the non-strict theory atom $c$ to true,
while $X$ excludes it to keep
$a\leftarrow b,\naf{c}$ applicable
for the (strict) defined theory atom~$a$.

To summarize the main principles of the $\mathbb{T}$-stable model concept,
strict theory atoms (for some $T\in\mathbb{T}$) must exactly match their
interpretation in a $\mathbb{T}$-solution $\mathcal{S}$,
while non-strict ones (not strict for any $T\in\mathbb{T}$) in~$X$ are only required not to exceed~$\mathcal{S}$.
Second,
external theory atoms that hold in~$\mathcal{S}$ are mapped to facts or choice rules,
while conditions on defined ones are enforced by means of integrity constraints.
As a result, $\mathbb{T}$-stable models are understood as regular stable models,
% in the sense of \cite{siniso02a}, 
yet relative to extensions of a given program~$P$
determined by underlying $\mathbb{T}$-solutions.
Notably, % since the regular notion of a stable model applies unaltered,
the concept of $\mathbb{T}$-stable models also carries on to logic programs allowing for
further constructs, such as weight constraints and disjunction,
which have not been discussed here for brevity (cf.~\cite{siniso02a}).

% Let $T$ be a theory in $\mathbf{T}$.
% A solution $S$ of $T$ induces the set of all theory atoms $X_S\subseteq\mathcal{T}$ satisfied by $S$ wrt $T$.
% Moreover, let $\mathcal{T}_T$ denote the set all theory atoms in~$\mathcal{T}$ that are relevant to~$T$ and thus possibly satisfied by solutions~$S$ of~$T$.
% \comment{MG: Rather solution $S=S_1\cup\dots\cup S_n$ for theories $\mathbf{T}=\{T_1,\dots,T_n\}$,
%              as theories may ``communicate'' through rules of $P$.\\
%              Maybe use $\mathbb{T}$ for set of theories to avoid potential clash with \Tsign\
%              for literals in assignment.}
% $X$ is a $\mathbf{T}$-stable model of $P$ and some theory $T\subseteq \mathbf{T}$ if
% \begin{itemize}
% \item $X$ is a stable model of
%   \begin{align}
% P
% \cup
% \{{a\leftarrow}       \mid a\in\mathcal{T}, a\in X_S\setminus\head{P}\}
% &\cup
% \{ {\leftarrow\naf{a}}\mid a\in\mathcal{T}, a\in X_S\cap     \head{P}\}
% \\&\cup
% %\{{\{a\}\leftarrow}       \mid a\in\mathcal{T}, a\in X_S\setminus\head{P}\}
% %&\cup
% \{ {\leftarrow{a}}\mid a\in\mathcal{T}, a\in \head{P}\setminus X_S    \}
%   \end{align}
%   \begin{align}
% P
% \cup
% \{{\{a\}\leftarrow}       \mid a\in\mathcal{T}, a\in X_S\setminus\head{P}\}
% &\cup
% \{ {\leftarrow{a}}\mid a\in\mathcal{T}, a\in \head{P}\setminus X_S    \}
%   \end{align}
% for some set $X_S\subseteq\mathcal{T}$ of induced theory atoms
% %
% \item $X$ is a stable model of
% \comment{MG: Modified definition considering partition $\mathcal{T}=\mathcal{T}_e\uplus\mathcal{T}_i$.}
% \begin{align}\nonumber
% P
% &{}\cup
% \{{a\leftarrow}       \mid a\in\mathcal{T}_e, a\in X_S\setminus\head{P}\}
% \cup
% \{ {\leftarrow\naf{a}}\mid a\in\mathcal{T}_e, a\in X_S\cap     \head{P}\}
% \\&{}\cup
% \{{\{a\}\leftarrow}   \mid a\in\mathcal{T}_i, a\in X_S\setminus\head{P}\}
% \cup
% \{ {\leftarrow{a}}\mid a\in\mathcal{T}, a\in \head{P}\setminus X_S    \}
% \label{eq:stable}
% \end{align}
% for some set $X_S\subseteq\mathcal{T}$ of induced theory atoms
% %
% \item Issues
% \begin{itemize}
%
% \item The above definition has two distinctive features
%   \begin{itemize}
%   \item different treatment of theory atoms in heads and bodies
%   \item strict and non-strict theory atoms
%     \comment{Clearly, The distinction should be made on a per-atom'' basis and not globally as above.} %
%
% distinguishes between strict and
% non-strict theory atoms. The truth values of strict atoms must be
% equivalent to their interpretation in the theory. For non-strict ones,
% the implication from true atoms to a true interpretion in the theory is
% enough. Whether atoms are strict or non-strict inherently depends on how
% they are handled by a corresponding theory solver.
%   \end{itemize}
% \item What about $\mathcal{T}\setminus X_S$?
%
% Are they false by default, or controllable like \verb+#external+s, that is, false, true, free?
% \end{itemize}
% \end{itemize}

%%% Local Variables:
%%% mode: latex
%%% TeX-master: "paper"
%%% End:
