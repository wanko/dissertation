
\section{Intermediate Language}\label{sec:aspif}
\newcommand\Space{\textvisiblespace}%

To accommodate the richer input language, a more general grounder-solver interface is needed.
Although this could be left internal to \clingo~5,
it is good practice to explicate such interfaces via an intermediate language.
This does not only provide a transparent interface, but also allows for using alternative downstream solvers or transformations.

Unlike the block-oriented \smodels\ format,
the \aspif\footnote{ASP Intermediate Format} format is line-based.
Notably, it abolishes the need of using symbol tables as in \smodels' format%
\footnote{\url{http://www.tcs.hut.fi/Software/smodels}}
for passing along meta-expressions and rather allows \gringo~5 to output information as soon as it is grounded.
An \aspif\ file starts with a header, beginning with the keyword \lstinline{asp}
along with version information and optional tags:
\[\texttt{asp} \Space v_m \Space v_n \Space v_r \Space t_1 \Space \dots \Space t_k \]
where $v_m,v_n,v_r$ are non-negative integers representing the version in terms of \textit{major}, \textit{minor}, and \textit{revision} numbers,
and each $t_i$ is a tag for $k\geq 0$.
Currently, the only tag is \lstinline{incremental}, meant to set up the underlying solver for multi-shot solving.
An example header is given in Line~1 of Figure~\ref{prg:ezy}(b) and~\ref{aspif:diff}.
%
The rest of the file is comprised of one or more logic programs.
Each logic program is a sequence of lines of \aspif\ statements followed by a \lstinline{0}, one statement or \lstinline{0} per line, respectively.
%
Positive and negative integers are used to represent positive or  negative literals, respectively.
Hence, \lstinline{0} is not a valid literal.

Let us now briefly describe the format of \aspif\ statements and illustrate them with a
simple logic program in Figure~\ref{prg:ezy} as well as the result of grounding a subset of Listing~\ref{prg:diff}
in Figure~\ref{aspif:diff}. % below.

% ------------------------------------------------------------
\begin{wrapfigure}{r}{0.4\textwidth}
\vspace{-15pt}%\frame%
{\begin{minipage}[t]{0.15\textwidth}
\lstinputlisting[numbers=left,basicstyle=\ttfamily\footnotesize]{code/ezy.lp}%
\end{minipage}}%
\hfill%\frame%
{\begin{minipage}[t]{0.2\textwidth}
\lstinputlisting[numbers=right,basicstyle=\ttfamily\footnotesize]{code/ezy.aspif}%
\end{minipage}}%
% \vspace{-5pt}%
\caption{\label{prg:ezy}(a) Simple logic program and its (b) \aspif\ representation}
\vspace{10pt}%
\end{wrapfigure}
% ------------------------------------------------------------
%
% --------------------------------------------------
\newcounter{DUID}%
\newcommand{\myparagraph}[1]{\par\emph{#1}}
% --------------------------------------------------
\myparagraph{Rule statements} have form
\addtocounter{DUID}{1}
\[\texttt{\theDUID} \Space H \Space B\]
in which head $H$ has form
\[h \Space m \Space a_1 \Space \dots \Space a_m\]
where
$h \in \{\texttt{0},\texttt{1}\}$ determines whether the head is a disjunction or choice,
$m \geq 0$ is the number of head elements, and
each $a_i$ is a positive literal.

Body $B$ has one of two forms:
\begin{itemize}
\item Normal bodies have form
  \[\texttt{0} \Space n \Space l_{1} \Space \dots \Space l_n\]
  where
  $n \geq 0$ is the length of the rule body, and
  each $l_i$ is a literal.
\item Weight bodies have form
  \[\texttt{1} \Space l \Space n \Space l_1 \Space w_1  \Space \dots \Space l_n \Space w_n\]
  where
  $l$ is a positive integer to denote the lower bound,
  $n \geq 0$ is the number of literals in the rule body, and
  each $l_i$ and $w_i$ are a literal and a positive integer.
\end{itemize}
All types of ASP rules are included in the above rule format.
Heads are disjunctions or choices, including the special case of singular disjunctions for representing normal rules.
As in the \smodels\ format,
aggregates are restricted to a singular body, just that in \aspif\ cardinality constraints are taken as special weight constraints.
Otherwise, a body is simply a conjunction of literals.

The three rules in Figure~\ref{prg:ezy}(a) are represented by the statements in Lines~2--4 of Figure~\ref{prg:ezy}(b).
For instance, the four occurrences of \lstinline{1} in Line~2 capture a rule with a choice in the head, having one element, identified by \lstinline{1}.
The two remaining zeros capture a normal body with no element.
For another example,
Lines~2--7 of Figure~\ref{aspif:diff} represent the four facts in Line~1 and~2 of Listing~\ref{prg:grd:diff}
along with the ones (comprising theory atoms) in Line~6 of Listing~\ref{prg:grd:diff}.

\myparagraph{Minimize statements} have form
\addtocounter{DUID}{1}
\[\texttt{\theDUID} \Space p \Space n \Space l_1 \Space w_1 \Space \dots \Space l_n \Space w_n\]
where
$p$ is an integer priority,
$n \geq 0$ is the number of weighted literals,
each $l_i$ is a literal, and
each $w_i$ is an integer weight.
%
Each of the above expressions gathers weighted literals sharing the same priority $p$
from all \lstinline{#minimize} directives and weak constraints in a logic program.
As before, maximize statements are translated into minimize statements.

\myparagraph{Projection statements} result from \lstinline{#project} directives and have form
\addtocounter{DUID}{1}
\[\texttt{\theDUID} \Space n  \Space a_1 \Space \dots \Space a_n\]
where
$n \geq 0$ is the number of atoms, and
each $a_i$ is a positive literal.

\myparagraph{Output statements} result from \lstinline{#show} directives and have form
\addtocounter{DUID}{1}
\[\texttt{\theDUID} \Space m \Space s \Space n  \Space l_1 \Space \dots \Space l_n\]
where
$n \geq 0$ is the length of the condition,
each $l_i$ is a literal, and
$m\geq0$ is an integer indicating the length in bytes of string $s$
(where $s$ excludes byte `\textbackslash0' and newline).
%
The output statements in Lines~5--7 of Figure~\ref{prg:ezy}(b) print the symbolic representation of atom
\lstinline{a}, \lstinline{b}, or \lstinline{c}, whenever the corresponding atom is true. % , respectively.
For instance, the string `\lstinline{a}' is printed  if atom `\lstinline{1}' holds.
Unlike this,
the statements in Lines~8--11 of Figure~\ref{aspif:diff} unconditionally print the symbolic representation
of the atoms stemming from the four     facts in Line~1 and~2 of Listing~\ref{prg:grd:diff}.

\myparagraph{External statements} result from \lstinline{#external} directives and have form
\addtocounter{DUID}{1}
\[\texttt{\theDUID} \Space a \Space v\]
where
$a$ is a positive literal, and
$v \in \{0,1,2,3\}$ indicates free, true, false, and release.

\myparagraph{Assumption statements} have form
\addtocounter{DUID}{1}
\[\texttt{\theDUID} \Space n \Space l_1 \Space \dots \Space l_n\]
where
$n\geq 0$ is the number of literals, and
each $l_i$ is a literal.
Assumptions instruct a solver to compute      stable models such that $l_1,\dots,l_n$ hold.
They are only valid for a single solver call.

\myparagraph{Heuristic statements} result from \lstinline{#heuristic} directives and have form
\addtocounter{DUID}{1}
\[\texttt{\theDUID} \Space m \Space a \Space k \Space p  \Space n  \Space l_1 \Space \dots \Space l_n\]
where
$m\in\{0,\dots,5\}$ stands for the ($m$+1)th heuristic modifier among % in the list
\lstinline{level}, \lstinline{sign}, \lstinline{factor}, \lstinline{init}, \lstinline{true}, and \lstinline{false}, % respectively,
$a$ is a positive literal,
$k$ is % a term evaluating to 
an integer,
$p$ is a non-negative integer priority,
$n \geq 0$ is the number of literals in the condition, and
the literals $l_i$ are the condition under which the heuristic modification should be applied.

\myparagraph{Edge statements} result from \lstinline{#edge} directives and have form
\addtocounter{DUID}{1}
\[\texttt{\theDUID} \Space u \Space v \Space n  \Space l_1 \Space \dots \Space l_n\]
where
$u$ and $v$ are integers representing an edge from node $u$ to node $v$,
$n \geq 0$ is the length of the condition, and
the literals $l_i$ are the condition for the edge to be present.

Let us now turn to the theory-specific part of \aspif.
% As mentioned, the grounder ignores the theory's semantics.
Once a theory expression is grounded,
\gringo~5 % confines itself with outputting 
outputs a serial representation of its syntax tree.
To illustrate this,
we give in Figure~\ref{aspif:diff} the (sorted) result of grounding all lines of Listing~\ref{prg:diff} related to difference constraints,
viz.\ Line~2/3, 11, 15/16, and 19, as well as Line~1 and~13.

\myparagraph{Theory terms} are represented using the following statements:
\addtocounter{DUID}{1}
\begin{align}
\texttt{\theDUID} \Space \texttt{0} & \Space u \Space w \label{eq:number}\\
\texttt{\theDUID} \Space \texttt{1} & \Space u \Space n \Space s \label{eq:symbols}\\
\texttt{\theDUID} \Space \texttt{2} & \Space u \Space t \Space n \Space u_1 \Space \dots \Space u_n \label{eq:compound-term}
\end{align}
where
$n \geq 0$ is a length,
index $u$ is a non-negative integer,
integer $w$ represents a numeric term,
% ------------------------------------------------------------
\begin{wrapfigure}{r}{0.25\textwidth}
%\vspace{-15pt}
\frame%
{\begin{minipage}{0.245\textwidth}
\lstinputlisting[numbers=right,basicstyle=\ttfamily\scriptsize,xleftmargin=1pt]{code/d.aspif}%
\end{minipage}}%
\caption{\label{aspif:diff}\aspif\ format}
\vspace{-42pt}
\end{wrapfigure}
% ------------------------------------------------------------
%
string $s$ of length $n$ represents a symbolic term (including functions) or an operator,
integer $t$ is either \texttt{-1}, \texttt{-2}, or \texttt{-3} for tuple terms in parentheses, braces, or brackets, respectively, or an index of a symbolic term or operator, and
each $u_i$ is an integer for a theory term.
%
Statements (\ref{eq:number}), (\ref{eq:symbols}), and (\ref{eq:compound-term})
capture
numeric terms,
symbolic terms, % and function symbols, 
as well as
compound terms (tuples, sets, lists, and terms over theory operators).
% respectively.

Fifteen theory terms are given in Lines~12--26 of Figure~\ref{aspif:diff}.
Each of them is identified by a unique index in the third spot of each statement.
While Lines~12--20 stand for primitive entities of type (\ref{eq:number}) or (\ref{eq:symbols}),
the ones beginning with '\lstinline[mathescape=t]{9$\Space$2}' represent compound terms.
For instance, Line~21 and 22 represent \lstinline{end(1)} or  \lstinline{start(1)}, respectively,
and Line~23 corresponds to \lstinline{end(1)-start(1)}.

\myparagraph{Theory atoms} are represented using the following statements:
\begin{align}
\texttt{\theDUID} \Space \texttt{4} & \Space v \Space n \Space u_1 \Space \dots \Space u_n \Space m \Space l_1 \Space \dots \Space l_m \label{eq:theory-element}\\
\texttt{\theDUID} \Space \texttt{5} & \Space a \Space p \Space n \Space v_1 \Space \dots \Space v_n \label{eq:theory-atom}\\
\texttt{\theDUID} \Space \texttt{6} & \Space a \Space p \Space n \Space v_1 \Space \dots \Space v_n \Space g \Space u_1 \label{eq:theory-atom-bounded}
\end{align}
where
$n \geq 0$ and $m \geq 0$ are lengths,
index $v$ is a non-negative integer,
$a$ is a positive literal or \texttt{0} for directives,
each $u_i$ is an integer for a theory term,
each $l_i$ is an integer for a literal,
integer~$p$ refers to a symbolic term,
each $v_i$ is an integer for a theory atom element, and
integer~$g$ refers to a theory operator.
%
Statement (\ref{eq:theory-element}) captures elements of theory atoms and directives, and
statements (\ref{eq:theory-atom}) and (\ref{eq:theory-atom-bounded}) refer to the latter.
% capture theory atoms and directives.

For instance,
Line~27 captures the (single) theory element  in `\lstinline+{ end(1)-start(1) }+',
and
Line~29 represents the theory atom `\lstinline[morekeywords={&diff},alsoletter={\&}]+&diff { end(1)-start(1) } <= 200+'.

\myparagraph{Comments} have form
\addtocounter{DUID}{1}
\[\texttt{\theDUID} \Space s\]
where $s$ is a string not containing a newline.

The \aspif\ format constitutes the default output of \gringo~5.
With \clasp~3.2,
ground logic programs can be read in both \smodels\ and \aspif\ format.
%%% Local Variables:
%%% mode: latex
%%% TeX-master: "paper"
%%% End:
