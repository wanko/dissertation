\com{RELATED WORK}
\begin{itemize}
\item 
A first step towards a more flexible ASP infrastructure,
was done with \clingo~4~\cite{gekakasc14b} by introducing APIs for \lua\ and \python\ for multi-shot solving.

Although this allows for fine-grained control of complex ASP reasoning processes,
its functionality provided no access to \clasp's propagation and was restricted to inspecting stable models.
\item several systems extended ASP with theories and can benefit from an uniform intermediate language
and easy to access propagation techniques
\item boosting development of \dlvhex{}
\item currently, \dlvhex~\cite{eifikrre12a} is build upon \clasp's internal interface 
\item the new interface can help to generalize and extend the propagation techniques of the \dlvhex{}~\cite{eifikrre12a} system
\item by having watches on non-theory atoms and also having an interface which is similar to a standard SMT solver interface
using the propagate and undo method
%\item currently it only implements a theory check on a complete assignment
%\item this is similar to our check method, while not having undo and propagate
\item our new interface can also be used to give access to thread local propagation techniques
%%%\item relation to \DLVHEX{}/CLINGCON
%%%  \begin{itemize}
%%%  \item provides the infrastructure for advanced hybrid solvers, like ...
%%%  \item hinders progress as witnessed by the fact that the newest \dlvhex{} version is based on clingo-3.0.4\comment{BK: NOT TRUE!!! Uses clasp-3.1.4 and gringo-4.5.4}
%%%	\begin{itemize}
%%%	\item \Dlvhex{}
%%%	\item following~\cite{eifikrre12a} \Dlvhex{} implements only theory check on a complete assignment
%%%	\item theory checking is therefore not thread local but global
%%%	\item one theory check regardless of number of threads, our interface supports multi-threading natively 
%%%	\item this is similar to our check method, while not having undo and propagate
%%%	\item it therefore does not allow for stateful propagators or theory propagation
%%%	\item uses \clasp's internal interface
%%%	\item has C++ and python interface
%%%	\item release 2.4.0  on 24. Sep 2014, gringo 4.4.0 clasp 3.1.0
%%%	\item following the webpage, since 12 of April new \Dlvhex{} has support for propagation on partial assignments,
%%%	uses gringo-4.5.4 and clasp 3.1.4
%%%	\item this has the functionality of our propagate method + check method
%%%	\item no equivalent to our undo method, which allows for stateful propagators
%%%	\item gives an easy to use interface to a small part of the internal clasp external propagator interface
%%%	\item no support for multishot solving
%%%	\end{itemize}
%%%  \item rely upon internal interfaces and inter-communication between developers
%%%  \item now: C- and Python interfaces
%%%  \item more efficient since use C++ interface
%%%  \item Python rapid prototyping
%%%  \item semantically well founded approach abstracting away from the implementation level
%%%  \item unlike this, we put forward a simple low-level interface
%%%  \item boost \DLVHEX{}
%%%  \end{itemize}
\item Our new intermediate language \aspif{} can be used to generalize several approaches
\item Default Reasoning with Constraints~\cite{bakakaossc16a} uses new \aspif{} format and translates from \aspif{} to \aspif,
which can be read by the \clingcon{} system
\item \clingcon
  \begin{itemize}
	\item new version of \clingcon{} already uses the \aspif{} format to separate the grounding from the solving process
	and to remove the necessity to extend the parsing frontend of \gringo
	\item grounding of the theory is now completely decoupled from \clingcon{} and actually independent of the theory
%	\item uses internal interface for complex propagator
%	\item comprehensive preprocessing library in C++
%	\item tight coupling with the solver to introduce new solver variables on the fly (this is not yet possible with our new \python\ interface) (future work)
	\end{itemize}
\item \sysfont{inca}~\cite{drewal12a} currently uses the internal interface of \clasp{} and could be updated to
use \aspif{} to separate the grounding and solving process,
which is currently interleaved in a \clingo{} like fashion
\item also, all features available in \sysfont{inca} can be captured using our propagator interface
\item tools like \sysfont{dingo}~\cite{jalini11a} which translates ASP to SMT,
and allow an easy incorporation of other constraints,
benefit from the use of a standardized (intermediate) language which is able to handle theories natively
\item ASPMT~\cite{barlee14b}\comment{crossref does not work} develops an approach combining ASP with theories,
while the intermediate format could be used to represent the grounded version of these programs,
the propagator interface could be used to complement existing implementations
\begin{itemize}
  \item future work:
  \begin{itemize}
   \item propagators for multiple theories might be used in conjunction like CSP and \DL{} to efficiently handle heterogeneous problems and benefit from the strength of both approaches
   \item optimization on the theory itself might also be realized using propagators, e.g. Pareto optimality for Design Space Exploration or minimization of latency for shop scheduling
   \item advantage of this method is avoiding the grounding bottleneck of a direct representation of the preference functions
   \item propagator interface might be extended to apply domain specific heuristics to the solving process
   \item extending propagator interface to dynamically add new atoms during the solving process
  \end{itemize}
  \item mining uses propagator? \comment{roland?}
\item 
The new interface structure cannot yet cater to all needs but it is a step into an ASP service architecture. Ho ho ho!
\end{itemize}

\end{itemize}

\com{EXPERIMENTS}
\begin{itemize}
 \item generality of interface with example Difference Logic (\DL)
 \item compare features inherent to ASP with \emph{defined} and \emph{external} theory atoms
 \item Difference Logic:
    \begin{itemize}
     \item we define a \DL\ problem $C$ as a set of constraints of the form $x_1 - x_2 \leq k$,
           where $x_1$ and $x_2$ are variables over $\mathbb{N}$ and $k\in \mathbb{N}$
     \item let $G(C)$ define the directed, weighted constraint graph, 
           where the nodes of the graph are all variables contained in $C$
           and for every $x_1 - x_2\leq k \in C$ there exists an edge from $x_1$ to $x_2$ with weight $k$ 
     \item $C$ is consistent in \DL\ iff $G(C)$ contains no negative cycles
     \item we identify \DL\ atoms in $\mathcal{T}^{\DL}$ with constraints in $C$ via a function $\gamma:\mathcal{T}^{\DL}\rightarrow C$ 
     \item a set of theory atoms $S^\DL\ \subseteq \mathcal{T}^{\DL}$ is \DL\ consistent 
           iff $G(\{\gamma(a) \mid a \in S^\DL\})$ has no negative cycle, \DL\ inconsistent otherwise
     \item we define the set of nogoods $\Delta_\DL=\{\{\Tsign{a}\mid a\in S\}\mid\ S \subseteq 2^{\mathcal{T}^{\DL}}, 
                S\text{ is \DL\ inconsistent, }\nexists S'\subset S.S'\text{ is \DL\ inconsistent} \}$
     \item given a (partial) assignment $A$, $\gamma(\{a \mid \Tsign{a} \in A\})$ is \DL\ consistent iff $A$ does not violate any nogood in $\Delta_\DL$
     \item this definition implies that we use a \emph{non-strict} semantics
     \item we use a conflict optimal propagator $\Pi_{\Delta_\DL}$ as defined in~\cite{drewal12a}
     \item \clingo~5's theory interface allows for stateful and stateless propagators:
     \begin{itemize}
	\item stateless
	\begin{itemize}
	 \item build constraint graph and check for negative cycles with single-source shortest path (SSSP) algorithm Bellman-Ford \cite{be58a,fofu62a}
	\end{itemize}
	\item stateful
	\begin{itemize}
	 \item add edges iteratively to the constraint graph
	 \item update potential function for each node/variable and check for negative cycles as proposed in \cite{coma06a}
	 \item constraint graph is maintained and updated until conflict or solution is found 
	\end{itemize}
	\item only literals representing edges/constraints in the negative cycle are returned as conflict clause
	\item the length of the shortest paths and the potential function values respectively 
	      represent valid assignments of the variables if constraint graph contains no negative cycle
     \end{itemize}
     \item no theory propagation as defined in \cite{niolti06a}
    \end{itemize}

 \item benchmark set on shop scheduling \cite{ta93a}, specifically created to be challenging and test scheduling algorithms:
    \begin{itemize}
     \item flow shop:
	\begin{itemize}
	 \item find permutation of $n$ jobs that have to be executed on $m$ machines that minimizes the make-span
	 \item each job has to be assigned to the machines in order $1\dots m$
	 \item the job order is the same on all machines
	\end{itemize}
     \item job shop:
	\begin{itemize}
	 \item $n$ jobs have to be placed on $m$ machines so that the make-span is minimal
	 \item each job has an assigned sequence in which it has to be scheduled on the machines
	 \item order of jobs on machines may differ
	\end{itemize}
     \item open shop:
	\begin{itemize}
	 \item $n$ jobs have to be placed on $m$ machines so that the make-span is minimal
	 \item same as job shop but sequence of job on different machines may be arbitrary
	 \item a job cannot be executed on two machines simultaneously 
	\end{itemize}
    \end{itemize}
    
  \item encodings:
    \begin{itemize}
     \item decision variant of the problems: find a valid schedule that assigns the lowest start times possible to each task on each machine given an ordering of the tasks
     \item ASP: optimized encodings without theory atoms
     \item ASP modulo \DL\ with theory atoms that are \emph{defined}:
	  \begin{itemize}
	    \item atoms over {\tt seq/3} signify order of tasks and execution time in between, decided by the ASP part, 
	    e.g. {\tt seq(t1,t2,3)} means start time of task t2 needs to be at least 3 time units after start time of task t1
	    \item {\tt \&diff\{T1-T2\} <= -Time :- seq(T1,T2,Time).}
	    \item since the theory atoms are strictly defined via the encoding, techniques like theory propagation may not be used
	    \item avoids symmetries and suboptimal solutions
	  \end{itemize}
     \item ASP modulo \DL\ with theory atoms that are \emph{external}:
	  \begin{itemize}
	    \item {\tt :- not \&diff\{T1-T2\} <= -Time, seq(T1,T2,Time).}
	    \item theory propagation is possible
	    \item introduces symmetric, possibly unwanted solutions
	  \end{itemize}
    \end{itemize}

  \item benchmarks on a Intel Xeon E5520 8x2.27GHz CPU and 24GB of memory
running Debian GNU/Linux 7.9 (wheezy),
600 seconds timeout and 6GB memory restriction
  \item Table~\ref{table:bench:encoding_comparisson_detailed} compares average run times, column \emph{T}, and number of timeouts, column \emph{TO}, for different encodings and propagator implementations
  \begin{itemize}
    \item the rows are groups of task and machine numbers and the second column shows the number of instances in the group\comment{only relevant if we take second larger table}
    \item groups are ordered by task number
    \item groups are partitioned by the timeouts of the different encodings and propagators except for job shop
    \item all group sizes are multiple of ten because ten instances have the same amount of tasks and machines for each class, 
          and a combination of encoding and propagation technique is either able to handle all instances of that size or none
    \item exponential scaling of average run time with increasing task number
  \end{itemize}
  \item stateful with \emph{defined} atoms the fastest, most efficient consistency check and no symmetric/unwanted solutions in the search space
  \item vanilla ASP cannot solve most of the instances since direct representation of numeric variables leads to large grounding
  \item largest advantage of \emph{defined} theory atoms for flow shop, because its solutions are the most constraint, many unnecessary choices with \emph{external} theory atoms
  \item advantage gets smaller in job shop and is virtually gone for open shop since more choices are open to place the tasks
  \item stateless approach leads to all timeout for job shop since even the smallest instances have many variables, 
        creating constraint graph from scratch and restarting SSSP algorithm at each propagate is inefficient
  

\end{itemize}
