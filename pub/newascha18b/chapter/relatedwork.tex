%\section{Related Work}
%\label{sec:relatedwork}
%%System level design space exploration (DSE) has been of great scientific interest in the last two decades. In this section, we will provide an overview of state-of-the-art \ac{DSE} techniques proposed throughout the years. \par
%%Essentially, \ac{DSE} approaches can be characterized into two types \cite{Pimentel2017}: First, (meta-)heuristics like evolutionary algorithms, particle swarm and ant colony optimization (e.g.~\cite{Thompson2013,Ferrandi2010}) and second, exact methods such as \ac{ILP} and branch-and-bound algorithms (e.g.~\cite{Lukasiewycz2008,Khalilzad2016}). \par 
%%Most of the works presented in the field of meta-heuristics extend basic techniques in order to respect domain specific characteristics. For example, in \cite{Thompson2013}, the authors extend genetic algorithms by utilizing domain knowledge. They state, that small differences in design decisions lead to similar system implementations and that symmetrical design points can be pruned from the search space. They overcome the problem of finding infeasible initial solutions by only considering homogeneous architectures where all tasks are executable on every processor. \par 
%%Another approach (e.g. \cite{Neubauer2016,Schlichter2006}) of handling the infeasibility problem is to integrate dedicated constraint solvers into a \ac{MOEA}. The work of Schlichter et al. \cite{Schlichter2006} integrates, for example, a \ac{SAT} solver into a \ac{MOEA}. Here, the decisions are not directly controlled by the randomized search algorithm of the \ac{MOEA} but the heuristic of the decision variables is subject to exploration. This way, solutions are guaranteed to be feasible.\par
%%Finally, fully exact methods have been developed to explore the design space systematically. While meta-heuristics normally only cover a limited portion of the design space, exact methods are guaranteed to find the optimal solutions. Nevertheless, for a long time those methods were restricted to single-objective optimization problems only. As one of the few exceptions, Lukasiewycz et al.  \cite{Lukasiewycz2008} present a complete multi-objective Pseudo-Boolean solver based on branch-and-bound algorithms. The results show that this technique is able to find the proven optimal solutions for small problems in a short time. However, in contrast to the work at hand, this approach does not allow the optimization of non-linear objectives. Moreover, exact methods are often replaced in favor of heuristic approaches as the complexity of large systems hinders reasonable employment of those techniques. To tackle this issue, in this work we apply domain specific heuristics that have been shown to steer \cite{Andres2015} the solver into promising design regions. \par
%%Jia et al. \cite{Jia2014} propose a two-phased approach to handle the complexity of embedded systems design more efficiently. In the first step, they evaluate design points with fast analytical models to prune the search space quickly. In a subsequent step, those design points are simulated to get more accurate estimations.
%\asprin{} (\emph{ASP} for \emph{pr}eference handl\emph{in}g) has been proposed in \cite{Brewka2015} as a framework to define and compute preferred (optimal) solutions among stable models of logic programs. In addition to single-objective preference types such as cardinality minimization, it also offers predefined composite (i.e.~multi-objective) preference types like Pareto optimization. As \asprin{} is based on the \ac{ASP} solver \clingo{} \cite{gekakaosscwa16a}, it is only efficient for linear constraints. In this work, we propose an optimization framework to \clingo{} that utilizes the input language of \asprin{} and allows non-linear preference types. Therefore, we integrate background theories that handle both non-linear constraints solving and Pareto filtering. 